\documentclass{article}

\usepackage{enumitem}
\usepackage{amsmath}
\usepackage{amsfonts}
\usepackage{amssymb}
\usepackage{textcomp}
\usepackage{gensymb}
\usepackage[dvipsnames]{xcolor}
\usepackage[margin=0.5in]{geometry}
\usepackage[hidelinks,bookmarks=false]{hyperref}

\usepackage{environ}
\NewEnviron{centerframebox}{\begin{center}\fbox{\parbox{0.92\textwidth}{\BODY}}\end{center}}

\hypersetup{
  colorlinks=true,
  urlcolor=MidnightBlue,
  linkcolor=WildStrawberry,
  % pdfborderstyle={/S/U/W 2}
}

\title{Cryptography \\ Exercise Sheet 6}
\author{
  AAAAAAAAAA AAAAAAA \\
  \href{mailto:AAAAAAAAAAAAAAAAAAAA}{AAAAAAAAAAAAAAAAAAAA}
}

\begin{document}
  \maketitle

  \setcounter{section}{6}
  \subsection{AtE and died: confidentially poisoned}
  \begin{centerframebox}
    Suppose we use an encryption scheme with encryption function $ENC_{K_e}$
    and a message authentication scheme with tagging function $MAC_{K_a}$.

    Now, suppose additionally that the encryption XORs something on the plaintext (like a one-time-pad),
    and define a variant $ENC^*_{K_e}$ of this encryption function as follows:
    first replace every 0-bit by two bits 00 and every 1-bit by two bits 01 or 10,
    choose randomly each time, next encrypt with $ENC_{K_e}$.
    For the decryption we translate 00 back to 0, 01, 10 and also 11 to 1. So we send $ENC^*_{K_e} (m| MAC_{K_a} (m))$.
  \end{centerframebox}
  \subsubsection{Prove (at least, argue) that $ENC^*_{K_e}$ is still IND-CPA secure.}
  We can use Exercise 5.1, where we had two schemes and combined them,
  and if one of them is IND-CPA, then the combined one is also IND-CPA.
  The first scheme is our original $ENC_{K_e}$, and the second ``scheme'' is the bit replacing code we use to construct $ENC^*_{K_e}$.
  So the combination, which is also $ENC^*_{K_e}$, is IND-CPA secure, because $ENC_{K_e}$ is IND-CPA secure.

  \subsubsection{Ruthless attacker}
  \begin{centerframebox}
    Suppose that a ruthless person, called Rudiger, has overheard the
    messages of your login to some server which was done by sending
    the password. Of course, your password was authenticated and encrypted, as all messages. Now, Rudiger takes the transmission of
    your password and resends it with a bit pair in the cipher text inverted.
    \begin{enumerate}[label=(\alph*)]
      \item How does the recipient react if the original bit was 0?
      \item How does the recipient react if the original bit was 1?
    \end{enumerate}
    Conclude that Rudiger learns the bit from the reaction of the server
    (and thus your passwords after enough trials).
  \end{centerframebox}
  If the original bit was zero, the bits 00 get inverted to 11, which translates to 1.
  So the message changes, and the recipient will notice and throw an error.
  But if the bit in one, then the bits 01 (or 10) get inverted to 10 (or 01), and this still remains 1.
  The message doesn't change, and the recipient accepts it.
  So the attacker, by changing one of the bit pairs at random, can figure out an arbitrary bit of the message,
  and one by one, the whole password, which is just sent as part of the message.

  \subsubsection{Show that the composed scheme is not IND-CCA secure.}
  For IND-CCA we are given an oracle to decrypt any ciphertext, except for the exact one we need to distinguish.
  But here we can flip one of the bit pairs, and with probability one half it's going to be a 1 bit,
  and thus feeding that message to the oracle will successfully decrypt it, without triggering the loss condition.
  And we get a non-negligible $\frac{1}{2}$ advantage.
  That's all we need, but we can try calling the oracle more times, to get a 100\% win rate.

  \subsubsection{Is SSH better?}
  \begin{centerframebox}
    In SSH we transmit $ENC_{K_e}(m) | MAC_{K_a} (m)$, so we authenticate and
    encrypt (rather than first authenticating and second encrypting). Is
    that better? [Try to use $ENC^*_{K_e}$ here.]
  \end{centerframebox}
  We can still recover all the bits of $m$, but now the bits of $MAC_{K_a} (m)$ are public.
  Which doesn't really change anything.
  Unless we don't authenticate with a password, but with some asymmetric key.

  \subsection{All the papers and youtube videos...}
  Yeah, I didn't read those...

\end{document}
