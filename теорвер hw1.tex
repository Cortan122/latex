\documentclass{article}
\usepackage{122}

\title{Теория вероятности \\ Домашнее задание №1}
\author{\AA{AAAAA AAAAAAA}{4} \\ \AA{AAAAAA}{11} \\ Вариант 3}

\begin{document}
  \maketitle

  \section{Какова вероятность угадать в спортлото 5 чисел? (из 49-ти чисел, среди которых 6 -- выигрышных, выбираются случайным образом 6 чисел)}
  мы можем допустим сказать что выигрышные числа у нас всегда первые 6 \\
  вероятность выигрыша от этого не меняется \\
  количество способов достать ровно 5 чисел будет $C^5_6 \cdot C^1_{\l(49-6\r)} = 43 \cdot 6 = 258$ \\
  всего способов вытащить 6 шаров будет $C^6_{49} = 13983816$ \\
  способов достать все 6 выигрышных шаров у нас очевидно 1 (но Федя сказал что это несчитается) \\
  ответ: \fbox{$\ds \f{258}{13983816} = \f{43}{2330636} \approx 0.0000184$}


  \section{Противник может применить ракеты трех типов ($A$, $B$ и $C$) с такой вероятностью: $P(A) = 0.3$; $P(B) = 0.6$; $P(C) = 0.1$. Вероятность сбить ракеты этих типов равны соответственно $P(A_s) = 0.6$; $P(B_s) = 0.8$ и $P(C_s) = 0.9$. Известно, что противник применил ракету одного из трех типов.}

  \subsection{Определить вероятность того, что ракета будет сбита}
  мы тут просто умножаем и складываем соответствующие вероятности \\
  $P(S) = P(AA_s) + P(BB_s) + P(CC_s) = P(A)P(A_s) + P(B)P(B_s) + P(C)P(C_s)$ \\
  потомучто у нас все события либо несовместны либо независимы \\
  ответ: \fbox{$P(S) = 0.3 \cdot 0.6 + 0.6 \cdot 0.8 + 0.1 \cdot 0.9 = 0.75$}

  \subsection{Если ракета сбита, то определить наиболее вероятный ее тип}
  $$ P(X|Y) = \f{P(XY)}{P(Y)} $$
  $\ds P(A|S) = \f{P(AA_s)}{P(S)} = \f{0.3 \cdot 0.6}{0.75} = \f{0.18}{0.75} = 0.24$ \\
  $\ds P(B|S) = \f{P(BB_s)}{P(S)} = \f{0.6 \cdot 0.8}{0.75} = \f{0.48}{0.75} = 0.64$ \\
  $\ds P(C|S) = \f{P(CC_s)}{P(S)} = \f{0.1 \cdot 0.9}{0.75} = \f{0.09}{0.75} = 0.12$ \\
  ответ: \fbox{$B$}


\end{document}
