\documentclass{article}

\usepackage{enumitem}
\usepackage{amsmath}
\usepackage{amsfonts}
\usepackage{amssymb}
\usepackage{textcomp}
\usepackage{gensymb}
\usepackage[dvipsnames]{xcolor}
\usepackage[margin=0.5in]{geometry}
\usepackage[hidelinks,bookmarks=false]{hyperref}

\usepackage{environ}
\NewEnviron{centerframebox}{\begin{center}\fbox{\parbox{0.92\textwidth}{\BODY}}\end{center}}

\hypersetup{
  colorlinks=true,
  urlcolor=MidnightBlue,
  linkcolor=WildStrawberry,
  % pdfborderstyle={/S/U/W 2}
}

\title{Cryptography \\ Exercise Sheet 4}
\author{
  \AA{AAAAAAAAAA AAAAAAA}{6} \\
  \href{mailto:\AA{AAAAAAAAAAAAAAAAAAAA}{7}}{\AA{AAAAAAAAAAAAAAAAAAAA}{7}}
}

\begin{document}
  \maketitle

  \setcounter{section}{4}
  \subsection{More encryption schemes: IND-CPA?}
  \begin{centerframebox}
    Let $F$ be a pseudorandom function and $G$ a pseudorandom generator with expansion factor $\ell(\kappa) = \kappa + 1$.
    For each of the following encryption schemes, classify whether the scheme is insecure, IND-POA secure (but not IND-CPA) or IND-CPA secure.
    In each case, the shared key $k$ is chosen uniformly random, $k \in \{0,\, 1\}^\kappa$.

    % TODO: define a pseudorandom function
  \end{centerframebox}
  \subsubsection{To encrypt $m \in \{0,\, 1\}^{2\kappa+2}$ send $m \oplus (G(k) \mid G(k + 2))$}
  This scheme is definitely not IND-CPA secure, because it is deterministic.
  It looks like a variation on the pseudorandom one-time-pad, except we call the generator twice,
  and concatenate the results.
  But because the second time we use a different key $k + 2$,
  there shouldn't be any significant corelation between $G(k)$ and $G(k + 2)$ that the attacker can exploit.

  \textbf{Verdict}: IND-POA secure.

  \subsubsection{To encrypt $m \in \{0,\, 1\}^{\kappa+1}$ choose a random $r \leftarrow \{0,\, 1\}^\kappa$ and send $[r,\, G(r) \oplus m]$}
  This is not even an encryption scheme. It doesn't use the key. At all!
  Anyone can decrypt it by grabbing the first half of the ciphertext, running it thru the generator, and XOR-ing that with the second half of the ciphertext.

  \textbf{Verdict}: insecure.

  \subsubsection{To encrypt $m \in \{0,\, 1\}^{\kappa}$ send $m \oplus F_k(0^\kappa)$}
  This scheme is also deterministic, and not IND-CPA secure.
  The function $F_k(0^\kappa)$ can be rewritten as just $G'(k)$,
  by first swapping the arguments for the pseudorandom function $F_k(0^\kappa) \Rightarrow F_{0^\kappa}(k)$,
  and then using the property\footnote{Literally the definition of a pseudorandom function},
  that for all keys $\forall k : F_k(x) \equiv G'(x)$ where $G': \{0,\, 1\}^{\kappa} \to \{0,\, 1\}^{\kappa}$ cannot be distinguished from a truly random function.
  %
  After all those transformations we are left with just the pseudorandom one-time-pad again\footnote{Not even modified this time},
  and we already know it is IND-POA secure.

  \textbf{Verdict}: IND-POA secure.

  \subsubsection{To encrypt $m \in \{0,\, 1\}^{2\kappa}$ choose a random $r \leftarrow \{0,\, 1\}^\kappa$ send $[r,\, m \oplus (F_k(r) \mid F_k(r + 1))]$}
  This scheme is nondeterministic and has a chance of being IND-CPA secure.
  We can assume that concatenating two separate outputs of a pseudorandom function doesn't lose us a significant amount of security.
  This is literally the way we construct our key when using CTR\footnote{CTR is counter mode. It uses a counter and encrypts it with $F_k$ to generate an XOR key for every block of the message. $F_k$ can be AES, for example.} mode (see Exercise \ref{sec:ctr}).

  For a more formal proof, we can use the hint from that exercise,
  instead of just assuming it is true\footnote{If that exercise came \textit{before} this one, we could just get away with that}.
  % TODO

  \textbf{Verdict}: IND-CPA secure.

  \subsection{PRF $\Rightarrow$ PRG}
  \begin{centerframebox}
    Let $F : \{0,\, 1\}^\kappa \to \{\{0,\, 1\}^\kappa \to \{0,\, 1\}^\kappa\}$, $k \mapsto F_k$
    be a pseudorandom function. Fix $w_0, w_1, w_2 \in \{0,\, 1\}^\kappa$, all different.
    Define $G(s) := F_s(w_0)|F_s(w_1)|F_s(w_2)$.
    Prove that $G$ is a pseudorandom generator.
  \end{centerframebox}
  We build a reduction from the pseudorandom generator game to the pseudorandom function game.
  This way, if the pseudorandom generator game has a big advantage,
  the pseudorandom function game will also that same advantage, and that will lead to a contradiction.
  If we can successfully build such a reduction,
  then $G$ can't be distinguished from a truly random function,
  because that will make $F$ not a pseudorandom function, which is a contradiction.

  Calling the pseudorandom function oracle with $w_0, w_1, w_2$ and feeding it to the distinguisher for $G$,
  and if it tells us that the value is pseudorandom,
  we can conclude that the function we were givin was also pseudorandom, and vice versa.

  \subsection{PRF argument swapping}
  \begin{centerframebox}
    Let $F$ be a keyed function. Define $H$ as the keyed function with `swapped arguments': $H_k(m) := F_m(k)$.
    Prove or disprove: If $F$ is a PRF, then $H$ also is a PRF.
  \end{centerframebox}
  % So if $H$ is not a pseudorandom function, there must exists a value $k=k'$, such that $H_{k'}(x)$ is not indistinguishable from a random function.
  % And when supplying $k'$ as an input to $F_k(k')$ there's some kind of correlation that between the values returned by all the different functions $F_k$
  % This means there exists an attacker $\mathcal{A}$ that when given an oracle $\mathcal{O}(x) = H_{k'}(x) = F_x(k')$ can distinguish that from a truly random oracle.

  Here we use the game hopping lemma, and a game $G_{HOF}$ that choses a hidden bit $h_{HOF}$ and supplies
  the pseudorandom function attacker $\mathcal{A}$ with either the function $F_k$ or $H_k$.
  This way we can bound the advantage of the $H$ game based on the advantage of the $F$ game,
  which we already know is negligible, because $F$ is a pseudorandom function.
  Thus $H$ will also be a pseudorandom function, as long as $\mathcal{A}$ cannot tell the difference between those two games.

  But because the attacker always receives a random looking string from the oracle,
  he can't tell if it came from the one or the other function.

  % \textcolor{ForestGreen}{
  %   i guess we can do some game hopping between the $H$ and $F$ games.
  %   and like the random bit will swap the order.
  %   but why would the advantage of the middle game be small?
  % }
  % TODO

  \subsection{Security of $F$-CTR}
  \begin{centerframebox}
    \textbf{Theorem.} If $F$ is a pseudorandom function then $F$-CTR, namely CTR
    mode with $F$ and the initial ctr is chosen randomly with the key, is IND-CPA secure.

    Prove this.

    \textit{Hint}: Combine the security proof for $\Pi^\textrm{rand}_F$
    with the idea for proving equivalence of IND-CPA and LOR-CPA\footnotemark{} security.
    Game hopping might be the nicest way to describe this.
  \end{centerframebox}
  \label{sec:ctr}
  \footnotetext{I think the difference between LOR and IND, is that in LOR the attacker can call the oracle has many times as she wants}
  The idea here is to use this ``equivalence of IND-CPA and LOR-CPA''
  to compare the case where in $\Pi^\textrm{rand}_F$ we only use the pseudorandom function once,
  to the case in $F$-CTR, where we can use it as many times as we want, or at least until the counter rolls over.
  So if IND-CPA is equivalent to LOR-CPA, and not just a subset, it means that being indistinguishable once,
  is the same as being indistinguishable with as many calls to the oracle as the attacker wants.

  We can copy the LOR-CPA game for $\Pi^\textrm{rand}_F$, but here, instead of letting the attacker be in control,
  we increase the counter each time he calls our oracle.
  And we can make use the game hopping lemma,
  because the attacker will not be able to tell whether he is dealing with a normal IND-CPA secure $\Pi^\textrm{rand}_F$ situation,
  or with our incrementing counter.

\end{document}
