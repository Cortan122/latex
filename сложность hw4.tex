\documentclass{article}
\usepackage{122}

\usepackage[many]{tcolorbox}
\newtcolorbox{cross}{blank,breakable,parbox=false,
  overlay={\draw[red,line width=5pt] (interior.south west)--(interior.north east);
    \draw[red,line width=5pt] (interior.north west)--(interior.south east);}}

\usepackage{algpseudocode}

% swap \epsilon (𝜀) and \varepsilon (𝜖)
\let\temp\epsilon
\let\epsilon\varepsilon
\let\varepsilon\temp

\usepackage{environ}
\NewEnviron{centerframebox}{\begin{center}\fbox{\parbox{0.92\textwidth}{\BODY}}\end{center}}

\newcommand{\RP}{\ensuremath{\mathsf{RP}}}
\newcommand{\PP}{\ensuremath{\mathsf{PP}}}
\newcommand{\BPP}{\ensuremath{\mathsf{BPP}}}
\newcommand{\ZPP}{\ensuremath{\mathsf{ZPP}}}
\newcommand{\coRP}{\ensuremath{\mathsf{coRP}}}
\newcommand{\QP}{\ensuremath{\mathsf{QP}}}

\newcommand{\NP}{\ensuremath{\mathsf{NP}}}
\newcommand{\coNP}{\ensuremath{\mathsf{coNP}}}
\newcommand{\TIME}{\ensuremath{\mathsf{TIME}}}
\newcommand{\EXP}{\ensuremath{\mathsf{EXP}}}
\newcommand{\NC}{\ensuremath{\mathsf{NC}^1}}
\newcommand{\PSPACE}{\ensuremath{\mathsf{PSPACE}}}
\renewcommand{\P}{\ensuremath{\mathsf{P}}}

\title{Теория сложности вычислений \\ Домашнее задание №9, №10 и №11}
\author{\AA{AAAAA AAAAAAA}{4} \\ \AA{AAAAAA}{13}}

\begin{document}
  \maketitle

  \setcounter{section}{8}
  \section{Домашнее задание №9}
  \setcounter{subsection}{8}
  % old version: \subsection{Show that the following problem is \NP-complete: given an oracle Turing machine $M$, decide whether there exists an oracle $A$, such that $M^A$ accepts the empty input.}
  \subsection{Show that the following problem is \NP-complete: given a number $t$ in unary and an oracle Turing machine $M$, decide whether there exists an oracle $A$, such that $M^A$ accepts the empty input in at most $t$ steps.}
  Поскольку мы должны рассмотреть все возможные варианты оракула $A$, можно притвориться, что машина $M^A$ -- недетерминированная.
  Каждый раз, когда машина $M$ делает запрос к оракулу $A$ с новым входом (повторяющиеся запросы, мы можем пропустить, потому что ответ оракула всегда будет одинаковый), она может пойти в одно из двух следующих состояний, примерно как недетерминированная машина Тьюринга.

  Представим дерево решений, которые может принимать машина $M^A$, основываясь на ответах $A$.
  Это дерево получается глубины $t$.
  Мы можем хранить оракул $A$ как список битов, на каждой развилке в дереве, и он будет длинны $t$.
  Этот список битов можно легко использовать в качестве сертификата, для проверки за \P{} время, потому что нам на вход даётся унарное число $t$ и длинна сертификата будет $O(n)$.
  Поэтому наша проблема в \NP.

  Теперь надо доказать что наша проблема \NP-hard.
  Тут мы можем взять любую проблему из \NP, и, поскольку она из \NP, существует недетерминированная машина Тьюринга $N$, которая решает эту проблему за полиномиальное время.
  Эту машину нам надо переделать из недетерминированной в оракульскую $B$.
  Для этого мы добавляем в неё счётчик, и каждый раз, когда машина $N$ делает недетерминированный шаг, машина $B$ увеличивает этот счётчик и отправляет его оракулу, и шаг в зависимости от его ответа.
  Так как машина $B$ будет работать за полиномиальное время, мы можем придумать такое число $t$, чтобы она гарантированно успела завершиться за столько шагов.
  Ещё надо не забывать, что у нас теперь есть входная строка, и надо её, уже во время исполнения, встроить в исходный код машины $B$, чтобы она заполняла ей пустую входную ленту.

  Таким образом мы можем привести любую проблему из \NP{} к нашей за полиномиальное время.
  $\NP \cap \NP\textrm{-hard} = \NP\textrm{-complete}$.
  $\blacksquare$

  \subsection{Show that $\QP^\QP = \QP$. Also prove that $\EXP^\EXP \neq \EXP$. Here $\QP = \bigcup^{+\infty}_{k=1} \TIME(2^{\log^k n}).$}
  Чтобы понять будет ли принадлежать оракульская комбинация двух классов какому-то классу, мы можем попробовать симулировать каждый вызов оракула, и оценить временные затраты этого процесса.
  Тогда затраченное время для симуляции $\TIME(f)^{\TIME(g)}$ будет меньше $O(f(n) \cdot g(f(n)))$, поскольку мы можем вызывать оракул максимально $f(n)$ раз, то есть на каждом шаге, и максимальная длинна входного слова, которое мы сможем в него передать, будет равна максимально возможной использованной памяти, то есть, в нашем случае, $f(n)$.

  % 2^{log^k n} * 2^{log^l m} = 2^{log^k n + l log^k n} = 2^{l log^k n} = 2^{log^k n}
  % m < 2^{log^k n}
  \noindent
  Попробуем этот приём на $\QP^\QP$. \\
  $\ds \QP^\QP$ \\
  $\ds \TIME\l(2^{\log^k n}\r)^{\TIME\l(2^{\log^l m}\r)}$ \\
  $\ds \TIME\l(2^{\log^k n} \cdot 2^{\log^l m}\r)$ \\
  $\ds \TIME\l(2^{\log^k n + \log^l m}\r)$ \\
  $\ds \TIME\l(2^{\log^k n + \log^l \l(2^{\log^k n}\r)}\r)$ \\
  $\ds \TIME\l(2^{\log^k n + l \log^k n}\r)$ \\
  $\ds \TIME\l(2^{(l+1) \cdot \log^k n}\r)$ \\
  $\ds \TIME\l(2^{\log^k n}\r)$ \\
  $\ds \QP$ \\
  $\blacksquare$

  % 2^n * 2^m = 2^{n+m} = 2^{2^n}
  % m < 2^n
  \noindent
  Теперь тоже самое, только для $\EXP^\EXP$. \\
  $\ds \EXP^\EXP$ \\
  $\ds \TIME\l(2^{n^k}\r)^{\TIME\l(2^{m^l}\r)}$ \\
  $\ds \TIME\l(2^{n^k} \cdot 2^{m^l}\r)$ \\
  $\ds \TIME\l(2^{n^k + m^l}\r)$ \\
  $\ds \TIME\l(2^{n^k + 2^{n^l}}\r)$ \\
  $\ds \TIME\l(2^{2^{n^l}}\r)$ \\
  $\ds 2\textrm{-}\EXP$ \\
  $\ds \neq \EXP$ \\
  $\blacksquare$

  \subsection{Show that there is a language $B \in \EXP$ such that $\coNP^B \neq \NP^B$.}
  Возьмём язык $B = \{(\langle M \rangle,\, w,\, t) \mid$ Turing machine $M$ halts on input word $w$ in at most $t$ steps$\}$,
  где $t$ записано в двоичной системе.
  Этот язык будет в \EXP, потому что для симуляции мишины Тьюринга нужно $t$ шагов, а, так как число $t$ у нас записано в двоичной, а не в унарной, системе, это будет $O(2^{\log t}) = O(2^n)$.

  Возьмём определение \coNP, где недетерминированная машина должна вернуть true на \textbf{всех} ветках своего выполнения, а не на хотябы одной, как в обычном \NP.
  Тогда мы можем определить язык $C = \{(\langle M \rangle,\, 1^k,\, t) \mid \forall w \in \sigma^k$ Turing machine $M$ halts on input word $w$ in at most $t$ steps$\}$, то есть этот язык содержит такие машины Тьюринга, которые завершаются на всех своих входных словах длинны $k$ за $t$ шагов.
  Здесь число $k$ надо записать в унарной системе, чтобы это можно было решить \NP{} машиной, и не надо было лезть в какой-то $\mathsf{NEXP}$.
  Такой язык легко решается недетерминированным алгоритмом: надо создать по одной ветке для каждого возможного слова $w$, и потом запустить оракул $B$.
  Поэтому мы можем надёжно положить его в класс $\coNP^B$.

  Из-за сложности данной проблемы, её нельзя будет решить в $\NP^B$, потому что там в недетерминированных машинах другой квантор.
  Нам просто не хватит полиномиально многих вызовов к оракулу, чтобы проверить все возможные входные слова, а проверить принадлежность языку $B$ нам никак, кроме оракула (потому что $B \in \EXP\textrm{-complete}$).
  Получается $C \in \coNP^B \land C \not\in \NP^B \Longrightarrow \coNP^B \neq \NP^B$.

  \section{Домашнее задание №10}
  \setcounter{subsection}{8}
  \subsection{Show that $\ZPP = \RP \cap \coRP$.}
  \begin{centerframebox}
    The class \ZPP{} contains all the languages $L$ for which there exists a probabilistic machine such that for every input $x$ the following holds:
    \begin{itemize}
      \item the expected running time is bounded by a polynomial in $|x|$,
      \item whenever $M$ halts, the output $M(x)$ is exactly $L(x)$ ($1$ if $x \in L$ and $0$ if $x \not\in L$).
    \end{itemize}
  \end{centerframebox}
  % https://en.wikipedia.org/wiki/ZPP_(complexity)#Intersection_definition
  Рассмотрим любой язык $L \in \RP \cap \coRP$.
  Для него есть \RP{} алгоритм $A$, и \coRP{} алгоритм $B$.
  Чтобы решить этот язык \ZPP{} алгоритмом, мы можем по очереди запускать алгоритмы $A$ и $B$, пока один из них не даст нам точный ответ ($A$ даёт точный ответ <<Да>>, а $B$ -- точный ответ <<Нет>>).
  С каждой итерацией этого цикла, шанс в неё попасть уменьшается по экспоненте.
  Таким образом средние время выполнение этого алгоритма будет полиномиальным, и $\RP \cap \coRP \subseteq \ZPP$.

  Теперь нам надо доказать обратное вхождение.
  Мы можем привести любой алгоритм из \ZPP{} в \RP.
  Для этого надо просто запустить его, и подождать в два раза больше его среднего времени.
  Если он вернёт ответ, то мы можем его вернуть, а если нет, то мы можем вернуть <<Нет>>, и это будет с достаточно маленькой вероятностью, чтобы войти в \RP.
  Получается $\ZPP \subseteq  \RP \cap \coRP$, и, поскольку мы уже доказали $\RP \cap \coRP \subseteq \ZPP$, то можем сказать что $\ZPP = \RP \cap \coRP$.
  $\blacksquare$

  \subsection{Show that $\NP \subseteq \PP$.}
  \begin{centerframebox}
    The class \PP{} contains all languages $L$ for which there exists a probabilistic machine $M$ that runs in polynomial time on all computational paths and such that for any $x$ we have $x \in L$ if and only if $\Pr[M(x) = 1] > 1/2$.
  \end{centerframebox}
  % https://en.wikipedia.org/wiki/PP_(complexity)#PP_compared_to_other_complexity_classes
  Чтобы доказать, что $\NP \subseteq \PP$, мы докажем, что $3\textrm{-SAT} \in \PP$, и, поскольку $3\textrm{-SAT} \in \NP\textrm{-complete}$, класс \PP{} будет включать в себя все проблемы из \NP, через полиномиальное приведение.
  При решение проблемы 3-SAT вероятностным алгоритмом, мы можем случайно выбрать подстановку переменных, и проверить получается ли формула true.
  Если формула вернула true, то мы можем сразу вернуть true, а если нет, то надо вернуть true с вероятностью $\f{1}{2} - \epsilon$, где $\epsilon = \f{1}{2^{n+1}}$.
  Такой $\epsilon$ мы взяли чтобы вероятность угадать была больше него.

  Получается, в случае, когда формула у нас satisfiable, вероятность угадать правильную подборку переменных $\f{1}{2^n}$ (в худшем случае у нас ровно $n$ переменных и только одна подходящая комбинация), и в сумме вероятность вернуть true будет $\l(\f{1}{2} - \f{1}{2^{n+1}}\r) \cdot \l(1-\f{1}{2^n}\r) + \f{1}{2^n} = \f{1}{2} + \f{1}{2^{2n+1}} > \f{1}{2}$.
  А если формула unsatisfiable, то вероятность будет $\f{1}{2} - \f{1}{2^{n+1}} < \f{1}{2}$.
  $\blacksquare$

  \subsection{Show that if $\NP \subseteq \BPP$, then $\NP = \RP$.}
  % \begin{cross}
  %   Недетерминированные машины можно проанализировать как особый случай рандомных.
  %   Для получения вероятностной машины из недетерминированной, надо просто заменить все недетерминированные ветки на рандомные, и поменять квантор $\exists$ в условие, на неравенство с вероятностью.
  %   Получает такое новое определение класса \NP.

  %   \begin{centerframebox}
  %     The class \NP{} contains all languages $L$ for which there exists a probabilistic machine $M$ that runs in polynomial time on all computational paths and such that for any $x$ we have $x \in L$ if and only if $\Pr[M(x) = 1] > 0$.
  %   \end{centerframebox}

  %   Мы можем сделать ещё лучше.
  %   Поскольку количество шагов машины из \NP{} ограничено полиномом, можно ограничить снизу вероятность правильного ответа не нулём, а степенью двойки.
  %   Получается $\Pr[M(x) = 1] \geq 2^{-\mathrm{poly}(n)}$, так как количество возможных веток ограничивается экспонентой от количества шагов.
  %   Если мы используем 3-SAT, то мы точно знаем что (как и в предыдущей задачке) вероятность угадать правильный ответ будет $\f{1}{2^n}$.
  % \end{cross}

  Мы тривиально знаем что $\RP \subseteq \NP$, так как недетерминированные машины можно проанализировать как частный случай рандомных, и нам для равенства нужно доказать только $\NP \subseteq \RP$.
  Чтобы доказать такое включение, нам не обязательно проверять все задачи в \NP, а можно просто доказать включение в \RP{} для какой-то одной \NP-complete задачки, например 3-SAT.
  Прелесть 3-SAT-а заключается в том, что используя алгоритм для satisfiability мы можем восстановить какое-то конкретное решение, по одной переменной за раз.
  Для этого надо будет просто сделать предположение для значения этой переменной, и если от него 3-SAT перестаёт быть satisfiable, то мы угадали неправильно.

  У \BPP{} алгоритмов есть интересное свойство, что их можно запускать много раз подряд, и, таким образом, экспоненциально увеличивать вероятность правильного ответа.
  Например после $k$ итераций вероятность ошибки при выданном ответе <<Да>> будет $2^{-k}$, то есть $x \in L \iff \Pr[M^k(x) = 1] > 1-2^{-k}$.

  Используя эти два свойства, мы можем запустить \BPP{} алгоритм для 3-SAT по $k$ раз для каждой из $n$ переменных,
  и тогда у нас будет подстановка, которая правильная с довольно большой вероятностью.
  Дальше мы обычным детерминированным полиномиальным алгоритмом проверяем её правильность, и если она оказывается правильной, то возвращаем <<Да>>. Вероятность того, что мы найдём неправильную подстановку в худшем случае равна $(2^{-k})^n$, и, чтобы она была меньше $1/2$, надо взять $k = n+2$.

  Наш алгоритм, как любой послушный алгоритм из \RP, говорит <<Да>> только когда в этом полностью уверен, и делает ошибки с вероятностью меньшей половины.
  Получается $\textrm{3-SAT} \in \RP \Longrightarrow \NP \subseteq \RP \Longrightarrow \RP = \NP$.
  $\blacksquare$

  \section{Домашнее задание №11}
  \setcounter{subsection}{5}
  \subsection{Show that there exists a streaming algorithm that, given a stream of numbers $a_1,\, \dots,\, a_m$ from $[m] = \{1,\, ...,\, m\}$, outputs the maximal length of an increasing substring $a_i,\, a_{i+1},\, ...,\,a_j$ and uses $O(\log m)$ memory.}
  Для решения этой задачки можно придумать алгоритм, использующий только три ячейки памяти, размер каждой из которых $\log m$.
  Одна ячейка для хранения предыдущего значения $a_{i-1}$, вторая для длинны текущей подстроки и третья для длинны максимальной подстроки.
  Длины подстрок можно хранить в $\log m$ бит, потому что возрастающая подстрока никогда не может быть длиннее $m$, так как она тогда будет использовать все доступные элементы.

  \begin{center}
    \begin{minipage}{10cm}
      \hrulefill
      \begin{algorithmic}[1]
        \State $A \gets \textsc{blank}$ \Comment{Предыдущий элемент}
        \State $B \gets 0$ \Comment{Текущая длина}
        \State $C \gets 0$ \Comment{Максимальная длина}

        \While{$a_i$}
          \If{$A < a_i \lor A = \textsc{blank}$}
            \State $B \gets B+1$
            \If{$C < B$}
              \State $C \gets B$
            \EndIf
          \Else
            \State $B \gets 1$
          \EndIf
          \State $A \gets a_i$
        \EndWhile

        \State \Return $C$
      \end{algorithmic}
      \hrulefill
    \end{minipage}
  \end{center}

  \subsection{Given a number $n$ and a stream of unknown length containing items from $[m]$, you need to find all the items that occur more than $\epsilon n$ times among the last $n$ items of the stream (if the length of the stream is $k < n$, you need to find the items that occur more than $\epsilon k$ times in the stream). Describe a streaming algorithm that uses $O(1/\epsilon \cdot \log(n + m))$ memory and returns a set containing all such items (perhaps, among others). Explain why your algorithm is correct}
  Тут очень похоже на алгоритм из лекции, где надо было найти тот элемент, который встречался больше половины раз.
  Мы тоже храним счётчик ``популярности'' данного числа, увеличиваем его каждый раз когда оно нам попадается, и уменьшаем его когда попадается что-то другое.

  Но теперь у нас есть $\epsilon$, и от этого возникают некоторые сложности.
  Допустим $\epsilon \in \mathbb{Q}$ (мы же каким-то образом его записали, а если бы он был иррациональным, то записать его было бы сложно), тогда его можно представить как $\epsilon = \f{a}{b}$, где $0 < a < b$.
  Мы резервируем массив для $\lceil 1/\epsilon \rceil$ ячеек, в каждой из которых мы будем хранить значение элемента и счётчик популярности, которые в сумме займут $O(\log (mn))$ бит.
  Изменять значения этого счётчика нам надо будет несимметрично: когда встречается правильный элемент, мы добавляем $1-\epsilon$, а когда неправильный -- вычитаем $\epsilon$.
  Также этот считчик надо ограничить сверху $\epsilon n$, чтобы в ответе учитывалась только последняя часть потока, а не он весь.
  Но делать счётчик в рациональных числах мы не можем, поэтому нам придётся всё домножить на знаменатель $b$.

  Чтобы ограничить использованием наших ячеек памяти, мы выкидываем из них все элементы, счётчик которых достигает нуля.
  То как работают наши счётчики, и как мы подобрали количество ячеек гарантирует что нам их хватит.

  Также тут есть подвох насчёт O-шки.
  В условиях везде написано $O(\log (n+m))$, а по логике должно получиться $O(\log n + \log m) = O(\log (mn))$.
  Это немного неочевидно, но под O-шкой эти две записи будут эквивалентны.
  Чтобы это доказать надо сначала привести всё к одной переменной, например допустим что $m \geq n$.
  Тогда $O(\log(n+m)) = O(\log 2m) = O(\log m)$ и $O(\log n + \log m) = O(2\log m) = O(\log m)$, и мы всё доказали.
  Есть ещё небольшая проблема, что там может появляться фактор числителя эпсилона под логарифмом.
  От этого можно избавиться заменив эпсилон на ближайшую под ним дробь с единичным числителем.
  Тогда нам даже не надо будет делать его рациональным.

  \begin{center}
    \begin{minipage}{13cm}
      \hrulefill
      \begin{algorithmic}[1]
        \State $\textrm{arr} \gets$ reserve $\lceil 1/\epsilon \rceil$ cells, $\log (mn)$ bits each

        \While{$c$} \Comment{$c$ это текущий элемент}
          \If{$c \in \textrm{arr}$} \Comment{есть ли он в списке}
            \State $\textrm{arr}_c \gets \textrm{arr}_c + b$ \Comment{будет $b-a$ после второго этапа алгоритма}
            \If{$\textrm{arr}_c > an$}
              \State $\textrm{arr}_c \gets an$
            \EndIf
          \Else
            \State $\textrm{arr}_c \gets b$ \Comment{выделяем новую ячейку}
          \EndIf

          \For{$d \in \textrm{arr}$}
            \State $\textrm{arr}_d \gets \textrm{arr}_d - a$
            \If{$\textrm{arr}_d \leq 0$}
              \State \textbf{delete} $\textrm{arr}_c$ \Comment{освобождаем ячейку}
            \EndIf
          \EndFor
        \EndWhile

        \State \Return $\textrm{arr}$
      \end{algorithmic}
      \hrulefill
    \end{minipage}
  \end{center}

\end{document}
