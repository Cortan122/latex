\documentclass{article}

\usepackage{amsmath}
\usepackage{amsfonts}
\usepackage{amssymb}
\usepackage[dvipsnames]{xcolor}
\usepackage[margin=0.5in]{geometry}
\usepackage[hidelinks, bookmarks=false]{hyperref}
\usepackage{enumitem}

\let\oldemptyset\emptyset
\let\emptyset\varnothing
\let\oldphi\phi
\let\phi\varphi
\let\oldepsilon\epsilon
\let\epsilon\varepsilon

\usepackage{environ}
\NewEnviron{centerframebox}{\begin{center}\fbox{\parbox{0.92\textwidth}{\BODY}}\end{center}}

\title{Algorithms for Data Science \\ Exercise Sheet 8}
\author{
  Vladislav Imashev \\ \href{mailto:s05vimas@uni-bonn.de}{s05vimas@uni-bonn.de} \and
  AAAAAAAAAA AAAAAAA \\ \href{mailto:AAAAAAAAAAAAAAAAAAAA}{AAAAAAAAAAAAAAAAAAAA} \and
  German Mikhelson \\ \href{mailto:s17gmikh@uni-bonn.de}{s17gmikh@uni-bonn.de} \and
  Aleksandra Volynets \\ \href{mailto:s02avoly@uni-bonn.de}{s02avoly@uni-bonn.de} \and
  Nikita Morev \\ \href{mailto:s99nmore@uni-bonn.de}{s99nmore@uni-bonn.de}
}

\begin{document}
  \maketitle

  \setcounter{section}{8}
  \subsection{Count-Min Sketch for Heavy Hitters}
  \begin{centerframebox}
    Using the method presented on Slides 32--36 of 2023-12-13, compute the $\phi$-heavy
    hitters for $\phi = 1/3$ for the following data stream:
    $$\sigma = \langle2,\, 1,\, 2,\, 2,\, 3,\, 3,\, 6,\, 6,\, 2,\, 5\rangle.$$

    In your solution, set the parameters of the Count-Min Sketch algorithm to $d = 3$
    and $w = 3$. Applying the method on Slide 38 of 2023-12-13 with $p = 3$ and
    $k = 2$, use the hash functions $h_{(1,2),1}, h_{(2,1),0}$, and $h_{(2,1),1}$.
    Give the state of all count-min sketches (i.e., the matrices) after the processing
    of the 1st, 2nd, 5th, and the last item. Give also all steps of the evaluation of
    the binary tree for the entire data stream (cf. Slide 32 of 2023-12-13).
  \end{centerframebox}
  TODO (Veronika; German)

  \subsection{The Median Trick}
  \begin{centerframebox}
    Prove Claim 4 on Slides 17--18 of 2023-12-20 for $s_1 = c \log\frac{1}{\delta}$ for some appropriate
    constant $c$.

    \textbf{Claim}: $\mathrm{Pr}\left[|\{i\in[s_1]:|Y_i-F_k| >\epsilon F_k\}| > \frac{s_1}{2}\right] \le \delta$

    \textbf{Remark}:
    The claim is stated for $c = 4$; the point is that $s_1 = O(\log\frac{1}{\delta})$ (cf. Slides 6 and 18 of 2023-12-20)

    \textit{Hint}: To prove the claim, you may use one of the Chernoff bounds stated below.
    In particular, one can show it for $c = 4$ by using (i).

    \textbf{Theorem}:
    Let $X_1,\, \dots,\, X_n$ be independent Poisson trials with
    $\mathrm{Pr}(X_i) = p_i$ (i.e. $\mathrm{Pr}(X_i = 1) = p_i$ and $\mathrm{Pr}(X_i = 0) = 1 - p_i$ for all $i \in [n]$).
    Let $X = \sum_{i=1}^n X_i$ and $\mu = \mathbb{E}[X]$.
    Then the following bounds hold:
    \begin{enumerate}[label=(\roman*)]
      \item For any $\gamma > 0$:
      \[ \mathrm{Pr}[X \geq (1+\gamma)\mu]  < \left(\frac{e^\gamma}{(1+\gamma)^{(1+\gamma)}}\right)^\mu\]

      \item For any $\gamma \in (0,\, 1)$:
      \[ \mathrm{Pr}[|X - \mu|\geq \gamma\mu] \leq 2e^{-\mu\gamma^2/3} \]
    \end{enumerate}
  \end{centerframebox}
  We know from Claim 3 that the probability we have is less then $\frac{1}{8}$,
  i.e. for every $i \in [s_1]$:
  \[ \mathrm{Pr}[|Y_i-F_k| > \epsilon F_k] \leq \frac{1}{8} \]
  We can define a boolean random variable $U_i$ to have that same $\frac{1}{8}$ probability:
  \[ U_i = \begin{cases}
    1 & \textrm{if } |Y_i-F_k| > \epsilon F_k \\
    0 & \textrm{if } |Y_i-F_k| \leq \epsilon F_k
  \end{cases} \]
  and apply The Theorem to it.
  Its expected value $\tilde{\mu} \leq \frac{1}{8}$ by claim 3,
  and its sum $\tilde{U} = |\{i\in[s_1]:|Y_i-F_k| >\epsilon F_k\}|$ be construction.
  By using (i), we also get:

  \[ \mathrm{Pr}\left[\tilde{U} \geq (1+\gamma)\tilde{\mu}\right] < \left(\frac{e^\gamma}{(1+\gamma)^{(1+\gamma)}}\right)^{\tilde{\mu}} \]
  for all $\gamma > 0$.
  Now we have to pick a correct $\gamma$ such that $(1+\gamma)\tilde{\mu} = \frac{s_1}{2}$.
  But first we'll need to use the $\tilde{\mu} \leq \frac{1}{8}$ claim to simplify this big inequality:

  \[ \mathrm{Pr}\left[\tilde{U} \geq (1+\gamma)\frac{1}{8}\right]
  \leq \mathrm{Pr}\left[\tilde{U} \geq (1+\gamma)\tilde{\mu}\right]
  < \left(\frac{e^\gamma}{(1+\gamma)^{(1+\gamma)}}\right)^{\tilde{\mu}}
  \leq \left(\frac{e^\gamma}{(1+\gamma)^{(1+\gamma)}}\right)^{\frac{1}{8}} \]
  \[ \mathrm{Pr}\left[\tilde{U} \geq \frac{1+\gamma}{8}\right] < \left(\frac{e^\gamma}{(1+\gamma)^{(1+\gamma)}}\right)^{\frac{1}{8}} \]
  Now $\frac{1+\gamma}{8} = \frac{s_1}{2}$ is a lot easier to think about,
  and we can safely set $\gamma = 4s_1-1$ and compute the right hand side fraction:

  \begin{align*}
    \mathrm{Pr}\left[\tilde{U} \geq (1+\gamma)\frac{1}{8}\right]
    &= \mathrm{Pr}\left[|\{i\in[s_1]:|Y_i-F_k| >\epsilon F_k\}| > \frac{s_1}{2}\right] \\
    &< \left(\frac{e^{4s_1-1}}{(4s_1)^{4s_1}}\right)^{\frac{1}{8}}\\
    &= \frac{e^{\frac{4s_1-1}{8}}}{(4s_1)^{\frac{4s_1}{8}}}\\
    &= \frac{e^{\frac{4s_1-1}{8}}}{(4s_1)^{\frac{4s_1}{8}}}\\
    &= e^{(4s_1 - 1 - \log (4s_1) 4s_1)/8}\\
    &= e^{(4s_1 (1 - \log (4s_1)) - 1 )/8}\\
    &< e^{-2s_1/8} \qquad\textrm{because }s_1\textrm{ is a positive integer} \\
    % i just plotted e^{4x(1-\ln(4x))-1} < e^{-2x} in desmos
    &= e^{-2c\log\frac{1}{\delta}/8}\\
    &= \delta \cdot e^{c/4}\\
  \end{align*}
  And if we pick $c=4$, how it says in The Hint, we well get our claim of:
  \[ \mathrm{Pr}\left[|\{i\in[s_1]:|Y_i-F_k| >\epsilon F_k\}| > \frac{s_1}{2}\right] < \delta \]
  which is the desired result, plus this time the bound is also not inclusive.
  % TODO (Sasha she/her; Vadim; Kostya)

  \subsection{The AMS Sketch}
  \begin{centerframebox}
    Generalize the AMS Sketch given on Slide 6 of 2023-12-20 to the \textit{cash register model},
    i.e., in which each token in the data stream over $[n]$ is of the
    form $(i, \Delta)$, where $i \in [n]$ and $\Delta > 0$ is some integer.
    For a data stream $\sigma = \langle(a_1, \Delta_1), \dots ,(a_m, \Delta_m)\rangle$ over $[n]$ in the cash register model, we define the
    frequency $f_i$ of $i$ by
    \[ f_i = \sum_{l\in[m],a_l=i} \Delta_l \]
    for all $i \in [n]$. Give the pseudocode of the generalized AMS Sketch for estimating
    $F_k (k \geq 1)$. Sketch the proof of the correctness of your algorithm. In your proof,
    you may rely on the proof of the vanilla case presented in 2023-12-20 (Claims 1--4),
    i.e., explain only what is new with respect to that proof.
  \end{centerframebox}

  \noindent
  \textbf{initialization}:
  \begin{enumerate}[topsep=0pt,itemsep=-1ex,partopsep=1ex,parsep=1ex]
    \item let $s_1 := 4 \log \frac{1}{\delta}$ and $s_2 := \frac{8kn^{1-1/k}}{\epsilon}$
    \item for $i = 1,\dots, s_1$ and for $j = 1,\dots, s_2$ do
    \begin{itemize}[topsep=0pt,itemsep=-1ex,partopsep=1ex,parsep=1ex]
      \item choose $p_{ij}$ independently at random from $[m]$, biased on the value of $\Delta_l$
      \item let $r_{ij} := 0$
    \end{itemize}
  \end{enumerate}
  \textbf{processing of $a_l$}:
  \begin{enumerate}[topsep=0pt,itemsep=-1ex,partopsep=1ex,parsep=1ex]
    \item forall $i \in [s_1]$ and forall $j \in [s_2]$: if $l \geq p_{ij}$ and $a_l = a_{p_{ij}}$ then increment $r_{ij}$ by $\Delta_l$
  \end{enumerate}
  \textbf{output:} return the median $F_k$ of $Y_1,\dots, Y_s$, where
  \[ Y_{i}=\frac{\sum_{j=1}^{s_{2}}X_{i j}}{s_{2}} \qquad\text{ and }\qquad
   X_{i j} = \left(r_{i j}^{k}-\left(r_{i j} - 1\right)^{k}\right) \sum_{l=1}^m \Delta_l \]

  TODO (Sasha he/him; Vlad)

\end{document}
