\documentclass{article}

\usepackage{enumitem}
\usepackage{amsmath}
\usepackage{amsfonts}
\usepackage{amssymb}
\usepackage[margin=0.5in]{geometry}
\usepackage[hidelinks, bookmarks=false]{hyperref}

\let\oldemptyset\emptyset
\let\emptyset\varnothing
\newcommand{\R}{\mathbb{R}}
\newcommand{\Rd}{\R^d}

\usepackage{environ}
\NewEnviron{centerframebox}{\begin{center}\fbox{\parbox{0.92\textwidth}{\BODY}}\end{center}}

\title{Discrete and Computational Geometry \\ Assignment 1}
\author{
  \AA{AAAAAAAAAA AAAAAAA}{6} \\
  \href{mailto:\AA{AAAAAAAAAAAAAAAAAAAA}{7}}{\AA{AAAAAAAAAAAAAAAAAAAA}{7}}
  \and
  Emilia Groß-Hardt \\
  \href{mailto:emilia.ghdt@uni-bonn.de}{emilia.ghdt@uni-bonn.de}
}

\usepackage{titlesec}
\titleformat{\section}
  {\normalfont\Large\bfseries}{Problem \thesection : }
  {0em}{\mdseries}

\begin{document}
  \maketitle

  \section{Linear dependence}
  \begin{centerframebox}
    Show that any set of $d + 1$ points in $\Rd$ is linearly dependent.
  \end{centerframebox}
  We can write those $d + 1$ points as columns of a $d \times (d+1)$ dimensional matrix.
  The rank of a matrix is defined as the number of linearly independent rows or columns.
  By the ``column rank = row rank'' theorem, that matrix cannot have a rank larger than $d$,
  and so our $d + 1$ points must be linearly dependent.
  % todo: maybe actually prove that theorem maybe ._.
  % (c) Clover

  \section{Hyperplanes}
  \begin{centerframebox}
    In the lecture, it was stated that each hyperplane can be represented either as the image
    of an affine mapping or as a solution of a linear equation. We are now interested in how to
    transform one representation into the other. Let $f: \R^{d-1} \to \Rd$ be an affine mapping with
    $f: y \mapsto By + c$ where $B$ is a $d \times (d-1)$ matrix and $c \in \Rd$. Let further $a \in \Rd \setminus \{0\}$ and
    $b \in \R$ such that:
    \[ \operatorname{Image}(f) = \{x \in \Rd \mid \langle a,x \rangle = b\} \]
    \begin{enumerate}[label=(\roman*)]
      \item Given $B$ and $c$, we want to find suitable $a$ and $b$. Construct a system of $(d - 1)$ linear
      equations that can be solved to determine $a$. How can we determine $b$ given $B$, $c$ and $a$?
      \item Given $a$ and $b$, find suitable $B$ and $c$.
    \end{enumerate}
  \end{centerframebox}

  \subsection{Mapping $\to$ Equation}
  Note that in both cases the hyperplane is defined by an ``orientation'' and an ``offset'' parameter.
  Changing both the $c$ and $b$ parameters doesn't rotate the plane, and only moves it around,
  and setting it to zero makes the transformation linear, instead of affine, and includes the point $0$ in the plane.

  We can think of this geometrically,
  with $a$ being the normal vector of the hyperplane and the $(d-1)$ columns of $B$ defining the basis vectors inside out hyperplane.
  So from that we can gather that $a$ must be perpendicular to every one of those vectors, i.e. have a zero dot product with them.
  The resulting system of equations can be written as:
  \[
    \left\{
    \begin{matrix}
      a_1 B_{11} &+& a_2 B_{21} &+& \cdots &+& a_d B_{d1} &=& 0 \\
      a_1 B_{12} &+& a_2 B_{22} &+& \cdots &+& a_d B_{d2} &=& 0 \\
      \vdots     && \vdots      && \ddots  && \vdots  \\
      a_1 B_{1n} &+& a_2 B_{2n} &+& \cdots &+& a_d B_{dn} &=& 0 \\
    \end{matrix}
    \right.
  \]
  with $B_{ij}$ being elements of the $B$ matrix, and $n = d-1$.

  To determine $b$ with everything else known, we can pick any vector $\gamma \in \R^{d-1}$, and compute $b = \langle a,f(\gamma) \rangle$.
  % todo: check if all that actually works
  % (c) Clover

  \subsection{Equation $\to$ Mapping}
  Here we can also use that trick of setting $c$ and $b$ to zero temporarily.
  That way we can find $d-1$ linearly independent solutions to $\langle a,x \rangle = 0$, and use them as columns for our $B$ matrix.
  They will all be perpendicular to the normal vector $a$.

  Using the linearity of the dot product:
  \[
    \langle a, By + c \rangle = \langle a, By \rangle + \langle a, c \rangle
  \]
  So we can say that $\langle a, c \rangle = b$ and $\langle a, By \rangle = 0$.
  Here we can just pick $c$ as one of the solutions to that equation.

  % the revelation came a bit too late...
  % (c) Clover

  \section{Open and closed sets}
  \begin{centerframebox}
    \begin{enumerate}[label=(\roman*)]
      \item For the following sets determine if they are open or closed or both or neither:
      (a) the empty set, (b) the interval $(0, 1]$, (c) the interval $[0, \infty)$
      \item Find an example of two disjoint closed convex sets that are not strictly separable.
    \end{enumerate}

    $A$ is closed if $\partial(A) \subseteq A$. \\
    $A$ is open if $\partial(A) \cap A = \emptyset$.
  \end{centerframebox}

  \subsection{The empty set}
  The boundary of the empty set is also empty. Which means that it is both closed and open.
  % (c) Clover

  \subsection{The interval $(0, 1]$}
  The boundary of this set is $\{0, 1\}$. It intersects the set, and is also not its subset.
  Thus, this interval is neither open nor closed.
  % (c) Clover

  \subsection{The interval $[0, \infty)$}
  The boundary of this set is $\{0\}$.
  It is fully included in the interval, so the set is closed.
  % (c) Clover

  \subsection{Not strictly separable sets}
  To find this kind of counterexample, both sets need to be unbounded, and thus infinite.
  % what...?

  \section{Disks in 2d}
  \begin{centerframebox}
    Each set $X \subset \R^2$ of diameter at most $1$ (i.e. any $2$ points have distance at most $1$) is
    contained in some disc of radius $\frac{1}{\sqrt{3}}$.

    \begin{enumerate}[label=(\roman*)]
      \item Prove the statement for $3$-element sets $X$.
      \item Prove the statement for all finite sets $X$.
    \end{enumerate}
  \end{centerframebox}

  \subsection{$3$-element sets}
  A $3$-element set is just a triangle, and the disk we're trying to inscribe it in is just the triangle's \textit{circumcircle}.
  We can use the \textit{Law of sines} to compute the radius of that circle:
  \[ \frac{a}{\sin{\alpha}} = \frac{b}{\sin{\beta}} = \frac{c}{\sin{\gamma}} = 2R \]

  Assume without loss of generality, that $a$ is the largest side of our triangle.
  That makes it the diameter of the set $X$, i.e. $a = 1$.
  The angle opposite the biggest side of the triangle $\alpha$ must always be larger than $\frac{\pi}{3} = 60^\circ$,
  by analogy from the equilateral triangle, where all angles are $60^\circ$,
  and making any side longer will also increase its corresponding angle.
  Transforming the equation above we get:
  \[ R = \frac{a}{2\sin{\alpha}} = \frac{1}{2\sin{\frac{\pi}{3}}} = \frac{1}{2 \cdot \frac{\sqrt{3}}{2}} = \frac{1}{\sqrt{3}} \]
  Which is exactly what we wanted to show!
  % (c) Clover

  \subsection{All finite sets}
  We can extend this to all finite sets, by picking 2 points that define the diameter,
  and then using the triangle result proven above on every other point in the set.

  % this is actually bullshit
  % (c) Clover

\end{document}
