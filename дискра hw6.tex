\documentclass{article}

\usepackage[table]{xcolor}
\usepackage{cmap} % поиск в pdf
\usepackage{mathtext} % русские буквы в формулах
\usepackage[english,russian]{babel} % локализация и переносы
\usepackage[T2A]{fontenc} % кодировка в pdf (магия)
\usepackage[utf8]{inputenc} % кодировка исходного текста
\usepackage{amsmath}
\usepackage{amsfonts}
\usepackage{amssymb}
\usepackage{gensymb}
\usepackage{headertable2}

\usepackage[hidelinks,bookmarks=false]{hyperref}
\usepackage{xurl}
\usepackage[margin=0.5in]{geometry}
\usepackage{fancyvrb}

\newcommand{\notimplies}{\mathrel{{\ooalign{\hidewidth$\not\phantom{=}$\hidewidth\cr$\implies$}}}}
\newcommand{\VCenter}[2]{\vcenter{\hbox{\scalebox{#1}{$#2$}}}}

\newcommand{\ds}{\displaystyle}
\newcommand{\DS}{\phantom{$0.5$}}
\newcommand{\N}{\mathbb{N}}
\newcommand{\Z}{\mathbb{Z}}
\newcommand{\Q}{\mathbb{Q}}
\newcommand{\R}{\mathbb{R}}
\newcommand{\range}{\underline}
\newcommand{\Aleph}{2^{\aleph_0}}
\newcommand{\Cnk}[2]{C_{#1}^{#2}}
\newcommand{\pe}[2]{({#1},\, {#2})}
\newcommand{\p}[2]{\pe{#1}{#2},\,}
\newcommand{\K}{\cellcolor{black}}
\renewcommand{\f}{\frac}
\renewcommand{\l}{\left}
\renewcommand{\r}{\right}
\renewcommand{\P}[1]{\mathcal{P}\l(#1\r)}
\renewcommand{\emptyset}{\varnothing}

\definecolor{red}{HTML}{d62728}
\definecolor{orange}{HTML}{ff7f0e}
\definecolor{green}{HTML}{2ca02c}
% \definecolor{blue}{HTML}{1f77b4}
\newcommand{\red}[1]{{\color{red}{#1}}}
\newcommand{\orange}[1]{{\color{orange}{#1}}}
\newcommand{\green}[1]{{\color{green}{#1}}}
\newcommand{\blue}[1]{{\color{blue}{#1}}}

% \pagecolor{black}
% \color{white}

\title{Дискретная математика \\ Домашнее задание №6}
\author{\AA{AAAAA AAAAAAA}{4} \\ AAAAAA}

\begin{document}
  \maketitle
  \HeaderTable{4}

  \section{Что более вероятно: что в четырех бросаниях (честной) кости хотя бы раз выпала единица, или что в двадцати четырех одновременнных бросаниях двух таких костей хотя бы раз выпали сразу две единицы?}
  чтобы "{}в четырех бросаниях кости хотя бы раз выпала единица"{} надо чтобы
  небыло "{}в четырех бросаниях кости всегда выпала не единица"{}. \\
  тоесть $\ds 1-\l(\f{5}{6}\r)^4 = 0.5177469135802468\dots$ \\
  в другом случаи будет $\ds 1-\l(\f{6^2-1}{6^2}\r)^{24} = 0.4914038761309034\dots$ \\
  получается что первый случай более вероятный

  \section{Есть девять коробок и один шарик. Равновозможно выбирается коробка. Затем с вероятностью $1/2$ в нее помещается шарик. Найти вероятность того, что в коробке $\textnormal{№}9$ шарик есть, при условии, что в остальных коробках его нет.}
  тут есть небольшая непонятка в условие.
  если я его правильно понял, то мы выбираем рандомную коробку и пытаемся туда положить шарик, пока он кудато не положится.
  получается что в коробку №9 он может попасть как и с первой попытки, так и со второй, третей и тд. \\
  получается такая бесконечная сумма $\ds \f{1}{9}\cdot\f{1}{2} + \f{1}{2}\cdot\f{8}{9}\cdot\f{1}{9}\cdot\f{1}{2} + \l(\f{1}{2}\cdot\f{8}{9}\r)^2\cdot\f{1}{9}\cdot\f{1}{2} + \cdots = \f{1}{18}\l(\l(\f{8}{18}\r)^0 + \l(\f{8}{18}\r)^1 + \l(\f{8}{18}\r)^2 + \cdots\r)$ \\
  это получается сумма геометрической прогрессии $\ds \f{1}{18}\l(\f{1}{1-\f{8}{18}}\r) = 0.1$ \\
  ответ \fbox{$0.1$}

  \newpage
  \section{Два игрока подбрасывают (одну) честную монету, причем первый выигрывает одно подбрасывание, если выпадает <<орел>>, а второй -- если <<решка>>. Денежный приз получает тот, кто первым выиграет $10$ подбрасываний. К некоторому моменту монета выпала $8$ раз <<орлом>> и $6$ раз <<решкой>>. Как игрокам справедливо (т. е. пропорционально своим шансам на выигрыш) разделить приз, если по каким-то причинам дальше они играть не могут?}
  <<орел>> $= 0$; <<решка>> $= 1$ \\
  чтобы второй игрок выиграл ему надо чтобы единичка выпала четыре раза подряд или чтобы выпали четыре единички и один ноль. \\
  тоесть $\ds \l(\f{1}{2}\r)^4 + 5\l(\f{1}{2}\r)^5 = 0.21875$ второму игроку и $0.78125$ первому

  НО ЭТО НЕПРАВИЛЬНО \\
  тут мы посчитали $1111$ и $11110$ как два разных события, хотя $11110$ произойти неможет тк начинается с $1111$.
  тоесть правильный ответ это \fbox{$0.1875$} и \fbox{$0.8125$}

  \section{$n$ зрителей оставили в театральном гардеробе свои шляпы. После спектакля -- под впечатлением, им произведенным, -- первый пришедший в гардероб зритель берет случайную шляпу. Каждый следующий пришедший действует так: забирает свою шляпу, если она на месте, иначе -- с досады берет наугад любую из оставшихся. С какой вероятностью последний зритель возьмет свою шляпу?}
  вот я написал код на питоне
  \begin{Verbatim}[frame=single]
from random import randint, choice

N = 100000
n = 20

def f():
  arr = [1]*n
  arr[randint(0, n-1)] = 0
  for i in range(1, n-1):
    if arr[i] == 0:
      arr[choice([i for i,e in enumerate(arr) if e == 1])] = 0
    else:
      arr[i] = 0
  return arr[n-1]

print(sum([f() for i in range(N)])/N)
  \end{Verbatim}
  и он мне сказал что при любом $n>1$ вероятность будет $0.5$

  для начала мы сделаем так чтобы шляпы шли в том же порядке что и посетители,
  чтобы ненадо было всегда говорить "{}шляпа которая должна была принадлежать энному посетителю"{},
  а можно было сказать "{}энная шляпа"{}.

  украденная шляпа это та, которую украли но её хозяйн ещё не ушёл.
  тут надо заметить что количество украденных шляп всегда либо $1$ либо $0$.
  тоесть когда человек узнаёт что его шляпу украли, то он ворует ещё чьюто шляпу, но у него есть шанс украсть первую шляпу,
  и тогда количество украденных шляп станет $0$.
  получается с того момента, когда ктото берёт первую шляпу, все идёт по плану и никто ничего больше не ворует.

  тогда в последний момент, когда за шляпой приходит последний человек, может быть только два случая:
  либо у нас осталась первая шляпа либо последняя.
  в первом случаи последний человек всегда уйдёт без своей шляпы, а во втором всегда с ней.
  эти два случая равновероятны, тк мы всегда рандомно выбираем шляпы и для рандома первая и последняя шляпа ничем не отличаются.
  поэтому у нас тут вероятность \fbox{$0.5$}

  % \vfill
  % \begin{thebibliography}{9}
    % \bibitem{dashkov} Дашков Е. В. Введение в математическую логику // URL: \url{https://drive.google.com/file/d/1wLW9t2UFJk9tTu8nM_YAYYr9zzZwKD1W/view}
    % \bibitem{comp} \url{http://www.problems.ru/view_problem_details_new.php?id=35352}
  % \end{thebibliography}
\end{document}
