\documentclass{article}

\usepackage[table]{xcolor}
\usepackage{cmap} % поиск в pdf
\usepackage{mathtext} % русские буквы в формулах
\usepackage[english,russian]{babel} % локализация и переносы
\usepackage[T2A]{fontenc} % кодировка в pdf (магия)
\usepackage[utf8]{inputenc} % кодировка исходного текста
\usepackage{amsmath}
\usepackage{amsfonts}
\usepackage{amssymb}
\usepackage{gensymb}
\usepackage{headertable2}

\usepackage[hidelinks,bookmarks=false]{hyperref}
\usepackage{xurl}
\usepackage[margin=0.5in]{geometry}
\usepackage{wrapfig}
\usepackage{tkz-berge}

\usepackage[many]{tcolorbox}
\newtcolorbox{cross}{blank,breakable,parbox=false,
  overlay={\draw[red,line width=5pt] (interior.south west)--(interior.north east);
    \draw[red,line width=5pt] (interior.north west)--(interior.south east);}}

\newcommand{\notimplies}{\mathrel{{\ooalign{\hidewidth$\not\phantom{=}$\hidewidth\cr$\implies$}}}}
\newcommand{\VCenter}[2]{\vcenter{\hbox{\scalebox{#1}{$#2$}}}}

\newcommand{\ds}{\displaystyle}
\newcommand{\DS}{\phantom{$0.5$}}
\newcommand{\N}{\mathbb{N}}
\newcommand{\Z}{\mathbb{Z}}
\newcommand{\Q}{\mathbb{Q}}
\newcommand{\R}{\mathbb{R}}
\newcommand{\range}{\underline}
\newcommand{\Aleph}{2^{\aleph_0}}
\newcommand{\Cnk}[2]{C_{#1}^{#2}}
\newcommand{\pe}[2]{({#1},\, {#2})}
\newcommand{\p}[2]{\pe{#1}{#2},\,}
\newcommand{\K}{\cellcolor{black}}
\renewcommand{\f}{\frac}
\renewcommand{\l}{\left}
\renewcommand{\r}{\right}
\renewcommand{\P}[1]{\mathcal{P}\l(#1\r)}
\renewcommand{\emptyset}{\varnothing}

\definecolor{red}{HTML}{d62728}
\definecolor{orange}{HTML}{ff7f0e}
\definecolor{green}{HTML}{2ca02c}
% \definecolor{blue}{HTML}{1f77b4}
\newcommand{\red}[1]{{\color{red}{#1}}}
\newcommand{\orange}[1]{{\color{orange}{#1}}}
\newcommand{\green}[1]{{\color{green}{#1}}}
\newcommand{\blue}[1]{{\color{blue}{#1}}}

% \pagecolor{black}
% \color{white}

\title{Дискретная математика \\ Домашнее задание №5}
\author{\AA{AAAAA AAAAAAA}{4} \\ \AA{AAAAAA}{11}}

\usepackage{titlesec}
\usepackage{enumitem}
\usepackage{soulutf8}
\setstcolor{red}

\begin{document}
  \maketitle
  \HeaderTable{14}

  \section{Существует ли граф}
  \subsection{c $8$ вершинами, $23$ ребрами и вершиной степени $1$}
  я тут немного затупил и подумал что в условие написано "{}граф вершинной степени $1$"{}
  и долго думал что такое "{}вершинная степень"{}.
  \begin{cross}
    есть тут такая вот "{}Лемма о рукопожатиях"{} \cite{lem}, которая говорит, что $\ds \sum_{v\in V}\deg(v)=2|E|$. \\
    а у нас тут $8$ вершин, и степень у них у всех $1$.
    тоесть сумма у нас будет $8$, а надо $46$.
    и поэтому такого графа нет.
    хотя тут непонятно что такое "{}вершинная степень"{}.
    я думаю что это степень всех вершин, но я неуверен.
  \end{cross}

  но теперь я знаю что тут написано "{}граф c $23$ ребрами и $8$ вершинами, степень одной из которых -- $1$"{}.
  мы можем выкинуть эту вершину и у нас будет граф с $22$ ребрами и $7$ вершинами.
  но в полном графе из $7$ вершин у нас только $\f{7 \cdot 6}{2}=21$ рёбер, а $21 < 22$.
  и вот мы нашли противоречие.
  и поэтому такого графа нет.

  \subsection{со степенной последовательностью $(3,\, 3,\, 3,\, 3,\, 2,\, 2)$}
  \begin{wrapfigure}{r}{11cm}
    \vspace{-.5cm}
    \hfill
    \begin{tikzpicture}
      \SetVertexMath
      \GraphInit[vstyle=Shade]

      \begin{scope}[rotate=45]
        \grEmptyCycle[RA=1.5,prefix=a]{4}
      \end{scope}
      \grEmptyCycle[RA=2,prefix=b]{2}

      \Edges(a0,a1,a3,a2,a0)
      \Edges(a0,b0,a3)
      \Edges(a2,b1,a1)
    \end{tikzpicture}
    \begin{tikzpicture}
      \SetVertexMath
      \GraphInit[vstyle=Shade]

      \begin{scope}[rotate=45]
        \grCycle[RA=1.5,prefix=a]{4}
        \grEmptyCycle[RA=.6,prefix=b]{2}
      \end{scope}

      \Edge[](b0)(a0)
      \Edge[](b1)(a2)
      \Edge[](a1)(a3)
      \Edge[](b0)(b1)
    \end{tikzpicture}
    \begin{tikzpicture}
      \SetVertexMath
      \GraphInit[vstyle=Shade]

      \grCycle[RA=.7,prefix=a]{4}
      \begin{scope}[yshift=1.4cm]
        \grEmptyCycle[RA=.7,prefix=b]{2}
      \end{scope}

      \Edge[](b0)(a0)
      \Edge[](b1)(a2)
      \Edge[](a1)(a3)
      \Edge[](b0)(b1)
    \end{tikzpicture}
    \vspace{-.5cm}
  \end{wrapfigure}
  вот он!
  ну тоесть у меня их тут даже 3, но последние 2 изоморфны.
  я нарисовал 3 чтобы у меня этот вот текст нормально поместился, и чтобы всё было красива.
  тут надеюсь понятно что все $a_i$ это те вершины со степенью $3$, а все $b_i$ это со степенью $2$.
  первый граф (как и второй) можно было переделать в несамопересекающейся, но мне лень и у меня уже один такой есть.

  \section{С точностью до изоморфизма опишите все графы, где любые два ребра имеют общую вершину}
  \begin{wrapfigure}{r}{6cm}
    \vspace{-1cm}
    \hfill
    \begin{tikzpicture}
      \SetVertexMath
      \GraphInit[vstyle=Shade]
      \grCycle[prefix=c,RA=1]{3}
    \end{tikzpicture}
    \begin{tikzpicture}
      \SetVertexMath
      \GraphInit[vstyle=Shade]
      \grStar[prefix=s,RA=1]{6}
    \end{tikzpicture}
    \vspace{-.5cm}
  \end{wrapfigure}
  тут у нас очевидно подходят графы, где у нас есть одна центральная вершина и все остальные к ней подсоединены.
  таких графов у нас бесконечно много ($n \in \N$) и тут нарисован такой граф для $n=6$.
  но это не единственные такие графы, у нас есть ещё одно исключение.
  и это тот самый $K_3$ из условия третьего номера (тоесть это просто такой треугольничек).
  для квадратов это уже не работает, и поэтому у нас больше нет таких графов.
  тут ещё можно к каждому из таких графов до бесконечности добавлять нискем несвязанные (одинокие) вершины.

  \section{Допустим, в графе $G$ $400$ вершин, причем каждая имеет степень $201$. Докажите, что в $G$ есть подграф, изоморфный $K_3$}
  \begin{wrapfigure}{r}{3cm}
    \vspace{-1cm}
    \hfill
    \begin{tikzpicture}
      \SetVertexMath
      \GraphInit[vstyle=Shade]
      \Vertices{square}{a,b,c_{200},d_{198}}
      \Edges(a,b)
      \Edges(a,c_{200})
    \end{tikzpicture}
    \vspace{-1cm}
  \end{wrapfigure}
  тут мы берём какуюто вершину $a$ и какуюто вершину $b$, которая соединена с $a$.
  все остальные $200$ вершин соединённых с $a$ мы называем $c$,
  а те $198$, которые с $a$ не соединены, называем $d$.
  у $b$ осталось ещё $200$ соединений, и она соединена с хотябы одной вершиной из $c$, тк $198 < 200$.
  пусть это будет $c_0$.
  тогда $a$, $b$ и $c_0$ образуют нужный нам треугольничек ($K_3$ это треугольничек).

  \clearpage
  \section{Постройте граф $G$ на $5$ или более вершинах, такой что $G \cong \bar{G}$}
  \begin{wrapfigure}{r}{8.5cm}
    \vspace{-.5cm}
    \hfill
    \begin{tikzpicture}
      \SetVertexMath
      \GraphInit[vstyle=Shade]
      \grCycle[prefix=c,RA=1]{5}
      \SetUpEdge[color=blue]
      \Edges(c0,c2,c4,c1,c3,c0)
    \end{tikzpicture}
    \begin{tikzpicture}
      \SetVertexMath
      \GraphInit[vstyle=Shade]
      \Vertices{circle}{c_0,c_2,c_4,c_1,c_3}
      \Edges(c_0,c_1,c_2,c_3,c_4,c_0)
      \SetUpEdge[color=blue]
      \Edges(c_0,c_2,c_4,c_1,c_3,c_0)
    \end{tikzpicture}
    \begin{tikzpicture}
      \SetVertexMath
      \GraphInit[vstyle=Shade]
      \Vertices{circle}{c_0,c_2,c_4,c_1,c_3}
      \Edges(c_0,c_2,c_4,c_1,c_3,c_0)
      \SetUpEdge[color=blue]
      \Edges(c_0,c_1,c_2,c_3,c_4,c_0)
    \end{tikzpicture}
    \vspace{-1cm}
  \end{wrapfigure}
  тут надо сначала понять что такое $\bar{G}$.
  я незнаю как это гооглить, но на лекции было сказано что это "{}дополнение графа"{}, хотя в википедии оно не $\bar{G}$ а $G'$.
  дополнение графа это когда мы меняем местами существующие и несуществующие рёбра.
  циферка $5$ в условие нам тут подсказывает: мы можем просто взять циклический граф с пятью вершинами
  (это наверно $K_5$ (нет я тупой это $C_5$)),
  и он будет самодополнительным.

  \section{Пусть $G$ -- граф на $100$ вершинах, причем $G \cong \bar{G}$. Может ли быть так, что в $G$ ровно одна вершина степени $50$?}
  если у нас ровно одна вершина степени $50$ ($A$), то у нас ровно одна вершина степени $49$ ($B$).
  если эти две вершины не соединены в $G$, то в $\bar{G}$ если какоето ребро,
  у которого на одном конце вершина со степенью $50$, а на другом конце вершина со степенью $49$.
  эти вершины у нас только одни, поэтому $A$ соединена с $B$.
  и вот у нас противоречие.
  будет тоже самое если мы предприложим что они соединены.
  ответ -- "{}нет"{}.

  \section{Из некоторых семи человек каждый имеет хотя бы трех братьев (среди этих семерых). Докажите, что любые двое среди них -- братья}
  тоесть надо доказать что любой граф на $7$ вершинах, где степень каждой вершины хотябы $3$ -- связный.
  связный граф -- это граф, содержащий ровно одну компоненту связности \cite{link}.
  поскольку у нас степени всех вершин $\ge 3$ все наши компоненты связности должны содержать $\ge 4$ вершин.
  но $7 < 4 \cdot 2$ и поэтому у нас граф всегда связный.

  \section{Города некоторой страны обладают таким свойством: если их как-либо разбить на две непустые группы, в разных группах всегда найдутся два города, соединенные дорогой. Докажите, что из любого города можно по дорогам добраться до любого другого.}
  если нельзя "{}из любого города можно по дорогам добраться до любого другого"{}, то наш граф будет несвязанный.
  а если он несвязанный, то его можно разбить на хотябы две группы так, чтобы из одной никак нельзя было попасть в другую.
  но это противоречие, тк в условие написано что
  "{}если их как-либо разбить на две непустые группы, в разных группах всегда найдутся два города, соединенные дорогой"{}
  и именно по этой дороге мы сможем попасть из одной группы в другую.
  (это типа было доказательство от обратного)

  \section{Вершинами графа $B_{n,\, r}$ служат всевозможные двоичные слова длины $n$. Две вершины-слова смежны, если они различаются ровно в $r$ разрядах. Связен ли граф $B_{1000,\, 400}$?}
  при инвертирование $400$ битиков у нас всегда сохраняется "{}чётность"{} числа, тоесть сумма всех битиков $\mod 2$.
  потомучто для инвертирования битика мы прибавляем к нему $1 \mod 2$ и $1 \cdot 400 \mod 2 = 0$.
  у $0 \dots 01$ чётность $1$, а у $0 \dots 0$ чётность $0$.
  поэтому наш граф несвязанный.

  \section{Докажите, что в каждом дереве на $2n$ вершинах можно найти $n$ попарно несмежных вершин}
  мы типа знаем что любое дерево это двудольный граф.
  но я всёравно докажу (в одно предложение): мы можем раскрасить дерево в два цвета взяв какойто рандомный корень,
  посчитав уровни \st{рекурсии} и раскрасив по чётности этих уровней.
  у нас есть два случая: либо у нас области (доли?) дерева одинакового размера по $n$ каждая, либо одна дольше другой.
  в первом случаи мы можем просто удалить любую и у нас останутся $n$ попарно несмежных вершин,
  а во втором надо удалить маленькую и останутся $>n$ попарно несмежных вершин, но это нестрашно, тк мы можем выкинуть лишние.

  \section{Является ли булев куб $B_n$ двудольным графом?}
  тут $B_n$ это $B_{n,\, 1}$ из номера 8.
  но в номере 8 у нас чётность менялось на $2$, а не на $1$.
  поэтому мы тут можем разбить на две области: в одной чётность $1$, а в другой $0$.
  вот и получился наш двудольный граф.

  \section{Можно ли на плоскости выбрать $26$ различных прямых и $43$ различные точки таким образом, что каждая из этих прямых содержит ровно $7$ (выбранных) точек, а каждая точка лежит ровно на $4$ прямых?}
  тут можно построить двудольный граф:
  в одной области будут все прямые, а в другой все точки.
  прямая соединена с точкой тогда-и-только-тогда-когда она лежит на ней.
  мы можем попробовать посчитать количество рёбер этого графа.
  с одной стороны будет $26 \cdot 7 = 182$, а с другой $4 \cdot 43 = 172$.
  вот мы нашли противоречие, и поэтому ответ у нас -- нет.

  \section{В классах <<A>> и <<Б>> вместе $26$ учеников. Каждый ученик одного класса подрался с некоторыми учениками другого (но ни с кем дважды). Всего было $169$ драк. Сколько учеников может быть в каждом классе?}
  в случаи, когда у нас максимальное количество драк классы разделены поровну.
  если мы так разделим то получим что у нас так и есть: $13 \cdot 13 = 169$.
  поэтому это единственная возможная конфигурация, и ответ у нас -- $13$ и $13$.

  \section{Пусть из любой вершины орграфа $G$ порядка $\geq 2$ в любую другую ведет ровно один простой (ориентированный) путь. Обязательно ли полустепень исхода каждой вершины в $G$ равна $1$?}
  \begin{wrapfigure}{r}{2.5cm}
    \vspace{-1cm}
    \begin{tikzpicture}
      \GraphInit[vstyle=Simple]
      \tikzset{EdgeStyle/.append style = {->}}
      \grCycle[RA=1]{6}
      \tikzset{EdgeStyle/.append style = {green}}
      \Edge(a1)(a4)
      \tikzset{EdgeStyle/.append style = {red}}
      \Edge(a4)(a2)
      \Edge(a0)(a4)
    \end{tikzpicture}
    \vspace{-1cm}
  \end{wrapfigure}
  графы, где полустепень исхода каждой вершины равна $1$, это вот такие вот циклы (то что чёрными стрелочками).
  но тут например мы можем по зелёной стрелочке склеить эти две вершины.
  у нас будет два цикла длинны $3$ с общей вершиной (то что красными стрелочками).
  это получается контрпример, тк у нас у одной вершины полустепень $2$.
  ещё можно придумать контрпример поменьше, но там не очень красивые двойные стрелочки.
  вот он:
  \raisebox{-.15cm}{\begin{tikzpicture}
    \GraphInit[vstyle=Simple]
    \tikzset{EdgeStyle/.append style = {<->}}
    \grPath[RA=1]{3}
  \end{tikzpicture}}
  настолько компактный что красива помещается прямо в текст.
  вот у меня есть два контрпримера и во втором вроде даже интуитивно понятно то,
  что из любой вершины в любую другую ведет ровно один простой путь.
  тоесть ответ -- "{}нет"{}.

  \newpage
  \section{Докажите, что в любом турнире есть простой (ориентированный) путь, включающий все вершины}
  турнир это ориентированный граф, полученный из полного неориентированного,
  тоесть каждому ребру полного графа назначили какое-то направление и получился турнир.
  такой граф у нас уже был в прошлей жизни (hw1):

  \noindent\fbox{
    \parbox{\textwidth}{
      \setcounter{section}{2}
      \titlespacing\section{0pt}{0pt}{10pt}
      \section{В некоторой стране лишь конечно много городов, причем любые два различных города соединены дорогой с односторонним движением. Докажите, что есть город, из которого можно добраться в любой другой по имеющимся дорогам}
      назовём город, из которого можно добраться в любой другой по имеющимся дорогам, столицей \\
      мы делаем индукцию для количества городов (назовём это $n$) \\
      стоит заметить, что для $n=0$ это неработает (столица неможет существовать, потомучто она город а городов нет) \\
      сначала нам надо доказать это для $n=1$, но тут всё очевидно, тк других городов нет \\
      потом нам надо доказать это для $n+1$, если нам дано это для $n$, \\
      тоесть если мы добавляем город (назовём его $B$) и столица (назовём её $A$) существует,
      то в получившемся графе тоже будет существовать столица, как бы мы не провели новые дороги (все остальные города мы назовём $C$) \\
      если дорога между $A$ и $B$ идет из $A$ в $B$, то $A$ остаётся столицей, тк из $A$ можно попасть и в $C$ и в $B$, \\
      а если она из $B$ в $A$, то $B$ станет новой столицей, тк из $B$ можно попасть в $A$, а из $A$ в $C$ \\
      $\blacksquare$ (а дальше по индукции)
    }
  }
  но тут у нас чуть более сильное утверждение получается.
  назовём наш путь $P$, и теперь столица $A$ это его начало, а $Z$ это его конец.
  если у нас из добавленной вершины $B$ есть дорога в $A$, то мы уже нашли наш новый путь: $B \to A \to \cdots \to Z$.
  но если её нет нам надо ещё поработать. надо разобрать два случая:
  \begin{enumerate}[topsep=0pt,itemsep=0pt,label=\textbf{\arabic*}]
    \item \textbf{все дороги направлены в $B$} \\
    тут мы можем просто прикрутить $B$ в конец нашего пути, будет $A \to \cdots \to Z \to B$.
    \item \textbf{есть дороги, которые ведут из $B$} \\
    мы берём первую такую, тоесть ту которая направлена в элемент $P$ с наименьшим индексом.
    назовём этот элемент $C$, и $C \neq A$ потомучто этот случай мы уже рассмотрели.
    мы точно знаем что до тот элемент $D$, который идёт до $C$, имеет дорогу в $B$.
    поэтому мы можем вставить $B$ между $D$ и $C$.
    будет $A \to \cdots \to D \to B \to C \to \cdots \to Z$
  \end{enumerate}
  $\blacksquare$ (а дальше по индукции)

  \vfill
  \begin{thebibliography}{9}
    % \bibitem{dashkov} Дашков Е. В. Введение в математическую логику // URL: \url{https://drive.google.com/file/d/1wLW9t2UFJk9tTu8nM_YAYYr9zzZwKD1W/view}
    \bibitem{lem} \url{https://ru.wikipedia.org/wiki/%D0%9B%D0%B5%D0%BC%D0%BC%D0%B0_%D0%BE_%D1%80%D1%83%D0%BA%D0%BE%D0%BF%D0%BE%D0%B6%D0%B0%D1%82%D0%B8%D1%8F%D1%85}
    \bibitem{comp} \url{https://en.wikipedia.org/wiki/Complement_graph}
    \bibitem{link} \url{https://ru.wikipedia.org/wiki/%D0%A1%D0%B2%D1%8F%D0%B7%D0%BD%D1%8B%D0%B9_%D0%B3%D1%80%D0%B0%D1%84}
    \bibitem{scomp} \url{https://ru.wikipedia.org/wiki/%D0%A1%D0%B0%D0%BC%D0%BE%D0%B4%D0%BE%D0%BF%D0%BE%D0%BB%D0%BD%D0%B8%D1%82%D0%B5%D0%BB%D1%8C%D0%BD%D1%8B%D0%B9_%D0%B3%D1%80%D0%B0%D1%84}
    \bibitem{path} \url{https://neerc.ifmo.ru/wiki/index.php?title=%D0%A2%D0%B5%D0%BE%D1%80%D0%B5%D0%BC%D0%B0_%D0%A0%D0%B5%D0%B4%D0%B5%D0%B8-%D0%9A%D0%B0%D0%BC%D0%B8%D0%BE%D0%BD%D0%B0}
    \bibitem{hw1} Первое Домашнее Задание По Дискретной Математики Для Первого Курса ПИ ФКН ВШЕ От Дашкова Е. В.
  \end{thebibliography}
\end{document}
