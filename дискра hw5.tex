\documentclass{article}

\usepackage[table]{xcolor}
\usepackage{cmap} % поиск в pdf
\usepackage{mathtext} % русские буквы в формулах
\usepackage[english,russian]{babel} % локализация и переносы
\usepackage[T2A]{fontenc} % кодировка в pdf (магия)
\usepackage[utf8]{inputenc} % кодировка исходного текста
\usepackage{amsmath}
\usepackage{amsfonts}
\usepackage{amssymb}
\usepackage{gensymb}
\usepackage{headertable2}

\usepackage[hidelinks,bookmarks=false]{hyperref}
\usepackage{xurl}
\usepackage[margin=0.5in]{geometry}
\usepackage{wrapfig}
\usepackage{tkz-berge}

\newcommand{\notimplies}{\mathrel{{\ooalign{\hidewidth$\not\phantom{=}$\hidewidth\cr$\implies$}}}}
\newcommand{\VCenter}[2]{\vcenter{\hbox{\scalebox{#1}{$#2$}}}}

\newcommand{\ds}{\displaystyle}
\newcommand{\DS}{\phantom{$0.5$}}
\newcommand{\N}{\mathbb{N}}
\newcommand{\Z}{\mathbb{Z}}
\newcommand{\Q}{\mathbb{Q}}
\newcommand{\R}{\mathbb{R}}
\newcommand{\range}{\underline}
\newcommand{\Aleph}{2^{\aleph_0}}
\newcommand{\Cnk}[2]{C_{#1}^{#2}}
\newcommand{\pe}[2]{({#1},\, {#2})}
\newcommand{\p}[2]{\pe{#1}{#2},\,}
\newcommand{\K}{\cellcolor{black}}
\renewcommand{\f}{\frac}
\renewcommand{\l}{\left}
\renewcommand{\r}{\right}
\renewcommand{\P}[1]{\mathcal{P}\l(#1\r)}
\renewcommand{\emptyset}{\varnothing}

\definecolor{red}{HTML}{d62728}
\definecolor{orange}{HTML}{ff7f0e}
\definecolor{green}{HTML}{2ca02c}
% \definecolor{blue}{HTML}{1f77b4}
\newcommand{\red}[1]{{\color{red}{#1}}}
\newcommand{\orange}[1]{{\color{orange}{#1}}}
\newcommand{\green}[1]{{\color{green}{#1}}}
\newcommand{\blue}[1]{{\color{blue}{#1}}}

% \pagecolor{black}
% \color{white}

\title{Дискретная математика \\ Домашнее задание №5}
\author{AAAAA AAAAAAA \\ AAAAAA}

\begin{document}
  \maketitle
  \HeaderTable{3}

  \section{Существует ли граф}
  \subsection{c $8$ вершинами, $23$ ребрами и вершиной степени $1$}
  есть тут такая вот "{}Лемма о рукопожатиях"{} \cite{lem}, которая говорит, что $\ds \sum_{v\in V}\deg(v)=2|E|$. \\
  а у нас тут $8$ вершин, и степень у них у всех $1$.
  тоесть сумма у нас будет $8$, а надо $46$.
  и поэтому такого графа нет.
  хотя тут непонятно что такое "{}вершинная степень"{}.
  я думаю что это степень всех вершин, но я неуверен.

  \subsection{со степенной последовательностью $(3,\, 3,\, 3,\, 3,\, 2,\, 2)$}
  \begin{wrapfigure}{r}{11cm}
    \vspace{-.5cm}
    \hfill
    \begin{tikzpicture}
      \SetVertexMath
      \GraphInit[vstyle=Shade]

      \begin{scope}[rotate=45]
        \grEmptyCycle[RA=1.5,prefix=a]{4}
      \end{scope}
      \grEmptyCycle[RA=2,prefix=b]{2}

      \Edges(a0,a1,a3,a2,a0)
      \Edges(a0,b0,a3)
      \Edges(a2,b1,a1)
    \end{tikzpicture}
    \begin{tikzpicture}
      \SetVertexMath
      \GraphInit[vstyle=Shade]

      \begin{scope}[rotate=45]
        \grCycle[RA=1.5,prefix=a]{4}
        \grEmptyCycle[RA=.6,prefix=b]{2}
      \end{scope}

      \Edge[](b0)(a0)
      \Edge[](b1)(a2)
      \Edge[](a1)(a3)
      \Edge[](b0)(b1)
    \end{tikzpicture}
    \begin{tikzpicture}
      \SetVertexMath
      \GraphInit[vstyle=Shade]

      \grCycle[RA=.7,prefix=a]{4}
      \begin{scope}[yshift=1.4cm]
        \grEmptyCycle[RA=.7,prefix=b]{2}
      \end{scope}

      \Edge[](b0)(a0)
      \Edge[](b1)(a2)
      \Edge[](a1)(a3)
      \Edge[](b0)(b1)
    \end{tikzpicture}
    \vspace{-.5cm}
  \end{wrapfigure}
  вот он!
  ну тоесть у меня их тут даже 3, но последние 2 изоморфны.
  я нарисовал 3 чтобы у меня этот вот текст нормально поместился, и чтобы всё было красива.
  тут надеюсь понятно что все $a_i$ это те вершины со степенью $3$, а все $b_i$ это со степенью $2$.
  первый граф (как и второй) можно было переделать в несамопересекающейся, но мне лень и у меня уже один такой есть.

  \section{С точностью до изоморфизма опишите все графы, где любые два ребра имеют общую вершину}
  \begin{wrapfigure}{r}{6cm}
    \vspace{-1cm}
    \hfill
    \begin{tikzpicture}
      \SetVertexMath
      \GraphInit[vstyle=Shade]
      \grCycle[prefix=c,RA=1]{3}
    \end{tikzpicture}
    \begin{tikzpicture}
      \SetVertexMath
      \GraphInit[vstyle=Shade]
      \grStar[prefix=s,RA=1]{6}
    \end{tikzpicture}
    \vspace{-.5cm}
  \end{wrapfigure}
  тут у нас очевидно подходят графы, где у нас есть одна центральная вершина и все остальные к ней подсоединены.
  таких графов у нас бесконечно много ($n \in \N$) и тут нарисован такой граф для $n=6$.
  но это не единственные такие графы, у нас есть ещё одно исключение.
  и это тот самый $K_3$ из условия третьего номера (тоесть это просто такой треугольничек).
  для квадратов это уже не работает, и поэтому у нас больше нет таких графов.


  \section{Допустим, в графе $G$ $400$ вершин, причем каждая имеет степень $201$. Докажите, что в $G$ есть подграф, изоморфный $K_3$}
  \begin{wrapfigure}{r}{3cm}
    \vspace{-1cm}
    \hfill
    \begin{tikzpicture}
      \SetVertexMath
      \GraphInit[vstyle=Shade]
      \Vertices{square}{a,b,c_{200},d_{198}}
      \Edges(a,b)
      \Edges(a,c_{200})
    \end{tikzpicture}
    \vspace{-1cm}
  \end{wrapfigure}
  тут мы берём какуюто вершину $a$ и какуюто вершину $b$, которая соединена с $a$.
  все остальные $200$ вершин соединённых с $a$ мы называем $c$,
  а те $198$, которые с $a$ не соединены, называем $d$.
  у $b$ осталось ещё $200$ соединений, и она соединена с хотябы одной вершиной из $c$, тк $198 < 200$.
  пусть это будет $c_0$.
  тогда $a$, $b$ и $c_0$ образуют нужный нам треугольничек ($K_3$ это треугольничек).

  \vfill
  \begin{thebibliography}{9}
    % \bibitem{dashkov} Дашков Е. В. Введение в математическую логику // URL: \url{https://drive.google.com/file/d/1wLW9t2UFJk9tTu8nM_YAYYr9zzZwKD1W/view}
    \bibitem{lem} \url{https://ru.wikipedia.org/wiki/%D0%9B%D0%B5%D0%BC%D0%BC%D0%B0_%D0%BE_%D1%80%D1%83%D0%BA%D0%BE%D0%BF%D0%BE%D0%B6%D0%B0%D1%82%D0%B8%D1%8F%D1%85}
  \end{thebibliography}
\end{document}
