\documentclass[a4paper]{article}
\usepackage[hidelinks,bookmarks=false]{hyperref}
\usepackage{122}

\renewcommand{\figureautorefname}{рисунок}
\addto\captionsrussian{\renewcommand{\figurename}{Рисунок}}
\addto\extrasrussian{\renewcommand{\figureautorefname}{рисунок}}

\usepackage{graphicx}
\usepackage{etoolbox} % for \patchcmd
\usepackage{verbatim}
\usepackage{fancyvrb}

\setlength{\parindent}{15pt}
\usepackage{indentfirst}

\usepackage{tocloft}
\renewcommand{\cftsecleader}{\cftdotfill{\cftdotsep}}

\usepackage{newunicodechar}
\newunicodechar{│}{|}
\newunicodechar{∞}{\ensuremath{\infty}}

\newcommand{\titlePage}{
  \begin{center}
    \Large
    \textbf{
      ПРАВИТЕЛЬСТВО РОССИЙСКОЙ ФЕДЕРАЦИИ \\
      НАЦИОНАЛЬНЫЙ ИССЛЕДОВАТЕЛЬСКИЙ УНИВЕРСИТЕТ \\
      <<ВЫСШАЯ ШКОЛА ЭКОНОМИКИ>> \\
    }
    \par\vspace{5mm}
    Факультет компьютерных наук \\
    Департамент программной инженерии \\
  \end{center}
  \vspace*{\fill}
  \begin{center}
    \Large\bf
    КОНСОЛЬНОЕ ПРИЛОЖЕНИЕ ДЛЯ НАХОЖДЕНИЯ ЗНАЧЕНИЯ ФУНКЦИИ ГИПЕРБОЛИЧЕСКОГО КОСИНУСА
    \par
    Пояснительная записка
  \end{center}
  \vspace*{\fill}
  \begin{flushright}
		Исполнитель: \\ студент группы AAAAAA \\
		\underline{\hspace{4cm}} AA AA AAAAAAA \\
		<<\underline{\hspace{1cm}}>> \underline{\hspace{4cm}} 2020 г. \\
  \end{flushright}
  \vspace*{\fill}
  \begin{center}Москва 2020\end{center}
}

\newcommand{\frameCode}[1]{
  \begin{center}
    \begin{minipage}{0.5\textwidth}
      \VerbatimInput[frame=single]{#1}
    \end{minipage}
  \end{center}
}

\usepackage[export]{adjustbox}
\newcommand{\CRTfigure}[2]{%
  \begin{figure}[h!]
    \centering
    \includegraphics[max width=\textwidth]{#1}
    \caption{#2}
    \label{fig:#1}
  \end{figure}%
}

\newcommand{\CRTfigref}[2]{(см. \autoref{fig:#1})\CRTfigure{#1}{#2}}

\graphicspath{ {../asm/screenshots/} }

\begin{document}
  \titlePage
  \newpage
  \tableofcontents
  \newpage

  \section{Текст задания}
  Разработать программу, вычисляющую с помощью степенного ряда с точностью не хуже $0.1\%$ значение функции
  гиперболического косинуса $\ds\cosh\l(x\r) = \f{e^x+e-x}{2}$ для конкретного параметра $x$ (использовать FPU) \cite{task}

  \newpage
  \section{Применяемые расчетные методы}
  \subsection{Алгоритм вычисления}
  Гиперболический косинус раскладывается в такой степенной ряд \cite{cosh}:
  $$ \cosh\l(x\r) = 1 + \f{x^2}{2!} + \f{x^4}{4!} + \f{x^6}{6!} + \cdots = \sum_{n=0}^\infty \f{x^{2n}}{(2n)!} $$
  Чтобы не вычислять каждый раз факториалы и степени заново мы можем хранить их во временных переменных. \\
  В идеале у нас был бы примерно такой код:
  \frameCode{../asm/mp01/algo.c}
  Но из-за ограничений ассемблера мне пришлось видоизменить алгоритм, чтобы код на ассемблере был более понятным:
  \frameCode{../asm/mp01/impl.c}
  Тут, для увеличения производительности, \texttt{i} каждый раз увеличивается на $2$ и во временных переменных хранится не вся дробь,
  а отдельно числитель в \texttt{tmpPow} и отдельно знаменатель в \texttt{tmpFactorial}.
  Таким образом у нас становится в два раза меньше операций деления, а деление чисел с плавающей точкой очень медленное \cite{fdiv}.
  Но, из-за нехватки точности, \texttt{tmpFactorial} и \texttt{tmpPow} быстро уходят в $\infty$,
  и поэтому посчитать гиперболический косинус от $\l|x\r| > 100$ получается плохо.

  \subsection{Считывание числа $x$}
  Число $x$ считывается с командной строки, или из первого аргумента \texttt{argv}.
  Чтобы функция \texttt{scanf} работала с 64-битными числами с плавающей точкой,
  используется \texttt{"\%lf"} вместо \texttt{"\%f"}.

  % \newpage
  \subsection{Вывод результата в консоль}
  Для вывода используется функция \texttt{printf} и \texttt{"\%23.20g"} вместо \texttt{"\%f"},
  потому что \texttt{"\%f"} для больших чисел выводит очень много лишних нулей.
  Также в \texttt{printf} числа с плавающей точкой всегда передаются 64-битные \cite{promo},
  а в \texttt{x86-32} инструкция \texttt{push} может работать только с 32-битными числами \cite{x86}.

  \subsection{Вывод промежуточных результатов}
  Если передать \texttt{debug} как второй аргумент, программа будет выводить промежуточный результат после каждой итерации
  (см. \autoref{fig:ch_debug.png}).
  Тут используется U+2502 (Box Drawings Light Vertical) вместо U+007C (Vertical Line) для того, чтобы не было пропусков в левой линии.
  \CRTfigure{ch_debug.png}{Вывод промежуточных результатов}

  \newpage
  \section{Тестовые примеры}
  Программа корректно работает на правильных данных \CRTfigref{ch_test_normal.png}{Правильные данные}.

  И выводит сообщения об ошибке при неправильных \CRTfigref{ch_test_err.png}{Неправильные данные}.

  Также программа запрашивает ввод данных, если не был предоставлен командный аргумент \CRTfigref{ch_test_stdin.png}{Запрос данных}.

  \newpage
  \section{Список использованной литературы}
  \patchcmd{\thebibliography}{\section*{\refname}}{}{}{} % убираем двойной заголовок
  \begin{thebibliography}{3}
    \bibitem{task} Инструкция по составлению пояснительной записки [Электронный ресурс].
      //URL: \url{http://softcraft.ru/edu/comparch/tasks/mp01/} (Дата обращения: 10.10.2020, режим доступа: свободный)
    \bibitem{x86} x86 Instruction Set Reference [Электронный ресурс].
      //URL: \url{https://c9x.me/x86/} (Дата обращения: 10.10.2020, режим доступа: свободный)
    \bibitem{cosh} Статья <<Hyperbolic functions>> Wikipedia.org
      //URL: \url{https://en.wikipedia.org/wiki/Hyperbolic_functions} (Дата обращения: 10.10.2020, режим доступа: свободный)
    \bibitem{fdiv} Why is float division slow? [Электронный ресурс].
      //URL: \url{https://stackoverflow.com/a/506345} (Дата обращения: 29.10.2020, режим доступа: свободный)
    \bibitem{promo} Default argument promotions [Электронный ресурс].
      //URL: \url{https://en.cppreference.com/w/c/language/conversion#Default_argument_promotions} (Дата обращения: 29.10.2020, режим доступа: свободный)
  \end{thebibliography}

  \newpage
  \section{Текст программы}
  \verbatiminput{../asm/mp01/ch.asm}

\end{document}
