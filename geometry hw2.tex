\documentclass{article}

\usepackage{enumitem}
\usepackage{amsmath}
\usepackage{amsfonts}
\usepackage{amssymb}
\usepackage[margin=0.5in]{geometry}
\usepackage[hidelinks, bookmarks=false]{hyperref}

\let\oldemptyset\emptyset
\let\emptyset\varnothing
\newcommand{\R}{\mathbb{R}}
\newcommand{\Rd}{\R^d}

\usepackage{environ}
\NewEnviron{centerframebox}{\begin{center}\fbox{\parbox{0.92\textwidth}{\BODY}}\end{center}}

\title{Discrete and Computational Geometry \\ Assignment 2}
\author{
  \AA{AAAAAAAAAA AAAAAAA}{6} \\
  \href{mailto:\AA{AAAAAAAAAAAAAAAAAAAA}{7}}{\AA{AAAAAAAAAAAAAAAAAAAA}{7}}
  \and
  Emilia Groß-Hardt \\
  \href{mailto:emilia.ghdt@uni-bonn.de}{emilia.ghdt@uni-bonn.de}
}

\usepackage{titlesec}
\titleformat{\section}
  {\normalfont\Large\bfseries}{Problem \thesection : }
  {0em}{\mdseries}

\begin{document}
  \maketitle

  \section{Convex hull size}
  \begin{centerframebox}
    We consider $n$ points $(x_1, y_1), (x_2, y_2), \dots , (x_n, y_n)$ in $\R^2$ which are chosen uniformly and
    independently at random from $[0, 1]^2$, i.e. each coordinate is chosen uniformly and
    independently at random from $[0, 1]$.
    Show that the expected number of vertices of the convex hull is $O(\log n)$.
  \end{centerframebox}
  Let $(x_m, y_m)$ be the point with the largest y coordinate. We will now consider the expected number of points on the convex hull, that have x coordinates larger than $x_m$ and are on the top side of the convex hull, as this will be a quarter and thus a linear factor off of the expected number of vertices on the convex hull.

  Sort the points with x-coordinates larger than $x_m$ by their x-coordinate. For every point $(x_a, y_a)$ in this sorted list if there is a point $(x_b, y_b)$ after it in the list with $y_b > y_a$, we can be sure, that the point is not on the top side of the convex hull as it will be below the line from $(x_m, y_m)$ to $(x_b, y_b)$, because $x_m < x_a < x_b$, $y_a < y_b$ and $y_a < y_m$. We can thus Stalin sort the list starting at the end and going to the beginning point by point eliminating points with a y-coordinate, that's smaller, than the largest we've seen so far.

  The number of remaining points, which can potentially be on the convex hull is thus bounded by the expected number of elements left in a list with $n$ elements, after the list has been Stalin sorted.
  This is just the $n$-th harmonic number, as the $i$-th element will not be eliminated iff. it is the largest element out of the first $i$ elements, which happens with probability $\frac1i$. It thus contributes $\frac1i$ to the expected number of elements. Applying linearity of expectation and the fact, that the $n$-th harmonic number is $O(\log n)$, we get the result.

  \section{Line transversals}
  \begin{centerframebox}
    A \textit{line transversal} for a set of \textit{vertical} line segments in the plane is a line that intersects every
    segment. Show that if $I_1, \dots , I_n$, $n \geq 3$ are vertical segments in the plane such that every
    three of them have a line transversal, then all of them have a line transversal.
  \end{centerframebox}
  We'll try to change the problem in such a way that it would be possible to apply Helly's theorem.
  Remember the concept of \textit{dual sets} form Lecture 4.
  They defined the halfspaces dual to each point $a$ by
  \[h^-(a) = \{x \in \R^2 \mid \langle a,x\rangle \leq 1\}\]
  \[h^+(a) = \{x \in \R^2 \mid \langle a,x\rangle \geq 1\}\]

  For a line to intersect our line segment $I$, it needs to include only one of the endpoints of $I$ in the halfspace it defines.
  Denote the endpoints of $I$ as $I^a$ and $I^b$.
  Then the dual set of all the lines that intersect $I$ can be constructed as
  \[ \left(h^-(I^a) \cap h^+(I^b)\right) \cup \left(h^+(I^a) \cap h^-(I^b)\right) \]
  This is not a convex set, so we can't apply Helly's theorem directly.
  Instead it is a union of two convex ``sectors'', one above $y=0$ and the other bellow.
  Because all our line segments are vertical, we can separately consider the top and bottom parts, as they will always connect at $y=0$.

  Given that for all triplets of lines segments have a \textit{line transversal}, i.e. that their dual sets have an intersection,
  those intersections must all be either on the top on the bottom, or both. Otherwise we get a contradiction.
  So we have to consider the 3 cases:
  in the first and second case we can just apply Helly's theorem to the top and bottom parts of each dual, because they are convex.
  And in the third case we can just throw away the bottom parts, because the top parts still have that triple-wise intersection.
  % the last paragraph is bullshit i give up
  % (c) Clover

  \section{Depth algorithm}
  \begin{centerframebox}
    Let $A$ be a set of $n$ points on the plane. Given a point $q$ on the plane, describe an algorithm
    that computes $\operatorname{depth}_A(q)$ in $O(n \log n)$ time.
    Show the correctness and analyze the running time of your algorithm.

    \[
      \operatorname{depth}_A(q) = \min_{a \in \Rd, b \in R, q \in H^\leq_{a,b}} |A \cap H^\leq_{a,b}|
    \]
  \end{centerframebox}
  To compute the depth of a point $q$, we need to sweep a hyperplane (line) center on $q$ thru all other points in $A$,
  and count how of the points are on the left side of that line (in the corresponding halfspace).
  We only need to check halfspaces with $q$ on their boundary, because we are looking for a minimum,
  and halfspaces with $q$ somewhere in the middle cannot contain less points from $A$ (they are ``larger'').
  We also only need to check halfspaces with a point from $A$ on their boundary, because otherwise the $|A \cap H|$ can't change.
  A naive approach would take $O(n^2)$ time, as we would need to check $n$ possible lines and count $n$ points each time.
  But we can do better, by sorting the points.

  The algorithm is as follows:
  \begin{enumerate}
    \item Double all the points in $A$, rotate them $180^\circ$ around $q$, and somehow mark them as ``negative''.
          Call that set $A'$.
    \item Sort the points in set $B := A' \cup A$ according to the angle component of their polar coordinates, measured from the point $q$.
    \item Draw the line thru $q$ and $b_1$.
          Count the number of points in $A$ that lie on the left side of that line, and store that as $m$.
          Allocate $k$ to store the minimum value of $m$.
    \item For each point $b_i$ in $B$ (in that sorted order, staring from $b_2$):
    \begin{enumerate}
      \item Rotate that line to go thru the point $b_i$.
      \item If that point is ``negative'', i.e. $b_i \in A'$, decrement $m$.
      \item Otherwise increment $m$.
      \item Check if $m$ is at its minimum value, and update $k$ if needed.
    \end{enumerate}
    \item Return $k$
  \end{enumerate}

  This algorithm takes $O(n \log n)$ time, because of the sorting, and everything else we do is linear.
  % (c) Clover

\end{document}
