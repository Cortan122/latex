\documentclass[12pt]{article}

\usepackage{cmap} % поиск в pdf
\usepackage[english,russian]{babel} % локализация и переносы

\usepackage[hidelinks]{hyperref}
\usepackage{xurl}
\usepackage{enumitem}

\usepackage[margin=.5in]{geometry}

\usepackage{xcolor}
\definecolor{purple}{HTML}{800080}
\newcommand{\red}[1]{{\color{red}{#1}}}
\newcommand{\orange}[1]{{\color{orange}{#1}}}
\newcommand{\brown}[1]{{\color{brown}{#1}}}
\newcommand{\teal}[1]{{\color{teal}{#1}}}
\newcommand{\purple}[1]{{\color{purple}{#1}}}

\title{История \\ Домашнее задание №10}
\author{AAAAA AAAAAAA \\ AAAAAA}
\date{15 мая 2020 г.}

\begin{document}
  \maketitle

  \setcounter{section}{1}
  \section{Причины первых неудач Красной Армии}
  \subsection{Просчеты высшего политического руководства СССР о сроках нападения Германии}
  Одной из серьезных ошибок советского руководства следует считать просчет
  в определении возможного времени нападения фашистской Германии на Советский Союз.
  Заключенный с Германией в 1939 году пакт о ненападении позволял Сталину и его ближайшему окружению считать,
  что Германия не рискнет нарушить его в ближайшие сроки,
  и у СССР еще есть время для планомерной подготовки к возможному отражению агрессии со стороны противника.
  Кроме того, Сталин считал, что Гитлер не начнет войну на два фронта -- на западе Европы и на территории СССР.
  Советское правительство считало, что до 1942 г. удастся воспрепятствовать вовлечению СССР в войну.
  Как видно, это убеждение оказалось ошибочным.

  Еще одним просчетом руководства СССР и генерального штаба РККА было неправильное определение направления главного удара сил вермахта.
  Основным ударом фашистской Германии считалось не центральное направление, по линии Брест-Минск-Москва,
  а юго-западное, в сторону Киева и Украины.
  В этом направлении буквально перед самой войной были переброшены основные силы Красной Армии, тем самым оголяя другие направления.

  Таким образом, противоречивая информация о сроках нападения Германии на СССР, надежды политического руководства страны на соблюдение противником достигнутых ранее договоренностей, недооценка планов вермахта относительно собственного государства не позволили вовремя подготовиться к отражению удара.

  \subsection{Запаздывание стратегического развертывания советских Вооруженных Сил}
  Дело в том, что ошибочная оценка времени возможного нападения Германии на Советский Союз обусловила запаздывание стратегического развертывания Вооруженных Сил Союза, а внезапность удара уничтожила много боевой техники и складов с боеприпасами.
  Неподготовленность в отражению нападения прежде всего проявлялась в плохой организации обороны.
  Значительная протяженность западной границы обусловила и растягивание сил Красной армии вдоль всей линии рубежа.

  Присоединение к СССР Западной Украины, Западной Белоруссии, Бессарабии, Прибалтийских государств в 1939-1940 г.г. привело к тому,
  что расформировались старые, хорошо организованные пограничные заставы и линии обороны.
  Пограничная структура отодвинулась на запад.
  Пришлось в спешном порядке строить и заново формировать всю пограничную инфраструктуру.
  Делалось это медленно, ощущалась нехватка средств.
  Кроме этого, необходимо было строить новые автомобильные дороги и прокладывать железнодорожные магистрали для подвоза материальных ресурсов, людей.
  Те железнодорожные пути, которые были на территории этих стран, были узкоколейными, европейскими.
  В СССР пути были с широкой колеей.
  В результате, подвоз материалов и техники, оборудование западных границ отставало от потребностей Красной Армии.

  \subsection{Качественное военное превосходство противника}
  Несмотря на договоренности между СССР и Германией о ненападении, никто не сомневался,
  что рано или поздно Советский Союз станет объектом нападения со стороны нацистов.
  Это был лишь вопрос времени.
  Страна старалась подготовиться к отражению агрессии.

  К середине 1941 г. СССР располагал материально-технической базой, обеспечивающей при ее мобилизации
  производство военной техники и вооружения.
  Осуществлялись важные мероприятия по перестройке промышленности и транспорта, готовых к выполнению оборонных заказов,
  развивались вооруженные силы, осуществлялось их техническое перевооружение, расширялась подготовка военных кадров.

  \subsubsection*{\textit{Что же помешало воспользоваться всей техникой и вооружением для отражения атаки фашистской армии?}}
  Количественное превосходство Красной Армии в боевой технике по многим позициям не означало качественного превосходства.
  Современный бой требовал и современного оружия.
  Но с ним было немало проблем.

  Новых моделей самолетов и танков, особенно танков Т-34 и тяжелых танков КВ, выпускалось слишком мало,
  освоить их производство к началу войны полностью не успели.
  К этому привело необдуманное решение ликвидировать крупные соединения бронетанковых войск и заменить их
  на более маневренные и управляемые отдельные бригады, основываясь на специфическом опыте военных действий в Испании 1936-1939 г.г.
  Такая реорганизация была осуществлена накануне войны, но надо признать,
  что Советское командование вскоре поняло ошибку и принялось исправлять ее.
  Вновь начали формировать крупные механизированные корпуса, но к июню 1941 г. они оказались неподготовленными к войне.

  \subsubsection*{\textit{Что же противопоставила фашистская Германия Вооруженным Силам СССР?}}
  Путем милитаризации экономики и всей жизни, захвата промышленности и запасов стратегического сырья других стран,
  принудительного использования дешевой рабочей силы оккупированных государств Германия создала огромный военно-технический потенциал.
  С 1934 г. по 1940 г. военное производство страны увеличилось в 22 раза.
  Почти в 36 раз (с 105 тыс. до 3755 тыс. чел) возросла численность немецких вооруженных сил.

  \subsubsection*{Подытожим:}
  Качественное превосходство германской армии было по стрелковому оружию.
  На вооружении немецких армий было значительное количество автоматического оружия (пистолет-пулемет, или автомат, МП-40).
  Это позволяло навязывать ближний бой, где превосходство автоматического оружия имело большое значение.

  В целом, оценивая боевые возможности советских приграничных округов к началу Великой Отечественной войны,
  можно констатировать их хорошие боевые возможности, хотя и уступающие в некоторых компонентах армии агрессора,
  которые при правильном использовании могли бы способствовать отражению первого удара Германии.

  \subsection{Репрессии в Красной Армии}
  Массовые репрессии конца 30-х годов в значительной степени ослабили командный и офицерский состав Вооруженных сил СССР,
  к началу войны приблизительно 70-75\% командиров и политруков находились на своих должностях не более одного года.
  По подсчетам современных исследователей войны только за 1937-1938 г.г. было репрессировано свыше 40 тыс. командиров Красной Армии
  и советского ВМФ, из них более 9 тыс. человек высшего и старшего командного состава, то есть примерно 60-70\%.
  Например, Тухачевский, во многом обязанный Л. Троцкому карьерой, был обвинен в измене Родине, терроре и военном заговоре,
  так как не возвеличивал имени Сталина, и таким образом, являлся неугодным ему лицом.

  Красная Армия осталась без закаленных в боях опытных командиров.
  Молодые кадры, хотя и были преданы Сталину и Советскому государству, но не обладали талантом и должным опытом.
  Опыт пришлось приобретать на начавшейся войне.
  Таким образом, массовые репрессии создавали тяжелую обстановку в армии, повлияли на боевые качества солдат и офицеров,
  которые оказались мало подготовленными к серьезной войне, ослабили моральные устои.

  \setcounter{section}{4}
  \section{Итоги Великой Отечественной войны}
  Великая Отечественная война завершилась полным поражением гитлеровской Германии и ее союзников.
  Ее основным результатом стало освобождение территории СССР от немецко-фашистской оккупации,
  защита суверенитета страны.

  Были разгромлены военные силы стран-агрессоров, сокрушены жестокие диктаторские режимы.
  Победа над Германией усилила симпатии к СССР во всем мире, подняла его авторитет.
  Советский Союз по праву стал одной из двух мировых держав.

  Советский народ добился не только военной, но и экономической победы.
  Располагая меньшей промышленной базой, СССР сумел ликвидировать временное преимущество фашистов в средствах ведения войны,
  а затем резко превзошел их по количественным и качественным показателям.
  К концу войны Красная армия превосходила врага в орудиях и минометах в четыре раза,
  танках и САУ -- более чем в три раза, а в боевых самолетах -- в восемь раз.
  Советская оборонная индустрия произвела 134 тыс. самолетов, почти 103 тыс. танков и САУ,
  свыше 825 тыс. орудий и минометов.

  \subsection{Последствия войны для СССР}
  Война унесла жизни почти 27 млн. человек\footnote{Оценки числа погибших граждан СССР доходят до 42 млн. человек} (из них примерно 10 млн. солдат и офицеров).
  Сражаясь в врагами, в тылу погибли 4 млн. партизан, подпольщиков, мирных жителей.
  Свыше 8.5 млн. человек оказались в фашистской неволе.
  Демографические последствия войны тяжелы.
  К концу войны из мужчин, родившихся в 1923 г., в живых осталось около 3\%.
  К 1945 г. женщин в возрасте от 20 до 29 лет в СССР было вдвое больше, чем мужчин.
  Резко упала рождаемость.
  В результате разрушения социальной инфраструктуры понизился уровень жизни населения.

  За время войны в СССР было разрушено 1710 городов и поселков, более 70 000 деревень.
  Некоторые из уже никогда не появятся на географических картах, как Хатынь.
  Эту белорусскую деревню немцы сожгли 22 марта 1943 г. вместе со всем ее населением от мала до велика.
  Стариков, женщин, детей загнали в колхозный сарай, заперли двери сарая, обложили его соломой, облили бензином и подожгли.
  Погибли 149 человек, из них 75 детей.
  Деревня была разграблена и сожжена дотла.

  За годы войны было уничтожено более 31 000 предприятий, 13 000 мостов, 65 000 километров железнодорожных путей.
  Прямой ущерб в рублевом эквиваленте составил около 678 млрд.

  На оккупированных территориях сильно пострадало сельское хозяйство.
  Больше всего потерь было в животноводческом секторе.
  СССР потерял более 30\% национального богатства.

  Последствием войны стало перебазирование промышленности из оккупированных районов на Урал,
  а также развитие на Востоке страны новых промышленных зон.

  \subsection{Результаты Великой Отечественной войны}
  Результаты Великой Отечественной войны неразрывно связаны с результатами Второй мировой,
  потому что без победы в Великой Отечественной войне возможно не было бы победы во Второй мировой.

  Несмотря на различные оценки российских и западных историков о роли СССР во Второй мировой войне, они сходятся в понимании того,
  что Великая Отечественная война закончилась полной военно-политической и экономической победой Советского Союза
  и полным идеологическим поражением нацистской Германии,
  предопределив исход Второй мировой войны в целом.
  Основным положительным результатом войны стал разгром нацистской Германии
  и освобождение оккупированных территорий и стран Европы от нацизма.

  \begin{enumerate}
    \item Главный итог войны -- уничтожение фашизма.
    Впервые нацизм был осужден на международном уровне.
    Были созданы условия для демократического развития стран.
    Начался распад колониальной системы.
    \item Создание Организации Объединенных Наций, образование системы коллективной безопасности, возникновение новой организации международных отношений.
    Уже в самом конце Второй мировой войны американцы создали и впервые применили ядерное оружие.
    Ракетно-ядерное оружие привело к резкому изменению общей ситуации в мире.
    С превращением в конце 1940-х г.г. СССР во вторую ядерную державу усилилась гонка вооружений.
    \item Была образована мировая система коммунизма в противовес капиталистической.
    Что можно также трактовать как образование мировой системы тоталитаризма советского типа,
    состоящей из сателлитов СССР\footnote{это так называемые <<страны варшавского договора>>},
    в противовес западной демократии.
    \item Начинается усиленное развития военного искусства и различных видов вооружения.
    Эпоха ускоренного развития ракетной, а затем и ракетно-космической техники.
    \item Во время войны\footnote{в 1943 году} было восстановлено Патриаршество, открыта Троице-Сергиева лавра и многие другие храмы.
    Храмы, открытые германской армией на оккупированной территории, не были закрыты и после освобождения от фашистов.
    Репрессии против церкви и верующих были ослаблены, хотя атеизм продолжал пропагандироваться.
    В 1948 году они были возобновлены, но никогда впоследствии не достигали довоенного уровня.
    \item После войны был закономерный рост преступности, который можно объяснить двумя причинами:
    резкое падение жизненного уровня вследствие послевоенной разрухи и большое количество оружия на руках у населения.
    \item Усиление культа личности Сталина.
    Его начали почитать как гения и главного виновника победы, а также называть <<отцом народов>>.
  \end{enumerate}

  % \begin{thebibliography}{9}
  %   \bibitem{причины1} \url{http://bagazhznaniy.ru/history/rossijskaya-revolyuciya-1917g-prichiny-zadachi-posledstviya}
  % \end{thebibliography}

\end{document}
