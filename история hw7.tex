\documentclass[12pt]{article}

\usepackage{cmap} % поиск в pdf
\usepackage[english,russian]{babel} % локализация и переносы

\usepackage[hidelinks]{hyperref}
\usepackage{xurl}
\usepackage{enumitem}

\usepackage[margin=1in]{geometry}

\usepackage{xcolor}
\newcommand{\red}[1]{{\color{red}{#1}}}
\newcommand{\orange}[1]{{\color{orange}{#1}}}
\newcommand{\brown}[1]{{\color{brown}{#1}}}
\newcommand{\teal}[1]{{\color{teal}{#1}}}

% \pagecolor{black}
% \color{white}

\title{История \\ Домашнее задание №7}
\author{\AA{AAAAA AAAAAAA}{4} \\ \AA{AAAAAA}{11}}

\begin{document}
  \maketitle

  \section{Деятельность III Отделения -- это <<утверждение благосостояния и спокойствия>> или воплощение <<апогея самодержавия>>?}
  Третье отделение Собственной Его Императорского Величества канцелярии -- это высший орган политической полиции Российской империи во времена правлений Николая I и Александра II (с 1826 по 1880 годы).

  <<Апогей самодержавия>> -- так называл А.Е. Пресняков время Николая I.
  Действительно, каждый день своего 30-летнего царствования Николай использовал для того, чтобы всемерно укреплять самодержавный режим.
  Прежде всего с целью заблаговременного обезвреживания революционных идей Николай усилил политический сыск.
  Именно он 3 июля 1826 г. образовал зловещее III отделение Собственной Его императорского величества канцелярии.

  III отделение разделялось на пять экспедиций, которые следили за революционерами, сектантами, уголовниками, иностранцами и прессой.
  Начальник III отделения являлся и шефом жандармов\footnote{Жандармерия или внутренние войска, -- полиция, имеющая военную организацию и выполняющая охранные функции внутри страны и в армии (полевая жандармерия или военная полиция)}.
  На этот пост выдвигались самые близкие к царю лица.
  Первым из них был граф А.Х. Бенкендорф -- услужливый царедворец и проницательный (хотя и ленивый) сыщик.

  Чтобы замаскировать репрессивную сущность III отделения, официальная пропаганда восхваляла его как блюстителя законности в стране,
  как орган, призванный стоять горой за <<бедных и сирых>>.
  С этой целью распространялась легенда о том, что Николай I вместо инструкции о руководстве III отделением протянул Бенкендорфу носовой платок и сказал: <<Вот тебе инструкция: чтоб ни один платок в России не был омочен слезами!>>
  Никто не верил таким легендам.
  За время царствования Николая каждый россиянин мог убедиться в том, что III отделение -- это,
  как назвал его Герцен, <<вооруженная инквизиция>>, которая стоит <<вне закона и над законом>>.

  \setcounter{section}{3}
  \section{Великие реформы: предпосылки, содержание, итоги.}
  В 1855 г. на престол взошел император Александр II, который на одном из своих открытых выступлений перед дворянством заявил о том,
  что необходимо срочно упразднить крепостничество указом сверху, пока это не сделали крестьяне снизу путем революции.
  Суть реформ Александра II заключалась в перестройке государства таким образом,
  чтобы оно могло бы эффективнее вести экономику по пути индустриализации и капитализма.

  Основные реформы Александра II:
  \begin{itemize}[noitemsep,parsep=0pt,partopsep=0pt]
    \item Отмена крепостного права (1861)
    \item Финансовая реформа (1863)
    \item Реформа образования (1863)
    \item Земская реформа (1864)
    \item Городская реформа (1864)
    \item Судебная реформа (1864)
    \item Реформа государственного управления (1870)
    \item Военная реформа (1874)
  \end{itemize}

  Ключевой была крестьянская реформа, которая провозгласила отмену крепостного права в 1861 году.
  Реформа готовилась на протяжении нескольких лет, и, хотя правящие классы не хотели свободы для крестьян,
  император понимал, что с крепостным правом дальше двигаться невозможно, поэтому изменения все же были осуществлены.
  В результате реформы крепостное право было упразднено, крестьяне получили свободу и могли выкупиться от своего помещика,
  получив при этом надел для ведения домашнего хозяйства.
  Для осуществления выкупа крестьянин мог взять ссуду в банке на 49 лет.
  Выкупившиеся крестьяне освобождались от административной и юридической зависимости от помещиков.
  Кроме того, свободные крестьяне получили ряд гражданских прав, могли вести торговлю и осуществлять сделки с недвижимостью.

  Реформы Александра II имели следующие последствия для России:
  \begin{itemize}
    \item
    Созданы предпосылки для построения капиталистической модели экономики.
    В стране был снижен уровень государственного регулирования экономики, а также создан свободный рынок рабочей силы.
    Тем не менее, промышленность не была на 100\% готова к восприятию капиталистической модели.
    Для этого требовалось больше времени.
    \item
    Заложены основы формирования гражданского общества. Население получило больше гражданских прав и свобод.
    Это касается всех сфер деятельности, начиная от образования, заканчивая реальными свободами передвижения и труда.
    \item
    Усиление оппозиционного движения.
    Основная часть реформ Александра II были либеральными, поэтому либеральные движения,
    которые были подавлены Николаем I, вновь начали набирать силу.
    % Именно в эту эпоху заложены ключевые аспекты, которые привели к событиям 1917 года.
  \end{itemize}


\end{document}
