\documentclass{article}

\usepackage{amsmath}
\usepackage{amsfonts}
\usepackage{amssymb}
\usepackage[dvipsnames]{xcolor}
\usepackage[margin=0.5in]{geometry}
\usepackage[hidelinks, bookmarks=false]{hyperref}
\usepackage{enumitem}

\let\oldemptyset\emptyset
\let\emptyset\varnothing
\let\oldphi\phi
\let\phi\varphi

\usepackage{multirow}
\newcommand{\MGinc}[1]{\textcolor{LimeGreen}{#1}}
\newcommand{\MGdec}[1]{\textcolor{Red}{#1}}

\usepackage{environ}
\NewEnviron{centerframebox}{\begin{center}\fbox{\parbox{0.92\textwidth}{\BODY}}\end{center}}

\title{Algorithms for Data Science \\ Exercise Sheet 6}
\author{
  Vladislav Imashev \\ \href{mailto:s05vimas@uni-bonn.de}{s05vimas@uni-bonn.de} \and
  \AA{AAAAAAAAAA AAAAAAA}{6} \\ \href{mailto:\AA{AAAAAAAAAAAAAAAAAAAA}{7}}{\AA{AAAAAAAAAAAAAAAAAAAA}{7}} \and
  German Mikhelson \\ \href{mailto:s17gmikh@uni-bonn.de}{s17gmikh@uni-bonn.de} \and
  Aleksandra Volynets \\ \href{mailto:s02avoly@uni-bonn.de}{s02avoly@uni-bonn.de} \and
  Nikita Morev \\ \href{mailto:s99nmore@uni-bonn.de}{s99nmore@uni-bonn.de}
}

\begin{document}
  \maketitle

  \setcounter{section}{6}
  \subsection{Reservoir Sampling I}
  \begin{centerframebox}
    Consider the algorithm on Slide 17 of 2023-11-29.

    \textbf{Algorithm} for selecting one element of the data stream uniformly at random: \\
    \textbf{initialization}: set $S := 0$ \\
    \textbf{processing of $a_l$}: // $l$-th element of the data stream
    \begin{enumerate}[topsep=0pt,itemsep=-1ex,partopsep=1ex,parsep=1ex]
      \item \textbf{if} $l = 1$ then $S = \{a_1\}$
      \item \textbf{else}
      \item ~~flip a (biased) coin with $\Pr[\textsc{Head}] = 1/l$
      \item ~~\textbf{if} outcome is \textsc{Head} \textbf{then} set $S = \{a_l\}$
    \end{enumerate}
    \textbf{output}: return $S$

    \begin{enumerate}[label=(\roman*)]
      \item Prove the theorem on Slide 17 of 2023-11-29. \\
      \textit{You are not allowed to use the theorem stated on Slide 20 of 2023-11-29!}

      \textbf{Theorem}: For any $m > 0$ integer and $l \in [m]: \Pr[S = \{a_l\}] = \frac{1}{m}$

      \item For a data stream of length $m$, what is the expected number of updates of the item in $S$?
    \end{enumerate}
  \end{centerframebox}
  \subsubsection*{(i) Uniformity}
  TODO% (Vadim; Vlad)

  \subsubsection*{(ii) Expected number of updates}
  The expected number of updates is just the sum of the probabilities of rolling \textsc{Heads} each time,
  because the expected value is linear.
  So the sum of the harmonic series up to $m$:
  \[ 1+\frac{1}{2}+\frac{1}{3} + \cdots + \frac{1}{m} = \sum_{i=1}^m \frac{1}{i} \approx O(\log m) \]

  \subsection{Reservoir Sampling II}
  \begin{centerframebox}
    Prove the theorem on Slide 20 of 2023-11-29.

    \textbf{Theorem}: after $I$ elements, $S$ has $\min\{s,\, l\}$ elements, each picked with probability $s/l$
  \end{centerframebox}
  The case where $l \leq s$ is obvious. The list $S$ will always include all the $l$ numbers it came across with probability $1$, i.e. with probability $s/l$, if we cap it at one.

  Now we can do induction on $l$, with $l > s$. The size of $S$ will always be $s$, and we know that the probability to see each item in the previous iteration of the algorithm was $\frac{s}{l-1}$, so all the previous $l-1$ items have an equal probability of appearing in $S$.
  Our new item $a_l$ will have a $s/l$ probability to replace one of the existing items,
  and because the situation is symmetrical, all the previous items will have the same probability $\beta$ after this removal step.
  Because there are $s$ items in total and each item can only be selected once, the sum of the probabilities for each item must equal $s$.
  \begin{align*}
    \beta &\cdot (l-1) + \frac{s}{l} = s \\
    \beta &= \frac{s - \frac{s}{l}}{l-1} \\
    \beta &= \frac{sl - s}{l \cdot (l-1)} \\
    \beta &= \frac{s}{l}
  \end{align*}
  So the probabilities for every item, including the new one, will be $\frac{s}{l}$.

  % But now the probability of the element will decrease, because there's a chance it will be replaced.
  % The algorithm decides to replace each item with probability $s/l$ and then has a $1/s$ chance to replace our item.
  % \[ \frac{s}{l-1} - \frac{s}{l}\frac{1}{s} = ... \]
  % TODO (Kostya; Sasha he/him)

  \subsection{The $\phi$-HH Problem: Lower Bounds}
  \begin{centerframebox}
    Prove the claim on Slide 25 of 2023-11-29.

    Let $S_1,\, S_2 \in \mathcal{S}$ be distinct sets and let the state of the storage be $\Sigma_1$
    (respectively $\Sigma_2$) after algorithm $\mathfrak{A}$ has processed the stream $\langle S_1\rangle$
    (respectively $\langle S_2\rangle$) \\
    \textbf{Claim}: $\Sigma_1 \neq \Sigma_2$
  \end{centerframebox}

  From the proof of this theorem, we know:
  \begin{center}
      let $n =2m$ and $S$ be the family of all $m$ subsets of $[n] = [2m]$
  \end{center}

  So, since $S_1,\,S_2\in \mathcal{S}$, this means that $S_1,\, S_2$ are m-subsets, i.e. they have the same cardinality, and they do not contain duplicate elements.

  Let's prove the opposite, let $\Sigma_1 = \Sigma_2$, i.e. the algorithm gives the same results for both sets. Then, consider 2 approaches:
  \begin{itemize}
      \item\textbf{Degenerate}, that is, $\phi = \frac{1}{m}$ this means that each of the elements satisfies the inequality of $\phi$-Heavy-Hitter. From this we conclude that when passing $S_1$ and $S_2$ through the algorithm, we get the same sets $S_1$ and $S_2$, and they are conditionally different, i.e. $\Sigma_1\neq\Sigma_2$. And in another case, when $\phi = 1$, this means that no element satisfies the $\phi$-Heavy-Hitter inequality and the result of the $\mathfrak{A}$ algorithm for $S_1$ and $S_2$ is empty, but our $\Sigma_1$ and $\Sigma_2$ are abstract and can contain other information, so we cannot claim that $\Sigma_1 = \Sigma_2$

      \item\textbf{Non-degenerate}, i.e. a certain threshold $\phi$ is set. Because $S_1$ and $S_2$ are distinct, they differ by at least 1 element. Let's copy one distinctive element $\beta\in S_1$ and $\beta\notin S_2$, and do it $x$ times in $S_1$ and $S_2$. Let's show what x is equal to:
      \[\phi \cdot (m + x) - 1 = x \iff x = \frac{1 - m \cdot \phi}{\phi - 1}\]
      Thus, we get the data streams: $\langle S_1, \alpha \rangle$ and $\langle S_2, \alpha \rangle$, where $\alpha$ is the letter $\beta$ copied $x$ times. It turns out that in $\langle S_1, \alpha \rangle$ the copied element occurs $x + 1$ times, which satisfies the original constraint and makes $\beta$ a $\phi$-Heavy-Hitter, while in $\langle S_2, \alpha \rangle$ this element occurs $x$ times, which does not satisfy, therefore the output results of $\mathfrak{A}$ must be different and different results can only be attained by different states $\Sigma_{1\alpha}\neq \Sigma_{2\alpha}$. But we sent the same $\alpha$ starting from $\Sigma_{1} = \Sigma_{2}$, so it should also be true $\Sigma_{1\alpha} = \Sigma_{2\alpha}$, and we get a contradiction.
  \end{itemize}

  In both task, we got a contradiction, hence $\Sigma_1\neq\Sigma_2$.

  \subsection{The Misra-Gries Algorithm}
  \begin{centerframebox}
    Using the Misra-Gries algorithm, output a superset of the elements in $[10]$ that
    have an absolute frequency greater than $10/3$ in the following data stream:
    \[ 2, 1, 2, 2, 3, 3, 6, 6, 2, 5 \]
    Give the state of the associative array for \textit{all} steps of the algorithm.
  \end{centerframebox}
  We need to find all items of frequency at least $\frac{10}{3}$. Given $m=10$, $k$ has to be greater than ${m}/{\frac{10}{3}}=\frac{3m}{10}=3$. Choose $k=4$.
  \\Misra-Gries Algorithm:

  \begin{center}
    \begin{minipage}{0.45\textwidth}
      \begin{center}
        % первая колонка
        \begin{tabular}{c|c|c|}
          \cline{2-3}
          \multirow{2}{*}{\textbf{\LARGE{$a_0=2$}}}
          & Key & \MGinc{2} \\\cline{2-3}
          & Counter & \MGinc{1} \\
          \cline{2-3}
        \end{tabular}\vspace{3mm}

        \begin{tabular}{c|c|c|c|}
          \cline{2-4}
          \multirow{2}{*}{\textbf{\LARGE{$a_1=1$}}}
          & Key & 2 & \MGinc{1} \\\cline{2-4}
          & Counter & 1 & \MGinc{1} \\
          \cline{2-4}
        \end{tabular}\vspace{3mm}

        \begin{tabular}{c|c|c|c|}
          \cline{2-4}
          \multirow{2}{*}{\textbf{\LARGE{$a_2=2$}}}
          & Key & \MGinc{2} & 1 \\\cline{2-4}
          & Counter & \MGinc{2} & 1 \\
          \cline{2-4}
        \end{tabular}\vspace{3mm}

        \begin{tabular}{c|c|c|c|}
          \cline{2-4}
          \multirow{2}{*}{\textbf{\LARGE{$a_3=2$}}}
          & Key & \MGinc{2} & 1 \\\cline{2-4}
          & Counter & \MGinc{3} & 1 \\
          \cline{2-4}
        \end{tabular}\vspace{3mm}

        \begin{tabular}{c|c|c|c|c|}
          \cline{2-5}
          \multirow{2}{*}{\textbf{\LARGE{$a_4=3$}}}
          & Key & 2 & 1 & \MGinc{3} \\\cline{2-5}
          & Counter & 3 & 1 & \MGinc{1} \\
          \cline{2-5}
        \end{tabular}

      \end{center}
    \end{minipage}
    \begin{minipage}{0.45\textwidth}
      \begin{center}
        % вторая колонка
        \begin{tabular}{c|c|c|c|c|}
          \cline{2-5}
          \multirow{2}{*}{\textbf{\LARGE{$a_5=3$}}}
          & Key & 2 & 1 & \MGinc{3} \\\cline{2-5}
          & Counter & 3 & 1 & \MGinc{2} \\
          \cline{2-5}
        \end{tabular}\vspace{3mm}

        \begin{tabular}{c|c|c|c|c|}
          \cline{2-4}
          \multirow{2}{*}{\textbf{\LARGE{$a_6=6$}}}
          & Key & \MGdec{2} & \MGdec{3} \\\cline{2-4}
          & Counter & \MGdec{2} & \MGdec{1} \\
          \cline{2-4}
        \end{tabular}\vspace{3mm}

        \begin{tabular}{c|c|c|c|c|}
          \cline{2-5}
          \multirow{2}{*}{\textbf{\LARGE{$a_7=6$}}}
          & Key & 2 & 3 & \MGinc{6} \\\cline{2-5}
          & Counter & 2 & 1 & \MGinc{1} \\
          \cline{2-5}
        \end{tabular}\vspace{3mm}

        \begin{tabular}{c|c|c|c|c|}
          \cline{2-5}
          \multirow{2}{*}{\textbf{\LARGE{$a_8=2$}}}
          & Key & \MGinc{2} & 3 & 6 \\\cline{2-5}
          & Counter & \MGinc{3} & 1 & 1 \\
          \cline{2-5}
        \end{tabular}\vspace{3mm}

        \begin{tabular}{c|c|c|c|c|c|}
          \cline{2-3}
          \multirow{2}{*}{\textbf{\LARGE{$a_9=5$}}}
          & Key & \MGdec{2} \\\cline{2-3}
          & Counter & \MGdec{2} \\
          \cline{2-3}
        \end{tabular}
      \end{center}
    \end{minipage}
  \end{center}\vspace{3mm}

  2nd pass to calculate the exact absolute frequencies:
  \begin{center}
    \begin{minipage}{0.45\textwidth}
      \begin{center}
        % первая колонка
        \begin{tabular}{c|c|c|}
          \cline{2-3}
          \multirow{2}{*}{\textbf{\LARGE{$a_0=2$}}}
          & Key & \MGinc{2} \\\cline{2-3}
          & Counter & \MGinc{1} \\
          \cline{2-3}
        \end{tabular}\vspace{3mm}

        \begin{tabular}{c|c|c|}
          \cline{2-3}
          \multirow{2}{*}{\textbf{\LARGE{$a_1=1$}}}
          & Key & 2 \\\cline{2-3}
          & Counter & 1 \\
          \cline{2-3}
        \end{tabular}\vspace{3mm}

        \begin{tabular}{c|c|c|}
          \cline{2-3}
          \multirow{2}{*}{\textbf{\LARGE{$a_2=2$}}}
          & Key & \MGinc{2} \\\cline{2-3}
          & Counter & \MGinc{2} \\
          \cline{2-3}
        \end{tabular}\vspace{3mm}

        \begin{tabular}{c|c|c|}
          \cline{2-3}
          \multirow{2}{*}{\textbf{\LARGE{$a_3=2$}}}
          & Key & \MGinc{2} \\\cline{2-3}
          & Counter & \MGinc{3} \\
          \cline{2-3}
        \end{tabular}\vspace{3mm}

        \begin{tabular}{c|c|c|}
          \cline{2-3}
          \multirow{2}{*}{\textbf{\LARGE{$a_4=3$}}}
          & Key & 2 \\\cline{2-3}
          & Counter & 3 \\
          \cline{2-3}
        \end{tabular}\vspace{3mm}

      \end{center}
    \end{minipage}
    \begin{minipage}{0.45\textwidth}
      \begin{center}
        % вторая колонка
        \begin{tabular}{c|c|c|}
          \cline{2-3}
          \multirow{2}{*}{\textbf{\LARGE{$a_5=3$}}}
          & Key & 2 \\\cline{2-3}
          & Counter & 3 \\
          \cline{2-3}
        \end{tabular}\vspace{3mm}

        \begin{tabular}{c|c|c|}
          \cline{2-3}
          \multirow{2}{*}{\textbf{\LARGE{$a_6=6$}}}
          & Key & 2 \\\cline{2-3}
          & Counter & 3 \\
          \cline{2-3}
        \end{tabular}\vspace{3mm}

        \begin{tabular}{c|c|c|}
          \cline{2-3}
          \multirow{2}{*}{\textbf{\LARGE{$a_7=6$}}}
          & Key & 2 \\\cline{2-3}
          & Counter & 3 \\
          \cline{2-3}
        \end{tabular}\vspace{3mm}

        \begin{tabular}{c|c|c|c|c|c|}
          \cline{2-3}
          \multirow{2}{*}{\textbf{\LARGE{$a_8=2$}}}
          & Key & \MGinc{2} \\\cline{2-3}
          & Counter & \MGinc{4} \\
          \cline{2-3}
        \end{tabular}\vspace{3mm}

        \begin{tabular}{c|c|c|}
          \cline{2-3}
          \multirow{2}{*}{\textbf{\LARGE{$a_9=5$}}}
          & Key & 2 \\\cline{2-3}
          & Counter & 4 \\
          \cline{2-3}
        \end{tabular}
      \end{center}
    \end{minipage}
  \end{center}\vspace{3mm}
  \textbf{Result}: 2.
\end{document}
