\documentclass[12pt]{article}

\usepackage{cmap} % поиск в pdf
\usepackage[english,russian]{babel} % локализация и переносы

\usepackage[hidelinks]{hyperref}
\usepackage{xurl}
\usepackage{enumitem}
\usepackage{framed}
\usepackage[many]{tcolorbox}

\newtcolorbox{cross}{blank,breakable,
  overlay={\draw[red,line width=5pt] (interior.south west)--(interior.north east);
    \draw[red,line width=5pt] (interior.north west)--(interior.south east);}}

\usepackage[margin=.5in]{geometry}

\usepackage{xcolor}
\newcommand{\red}[1]{{\color{red}{#1}}}
\newcommand{\orange}[1]{{\color{orange}{#1}}}
\newcommand{\brown}[1]{{\color{brown}{#1}}}
\newcommand{\teal}[1]{{\color{teal}{#1}}}

\newcommand{\block}[1]{\vspace{-1em}\begin{framed}#1\end{framed}\vspace{-\lastskip}\vspace{-.1cm}\hfill}

\bgroup\catcode`\%=12\catcode`\#=12
\gdef\wurl{\href{https://ru.wikipedia.org/wiki/%D0%91%D0%BE%D0%BB%D1%8C%D1%88%D0%BE%D0%B9_%D1%82%D0%B5%D1%80%D1%80%D0%BE%D1%80#%D0%9C%D0%BD%D0%B5%D0%BD%D0%B8%D1%8F_%D0%BE_%D0%B2%D0%BE%D0%B7%D0%BC%D0%BE%D0%B6%D0%BD%D1%8B%D1%85_%D0%BF%D1%80%D0%B8%D1%87%D0%B8%D0%BD%D0%B0%D1%85_%D0%B1%D0%BE%D0%BB%D1%8C%D1%88%D0%BE%D0%B3%D0%BE_%D1%82%D0%B5%D1%80%D1%80%D0%BE%D1%80%D0%B0}{Source}}
\gdef\vurl{\href{https://www.vedomosti.ru/opinion/articles/2017/07/06/712528-bolshogo-terrora}{Source}}
\egroup

% \pagecolor{black}
% \color{white}

\title{История \\ Домашнее задание №9}
\author{AAAAA AAAAAAA \\ AAAAAA}

\begin{document}
  \maketitle

  \setcounter{section}{3}
  \section{<<Большой террор>> 1937--1938 гг.: причины, механизмы реализации, результаты}
  Сталинские репрессии не ограничились 1937--1938 годами.
  Большой террор это просто период наиболее массовых репрессий.
  \subsection{Причины}
  % В научной среде нет согласия по поводу удовлетворительного объяснения причин террора.
  В научных кругах до сих пор нет единого мнения, что было причиной террора.
  Практически все историки сходятся во мнении, что Сталин сыграл ключевую роль в организации и проведении репрессий.

  По мнению историка О. В. Хлевнюка, террор 1937--1938 года связан с обозначившейся угрозой войны.
  Содержание приказов, регулирующих массовые операции, демонстрировало стремление сталинского руководства ликвидировать воображаемую <<пятую колонну>>.
  Именно подготовкой к войне объясняли массовые операции сами сотрудники НКВД \teal{(ранняя версия КГБ)} в своей среде.
  При этом заговоры и <<пятая колонна>> существовали именно в представлениях Сталина и его соратников, но не в реальности.

  Советолог Шейла Фицпатрик проводит параллели между Большим террором и якобинским террором 1794 года Великой Французской революции, так как в обоих случаях в первую очередь уничтожались старые революционеры, под предлогом подавления <<внутренней контрреволюции>>, необходимости обновления общества принесением в жертву тысяч голов изменников.
  При этом масштабы террора были несопоставимы и, в отличие от якобинского террора, в ходе Большого террора уничтожались не оппозиционные политические структуры (они были давно ликвидированы), а не связанные между собой люди.

  Историк Ю. Н. Жуков выдвинул гипотезу о связи Большого террора с принятием Конституции СССР 1936 года и выборами в Верховный Совет СССР в декабре 1937 года.
  По его утверждениям, Сталин намеревался провести первые выборы в Верховный Совет СССР как альтернативные, состязательные, хотел <<вообще отстранить партию от власти>>, но ему в этом помешали руководители региональных партийных органов, которые якобы боялись лишиться своих постов в ходе этих выборов и потому выступили инициаторами Большого террора.
  И. В. Павлова критиковала эту концепцию, доказывая, что \textbf{инициатором террора был именно Сталин}, а выборы изначально были лишь имитацией демократии.
  Специалист по истории СССР 1920--1950-х годов О. В. Хлевнюк критично отозвался о теории Жукова, ссылаясь на сверхвольное обращение с источниками.
  По утверждению О. В. Хлевнюка, \textbf{исключительная роль Сталина в организации этого всплеска террора не вызывает сомнений и подтверждается документально.
  Например имеются записи, в которых он требовал применять методы физического насилия против арестованных людей.}
  \\ \teal{тут я ещё хотел написать что называть Большой террор ежовщиной немного нечестно.
  массовые репрессии небыли инициативой и виной Ежова и его прислужников.
  Ежов was Just Following Orders и Сталин и его тоже растрелял}

  Германский историк Йорг Баберовски считал причиной перехода к Большому террору сопротивление Сталину со стороны региональных и местных элит, саботировавших указания центра.
  Французский историк Николя Верт \textbf{рассматривал Большой террор как осуществляемый властью механизм социальной инженерии, окончательное завершение политики <<раскулачивания>> и депортаций <<вредных элементов>>.}
  Американский историк Дэвид Ширер связывал возникновение Большого террора с преодолением общего хозяйственно-политического хаоса первой половины и середины 1930-х годов.

  Некоторые специалисты усматривают в массовых операциях орудие социальной инженерии, насильственной унификации общества.
  Такие мотивы, несомненно, характерны для любого диктатора.
  Однако, полагаясь только на них, мы не сможем понять, почему Сталин решил в ударном порядке <<унифицировать>> общество именно в 1937 г.
  Много пишут о стремлении кремлевской власти найти козлов отпущения среди чиновников, объявить их ответственными за огромные тяготы жизни народа, выпустить пар социального напряжения.
  И этот мотив, несомненно, присутствовал во всех сталинских репрессиях, в том числе в чистках номенклатуры в 1937--1938 гг.
  Однако он не объясняет массовые операции, которые обрушились в основном на рядовых граждан страны.
  Обращается внимание на необходимость обеспечения рабочей силой растущей экономики ГУЛАГа.
  Однако применительно к 1937--1938 гг. и это объяснение не работает.
  Среди 700 000 расстрелянных и замученных в ходе следствия подавляющее большинство составляли люди в трудоспособном возрасте.
  Их просто уничтожили, а не послали на хозяйственные объекты ГУЛАГа.

  \subsection{Механизмы реализации}
  \teal{тут было написано что доносы это миф}

  Первые серьезные сомнения по поводу доносов у историков появились в начале 1990-х гг.,
  когда ненадолго открылся доступ к материалам следственных дел 1937--1938 гг.
  Выяснилось, что основой обвинительных материалов в следственных делах были признания, полученные во время следствия.
  При этом заявления и доносы как доказательство вины арестованного в следственных делах встречаются крайне редко.
  Глубокое исследование механизмов <<большого террора>> вполне разъяснило причины такого положения.
  Организация массовых операций 1937--1938 гг. не предусматривала широкого использования доносов как основы для арестов.
  Изъятия антисоветских элементов проводились первоначально на основе картотек НКВД, а затем на основе показаний, выбитых на следствии.
  Запустив конвейер допросов с применением пыток, чекисты в избытке были обеспечены <<врагами>> и не нуждались в подсказках доносчиков.

  \subsection{Результаты}
  Террор привел к дезорганизации партийной системы управления страной.
  НКВД господствовало над партийно-государственной структурой.

  Под давлением уже новой партийной элиты, опасавшейся следующей волны репрессий, были официально запрещены допускавшиеся ранее физические пытки.
  Сталин прекратил массированное уничтожение правящего слоя.
  Возникшее в период террора господство органов НКВД над партийными структурами постепенно было ликвидировано.
  У обеих структур остался только один хозяин -- Вождь.

  \teal{тут можно ещё сказать что Советско-финляндская война была настолько провальной, потомучто небыло нармальных генералов, тк их всех растреляли}

  % \begin{thebibliography}{9}
  %   \bibitem{причины1} \url{http://bagazhznaniy.ru/history/rossijskaya-revolyuciya-1917g-prichiny-zadachi-posledstviya}
  % \end{thebibliography}

\end{document}
