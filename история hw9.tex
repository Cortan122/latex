\documentclass[12pt]{article}

\usepackage{cmap} % поиск в pdf
\usepackage[english,russian]{babel} % локализация и переносы

\usepackage[hidelinks]{hyperref}
\usepackage{xurl}
\usepackage{enumitem}
\usepackage{framed}
\usepackage[many]{tcolorbox}

\newtcolorbox{cross}{blank,breakable,
  overlay={\draw[red,line width=5pt] (interior.south west)--(interior.north east);
    \draw[red,line width=5pt] (interior.north west)--(interior.south east);}}

\usepackage[margin=.5in]{geometry}

\usepackage{xcolor}
\newcommand{\red}[1]{{\color{red}{#1}}}
\newcommand{\orange}[1]{{\color{orange}{#1}}}
\newcommand{\brown}[1]{{\color{brown}{#1}}}
\newcommand{\teal}[1]{{\color{teal}{#1}}}

\newcommand{\block}[1]{\vspace{-1em}\begin{framed}#1\end{framed}\vspace{-\lastskip}\vspace{-.1cm}\hfill}

\bgroup\catcode`\%=12\catcode`\#=12
\gdef\wurl{\href{https://ru.wikipedia.org/wiki/%D0%91%D0%BE%D0%BB%D1%8C%D1%88%D0%BE%D0%B9_%D1%82%D0%B5%D1%80%D1%80%D0%BE%D1%80#%D0%9C%D0%BD%D0%B5%D0%BD%D0%B8%D1%8F_%D0%BE_%D0%B2%D0%BE%D0%B7%D0%BC%D0%BE%D0%B6%D0%BD%D1%8B%D1%85_%D0%BF%D1%80%D0%B8%D1%87%D0%B8%D0%BD%D0%B0%D1%85_%D0%B1%D0%BE%D0%BB%D1%8C%D1%88%D0%BE%D0%B3%D0%BE_%D1%82%D0%B5%D1%80%D1%80%D0%BE%D1%80%D0%B0}{Source}}
\gdef\vurl{\href{https://www.vedomosti.ru/opinion/articles/2017/07/06/712528-bolshogo-terrora}{Source}}
\egroup

% \pagecolor{black}
% \color{white}

\title{История \\ Домашнее задание №9}
\author{AAAAA AAAAAAA \\ AAAAAA}
\date{24 апреля 2020 г.}

\begin{document}
  \maketitle

  \setcounter{section}{2}
  \section{При помощи каких институтов и практик формировалась тоталитарная система?}
  \subsection{Усиление роли партийного аппарата}
  Сочетание экономических, политических и культурных фактором послужило мощным толчком к формированию тоталитарного режима на территории СССР в 30-е годы.

  Основной отличительной чертой тоталитарного режима стал перенос центра тяжести на партийные, чрезвычайные и карательные органы.
  После очередного съезда ВКП(б) было принято решение, касательно значительного усиление роли партийного аппарата.
  Теперь ему было дано право на занятие вопросами государственного и хозяйственного управления, а партийное руководство получило неограниченную свободу, рядовые коммунисты в свою очередь получили приказ, который заключался в строгом подчинении руководящим центрам партийной иерархии.

  Врастание партии в экономику и государственную сферу на тот момент стало отличительной чертой политической системы.
  Существовала пирамида партийно-государственного управления, во главе которой стоял Сталин, как Генеральный секретарь ЦК ВКП(б).
  \textbf{Таким образом, ранее второстепенная должность генерального секретаря, теперь получила первостепенное значение,
  давая тем самым занимающему должность человеку полную власть в государстве.}

  \subsection{Ликвидация НЭПа}
  Со второй половины 1920-х годов начались первые попытки свёртывания НЭПа.
  Ликвидировались синдикаты в промышленности, из которой административно вытеснялся частный капитал,
  создавалась жёсткая централизованная система управления экономикой (хозяйственные наркоматы).

  Непосредственным поводом для полного сворачивания НЭПа послужил срыв государственных хлебозаготовок в конце 1927 года.
  В конце декабря по отношению к кулачеству впервые после окончания <<военного коммунизма>> были применены меры принудительной конфискации хлебных запасов.
  Летом 1928 года они были временно приостановлены, но затем возобновились осенью того же года.

  В октябре 1928 года началось осуществление первого пятилетнего плана развития народного хозяйства,
  руководство страны взяло курс на форсированную индустриализацию и коллективизацию.
  Хотя официально НЭП никто не отменял, к тому времени он был уже фактически свёрнут.

  Юридически НЭП был прекращён только 11 октября 1931 года, когда было принято постановление о полном запрете частной торговли в СССР.

  \subsection{Уничтожение внутрипартийной оппозиции}
  Левая оппозиция начала оформляться в ходе внутрипартийной борьбы в период болезни Ленина и особенно после его смерти в январе 1924 года.
  Противостояние происходило между Троцким и его сторонниками, в том числе подписавшимися в октябре 1923 года под <<Заявлением 46-ти>>,
  с одной стороны, и триумвиратом в составе Зиновьева, Сталина и Каменева и их сторонниками, с другой.
  Подписей многих известных сторонников Троцкого -- Христиана Раковского, Карла Радека, Николая Крестинского, Адольфа Иоффе и других --
  под <<Заявлением>> нет, при этом как в составлении документа, так и в оппозиции в целом важную роль играли бывшие <<децисты>>,
  в частности Владимир Смирнов и Тимофей Сапронов, в 1926 году образовавшие самостоятельную группу.
  Левая оппозиция в 1923-1924 годы не была ни <<троцкистской>>, ни <<фракционной>> организацией.
  В новейшей литературе высказывается мнение, что <<оппозиция в широком смысле представляла собой внутрипартийную тенденцию,
  сторонников которой ситуационно объединяло критическое отношение к политике партии и поддержка более решительной <<демократизации>> внутрипартийного режима>>.

  Практически все члены внутрипартийной оппозиции были впоследствии расстреляны в период большого террора.

  \subsection{Уничтожение состязательного судебного процесса}
  Тоталитаризм был бы невозможен если не была бы уничтожена нормальная судебная система.
  Поэтому был утверждён упрощенный порядок расследования и рассмотрения дел о террористических организациях:
  дела слушались без участия сторон и приговор приводился в исполнение немедленно.
  Идеологическим обоснованием этого явилась теория судебных доказательств, разработанная Прокурором СССР А. Я. Вышинским.
  Вышинский считал, что формальные требования закона могли вступать в противоречие с требованиями жизни.
  Поэтому <<задача правосудия заключается в умении судьи>> дать оценку преступного деяния <<с точки зрения не только формальных требований закона,
  но и с точки зрения интересов дела социалистического строительства>>.

  Согласно утверждениям Вышинского принцип презумпции невиновности, характерный для буржуазного права, не мог быть использован в социалистическом государстве.
  По его теории обострение классовой борьбы могло вызвать сжатие, свертывание процессуальной формы и связанных с ней процессуальных гарантий.
  Вышинский возродил теорию формальных доказательств, объявив главным доказательством (<<царицей доказательств>>) признание обвиняемого.
  За монографию, посвященную теории судебных доказательств, Вышинский был удостоен Сталинской премии 1-й степени и объявлен классиком советского правоведения.

  Поподробнее об этом в пункте \textbf{4.2}.

  % \subsection{Военный коммунизм}
  % Экономическая политика большевиков периода Гражданской войны получила название <<военного коммунизма>>.
  % Данное понятие включало в себя не просто экономическую политику в условиях военного времени,
  % но и определенную доктринальную концепцию построения социализма в одной стране.

  % В развитии некоторых <<военно-коммунистических>> мер сыграла свою роль и Гражданская война.
  % Война поставила перед большевиками задачу создания огромной армии, максимальной мобилизации всех ресурсов,
  % а отсюда -- чрезмерной централизации власти и подчинения ее контролю всех сфер жизнедеятельности государства.
  % При этом задачи военного времени совпали с представлениями большевиков о социализме как бестоварном, без рыночном, централизованном обществе.


  \section{<<Большой террор>> 1937--1938 гг.: причины, механизмы реализации, результаты}
  Сталинские репрессии не ограничились 1937--1938 годами.
  Большой террор это просто период наиболее массовых репрессий.
  \subsection{Причины}
  % В научной среде нет согласия по поводу удовлетворительного объяснения причин террора.
  В научных кругах до сих пор нет единого мнения, что было причиной террора.
  Практически все историки считают, что Сталин сыграл ключевую роль в организации и проведении репрессий.

  По мнению историка О. В. Хлевнюка, террор 1937--1938 года связан с обозначившейся угрозой войны.
  Содержание приказов, регулирующих массовые операции, демонстрировало стремление сталинского руководства ликвидировать воображаемую <<пятую колонну>>.
  Именно подготовкой к войне объясняли массовые операции сами сотрудники НКВД в своей среде.
  При этом заговоры и <<пятая колонна>> существовали именно в представлениях Сталина и его соратников, но не в реальности.

  Советолог Шейла Фицпатрик проводит параллели между Большим террором и якобинским террором 1794 года Великой Французской революции, так как в обоих случаях в первую очередь уничтожались старые революционеры, под предлогом подавления <<внутренней контрреволюции>>, необходимости обновления общества принесением в жертву тысяч голов изменников.
  При этом масштабы террора были несопоставимы и, в отличие от якобинского террора, в ходе Большого террора уничтожались не оппозиционные политические структуры (они были давно ликвидированы), а не связанные между собой люди.

  Историк Ю. Н. Жуков выдвинул гипотезу о связи Большого террора с принятием Конституции СССР 1936 года и выборами в Верховный Совет СССР в декабре 1937 года.
  По его утверждениям, Сталин намеревался провести первые выборы в Верховный Совет СССР как альтернативные, состязательные, хотел <<вообще отстранить партию от власти>>, но ему в этом помешали руководители региональных партийных органов, которые якобы боялись лишиться своих постов в ходе этих выборов и потому выступили инициаторами Большого террора.
  И. В. Павлова критиковала эту концепцию, доказывая, что \textbf{инициатором террора был именно Сталин}, а выборы изначально были лишь имитацией демократии.
  Специалист по истории СССР 1920--1950-х годов О. В. Хлевнюк критично отозвался о теории Жукова, ссылаясь на сверхвольное обращение с источниками.
  По утверждению О. В. Хлевнюка, \textbf{исключительная роль Сталина в организации этого всплеска террора не вызывает сомнений и подтверждается документально.
  Например имеются записи, в которых он требовал применять методы физического насилия против арестованных людей.}

  Здесь надо заметить, что называть большой террор ежовщиной не совсем правильно,
  поскольку массовые репрессии не были инициативой и виной Ежова и его прислужников.
  Он просто выполнял приказы Сталина, и впоследствии сам был расстрелян.

  Германский историк Йорг Баберовски считал причиной перехода к Большому террору сопротивление Сталину со стороны региональных и местных элит, саботировавших указания центра.
  Французский историк Николя Верт рассматривал Большой террор как осуществляемый властью механизм социальной инженерии, окончательное завершение политики <<раскулачивания>> и депортаций <<вредных элементов>>.
  Американский историк Дэвид Ширер связывал возникновение Большого террора с преодолением общего хозяйственно-политического хаоса первой половины и середины 1930-х годов.

  Некоторые специалисты усматривают в массовых операциях орудие социальной инженерии, насильственной унификации общества.
  Такие мотивы, несомненно, характерны для любого диктатора.
  Однако, полагаясь только на них, мы не сможем понять, почему Сталин решил в ударном порядке <<унифицировать>> общество именно в 1937 г.
  Много пишут о стремлении кремлевской власти найти козлов отпущения среди чиновников, объявить их ответственными за огромные тяготы жизни народа, выпустить пар социального напряжения.
  И этот мотив, несомненно, присутствовал во всех сталинских репрессиях, в том числе в чистках номенклатуры в 1937--1938 гг.
  Однако он не объясняет массовые операции, которые обрушились в основном на рядовых граждан страны.
  Обращается внимание на необходимость обеспечения рабочей силой растущей экономики ГУЛАГа.
  Однако применительно к 1937--1938 гг. и это объяснение не работает.
  Среди 700 000 расстрелянных и замученных в ходе следствия подавляющее большинство составляли люди в трудоспособном возрасте.
  Их просто уничтожили, а не послали на хозяйственные объекты ГУЛАГа.

  \subsection{Механизмы реализации}
  Первоначально считалось что одним из основных механизмов реализации были доносы, однако это мнение оказалось далёким от истины.

  Первые серьезные сомнения по поводу доносов у историков появились в начале 1990-х гг.,
  когда ненадолго открылся доступ к материалам следственных дел 1937--1938 гг.
  Выяснилось, что основой обвинительных материалов в следственных делах были признания, полученные во время следствия.
  При этом заявления и доносы как доказательство вины арестованного в следственных делах встречаются крайне редко.
  Глубокое исследование механизмов <<большого террора>> вполне разъяснило причины такого положения.
  Организация массовых операций 1937--1938 гг. не предусматривала широкого использования доносов как основы для арестов.
  Изъятия антисоветских элементов проводились первоначально на основе картотек НКВД, а затем на основе показаний, выбитых на следствии.
  Запустив конвейер допросов с применением пыток, чекисты в избытке были обеспечены <<врагами>> и не нуждались в подсказках доносчиков.

  Большую роль в формированииправовой основы политических репрессий сыграли изменения, внесенные в уголовное законодательство в 30-е гг.
  Были уточнены составы ряда преступлений и ужесточена система уголовных наказаний.

  Серьезные изменения произошли и в уголовно-процессульном законодательстве.
  После убийства С. М. Кирова 1 декабря 1934 г. было принято постановление ЦИК СССР <<О внесении изменений в действующие уголовно-процессуальные кодексы союзных республик>>,
  по которому вводился упрощенный порядок расследования и рассмотрения дел о террористических организациях и террористических актах против работников советской власти.
  Следствие по этим делам заканчивалось в срок не более 10 дней, обвинительное заключение вручалось обвиняемым за сутки до рассмотрения дела в суде,
  дела слушались без участия сторон, кассационные обжалования приговоров и подачи ходатайств о помиловании не допускались,
  приговор к высшей мере наказания приводился в исполнение немедленно.
  Этот антидемократический по своей сути закон послужил основанием для начала массовых репрессий.

  Планировалось пропустить через судебный конвейер огромное количество людей, но никакая судебная система не справилась бы с таким количеством дел.
  Поэтому были созданы новые органы -- тройки.
  Состояли они из трех человек: начальник НКВД, прокурор, секретарь обкома/первый секретарь.
  Механизм работы был обкатан на <<милицейских тройках>>, созданных за несколько дней до террора (их функцией было упрощенное рассмотрение дел о нарушениях паспортного режима).

  Тройки не проводили судов.
  Решения выносились тройкой заочно -- по материалам дел, представляемым органами НКВД,
  а в некоторых случаях и при отсутствии каких-либо материалов -- по представляемым спискам арестованных.
  Процедура рассмотрения дел была свободной, протоколов не велось.
  Характерным признаком дел, рассматриваемых <<тройками>>, было минимальное количество документов,
  на основании которых выносилось решение о применении репрессии.


  \subsection{Результаты}
  Террор привел к дезорганизации партийной системы управления страной.
  НКВД господствовало над партийно-государственной структурой.

  Под давлением уже новой партийной элиты, опасавшейся следующей волны репрессий, были официально запрещены допускавшиеся ранее физические пытки.
  Сталин прекратил массированное уничтожение правящего слоя.
  Возникшее в период террора господство органов НКВД над партийными структурами постепенно было ликвидировано.
  У обеих структур остался только один хозяин -- Вождь.

  Одним из результатов выло ослабление генералитета армии, и, как следствие,
  поражение в Советско-финляндской войне, а также в первых битах Великой Отечественной.

  % \begin{thebibliography}{9}
  %   \bibitem{причины1} \url{http://bagazhznaniy.ru/history/rossijskaya-revolyuciya-1917g-prichiny-zadachi-posledstviya}
  % \end{thebibliography}

\end{document}
