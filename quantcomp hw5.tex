\documentclass[aspectratio=169]{beamer}
\beamertemplatenavigationsymbolsempty
\usetheme{Boadilla}
\usepackage{textpos} % package for the positioning
\usepackage{enumitem}
\usepackage{amsmath,amsfonts,amssymb}

\usepackage{listings}
\usepackage{xcolor}
\usepackage{tasks}
\usepackage{braket}

% Define color values for the theme
\definecolor{backcolour}{rgb}{0.97, 0.97, 1.0}    % Light purple background
\definecolor{codegreen}{rgb}{0.0, 0.6, 0.0}       % Green for comments
\definecolor{codegray}{rgb}{0.5, 0.5, 0.5}        % Gray for line numbers
\definecolor{codepurple}{rgb}{0.58, 0.0, 0.83}    % Purple for strings
\definecolor{keywordcolor}{rgb}{0.5, 0.0, 0.5}    % Dark purple for keywords

% Define the style for the listings
\lstdefinestyle{mystyle}{
    backgroundcolor=\color{backcolour},
    commentstyle=\color{codegreen},
    keywordstyle=\color{keywordcolor},
    numberstyle=\tiny\color{codegray},
    stringstyle=\color{codepurple},
    basicstyle=\ttfamily\footnotesize,
    breakatwhitespace=false,
    breaklines=true,
    captionpos=b,
    keepspaces=true,
    numbers=left,
    numbersep=5pt,
    showspaces=false,
    showstringspaces=false,
    showtabs=false,
    tabsize=2
}

\lstset{style=mystyle}

\definecolor{uwopurple}{RGB}{79,38,131} % official purple color for uwo


\title{Foundations of Quantum Computing}
\author[]{Zyad Altahan (50238152) \and \AA{AAAAAAAAAA AAAAAAA}{6} (\AA{AAAAAAAA}{10}) \and Tobias Mette (3105540) \and Aksa Aksa (50146305) \and Mohamed Sonbol (50262724)}
\institute[]{Department of Computer Science \\ University of Bonn}
\date{\today}

\usepackage[listings,many]{tcolorbox}
\makeatletter
\newtcblisting{mylisting}{
  listing only,
  breakable, enhanced jigsaw,
  listing engine=listings,
  colback=gray!20,
  listing options={
    language=python,
    keywordstyle=\color{blue},
    basicstyle=\ttfamily,
    stringstyle=\color{ForestGreen},
    commentstyle=\color{gray},
    ndkeywordstyle={\color{orange}},
    identifierstyle=\color{black},
    numbers=none,
    showstringspaces=false,
    aboveskip={0\p@ \@plus 6\p@}, belowskip={0\p@ \@plus 6\p@},
    columns=fullflexible, keepspaces=true,
    breaklines=true, breakatwhitespace=true,
    extendedchars=false,
    inputencoding=utf8,
    upquote=true,
    xleftmargin=-25pt,
  }
}
\makeatother

% set color
\setbeamercolor{title in head/foot}{bg=white}
\setbeamercolor{author in head/foot}{bg=white}
\setbeamercolor{date in head/foot}{fg=uwopurple}
\setbeamercolor{date in head/foot}{bg=white}
\setbeamercolor{title}{fg=uwopurple}
\setbeamerfont{title}{series=\bfseries}
\setbeamercolor{frametitle}{fg=uwopurple}
\setbeamerfont{frametitle}{series=\bfseries}
\setbeamercolor{block title}{bg=uwopurple!30,fg=black}
\setbeamercolor{item}{fg=uwopurple}
\setbeamercolor{caption name}{fg=uwopurple!70!}
\usefonttheme[onlymath]{serif}


\begin{document}


\begin{frame}
    \titlepage
    \begin{textblock*}{8cm}(5.0cm,-7.0cm)
        {\large \color{uwopurple}\hspace{0.66cm} \textbf{Exercise Sheet 05}} % Change the lecture # right here!
    \end{textblock*}
\end{frame}


\begin{frame}{Task 5.5: Tensor products of complex unit vectors}
  We know that $\sum_{j=0}^{1} |\alpha_j|^2 = 1$ and $\sum_{k=0}^{1} |\beta_k|^2 = 1$ $\Leftrightarrow$   \textbf{Born Rule}\\
  To prove $\sum_{j=0}^{1} \sum_{k=0}^{1} |\gamma_{jk}|^2 = 1$, we substitute the value of $\gamma_{jk} = \alpha_j \beta_k$:
  \[
    \sum_{j=0}^{1} \sum_{k=0}^{1} |\alpha_j \beta_k|^2  = \sum_{j=0}^{1} |\alpha_j|^2 \sum_{k=0}^{1} |\beta_k|^2 = 1 * 1 = 1
  \]
\end{frame}

\begin{frame}{Task 5.6: Effects of the Born rule}
  a) If $\alpha_0 = 1/\sqrt{2}, |\alpha_0| = \sqrt{(1/\sqrt{2})^2 + 0^2} = 1/\sqrt{2}$, then $|\alpha_{1}|^2 = 1 - |\alpha_{0}|^2$ $\Rightarrow$ \textbf{Born rule} \\
  $|\alpha_{1}|^2 = 1 - |1/\sqrt{2}|^2  = 1 - 1/2 = 1/2$ $\Rightarrow$  $\alpha_{1} = \pm 1/\sqrt{2}$\\

  b) if $\alpha_0 = 1/2, |\alpha_0| = 1/2$, then $\alpha_{1} = \pm \sqrt{3}/2$\\
  c) if $\alpha_1 = 1/4, |\alpha_0| = 1/4$, then $\alpha_{0} = \pm \sqrt{15}/4$\\
  d) if $\alpha_1 = i, |\alpha_1| = \sqrt{0^2 + 1^2} = \sqrt{1} = 1$, then $\alpha_{0} = \sqrt{1 - 1} = 0$\\
  e) if $\alpha_1 = i/\sqrt{2}, |\alpha_1| = \sqrt{0^2 + (1/\sqrt{2})^2} = \sqrt{1/2}, |\alpha_1|^2 = 1/2$, then $\alpha_{0} = \sqrt{1 - 1/2} = \sqrt{1/2} = \pm 1/\sqrt{2}$\\

\end{frame}

\begin{frame}{Task 5.7}

{\tiny

\textbf{(a) Prove that \(X^2 = Y^2 = Z^2 = I\):}

\[
X = |1\rangle\langle0| + |0\rangle\langle1|, \quad
X^2 = (|1\rangle\langle0| + |0\rangle\langle1|)(|1\rangle\langle0| + |0\rangle\langle1|),
\]
\[
X^2 = |1\rangle\langle0|1\rangle\langle0| + |1\rangle\langle0|0\rangle\langle1| + |0\rangle\langle1|1\rangle\langle0| + |0\rangle\langle1|0\rangle\langle1|,
\]
\[
X^2 = |1\rangle\langle1| + |0\rangle\langle0| = I.
\]

\[
Y = i|1\rangle\langle0| - i|0\rangle\langle1|, \quad
Y^2 = (i|1\rangle\langle0| - i|0\rangle\langle1|)(i|1\rangle\langle0| - i|0\rangle\langle1|),
\]
\[
Y^2 = i^2|1\rangle\langle0|1\rangle\langle0| - i^2|1\rangle\langle0|0\rangle\langle1| - i^2|0\rangle\langle1|1\rangle\langle0| + i^2|0\rangle\langle1|0\rangle\langle1|,
\]
\[
Y^2 = +1(|1\rangle\langle1| + 1|0\rangle\langle0|) = I.
\]

\[
Z = |0\rangle\langle0| - |1\rangle\langle1|, \quad
Z^2 = (|0\rangle\langle0| - |1\rangle\langle1|)(|0\rangle\langle0| - |1\rangle\langle1|),
\]
\[
Z^2 = |0\rangle\langle0|0\rangle\langle0| - |0\rangle\langle0|1\rangle\langle1| - |1\rangle\langle1|0\rangle\langle0| + |1\rangle\langle1|1\rangle\langle1|,
\]
\[
Z^2 = |0\rangle\langle0| + |1\rangle\langle1| = I.
\]

\textbf{Result:} \(X^2 = Y^2 = Z^2 = I.\)

\vspace{0.3cm}

\textbf{(b) Prove that \(-iXYZ = I\):}
To show that \(-iXYZ = I\), first compute \(XY\) using the definitions:
\[
XY = (|1\rangle\langle0| + |0\rangle\langle1|)(i|1\rangle\langle0| - i|0\rangle\langle1|).
\]
\[
XY = -i|1\rangle\langle0|1\rangle\langle0| + i|1\rangle\langle0|0\rangle\langle1| + i|0\rangle\langle1|1\rangle\langle0| - i|0\rangle\langle1|0\rangle\langle1|.
\]

\[
XY = -i|1\rangle\langle1| + i|0\rangle\langle0|.
\]
\[
XY = i(1.|0\rangle\langle0| - 1.|1\rangle\langle1|) = iZ
\]

}

\end{frame}

\begin{frame}{Task 5.7 (Continue)}

{\tiny

\[
-iXYZ = I \]
\[-i(iZ)Z = I \]
\[
-i^2Z^2 = I
\]
\( i^2 = -1 \) and \(Z^2 = I\). So,
\[
-1(-1).I = I
\]



\vspace{0.3cm}

\textbf{(c) Prove that \(XY = -YX = iZ, \quad YZ = -ZY = iX, \quad ZX = -XZ = iY\):}

\textbf{Step 1: Prove \(XY = -YX = iZ\):}
from:
\[
-iXYZ = I
\]
\[
-i \cdot XY Z \cdot Z = Z  \quad \text{(multiplying \(Z\) on both sides)}.
\]
Multiply by \(i\):
\[
-i^2 \cdot XY = i Z \implies XY = iZ.
\]
Moreover:
\[
XY = -YX \quad \text{(from anticommutation relation)}.
\]
\[
XY = -YX = iZ.
\]

\vspace{0.3cm}

\textbf{Step 2: Prove \(YZ = -ZY = iX\):}
from:
\[
-iXYZ = I
\]
\[
-iX.XYZ = XI \quad \text{(multiplying \(X\) on both sides)}.
\]
Multiply by \(i\):
\[
-i^2YZ = iX \quad \implies YZ = iX.
\]
Using the anticommutation relation:
\[
YZ = -ZY = iX.
\]

}

\end{frame}

\begin{frame}{Task 5.7 (Continue)}

{\tiny

\textbf{Step 3: Prove \(ZX = -XZ = iY\):}

\[
ZX = -XZ = iY.
\]
from:
\[
-iXYZ = I.
\]

\[
-iZXY = I. \quad \text {Using cyclic property}
\]
\[
-iZXY \cdot Y = IY \quad \text{(multiplying \(Y\) on both sides)}.
\]
\[
ZX = iY \quad \text{(multiplying by i on both sides using \(i^2 =-1\))}
\]
%Using the anticommutation relation:
%\[
%ZX = -XZ = iY.
%\]
\textbf{Final Result:}
\[
XY = -YX = iZ, \quad YZ = -ZY = iX, \quad ZX = -XZ = iY.
\]

\textbf{(d) Prove that \(HXH = Z\)}

\[
HXH = \frac{1}{\sqrt{2}}[X + Z] X \frac{1}{\sqrt{2}}[X + Z]
\]
\[
\frac{1}{2}[X^2 + ZX][X+Z]
\]
\[
\frac{1}{2}[I + ZX][X+Z] \quad \text{Using \(X^2 = I\)}
\]
\[
\frac{1}{2}[I \cdot X + I \cdot Z + ZX \cdot X + ZX \cdot Z]
\]

\[
\frac{1}{2}[X + Z + Z\cdot I - ZZ \cdot X] = \frac{1}{2}[X + 2Z - Z^2 \cdot X] \quad \text{Using \(X^2 = I\) and \(XZ = -ZX\)}
\]
\[
\frac{1}{2}[X + 2Z - X] = \frac{1}{2}(2Z) = Z
\]
}

\end{frame}



\begin{frame}{Task 5.8}
  a) We can prove that the two expressions are equal using basic algebra with the tensor product:
\begin{align*}
  &I \otimes I + \ket{1}\bra{1} \otimes (X - I) =
  I \otimes I + \ket{1}\bra{1} \otimes X - \ket{1}\bra{1} \otimes I = \\
  &(I - \ket{1}\bra{1}) \otimes I + \ket{1}\bra{1} \otimes X =
  \ket{0}\bra{0} \otimes I + \ket{1}\bra{1} \otimes X
\end{align*}
  And computing the two Kronecker products gives us the correct $CX$ matrix.
  \vspace{0.5cm}

  b) As expected, \(CX \ket{\psi}\) produces the correct output for all \(\ket{00},\, \ket{01},\, \ket{10},\, \ket{11}\) inputs. Actually, the matrix multiplication always outputs the $n$-th column of the $CX$ matrix. It's just a bit hard to convert in back to a ket.
\end{frame}

\begin{frame}{Task 5.9}
  The goal of this task was to compute the matrix $CNOT_{2 \to 1}$ that results if we switch the target and control qubits of the $CNOT_{1 \to 2}$ gate, this results in the following truth table.

\[
\begin{array}{|c|c|}
\hline
\text{Input } |x_1 x_2\rangle & \text{Output } |y_1 y_2\rangle \\
\hline
|00\rangle & |00\rangle \\
|01\rangle & |11\rangle \\
|10\rangle & |10\rangle \\
|11\rangle & |01\rangle \\
\hline
\end{array}
\]

We can use the results of this truth table to compute the $CNOT_{2 \to 1}$ gate matrix, which results in the following matrix.
\[
\text{CNOT}_{2 \to 1} =
\begin{bmatrix}
1 & 0 & 0 & 0 \\
0 & 0 & 0 & 1 \\
0 & 0 & 1 & 0 \\
0 & 1 & 0 & 0
\end{bmatrix}.
\]
Which is just the output vectors arranged as columns or rows.
\end{frame}

\begin{frame}{Task 5.10}
\begin{align*}
SWAP &= CNOT_{1\rightarrow2}\cdot CNOT_{2\rightarrow1}\cdot CNOT_{1\rightarrow2} \\
&= \begin{pmatrix}
1 & 0 & 0 & 0\\
0 & 1 & 0 & 0\\
0 & 0 & 0 & 1\\
0 & 0 & 1 & 0\\
\end{pmatrix}\cdot \begin{pmatrix}
1 & 0 & 0 & 0\\
0 & 0 & 0 & 1\\
0 & 0 & 1 & 0\\
0 & 1 & 0 & 0\\
\end{pmatrix}\cdot \begin{pmatrix}
1 & 0 & 0 & 0\\
0 & 1 & 0 & 0\\
0 & 0 & 0 & 1\\
0 & 0 & 1 & 0\\
\end{pmatrix} = \begin{pmatrix}
1 & 0 & 0 & 0\\
0 & 0 & 1 & 0\\
0 & 1 & 0 & 0\\
0 & 0 & 0 & 1\\
\end{pmatrix}\\
\end{align*}

\begin{columns}
\begin{column}{0.5\textwidth}
\begin{align*}
SWAP \cdot \ket{00} = \ket{00}\\
SWAP \cdot \ket{01} = \ket{10}\\
SWAP \cdot \ket{10} = \ket{01}\\
SWAP \cdot \ket{11} = \ket{11}\\
\end{align*}
\end{column}
\begin{column}{0.5\textwidth}
The \textit{SWAP}-operator swaps the input, it projects $x_1 \rightarrow y_2$ and $x_2 \rightarrow y_1$. We could also say it flips $x_1$ and $x_2$ if they are different.

\end{column}
\end{columns}

\end{frame}

\end{document}
