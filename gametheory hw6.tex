\documentclass{article}

\usepackage{enumitem}
\usepackage{amsmath}
\usepackage{amsfonts}
\usepackage[dvipsnames]{xcolor}
\usepackage{amssymb}
\usepackage[margin=0.5in]{geometry}
\usepackage[thinc]{esdiff}
\usepackage[hidelinks, bookmarks=false]{hyperref}

\mathchardef\simsym"0218
\let\oldphi\phi
\let\phi\varphi
\newcommand{\E}{\mathbf{\mathbb{N}}}

\usepackage{environ}
\NewEnviron{centerframebox}{\begin{center}\fbox{\parbox{0.92\textwidth}{\BODY}}\end{center}}

\title{Algorithmic Game Theory\\ Exercise Set 6}
\author{
  \AA{AAAAAAAAAA AAAAAAA}{6} \\
  \href{mailto:\AA{AAAAAAAAAAAAAAAAAAAA}{7}}{\AA{AAAAAAAAAAAAAAAAAAAA}{7}}
}

\begin{document}
  \maketitle

  \setcounter{section}{6}
  \subsection{All-pay auction}
  \begin{centerframebox}
    An \textit{all-pay auction} is a single-item auction defined in almost the same manner as a first-price
    auction: Each bidder reports a bid $b_i \geq 0$. The bidder with the highest bid wins the item.
    However, every bidder must pay their own bid regardless of whether they win the item or not.

    Be inspired by the steps of Section 2 in the notes of Lecture 15 to derive the symmetric
    Bayes-Nash equilibrium of an all-pay auction with $n$ bidders and identical distributions.
  \end{centerframebox}

  The lecture notes for finding the Bayes-Nash equilibrium in a First-price auction (given all the same assumption we have here) started with this massive formula:
  \[
    \mathbb{E}_{v_{-i} \sim \mathcal{D}_{-i}} \left[ u_i\left( \left( y, \beta_{-i} (v) \right), v_i \right) \right] = (v_i - y) \Pr \left[ \bigwedge_{j \neq i} \beta_j (v_j) < y \right]
  \]
  where we are considering the expected utility player $i$ gets by deviating from their betting function $\beta_i(v_i)$ and instead using $y$ as their bet.

  In our case, the players will always have to pay their bets, so the big $\Pr$ term can be separated out and the expected utility can we rewritten like this:
  \[
    \mathbb{E}_{v_{-i} \sim \mathcal{D}_{-i}} \left[ u_i\left( \left(y, \beta_{-i} (v) \right), v_i \right) \right] =
    v_i \Pr \left[ \bigwedge_{j \neq i} \beta_j (v_j) < y \right] - y
  \]
  Following further transformations from the lecture we can rewrite the $\Pr$ term like this:
  \[
    \Pr \left[ \bigwedge_{j \neq i} \beta_j (v_j) < y \right] =
    \prod_{j\neq i} \Pr \left[ \beta_j (v_j) < y \right] =
    \prod_{j\neq i} \Pr \left[ v_j < \phi(y) \right] =
    \prod_{j\neq i} F(\phi(y)) =
    F(\phi(y))^{n-1} =
    G(\phi(y))
  \]
  where $\phi$ is the inverse of $\beta_i$, $F$ is the cumulative distribution function and $G$ is just $F(y)^{n-1}$.
  The final expected utility function looks like this:
  \[
    \mathbb{E}_{v_{-i} \sim \mathcal{D}_{-i}} \left[ u_i\left( \left(y, \beta_{-i} (v) \right), v_i \right) \right] =
    v_i G(\phi(y)) - y
  \]

  Now if we want $y$ to be part of our Bayes-Nash equilibrium (i.e. $\beta_i(v_i) = y$), it has to be a maximum of the function above, so we set the derivative to 0.
  \begin{align*}
    \diff{}{y} (v_i G(\phi(y)) - y) &= 0 \\
    v_i \diff{}{y} G(\phi(y)) - 1 &= 0 \\
    v_i G'(\phi(y)) \phi'(y) &= 1
  \end{align*}
  By the inverse function theorem, we have $\phi'(y) = \frac{1}{\beta_i'(\phi(y))}$, and we can also substitute $\beta_i(v_i) = y$:
  \begin{align*}
    v_i G'(\phi(y)) \phi'(y) &= 1 \\
    v_i G'(\phi(y)) \frac{1}{\beta_i'(\phi(y))} &= 1 \\
    v_i G'(v_i) \frac{1}{\beta_i'(v_i)} &= 1 \\
    v_i G'(v_i) &= \beta_i'(v_i)
  \end{align*}
  The right-hand side is just the derivative of $\beta_i$, so we can use an integral and the fact that $\beta_i(0) = 0$ to define it.
  \[
    \beta_i(v_i) = \int v_i G'(v_i) \mathrm{d}v_i = \int_0^{v_i} t G'(t) \mathrm{d}t
  \]

  \noindent\rule{\textwidth}{0.4pt}

  It looks like for a basic uniform $0\simsym 1$ distribution, it will be
  \[ \beta_i(v_i) = \frac{n-1}{n} \cdot v_i^n \]
  for $n=2$
  \[ \beta_i(v_i) = \frac{v_i^2}{2} \]
  because $v_i$ in this case is always $\leq 1$, the square will always make our bet smaller.

\end{document}
