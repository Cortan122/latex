\documentclass{article}

\usepackage{enumitem}
\usepackage{amsmath}
\usepackage{amsfonts}
\usepackage{amssymb}
\usepackage{textcomp}
\usepackage{gensymb}
\usepackage[dvipsnames]{xcolor}
\usepackage[margin=0.5in]{geometry}
\usepackage[hidelinks,bookmarks=false]{hyperref}

\usepackage{environ}
\NewEnviron{centerframebox}{\begin{center}\fbox{\parbox{0.92\textwidth}{\BODY}}\end{center}}

\let\oldphi\phi
\let\phi\varphi
\newcommand{\Z}{\mathbb{Z}}

\hypersetup{
  colorlinks=true,
  urlcolor=MidnightBlue,
  linkcolor=WildStrawberry,
  % pdfborderstyle={/S/U/W 2}
}

\usepackage[listings,many]{tcolorbox}
\makeatletter
\newtcblisting{mylisting}{
  listing only,
  breakable, enhanced jigsaw,
  listing engine=listings,
  colback=gray!20,
  listing options={
    language=python,
    keywordstyle=\color{blue},
    basicstyle=\ttfamily,
    stringstyle=\color{ForestGreen},
    commentstyle=\color{gray},
    ndkeywordstyle={\color{orange}},
    identifierstyle=\color{black},
    numbers=none,
    showstringspaces=false,
    aboveskip={0\p@ \@plus 6\p@}, belowskip={0\p@ \@plus 6\p@},
    columns=fullflexible, keepspaces=true,
    breaklines=true, breakatwhitespace=true,
    extendedchars=false,
    inputencoding=utf8,
    upquote=true,
    xleftmargin=-25pt,
    mathescape=true,
  }
}
\makeatother

\title{Cryptography \\ Exercise Sheet 10}
\author{
  AAAAAAAAAA AAAAAAA \\
  \href{mailto:AAAAAAAAAAAAAAAAAAAA}{AAAAAAAAAAAAAAAAAAAA}
}

\begin{document}
  \maketitle

  \setcounter{section}{10}
  \subsection{ElGamal encryption}
  \begin{centerframebox}
    Consider the ElGamal encryption scheme (slide 278+54). Basically, it looks
    like a variant of the Diffie-Hellman key exchange. The sender uses $A^t$
    for encrypting the message $m$ and the recipient uses $T^a$ for decrypting.

    \begin{enumerate}[label=(\roman*)]
      \item Perform a toy example of ElGamal encryption with $(G,\, g,\, q)$ as follows:
      Take $G = \Z^\times_p$  with the last 32-bit prime $p = \operatorname{prevprime}(2^{32}) = 2^{32} - 5 = 4 294 967 291$.
      Then $g = 38$ has prime order $q = 22 605 091 \in [2^{25-1}, 2^{25}-1]$.
    \end{enumerate}
  \end{centerframebox}
  \[ g = 38 \qquad q = 22 605 091 \qquad p = 4 294 967 291 \]
  Before we can encrypt or decrypt anything, we also have to pick a random $a \in \Z_q$ and compute $A = g^a$.
  \[ a = 13 130 822 \qquad A = 1 536 924 081 \]
  Now to encrypt some message $m$, we also need to pick a random number $t \in \Z_q$
  and then compute the ciphertext $(g^t,\, m \cdot A^t)$.
  \[ m = 420 122 069 \qquad t = 21 446 789 \qquad  g^t = 2 527 812 898 \qquad m \cdot A^t = 3 605 983 895
  \qquad c = (2 527 812 898,\, 3 605 983 895 ) \]
  Then we can try to decrypt this message as $m' = c_0^{-1} \cdot c_1$.
  \[ c_0^{-1} = 3 921 537 347 \qquad m' = 420 122 069 \]
  Everything lines up!!

  \begin{mylisting}
    p = 2*32 - 5
    g = 38
    q = 22_605_091
    a = random.randint(0, q-1)
    A = pow(g, a, p)

    m = 420_122_069
    t = random.randint(0, q-1)
    c0 = pow(g,t,p)
    c1 = m * pow(A,t,p) % p

    m_prime = pow(c0, -a, p)*c1 % p
  \end{mylisting}

\end{document}
