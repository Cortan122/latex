\documentclass{article}
\usepackage[hidelinks,bookmarks=false]{hyperref}
\usepackage{122}


\title{Теория вероятности \\ Домашнее задание №4}
\author{\AA{AAAAA AAAAAAA}{4} \\ \AA{AAAAAA}{11} \\ Вариант 3}

\begin{document}
  \maketitle

  \setcounter{section}{6}
  \section{В результате 15-ти независимых измерений давления в топливном баке найдена оценка дисперсии давления, равная $0.2$ Па. Построить доверительный интервал для дисперсии, если математическое ожидание значения давления неизвестно, а доверительная вероятность $\gamma = 0.8$}
  $$ n = 15 $$
  $$ S^2 = 0.2 $$
  $$ \beta = 1-\gamma = 0.2 $$
  $$ \chi^2\l(m,\, \alpha\r) \text{ это }\alpha\text{-ый квантиль } m\text{-ого }\chi^2\text{-а} $$
  у нас выборочная дисперсия распределенная по $\chi^2$-у, и поэтому наш доверительный интервал будет равен
  $$ \l(\f{nS^2}{\ds\chi^2\l(n-1,\, 1-\f{\beta}{2}\r)},\quad \f{nS^2}{\ds\chi^2\l(n-1,\, \f{\beta}{2}\r)}\r) $$
  из таблицы (на \href{https://ru.wikipedia.org/wiki/%D0%9A%D0%B2%D0%B0%D0%BD%D1%82%D0%B8%D0%BB%D0%B8_%D1%80%D0%B0%D1%81%D0%BF%D1%80%D0%B5%D0%B4%D0%B5%D0%BB%D0%B5%D0%BD%D0%B8%D1%8F_%D1%85%D0%B8-%D0%BA%D0%B2%D0%B0%D0%B4%D1%80%D0%B0%D1%82#%D0%A2%D0%B0%D0%B1%D0%BB%D0%B8%D1%86%D0%B0_%D0%BA%D0%B2%D0%B0%D0%BD%D1%82%D0%B8%D0%BB%D0%B5%D0%B9}{\underline{википедии}} очень красивая) мы получаем что
  $$ \chi^2\l(14,\, 0.9\r) = 21.0641 $$
  $$ \chi^2\l(14,\, 0.1\r) = 7.7895 $$
  и получаем ответ
  $$ \l(\f{3}{21.0641},\quad \f{3}{7.7895}\r) $$
  $$ \l(0.1424,\, 0.3851\r) $$

  \section{Будем считать, что наблюдаемая в задаче №6 СВ имеет гауссовское распределение.}
  $$ \bar{X} = 24.8357 $$
  $$ S^2 = 13.3746 $$
  $$ n = 196 $$
  \subsection{Постройте двусторонние доверительные интервалы уровня надёжности $0.99$ для математического ожидания и дисперсии наблюдаемой случайной величины.}
  \subsubsection{Для математического ожидания}
  у нас всегда -- по теореме Фишера -- такая вот вещь распределена по Стьюденту
  $$ \f{\mu-\bar{X}}{\sqrt{\f{S^2}{n-1}}} \sim \mathrm{t}(n-1) $$
  и интервал у нас будет
  $$ \l(\bar{X} - \mathrm{t}(n-1,\, 0.995)\cdot\sqrt{\f{S^2}{n-1}},\quad \bar{X} + \mathrm{t}(n-1,\, 0.995)\cdot\sqrt{\f{S^2}{n-1}}\r) $$
  и тут квантиль (посчитанная на питоне) $\mathrm{t}(195,\, 0.995) \approx 2.6013$ и финальный ответ
  $$ \l(24.1544,\, 25.5170\r) $$

  \subsubsection{Для дисперсии}
  тут у нас тоже самое что и в №7
  $$ \l(\f{nS^2}{\ds\chi^2\l(n-1,\, 0.995\r)},\quad \f{nS^2}{\ds\chi^2\l(n-1,\, 0.005\r)}\r) $$
  считаем квантили на питоне
  $$ \chi^2\l(195,\, 0.995\r) = 249.6160 $$
  $$ \chi^2\l(195,\, 0.005\r) = 147.8890 $$
  и получаем ответ
  $$ \l(10.5018,\, 17.7256\r) $$

  \subsection{Проверьте на уровне значимости $0.05$ гипотезу о том, что математическое ожидание наблюдаемой СВ равно $25$, а дисперсия равна $12$.}
  \subsubsection{Для математического ожидания}
  \begin{enumerate}
    \item \textbf{Формулируем основную и альтернативную гипотезы} \\
      $H_0: m = 25$ \\
      $H_1: m \neq 25$
    \item \textbf{Выбираем уровень значимости} \\
      $\alpha = 0.05$
    \item \textbf{Выбираем статистику} \\
      $\ds T = \f{\l(\bar{X} - 25\r)\sqrt{n}}{\tilde{S}}
      = \f{\l(\bar{X} - 25\r)\sqrt{n}}{\sqrt{\f{S^2}{n-1} \cdot n}}
      = \f{\l(\bar{X} - 25\r)\sqrt{n-1}}{\sqrt{S^2}}$
    \item \textbf{Определяем, какое будет распределение у статистики, если верна основная гипотеза} \\
      $\ds T|H_0 \sim \mathrm{t}(n-1)$
    \item \textbf{Строим доверительную и критическую области (определяем их границы)} \\
      тут нам надо просто найти $\f{\alpha}{2}$ и $1-\f{\alpha}{2}$-ые квантили полученного распределения
      $$ \l(\mathrm{t}\l(n-1,\, \f{\alpha}{2}\r),\quad \mathrm{t}\l(n-1,\, 1-\f{\alpha}{2}\r)\r) $$
      $$ \l(-1.9722,\, 1.9722\r) $$
    \item \textbf{Вычисляем реализацию статистики от нашей выборки} \\
      $\ds \text{т} = \f{\l(24.8357 - 25\r)\sqrt{195}}{\sqrt{13.3746}} \approx -0.6274$
    \item \textbf{Смотрим, куда попало значение статистики и принимаем решение, отвергнуть нулевую гипотезу или нет} \\
      статистика у нас в доверительный интервал попала, и, поэтому, мы не можем отвергнуть гипотезу $H_0$
  \end{enumerate}

  \subsubsection{Для дисперсии}
  \begin{enumerate}
    \item \textbf{Формулируем основную и альтернативную гипотезы} \\
      $H_0: \sigma^2 = 12$ \\
      $H_1: \sigma^2 \neq 12$
    \item \textbf{Выбираем уровень значимости} \\
      $\alpha = 0.05$
    \item \textbf{Выбираем статистику} \\
      $\ds T = \sum_{i=1}^n \f{(X_i - \bar{X})^2}{12}$
    \item \textbf{Определяем, какое будет распределение у статистики, если верна основная гипотеза} \\
      $\ds T|H_0 \sim \chi^2(n-1)$
    \item \textbf{Строим доверительную и критическую области (определяем их границы)} \\
      тут нам надо просто найти $\f{\alpha}{2}$ и $1-\f{\alpha}{2}$-ые квантили полученного распределения
      $$ \l(\chi^2\l(n-1,\, \f{\alpha}{2}\r),\quad \chi^2\l(n-1,\, 1-\f{\alpha}{2}\r)\r) $$
      $$ \l(158.2215,\, 235.5643\r) $$
    \item \textbf{Вычисляем реализацию статистики от нашей выборки} \\
      $\ds \text{т} = \f{n \cdot S^2}{12} = \f{13.3746 \cdot 196}{12} \approx 218.4518$
    \item \textbf{Смотрим, куда попало значение статистики и принимаем решение, отвергнуть нулевую гипотезу или нет} \\
      статистика у нас в доверительный интервал попала, и, поэтому, мы не можем отвергнуть гипотезу $H_0$
  \end{enumerate}

\end{document}
