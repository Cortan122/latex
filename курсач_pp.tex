\documentclass[conference]{IEEEtran}
\usepackage[numbers]{natbib}
\usepackage{amsmath,amssymb,amsfonts}
\usepackage{algorithmic}
\usepackage{graphicx}
\usepackage{textcomp}
\usepackage{xcolor}
\usepackage{listings}

\def\myname{\AA{AAAAAAAAAA AAAAAAA}{6}}
\def\mytitle{Project Proposal 2023}
\def\mykeywords{compiler, C language, paradigm, procedural-parametric, syntax, polymorphism}
\usepackage[pdftex, pdfauthor={\myname}, pdftitle={\mytitle}, pdfkeywords={\mykeywords}, hidelinks]{hyperref}

\title{Development of C Compiler Extensions for the Procedural-Parametric Paradigm}

\author{
  \IEEEauthorblockN{\myname}
  \IEEEauthorblockA{\textit{Faculty of Computer Science, Department of Software Engineering} \\
  \textit{National Research University Higher School of Economics}\\
  Moscow, Russia \\
  \AA{AAAAAAAAAAAAAAAAAAAA}{8}}
  % \IEEEauthorblockA{\textit{Faculty of Computer Science} \\
  % \textit{Higher School of Economics}\\
  % \IEEEauthorblockA{\textit{Faculty of Computer Science, Department of Software Engineering} \\
  % \textit{National Research University Higher School of Economics}\\
% \and
%   \IEEEauthorblockN{Legalov Aleksandr}
%   \IEEEauthorblockA{\textit{Department of Software Engineering} \\
%   \textit{Higher School of Economics}\\
%   Moscow, Russia \\
%   alegalov@hse.ru}
}

\begin{document}

\maketitle

\begin{abstract}
  This paper presents a C compiler extension based on the Procedural-Parametric Paradigm, a programming paradigm that allows the definition of parametric types and functions.
  The extension is implemented using the LibTooling and LibASTMatchers libraries from the Clang ecosystem.
  The paper presents an experimental evaluation of the extension compared to other approaches to compilation and syntactic variations, using metrics such as compilation time, code size and extension effort.
  Overall, the proposed extension provides a way to extend C with advanced features while retaining the performance benefits of the C language.
\end{abstract}

\begin{IEEEkeywords}\mykeywords\end{IEEEkeywords}

\section{Introduction}
% ~300 words

% What is the area of your research? (1--3 sentences)
Developing a C compiler is a complex task that involves several areas of research:
parsing, language design, semantic analysis and code generation.
More specifically, my research focuses on the
development of a new programming language and its associated compiler to meet the needs of a specific programming paradigm.

% What is the problem you're trying to solve, why is it important? (2--4 sentences)
The lack of advanced language features in C~\cite{kernighan1988c}, such as polymorphism and function overloading,
can be a limitation in certain programming scenarios where object-oriented programming paradigms are highly desirable.
However, on many high-performance systems and embedded systems, the use of full-fledged object-oriented programming languages
may not be practical or desirable due to the additional overhead and complexity introduced by them.
The lack of these features in C means that developers need to rely on other programming techniques to achieve similar functionality.
While these techniques can be effective, they can also be more cumbersome and error-prone compared to native language features.

% How are you going to solve it? (1--2 sentences)
The procedural-parametric paradigm is a programming paradigm that combines elements of procedural programming and parametric polymorphism.
Parametric polymorphism allows the same code to be used with different types, providing a way to write more generic and reusable code~\cite{cardelli1985understanding}.

% On which studies do you base your research? (an appropriate number of sentences)
% What makes your approach new as compared to the existing ones? (2--4 sentences)
The procedural-parametric paradigm has been implemented in a few programming languages, including Haskell~\cite{hall1996type}, Oberon-2~\cite{mossenbock1991differences}, and Idris~\cite{brady2013idris}, which are not as widely used as C~\cite{bissyande2013popularity}.
By implementing the procedural-parametric paradigm as an extension to C, it would be possible to bring the benefits of this programming paradigm to a much larger audience.
Additionally, an extension to the C compiler can help solve some of the problems associated with the lack of object-oriented features in C by adding new language features and capabilities to the language.
These extensions can be implemented in a way that is compatible with existing C code and toolchains, allowing developers to take advantage of the new features without completely abandoning C or its ecosystem.

% What are your [expected] results? (1--4 sentences)
Overall, the expected results of my research would provide valuable information about the benefits and limitations of using the procedural-parametric paradigm in the context of the C programming language.

% How your paper is organized? (3--4 very short sentences)

\section{Subject Area}
The procedural-parametric paradigm is a programming paradigm that combines elements of procedural programming and parametric polymorphism.
In this paradigm, programs are structured around procedures, or functions, that operate on data.
However, these procedures can be parameterized by one or more types, allowing them to operate on different types of data.
The language implementation will automatically select the appropriate implementation at runtime based on the underlying types.

Ordinarily, the same functionality can be achieved in the procedural programming paradigm using a `\texttt{switch}' statement to manually handle each underlying type.
This can be useful when you have a small number of different cases and methods to handle, but it can become unwieldy and difficult to maintain when the number of cases grows large.
Additionally, this approach does not support extending an existing type form an external module~\cite{legalov2016evolutionary}.

In contrast, the procedural-parametric paradigm provides a more flexible and powerful way of working with different types of data.
By using parametric types and multiple dispatch methods, it becomes possible to define generic procedures that can handle many different types of data, without the need for manual `\texttt{switch}' statements, avoiding a lot of boilerplate code.
This can lead to cleaner, more modular code that is easier to understand and maintain.
Additionally, the use of parametric types allows for more flexible and extensible code, as new types can be easily added to the system without the need to modify existing code.

\section{Proposed Syntax}
The procedural-parametric paradigm involves two main syntactic constructions: the parametric type (see Fig.~\ref{fig:typedef}) and the parametric method (see Fig.~\ref{fig:funcdef}).

\subsection{Parametric type} % Параметрическое обобщение
The basic idea behind a parametric type is to create a generic data structure that can hold values of different underlying types.
Behind the scenes, the structure uses a tag field to identify the type of data that is currently stored in the structure, and a union field to hold the data itself (see Fig.~\ref{fig:typedefpurec}).

\begin{figure}[htbp]
  \begin{lstlisting}[frame=single,basicstyle=\ttfamily]
union Number switch {
  int;
  float;
  struct Fraction;
};
  \end{lstlisting}
  \caption{Example of a parametric type definition.}
  \label{fig:typedef}
\end{figure}

\begin{figure}[htbp]
  \begin{lstlisting}[frame=single,basicstyle=\ttfamily]
struct Number {
  int tag;
  union {
    int i;
    float f;
    struct Fraction frac;
  } value;
};
  \end{lstlisting}
  \caption{Parametric type in standard C.}
  \label{fig:typedefpurec}
\end{figure}

By using this approach, we can define a single data type that can represent multiple types of data, such as integers, floats, and structures.
This is useful in situations where we want to write functions that can operate on different types of data without having to write separate functions for each type.

\subsection{Parametric method} % Обобщающая параметрическая процедура
A parametric method allows the programmer to define a single function that can accept different types of input data, and still perform the same operation on that data.
The function achieves this by using the tag field of a parametric type to determine the underlying type of the input data at runtime, and then applying the appropriate operations for that data type.

\begin{figure}[htbp]
  \begin{lstlisting}[frame=single,basicstyle=\ttfamily]
void increment(Number* num) = 0;

void increment(int* i) {++*i;}
void increment(float* f) {++*f;}
void increment(Fraction* frac) {
  frac->numerator += frac->denominator;
}
  \end{lstlisting}
  \caption{Example of a parametric method definition.}
  \label{fig:funcdef}
\end{figure}

To define a parametric method, one must first specify the header using the ``\texttt{= 0}'' syntax.
In the header, the parametric type must be defined as pointer, which will later be replaced with pointers to the underlying types.
Then, each type that the method will operate on must be implemented using the regular function syntax, with the actual type name replacing the parametric placeholder.

The low-level implementation of a parametric method is similar to that of a switch statement (see Fig.~\ref{fig:funcdefpurec}), in which the compiler generates a table of function pointers that correspond to each possible type of input parameter~\cite{legalov2016evolutionary}.
When the method is called, the program uses this table to look up the appropriate function pointer for the type of the input parameter, and then calls the function using that pointer.

\begin{figure}[htbp]
  \begin{lstlisting}[frame=single,basicstyle=\ttfamily]
void increment(Number* num) {
  switch (num->tag) {
    case INT_TYPE:
      ++*num->i;
    case FLOAT_TYPE:
      ++*num->f;
    case FRACTION_TYPE:
      num->frac.numerator +=
        num->frac.denominator;
  }
}
  \end{lstlisting}
  \caption{Parametric method in standard C.}
  \label{fig:funcdefpurec}
\end{figure}

\subsection{Extensibility}
The key advantage of incorporating support for the pro\-ce\-dur\-al-parametric paradigm into a programming language is the ability to extend the functionality of types through external modules.
By allowing programmers to create external modules that define new implementations of existing parametric types (see Fig.~\ref{fig:extend}), the language becomes more flexible and easier to extend over time.
This feature allows developers to build on existing codebases and to create reusable libraries that can be shared across different projects, which can significantly improve code quality, maintainability, and productivity.

\begin{figure}[htbp]
  \begin{lstlisting}[frame=single,basicstyle=\ttfamily]
Number += {struct Complex;};
  \end{lstlisting}
  \caption{Extending a parametric type.}
  \label{fig:extend}
\end{figure}

\section{Methodology}
% ~1200-1500w/2

\subsection{Implementation Details}
To simplify the task of implementing a C compiler with new language features, I opted to use the LibTooling and LibASTMatchers libraries from the Clang ecosystem.
These libraries provide a range of high-level tools and abstractions that allow for more efficient development of custom compiler extensions~\cite{LibTooling}.
By using these libraries, I can focus on implementing the new language features I want to add, rather than worrying about the intricacies of building a C compiler from the ground up.

There are two main approaches to implementing a new language extension in the context of C: source-to-source translation and compilation to LLVM~\cite{llvm} intermediate representation (IR).

The source-to-source approach involves writing a preprocessor that reads in the original source code, applies the desired transformations, and then outputs the transformed source code.
This approach has the advantage of being relatively simple and straightforward, since it does not require extensive knowledge of the underlying compiler toolchain.
It can also make it easier to understand and debug the transformed code.
However, the downside is that errors from the original code may propagate to the resulting code, leading to incorrect line numbers being reported by the compiler.

The compilation to LLVM IR approach involves modifying the front-end of the compiler to recognize the new language constructs, and then outputting LLVM IR instead of machine code.
This has the advantage of allowing for more sophisticated optimizations and potentially higher performance, since the IR can be optimized using the LLVM toolchain.
However, this approach can be more complex and may require more knowledge of the underlying toolchain, and the resulting IR can be more difficult to read and debug than the original source code.

\subsection{Planned Experiments}
As part of my research, I intend to conduct experimental comparisons between the different approaches to compilation, as well as evaluate various syntactic variations in the context of the procedural-parametric paradigm.

These experiments would involve porting a number of existing examples from standard C to the C extension with support for the procedural-parametric paradigm.
This could help to identify any issues that arise during the porting process, and could also provide insight into the practical benefits of using the C extension in a real-world setting.
% For example, the experiment could show that porting the codebase to the C extension requires a significant amount of time and effort, but that the resulting code is more flexible and easier to maintain.

The resulting programs will be compared using the following metrics:
\begin{itemize}
  \item \textbf{Compilation time} may be impacted by the added complexity of my extension and the amount of code that needs to be processed. Longer compilation times can be a hindrance to developer productivity and can make the development process less agile.
  \item \textbf{Code size} may be impacted by the additional data structures or runtime instructions required to support my extension.
  \item \textbf{Extension effort}, i.e. the number of lines changed to add a new type, could be considered an important metric, especially in terms of ease of use and maintainability.
\end{itemize}

By measuring these metrics and comparing them to other approaches, I can gain insights into the strengths and weaknesses of my C compiler extension and identify areas for further optimization.

\section{Conclusion}
% ~200 words

The development of a compiler extension to support the procedural-parametric paradigm in C has the potential to enhance the language's expressiveness, while still maintaining the performance benefits of C.
By providing support for parametric types and functions, along with multiple dispatch methods, it is possible to create more modular and extensible code that is easier to maintain.

In order to evaluate the effectiveness of this extension, a series of experiments can be conducted, comparing the compilation time and size of the generated code to that of other approaches, as well as measuring the extension effort required to add a new type.
With these metrics, it will be possible to assess the tradeoffs between using this approach versus other available options.

While further research and experimentation is required to fully evaluate its effectiveness, the potential benefits make it a promising area of exploration for those interested in improving the capabilities of the C language.

\bibliographystyle{IEEEtran}
\bibliography{references}

\begin{flushright}
  Word count: 1666
\end{flushright}

\end{document}
