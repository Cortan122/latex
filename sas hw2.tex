\documentclass[12pt,a4paper]{article}
\usepackage{sasstyle}

\graphicspath{ {./sas/hw2/} }

\begin{document}
  \titlePage{2}{Регрессия}

  \section{Задание 1: Exploratory Data Analysis}
  \subsection{Постановка задачи}
  Исследовать данные с помощью PROC SGSCATTER и PROC CORR.
  Дать общее описание наличия и характера <<связи>> между различными переменными и ценой (Price).
  Линейная/нелинейная? Если нелинейная, то какая?

  \subsection{Описание работы программы}
  Используя процедуру <<proc sgscatter>>,
  программа сравнивает переменную Price со всеми другими числовыми переменными, которые есть в датасете.
  Для этого она строит график зависимости Price от каждой переменной, и строит приближённый к множеству точек В-сплайн
  (см. рисунок \ref{fig:task1_plot1.png}).

  Процедура <<proc corr>> для каждой переменной вычисляет значения минимума, максимума, математического ожидания и стандартного отклонения.
  Она также строит матрицу корреляции (см. рисунок \ref{fig:task1-result.html.png}) и проверяет P-value для каждой её ячейки.
  К тому же мы получаем гистограммную матрицу точечных диаграмм (см. рисунок \ref{fig:task1_plot2.png}).

  Также программа создаёт таблицу <<cars2>> с добавленными переменными PriceLog, CitympgLog и HwympgLog,
  которые являются логарифмами переменных Price, Citympg и Hwympg соответственно.

  \subsection{Результат выполнения программы}
  \CRTfigure{task1-result.html.png}{Результаты выполнения процедуры <<proc corr>>}
  \CRTfigure{task1_plot1.png}{Результаты выполнения программы}
  \CRTfigure{task1_plot2.png}{Матрица графиков корреляции}

  \subsection{Фрагменты из журнала}
  \CRTfigure{task1-log.html.png}{Фрагменты из журнала}

  \subsection{Ответы на вопросы}
  Связь цены (Price) с объёмом двигателя (EngineSize), количеством цилиндров в двигателе (Cylinders)
  вместимостью топливного бака (FuelTank), весом (Weight) и лошадиной силой (Horsepower)
  похожа на линейную, только разброс увеличивается с увеличением цены (см. рисунок \ref{fig:task1_plot1.png}).
  Опираясь на эту увеличивающиеся дисперсию, можно предположить что стоит моделировать не саму цену, а её логарифм.

  Связь цены с эффективностью потребления топлива в городе (Citympg) и за городом (Hwympg) не линейная.
  Возможно это обратная или экспоненциальная связь.
  Между этими параметрами существует очень чёткая линейная зависимость,
  как можно заметить на графиках в позициях $(3, 2)$ и $(2, 3)$ на рисунке \ref{fig:task1_plot2.png}.

  Связи цены с объёмом багажника (Luggage) и вместимостью пассажиров (Passengers)
  не наблюдается.

  \newpage
  \section{Задание 2: Candidate Model Selection}
  \subsection{Постановка задачи}
  Протестировать различные методы выбора переменных (пошаговые и перебор подмножеств).
  Для пошаговых методов исследовать влияние порогов для p-value для входных (включаемых в регрессию)
  и выходных (исключаемых из регрессии) переменных.
  Выбрать <<наилучшую>> модель на основе любого из исследованных методов.

  \subsection{Описание работы программы}
  Программа содержит три линейных модели.
  \begin{enumerate}
    \item Первая модель используется для прямого пошагового выбора переменных.
      В неё с каждым шагом добавляется та переменная, которая улучшает её лучше всех.
    \item Вторая модель используется для обратного исключения переменных.
      С каждым шагом из неё удаляется самая бесполезная переменная.
    \item Третья модель используется для полного перебора всех множеств переменных.
  \end{enumerate}

  \subsection{Результат выполнения программы}
  \CRTfigure{task2-plot1.png}{Прогресс параметра Adjusted R-Square с каждым шагом выбора переменных}
  \CRTfigure{task2-plot4.png}{Прогресс параметра Adjusted R-Square с каждым шагом обратного исключения переменных}
  \CRTfigure{task2-plot7.png}{Значение параметра Adjusted R-Square для всех множеств переменных, в зависимости от их размера}
  \CRTfigure{image20220313164913.png}{Наилучшая найденная модель}

  \subsection{Фрагменты из журнала}
  \CRTfigure{task2-log.html.png}{Фрагменты из журнала}

  \subsection{Ответы на вопросы}
  Пошаговые методы выбора переменных физически не могут дать результат лучше полного перебора всех множеств переменных.
  Прямой выбор переменных достиг максимального результата на своём последнем шестом шагу (см. рисунок \ref{fig:task2-plot1.png}),
  а метод обратного исключения --- на пятом (см. рисунок \ref{fig:task2-plot4.png}).

  Наилучшая модель была найдена методом полного перебора, использует переменные Citympg, EngineSize, Horsepower, FuelTank, Luggage, Weight и CitympgLog (см. рисунок \ref{fig:image20220313164913.png}).

  \newpage
  \section{Задание 3: Model Assumption Validation}
  \subsection{Постановка задачи}
  Для выбранной в предыдущем пункте <<наилучшей>> модели проверить предположения регрессионного анализа.
  Если вы не заметили раньше, то теперь почти точно нужно моделировать не Price, а log(Price) --- все из-за дисперсий.

  \subsection{Описание работы программы}
  Программа проводит диагностику лучшей модели, выбранной в задание 2, и строит графики для проверки предположений.

  \subsection{Результат выполнения программы}
  \CRTfigure{task3-result.html.png}{Результаты выполнения программы}
  \CRTfigure{task3-plot1.png}{Диагностические графики для модели}

  \subsection{Фрагменты из журнала}
  \CRTfigure{task3-log.html.png}{Фрагменты из журнала}

  \subsection{Ответы на вопросы}
  Как можно заметить на нижнем левом графике рисунка \ref{fig:task3-plot1.png}, остатки распределены нормально.
  Такой-же вывод можно сделать и используя график квантиль--квантиль,
  который находится посередине слева, над графиком нормального распределения.
  Поскольку наша модель не использует переменные Manufacturer, Model, Type и Origin,
  проверять равенство дисперсий классов не представляется необходимым.
  Независимость наблюдений проверяется создателями датасета.
  Итак мы можем сделать вывод, что все предположения проверены и модель можно использовать.


  \newpage
  \section{Задание 4: Collinearity and Influential Variables Detection}
  \subsection{Постановка задачи}
  Проверить наличие коррелирующих переменных среди выбранных, используя различные статистики в моей презентации.

  Прочитать главу <<3 Linear Methods for Regression>> из вот этой книги \cite{book}.
  Уделить особое внимание разделам <<3.4 Shrinkage Methods>>!
  (они могут помочь справиться с проблемой коллинеарности).
  Протестировать 3 метода в SAS/STAT: (1) PROC REG …. RIDGE ….
  (пример) и (2,3) PROC GLMSELECT опция SELECTION={LAR, LASSO} в операторе MODEL.

  Исследовать данные на предмет наличия <<влиятельных>> наблюдений (Influential Observation), используя методы в моей презентации.

  Убедившись, что теперь-то модель в порядке, применить ее к новому набору данных
  (для простоты - к тому же самому, из которого удалены значения переменной Price).
  Для этого использовать PROC SCORE (пример).

  \subsection{Описание работы программы}
  Программа проверяет три метода сужения регрессионных моделей.
  \begin{enumerate}
    \item Ridge регрессия используется для оценки коэффициентов линейной модели в случаях, когда в ней присутствуют коллинеарные переменные.
    \item Метод выбора переменных Lasso.
    \item Метод выбора переменных LAR.
  \end{enumerate}

  Также программа использует коэффициенты, полученные при диагностике наилучшей модели в пункте 3,
  чтобы сделать предсказания и сравнить их с истинными значениями.
  Для этого отдельным шагом надо конвертировать логарифм цены обратно в цену.

  \subsection{Результат выполнения программы}
  % \CRTfigure{task4-result.html.png}{Результаты выполнения программы}
  \CRTfigure{task4-plot1.png}{График Ridge анализа регрессии}
  \CRTfigure{task4-plot2.png}{Диагностические графики для модели}
  \CRTfigure{task4-gmselect1.png}{Результаты работы метода LAR}
  \CRTfigure{task4-gmselect2.png}{Результаты работы метода Lasso}
  \CRTfigureLong{task4-table.png}{Таблица результатов модели, применённой к набору данных}

  \subsection{Фрагменты из журнала}
  \CRTfigure{task4-log.html.png}{Фрагменты из журнала}

  \subsection{Ответы на вопросы}
  Методы Lasso (см. рисунок \ref{fig:task4-gmselect2.png}) и LAR (см. рисунок \ref{fig:task4-gmselect1.png})
  пришли к одному и тому же определению модели: <<Intercept Horsepower FuelTank Weight CitympgLog>>.

  Как можно видеть на рисунке \ref{fig:task4-table.png} в большинстве случаев модель угадывает примерно правильную цену.

  \newpage
  \section{Список использованной литературы}
  \begin{thebibliography}{3}
    \bibitem{sas} Programming Documentation for SAS [Электронный ресурс].
      //URL: \url{https://documentation.sas.com/doc/en/pgmsascdc/9.4_3.3/pgmsaswlcm/home.htm}
      (Дата обращения: 22.02.2022, режим доступа: свободный)
    \bibitem{book} Hastie, T., Tibshirani, R., Friedman, J. H., \& Friedman, J. H. (2009). The elements of statistical learning: data mining, inference, and prediction (Vol. 2). New York: springer.
  \end{thebibliography}

\end{document}
