\documentclass[12pt,a4paper]{article}
\usepackage{sasstyle}

\graphicspath{ {./sas/hw1/} }

\begin{document}
  \titlePage{1}{Дисперсионный анализ ANOVA}

  \section{Задание 1}
  \subsection{Постановка задачи}
  \label{sec:1.1}
  Используя процедуру MEANS, познакомиться с данными.
  [P1] Вывести средние в новый набор данных.
  [P2] Построить график процедурой SGPLOT, используя полученный набор данных:
       Strength по оси Y, Additive по оси X, группировать по переменной Brand.
  [P3] Что вы можете сказать о данных?
  [P4] Основываясь на графике, нужно ли использовать пересечение факторов Additive и Brand в модели?

  \subsection{Описание работы программы}
  Используя процедуру <<proc means>>, программа вычисляет среднее значение прочности бетона,
  сгруппировав его по параметрам Additive и Brand.
  Полученный набор данных сохраняется в таблицу <<averages>>.
  Затем по этой таблице строится точечная диаграмма,
  в которой значение Strength расположено по оси Y, Additive --- по оси X, а также цвет каждой точки определяется параметром Brand.

  Процедура <<proc means>> также высчитывает cреднеквадратическое отклонение, минимум и максимум
  для каждого класса (комбинации бренда и добавки) бетона.

  \subsection{Результат выполнения программы}
  \CRTfigure{task1-result.html.png}{Результаты выполнения программы}

  \subsection{Фрагменты из журнала}
  \CRTfigure{task1-log.html.png}{Фрагменты из журнала}

  \subsection{Ответы на вопросы}
  \label{sec:describe}
  Как показывает график (см. рисунок \ref{fig:task1-result.html.png}),
  средние прочности бетона действительно зависят наличия в нём добавки.
  Также можно заметить, что самый сильный из стандартных бетонов по прочности примерно совпадает с самым слабым из усиленных.
  Для каждого бренда, усиленный вариант бетона с добавкой всегда прочнее обычного бетона того же бренда.
  Но стоит учесть, что для разных брендов разница в прочности разная:
  у бренда Graystone она почти в два раза больше.
  Это значит что в модели нужно будет использовать пересечение факторов Additive и Brand.
  Данная модель будет в дальнейшем обсуждаться в пункте \ref{sec:model}.

  \newpage
  \section{Задание 2}
  \subsection{Постановка задачи}
  [P1] Проверьте гипотезу о том, что средняя прочность одинакова для всех марок. Проверить предположения. Если возможно, сравните все марки с маркой Graystone.
  [P2] Добавьте оставшийся фактор --- Additive. Какие выводы можно сделать сейчас?
  [P3] Если графики из п.1 говорят, что нужно использовать пересечение, то добавьте его. Какие выводы вы можете сделать на данном шаге анализа?

  \subsection{Описание работы программы}
  Программа содержит три линейных модели.
  \begin{enumerate}
    \item Первая модель предсказывает прочность бетона используя только его бренд.
    \item Вторая модель предсказывает прочность бетона используя линейную комбинацию его бренда и наличия в нём добавки.
    \item Третья модель предсказывает прочность бетона используя линейную комбинацию его бренда и наличия в нём добавки,
      а также произведение (пересечение) бренда на добавку.
  \end{enumerate}

  \subsection{Результат выполнения программы}
  \CRTfigure{task2-result1.html.png}{Результат выполнения первой части программы, которая не использует фактор Additive}
  \CRTfigure{task2-result2.html.png}{Результат выполнения второй части программы, которая использует фактор Additive}
  \CRTfigure{task2-result3.html.png}{Результат выполнения второй части программы, которая пересечение факторов}
  \CRTfigure{task2_plot_0.png}{Диагностические графики для модели без дополнительного фактора Additive}
  \CRTfigure{task2_plot_3.png}{Диагностические графики для модели с дополнительным фактором Additive}
  \CRTfigure{task2_plot_6.png}{Диагностические графики для модели с пересечением фактов}
  \CRTfigure{task2_plot_4.png}{График зависимости для линейной модели}
  \CRTfigure{task2_plot_7.png}{График зависимости для нелинейной модели}

  \subsection{Фрагменты из журнала}
  \CRTfigure{task2-log.html.png}{Фрагменты из журнала}

  \subsection{Ответы на вопросы}
  \label{sec:model}
  Чтобы проверить гипотезу о том, что средняя прочность одинакова для всех марок, нужно сначала проверить предположения,
  необходимые для выдвижения данной гипотезы.
  \begin{enumerate}
    \item \textbf{Нормальное распределение остатков (residuals)}: \\
      Как показывает рисунок \ref{fig:task2_plot_0.png}, а именно нижний левый график на нём, распределения остатков нормальное.
      Это также можно понять по графику квантиль--квантиль, который находится посередине слева, над графиком нормального распределения.
    \item \textbf{Равенство дисперсий всех марок бетона}: \\
      Наша модель провела тест на равенство дисперсий, результат которого можно найти в таблице <<Levene's Test for Homogeneity of Strength Variance ANOVA of Squared Deviations from Group Means>> на рисунке \ref{fig:task2-result1.html.png}.
      Значение F-value равно $0,72$, что сильно меньше единицы, и поэтому мы не можем отвергнуть базовую гипотезу о том,
      что дисперсии всех классов равны.
  \end{enumerate}
  Теперь, когда проверены все предположения, мы можем приступить к проверке самой гипотезы.
  Для этого нам понадобится третья таблица на рисунке \ref{fig:task2-result1.html.png}.
  На этот раз, значение F-value равно $3,58$, что сильно больше единицы, и мы можем отвергнуть гипотезу.
  Получатся, если у нас отвергнута гипотеза о равенстве средней прочности всех марок,
  мы можем сказать, что средние прочности у всех марок разные.
  Самый прочный бренд получается Graystone.

  Теперь перейдём ко второй модели.
  Как видно на рисунке \ref{fig:task2_plot_3.png}, распределение остатков всё ещё нормальное и предположения остаются верными.
  На рисунке \ref{fig:task2-result2.html.png} можно заметить, что значение F-value выросло до $9,76$.
  Это означает что в этой модели группы более стильно различаются.
  Также стоит обратить внимание на значение R-Square: оно поднялось с $0,209758$ для первой модели до $0,529717$ для второй.
  Из этого можно сделать вывод, что добавление фактора Additive сильно улучшило модель.
  Такой вывод идеально сходится с нашими наблюдениями в пункте \ref{sec:describe}.

  Наконец-то можно обсудить третью модель.
  Как мы решили в пункте \ref{sec:describe}, пересечение параметров Brand*Additive не будет неуместно в модели.
  Но, не смотря на это, все метрики, кроме R-Square, который увеличился всего на $0,02$, уменьшились.
  Даже распределение остатков стало менее похожим на нормальное.
  Может быть пересечение было не так необходимо, как мы думали.
  Или у нас просто недостаточно данных.

  \newpage
  \section{Задание 3}
  \subsection{Постановка задачи}
  Выполните подходящие множественные сравнения для статистически значимых переменных.

  \subsection{Описание работы программы}
  Программа проводит множественные сравнения влияний переменной Brand и переменной Additive на переменную Strength,
  используя методы корректирования Tukey и Dunnett.
  Результатны для статистически значимых переменных на графиках будут выделены синим,
  а для статистически незначимых переменных --- оранжевым.

  \subsection{Результат выполнения программы}
  \CRTfigure{task3-result.html.png}{Результаты выполнения программы}
  \CRTfigure{task3_plot_1.png}{График, сравнивающий бетон с и без добавки}
  \CRTfigure{task3_plot_0.png}{График, сравнивающий бетон с и без добавки, используя метод коррекции Tukey}
  \CRTfigure{task3_plot_3.png}{График, сравнивающий прочность бетона с добавкой относительно бетона без добавки, используя метод коррекции Dunnett}
  \CRTfigure{task3_plot_2.png}{График, попрано сравнивающий прочности разных брендов бетона}
  \CRTfigure{task3_plot_4.png}{График, сравнивающий прочность брендов бетона относительно бренда Graystone, используя метод коррекции Dunnett}

  \subsection{Фрагменты из журнала}
  \CRTfigure{task3-log.html.png}{Фрагменты из журнала}

  \subsection{Ответы на вопросы}
  Метод Tukey делает попарное сравнение каждого возможного значения класса,
  а метод Dunnett сравнивает с каким-то одним контрольным значением.
  Поскольку значений переменной Additive может быть только два: standard и reinforced,
  результаты этих методов содержат одну и туже информацию.
  Если сравнить рисунок \ref{fig:task3_plot_0.png} с рисунком \ref{fig:task3_plot_3.png},
  то можно увидеть что результаты получились одинаковые,
  и графики содержат одинаковую информацию, только в разном формате.
  Все три метода сравнения делают вывод, что разница между двумя типами Additive статистически значимая.

  При сравнение значений переменной Brand методы дают разную информацию.
  Метод Dunnett говорит, что от бренда Graystone статистически значимо отличается только бренд Consolidated,
  а бренд EZ Mix --- нет.
  Метод Tukey также выясняет, что бренды Consolidated и EZ Mix статистически значимо не отличается.

  \newpage
  \section{Список использованной литературы}
  \begin{thebibliography}{3}
    \bibitem{sas} Programming Documentation for SAS [Электронный ресурс].
      //URL: \url{https://documentation.sas.com/doc/en/pgmsascdc/9.4_3.3/pgmsaswlcm/home.htm}
      (Дата обращения: 22.02.2022, режим доступа: свободный)
  \end{thebibliography}

\end{document}
