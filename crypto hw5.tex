\documentclass{article}

\usepackage{enumitem}
\usepackage{amsmath}
\usepackage{amsfonts}
\usepackage{amssymb}
\usepackage{textcomp}
\usepackage{gensymb}
\usepackage[dvipsnames]{xcolor}
\usepackage[margin=0.5in]{geometry}
\usepackage[hidelinks,bookmarks=false]{hyperref}

\usepackage{environ}
\NewEnviron{centerframebox}{\begin{center}\fbox{\parbox{0.92\textwidth}{\BODY}}\end{center}}

\hypersetup{
  colorlinks=true,
  urlcolor=MidnightBlue,
  linkcolor=WildStrawberry,
  % pdfborderstyle={/S/U/W 2}
}

\title{Cryptography \\ Exercise Sheet 5}
\author{
  \AA{AAAAAAAAAA AAAAAAA}{6} \\
  \href{mailto:\AA{AAAAAAAAAAAAAAAAAAAA}{7}}{\AA{AAAAAAAAAAAAAAAAAAAA}{7}}
}

\begin{document}
  \maketitle

  \setcounter{section}{5}
  \subsection{Dilemma}
  \begin{centerframebox}
    Let $\Pi_0$ and $\Pi_1$ be two encryption schemes for which it is known that at
    least one of them is IND-CPA secure. The problem is that you don't know
    which one is IND-CPA-secure and which one may not be. Show how to
    construct an encryption scheme $\Pi$ that is guaranteed to be IND-CPA secure.

    Provide a full proof of your answer.

    You may assume that plaintext and ciphertext spaces are such that you can
    combine the two ciphers.
  \end{centerframebox}
  The obvious solution is to simply combine the schemes by calling one with the output of the other: $c = \Pi_0(\Pi_1(m))$.
  It looks good, and even works if one of the schemes is entirely insecure%
  \footnote{Passing the message unaffected, for example. But the schemes do have to be reversible. And, I think, they are reversible just by virtue of being encryption schemes},
  but there is a counterexample.
  Imagine that $\Pi_0$ is the exact inverse of $\Pi_1$, i.e. $\operatorname{Enc}_{\Pi_0} = \operatorname{Dec}_{\Pi_1}$.
  In this case our combined scheme will simply return the unaltered message, which is quite obviously not secure.

  Because of this counterexample we must pass different keys to our schemes,
  and the total key length of the combined scheme $\Pi_{01}$ must be twice as long.

  Now to prove the IND-CPA security of $\Pi_{01}$, we assume the opposite:
  an attacker $\mathcal{A}$ exists and has a non-negligible advantage in the IND-CPA game for $\Pi_{01}$.
  But then the reduction $\mathcal{A} \circ \Pi_0$ will break the IND-CPA security of $\Pi_1$,
  and it contradicts the case of $\Pi_1$ being the secure one of the pair.
  We can use $\Pi_1$ in the reduction with a random key, because the keys are independent,
  and will not affect the game's success probability.

  For the second case, where $\Pi_0$ is IND-CPA secure, the reduction is a bit more tricky.
  We have to modify the oracles to first run the attacker supplied message through $\Pi_1$,
  again with a fixed random key, and then pass the result as a message to the attacked scheme.
  Either way, we get a contradiction\footnote{I think actually the reductions work just by themselves, without needing to declare an explicit contradiction} in both cases, and $\Pi_{01}$ must be IND-CPA secure.

  % XORing the outputs of the two schemes will also not work, because what if $\operatorname{Enc}_{\Pi_0} = \lnot \operatorname{Enc}_{\Pi_1}$ and will always result in a zero ciphertext (also you can't even decrypt that shit)
  % TODO

  \subsection{MAC cross over}
  \begin{centerframebox}
    Consider the simple, fixed-length CBC-MAC from the lecture.

    Let's ignore the length restriction: Given a message $m = m^{(0)}|m^{(1)}$ of
    double length $2\kappa \cdot \ell(\kappa)$. Construct a CBC-MAC tag for $m$ by calling $\mathcal{O}^\mathrm{Mac}$
    twice for messages of length $\kappa \cdot \ell(\kappa)$.

    Conclude that CBC-MAC without length restriction is not EUF-CMA secure.
  \end{centerframebox}
  The usual tag for CBC-MAC is based on the value of the previous tag XORed with the each block of the message.
  If we obtain the first tag by calling $t^{(0)} = \mathcal{O}^\mathrm{Mac}\left(m^{(0)}\right)$, we can modify the first block of $m^{(1)}$,
  such that the process of producing intermediate tags continues as normal.
  For this we need to make sure that in the second call of the oracle the input to the pseudorandom function is exactly
  $t^{(0)} \oplus m^{(1)}_0$.
  But it is normally calculated as $\mathsf{IV} \oplus m^{(1)}_0$, and if we compute $m_0' = m^{(1)}_0 \oplus t^{(0)} \oplus \mathsf{IV}$
  and replace the block $m^{(1)}_0$ with $m_0'$, then it will also be $\mathsf{IV} \oplus m^{(1)}_0 \oplus t^{(0)} \oplus \mathsf{IV} = t^{(0)} \oplus m^{(1)}_0$.
  So we got the correct value and can break the scheme with:
  \[ \operatorname{Mac}(m) = \mathcal{O}^\mathrm{Mac}\left(
      \mathcal{O}^\mathrm{Mac}\left(m^{(0)}\right) \oplus m^{(1)}_0 \oplus \mathsf{IV}
      \;\middle|\;
      m^{(1)}_1 \dots m^{(1)}_{\ell(\kappa)}
  \right) \]

\end{document}
