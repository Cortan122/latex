\documentclass{article}

\usepackage{enumitem}
\usepackage{amsmath}
\usepackage{amsfonts}
\usepackage{amssymb}
\usepackage{textcomp}
\usepackage{gensymb}
\usepackage[dvipsnames]{xcolor}
\usepackage[margin=0.5in]{geometry}
\usepackage[hidelinks]{hyperref}

\usepackage{environ}
\NewEnviron{centerframebox}{\begin{center}\fbox{\parbox{0.92\textwidth}{\BODY}}\end{center}}

\hypersetup{
  colorlinks=true,
  urlcolor=MidnightBlue,
  % pdfborderstyle={/S/U/W 2}
}

\usepackage[listings,many]{tcolorbox}
\makeatletter
\newtcblisting{mylisting}{
  listing only,
  breakable, enhanced jigsaw,
  listing engine=listings,
  colback=gray!20,
  listing options={
    language=python,
    keywordstyle=\color{blue},
    basicstyle=\ttfamily,
    stringstyle=\color{ForestGreen},
    commentstyle=\color{gray},
    ndkeywordstyle={\color{orange}},
    identifierstyle=\color{black},
    numbers=none,
    showstringspaces=false,
    aboveskip={0\p@ \@plus 6\p@}, belowskip={0\p@ \@plus 6\p@},
    columns=fullflexible, keepspaces=true,
    breaklines=true, breakatwhitespace=true,
    extendedchars=false,
    inputencoding=utf8,
    upquote=true,
    xleftmargin=-25pt,
  }
}
\makeatother

\newtcolorbox{cross}{blank,breakable,parbox=false,
  overlay={\draw[red,line width=5pt] (interior.south west)--(interior.north east);
    \draw[red,line width=5pt] (interior.north west)--(interior.south east);}}

\title{Cryptography \\ Exercise Sheet 2}
\author{
  AAAAAAAAAA AAAAAAA \\
  \href{mailto:AAAAAAAAAAAAAAAAAAAA}{AAAAAAAAAAAAAAAAAAAA}
}

\begin{document}
  \maketitle

  \setcounter{section}{1}
  \subsection{One-time pad or Vernam's cipher}
  \begin{centerframebox}
    When using the one-time pad (Vernam's cipher) with the key $k = 0^\kappa$, it
    follows that $\operatorname{Enc}_k(m) = m$ and the message is effectively sent in the clear.
    It has therefore been suggested to improve the one-time pad by only encrypting with a key $k \neq 0^\kappa$,
    i. e. to have KeyGen choose $k$ uniformly at random from the set of non-zero keys of length $\kappa$.
    Is this an improvement? In particular, is it still perfectly secret? Prove your answer.

    If your answer is positive, explain why the one-time pad is not described
    in this way. If your answer is negative, reconcile this with the fact that
    encrypting with $0^\kappa$ doesn't change the plaintext.
  \end{centerframebox}
  Removing the all zero key $0^\kappa$ actually breaks perfect secrecy.
  To prove this, we just need to find one counterexample pair of $m$ and $c$,
  for which the probability of seeing the message $m$ depends of seeing the ciphertext $c$.
  Let's just pick $c = m$.
  Because we made it so the key $0^\kappa$ is impossible,
  encrypting $m$ could not have resulted in the ciphertext $c$, and $\operatorname{prob}(m \mid c) = 0$.
  Which is obviously not the same as $\operatorname{prob}(m)$, because that depends on our message distribution.
  In simpler terms, if the key $0^\kappa$ is impossible, when an attacker receives a ciphertext, that could be a plausible message,
  they will automatically know that that message is the the message that was sent.

  Sometimes not encrypting the message is fine, because
  (1) in practice the probability of the zero key $0^\kappa$ is negligibly small, and
  (2) it is also the same probability of randomly hitting a different plausible message just by chance.
  When an attacker receives a ciphertext, that could be a plausible message, they can't know if the key was zero,
  or if the ciphertext came from encrypting a completely different plausible message.

  But it the real word, the attacker should probably assume the encryption algorithm has a bug,
  and generates zero keys with an abnormally high probability.
  A bug that makes zero keys more common is much more likely to occur in actual software
  (and I think there are historical examples of this very thing, that we unnoticed for years),
  than hitting the miniscule probabilities for one of the theoretical scenarios described above.

  \subsection{Game definition for pseudo-randomness}
  \begin{centerframebox}
    Show that the definition for pseudo-randomness coincides with the game-based one.

    Pseudo-randomness is defined as:
    For each probabilistic polynomial-time distinguisher $\mathcal{D}$ the advantage
    \[ \operatorname{adv}_G(\mathcal{D}) = \frac{1}{2}\left|
      \vphantom{\frac{1}{2}}
      \operatorname{prob}(\mathcal{D}(G(s)) = 1) -
      \operatorname{prob}(\mathcal{D}(r) = 1)
    \right| \]
    is negligible. Here, $s \leftarrow \{0,\, 1\}^\kappa$ and $r \leftarrow \{0,\, 1\}^{\ell(k)}$ are chosen uniformly at random.

    The game based definition involves the same distinguisher $\mathcal{D}$, and feeds it either $G(s)$ or $r$ at random,
    determined by the random bit $h_\textsc{prg}$, and then expects the distinguisher to guess it.
  \end{centerframebox}
  First prove that the game based definition follows from the normal one.
  Then we can use the fact that the probabilities $\operatorname{prob}(\mathcal{D}(G(s)) = 1)$ and $\operatorname{prob}(\mathcal{D}(r) = 1)$ are significantly different.
  We can rewrite them to include the our random bit $h_\textsc{prg}$:
  \begin{align*}
    \operatorname{prob}(\mathcal{D}(G(s)) = 1) &= \operatorname{prob}(\mathcal{D}(w_{h_\textsc{prg}}) = 1 \mid h_\textsc{prg} = 0)
    = \operatorname{prob}(G^\textsc{prg}(\mathcal{D}) = \textcolor{BrickRed}{REJECT} \mid h_\textsc{prg} = 0)\\
    \operatorname{prob}(\mathcal{D}(r) = 1) &= \operatorname{prob}(\mathcal{D}(w_{h_\textsc{prg}}) = 1 \mid h_\textsc{prg} = 1)
    = \operatorname{prob}(G^\textsc{prg}(\mathcal{D}) = \textcolor{OliveGreen}{ACCEPT} \mid h_\textsc{prg} = 1)
  \end{align*}

  From the lemma on page 84 of the lecture slides we know, that for the indistinguishability game:
  \[ \frac{1}{2}\left|
    \vphantom{\frac{1}{2}}
    \operatorname{prob}(G(\dots) = \textcolor{OliveGreen}{ACCEPT} \mid h = 0) -
    \operatorname{prob}(G(\dots) = \textcolor{BrickRed}{REJECT} \mid h = 1)
  \right| = \left|
    \operatorname{prob}(G(\dots) = \textcolor{OliveGreen}{ACCEPT}) - \frac{1}{2}
  \right| \]
  This is derived form basic algebra, and can be safely used here, with a completely different game,
  as long as it also has this random bit $h$.
  We can also swap the $h=0$ and $h=1$ states, because that will normally flip the sign, and we are under the absolute value operator.
  Applying it to our game $G^\textsc{prg}(\mathcal{D})$, along with the equalities we derived in the previous paragraph, we get:

  \[ \operatorname{adv}_G(\mathcal{D}) = \frac{1}{2}\left|
    \vphantom{\frac{1}{2}}
    \operatorname{prob}(\mathcal{D}(G(s)) = 1) -
    \operatorname{prob}(\mathcal{D}(r) = 1)
  \right| = \left|
    \operatorname{prob}(G^\textsc{prg}(\mathcal{D}) = \textcolor{OliveGreen}{ACCEPT}) - \frac{1}{2}
  \right| = \operatorname{adv}^\textsc{prg}_G(\mathcal{D}) \]

  So if the probabilities on the left differ by a non-negligible advantage, the ones on the right will also have that advantage.
  And we don't even need to consider the inverse case, because we proved they are perfectly equal.

  \subsection{A toy generator}
  \begin{centerframebox}
    Consider the (one-size) generator given by the following table:

    \begin{center}
      \begin{tabular}{r|r}
        $s$ & $G(s)$ \\ \hline
        000 & 000000 \\
        001 & 010001 \\
        010 & 111001 \\
        011 & 101110 \\
        100 & 010111 \\
        101 & 101101 \\
        110 & 110011 \\
        111 & 010100 \\
      \end{tabular}
    \end{center}
  \end{centerframebox}

  \subsubsection*{(i) Determine the advantage of the distinguisher that on input $w$ returns whether $\operatorname{bit}_0 w$ equals $\operatorname{bit}_2 w$.}
  Let's look at the values of the $f(w) = \operatorname{bit}_0 w == \operatorname{bit}_2 w$ function for all of our 8 possible outputs:

  Update: I was actually working with the old version of the Exercise Sheets.
  In the updated version we actually had to compare $g(w) = \operatorname{bit}_3 w == \operatorname{bit}_4 w$

  \begin{center}
    \begin{tabular}{r|r|c|c}
      $s$ & $G(s)$ & $f(G(s))$ & $g(G(s))$\\ \hline
      000 & 000000 & $0=0$ & $0 = 0$\\
      001 & 010001 & $0=0$ & $0 = 0$\\
      010 & 111001 & $1=1$ & $0 = 0$\\
      011 & 101110 & $1=1$ & $1 = 1$\\
      100 & 010111 & $0=0$ & $1 = 1$\\
      101 & 101101 & $1=1$ & $1 \neq 0$\\
      110 & 110011 & $1 \neq 0$ & $0 \neq 1$\\
      111 & 010100 & $0=0$ & $1 \neq 0$\\
    \end{tabular}
  \end{center}

  We can see that it is true in 7 out of the 8 possible cases. And for the random values it will just be $\frac{1}{2}$.
  \begin{align*}
    \operatorname{prob}(\mathcal{D}(G(s)) = 1) &= \frac{7}{8} \\
    \operatorname{prob}(\mathcal{D}(r) = 1) &= \frac{1}{2}
  \end{align*}
  So the advantage of this distinguisher $\mathcal{D}$ will be:
  \[ \operatorname{adv}_G(\mathcal{D}) = \frac{1}{2}\left|\frac{7}{8} - \frac{1}{2}\right| = \frac{3}{16} \]
  Sadly this is still smaller then $\frac{1}{4}$, and we have to construct a better distinguisher for part (ii).

  And for the function $g$, we get the advantage:
  \[ \operatorname{adv}_G(\mathcal{D}) = \frac{1}{2}\left|\frac{5}{8} - \frac{1}{2}\right| = \frac{1}{16} \]

  \subsubsection*{(ii) Construct a distinguisher with advantage $\frac{1}{4}$ or better.}
  Here we can take the obvious brute force distinguisher, because I don't think we can get anything better then $\frac{7}{8}$ with a hand crafted bit function.
  The brute force distinguisher $\mathcal{B}$ just checks if the input
  \[w \in \{000000,\, 010001,\, 111001,\, 101110,\, 010111,\, 101101,\, 110011,\, 010100\},\]
  and returns 0 otherwise.
  The probabilities are a lot easier here, because $\mathcal{B}$ never\footnote{not really \texttt{:)}} makes mistakes.
  \clearpage
  \[\operatorname{prob}(\mathcal{B}(G(s)) = 1) = 1  \]
  \[\operatorname{prob}(\mathcal{B}(r) = 1) = \frac{2^3}{2^6} = \frac{8}{64} = \frac{1}{8} \]
  And the advantage of this distinguisher $\mathcal{B}$ will be:
  \[ \operatorname{adv}_G(\mathcal{B}) = \frac{1}{2}\left|1 - \frac{1}{8}\right| = \frac{7}{16} > \frac{1}{4} \]
  It is impossible to construct anything with a better advantage for this generator.

\end{document}
