\documentclass{article}

\usepackage{enumitem}
\usepackage{amsmath}
\usepackage{amsfonts}
\usepackage{amssymb}
\usepackage[margin=0.5in]{geometry}
\usepackage[hidelinks, bookmarks=false]{hyperref}

\usepackage[dvipsnames]{xcolor}
\usepackage{algorithm}
\usepackage{algpseudocode}
\usepackage{algorithmicx}

\let\oldemptyset\emptyset
\let\emptyset\varnothing
\newcommand{\R}{\mathbb{R}}
\newcommand{\Rd}{\R^d}

\usepackage{environ}
\NewEnviron{centerframebox}{\begin{center}\fbox{\parbox{0.92\textwidth}{\BODY}}\end{center}}

\title{Discrete and Computational Geometry \\ Assignment 4}
\author{
  \AA{AAAAAAAAAA AAAAAAA}{6} \\
  \href{mailto:\AA{AAAAAAAAAAAAAAAAAAAA}{7}}{\AA{AAAAAAAAAAAAAAAAAAAA}{7}}
  \and
  Emilia Groß-Hardt \\
  \href{mailto:emilia.ghdt@uni-bonn.de}{emilia.ghdt@uni-bonn.de}
}

\usepackage{titlesec}
\titleformat{\section}
  {\normalfont\Large\bfseries}{Problem \thesection : }
  {0em}{\mdseries}

\newcommand{\algorithmRuledHeader}[2]{\hrule height.8pt \vspace{2pt}\textbf{Algorithm #1:} #2\vspace{2pt}\hrule}

\begin{document}
  \maketitle
  \begin{center}
    { \bfseries Deadline: 23 Nov 2024, 23:55 }
  \end{center}

  \section{}
  Let $P = \{a=(0, 0), b=(\cos{0^\circ, \sin{0^\circ}}), c=(\cos{120^\circ, \sin{120^\circ}}), d=(\cos{240^\circ, \sin{240^\circ}})\}$. It follows, that $\Delta_{bcd}$ is part of the convex hull of $P'$ since every possible triangle in $P'$ is part of the convex hull as $|P'| = 4$. However $\Delta_{bcd}$ isn't a delaunay triangle as $a$ is obviously contained in the circle with $b$, $c$, and $d$, since this circle is just the circle with radius $1$ around $a$.
  
  \section{}
  Each edge of a Voronoi diagram region must be perpendicular to the line drawn thru the \textit{point sites} of the two regions it divides. The point sites must also be equidistant from this edge.

  Additionally, this problem doesn't necessarily have a unique solution. Imagine the simplest case of the plane divided into two infinite half-planes. We can place one point site anywhere in the left halfplane, and then determine the position of the second, symmetrical on the dividing line. 

  In the same way, we can extend to a full solution, as long as we have one known point.
  However, in the general case, not all choices of the starting point will lead to a solution. We might end up with a region having two or more points in it. 

  Is is also possible to use the fact that the Delaunay triangulation is the dual of the Voronoi diagram to recover one of the point sites.
  
  \section{}
  Let $a = (0, 0), b = (\cos{0^\circ, \sin{0^\circ}}), c = (\cos{1^\circ, \sin{1^\circ}}), d = (\cos{-1^\circ, \sin{-1^\circ}})$.
  Since $a$ is contained in the circle defined by $b$, $c$, and $d$, $\Delta_{bcd}$ is not a delaunay triangle, thus the delaunay triangulation contains exactly the edges $ac$, $cb$, $bd$, $da$, and $ab$. However the minimum weight triangulation would contain $cd$ instead of $ab$ since $\sin{(1^\circ)} - \sin{(-1^\circ)} < 1$.

\end{document}
