\documentclass[12pt]{article}
\usepackage{122}

% \geometry{margin=2cm}
\geometry{reset}
\setlength{\parindent}{1.5em}

\title{Базы данных \\ Эссе \\ Информационная модель для сети фастфуд ресторанов}
\author{\AA{AAAAA AAAAAAA}{4} \\ БПИ197}

\begin{document}
  \maketitle

  \section{SQL vs NoSQL}
  В последние несколько лет стал популярным новый подход к построению баз данных -- NoSQL.
  Сказать что-то общее про NoSQL довольно сложно, потому что это очень широкий термин
  и под него попадает большое количество разнородных СУБД.
  По определению, NoSQL это буквально всё кроме SQL:
  от графовых систем, в которых всё построено на связях, до key-value систем, в которых связей просто нет.

  Такое странное разделение всех баз данных на два типа, реляционные и все остальные, сложилось исторически.
  Всемирное сообщество программистов в 1970-х годах прошлого века пришло именно к этой парадигме,
  и до сих пор реляционные СУБД и язык запросов SQL остаются очень популярными.

  Некоторые типы данных плохо подходят для хранения в таблицах, но из-за популярности реляционных систем,
  люди находили способы адаптировать их под строгие требования данной парадигмы.
  Так, со временем, язык SQL становился всё сложнее и сложнее, и в конце концов это привило к тому,
  что SQL инъекции стали одним из самых распространённых способов взлома информационных систем.
  Большинству NoSQL не нужен такой сложный язык запросов, хотя некоторые используют свои аналоги SQL-а.
  Например в MongoDB в качестве языка запросов используется сам JSON, в который невозможно сделать инъекцию.

  \section{MongoDB}
  MongoDB относится к документоориентированным СУБД.
  Она хранит свои данные в JSON документах.
  Это звучит очень привлекательно, учитывая тот факт,
  что в современной мобильной и web разработке принято все взаимодействие с базой данных производить через CURD API.
  Обычно это очень скучные программы, которые просто используют ORM библиотеки для автоматической конвертации запросов в SQL,
  и возвращают JSON со всеми релевантными к запросу данными.
  Если всё уже итак хранится в JSON-е, то этого можно избежать.

  Хранение всех связанных данных в одном месте конечно очень удобно для записи,
  но из-за отсутствия нормализации, происходит много дупликации данных, и эти дуплицированные данные будет потом сложно поменять.
  Вообще нормализация это концепт придуманный специально для SQL-ых СУБД и на NoSQL ложится плохо.
  Хотя MongoDB в недавних версиях поддерживает даже join, и если очень захотеть,
  то её можно использовать для хранения нормализованных данных.

  Ещё одно преимущество MongoDB это её скорость.
  При обычном использование она работает в от 10 до 100 раз быстрее чем реляционные базы данных,
  не смотря на то что у них было 50 лет на тщательную оптимизацию всех возможных деталей.

  От использования MongoDB может отпугнуть тот факт, что код серверной части распространяется под проприетарной лицензией.
  Реляционных БД много, они мало отличаются и всегда можно найти open source аналог.

  \section{Сеть ресторанов}
  Мне кажется, что в данной ситуации лучше использовать MongoDB.
  Нам больше важна масштабируемость и плавность работы приложений чем точная гарантия целостности данных и максимальная экономия места.
  Фастфуд ресторан это не банк: если одна транзакция не пройдёт, то ничего страшного.
  Изменения, которые сможет делать рядовой пользователь, не будут касаться таких продублированных данных, как адреса заведений,
  и, в тех редких случаях, когда их всё таки надо будет поменять, это сможет сделать администратор.

  Но выбор системы устройства БД это больше субъективная вещь.
  Очень много споров, про то какая система лучше, и в каких ситуациях какую стоит использовать.
  Его лучше делать основываясь на персональных предпочтениях тех программистов, которые будут работать над данной системой.

\end{document}
