\documentclass{article}

\usepackage{cmap} % поиск в pdf
\usepackage{mathtext} % русские буквы в формулах
\usepackage[english,russian]{babel} % локализация и переносы
\usepackage[T2A]{fontenc} % кодировка в pdf (магия)
\usepackage[utf8]{inputenc} % кодировка исходного текста
\usepackage{amsmath}
\usepackage{amsfonts}
\usepackage{amssymb}
\usepackage{headertable}

\usepackage{url}
\usepackage{relsize}
\usepackage{graphicx}
\usepackage{tasks} % horizontal lists
\usepackage{mathdots} % iddots
\usepackage{varwidth}

\usepackage[margin=0.5in]{geometry}

\usepackage{xcolor}
\newcommand{\orange}[1]{{\color{orange}{#1}}}

\renewcommand{\vec}{\overrightarrow}
\newcommand{\ds}{\displaystyle}
\newcommand{\abs}[1]{\left|#1\right|}
\newcommand{\Arg}[1]{\arg\left(#1\right)}
\newcommand{\TrigFrom}[1]{
  \left(\cos\left({#1}\right)+i\sin\left({#1}\right)\right)
}
\newcommand{\DS}{\phantom{$0.5$}}

\newcommand{\Mod}[1]{\pmod{#1}}
\renewcommand{\mod}{\operatorname{mod}}
\newcommand{\floor}[1]{\left\lfloor {#1} \right\rfloor}
\newcommand{\bigO}{\mathcal{O}}

\title{Дискретная математика \\ Домашнее задание №1}
\author{AAAAA AAAAAAA \\ AAAAAA}

\begin{document}
  \maketitle
  \HeaderTable{22}

  \section{Докажите, что для каждого натурального $n \geq 1$ верно}
  $$\sum_{k=1}^n k^2 = \frac{n(n+1)(2n+1)}{6}$$
  сначала нам надо доказать это для $n=1$, будет $\ds 1^2 = \frac{1 \cdot 2 \cdot 3}{6}$ \\
  потом нам надо доказать это для $n+1$, если нам дано это для $n$, \\
  тоесть $\ds \frac{n(n+1)(2n+1)}{6} + (n+1)^2 = \frac{(n+1)(n+2)(2(n+1)+1)}{6}$ \\
  $\ds \frac{n(n+1)(2n+1)}{6} + n^2 + 2n + 1 = \frac{(n+1)(n+2)(2n+3)}{6}$ \\
  $\ds n(n+1)(2n+1) + 6n^2 + 12n + 6 = (n+1)(n+2)(2n+3)$ \\
  $\ds 2n^3 + 3n^2 + n + 6n^2 + 12n + 6 = 2n^3 + 9n^2 + 13n + 6$ \\
  $\ds 2n^3 + 9n^2 + 13n + 6 = 2n^3 + 9n^2 + 13n + 6$ \\
  $\blacksquare$ (а дальше по индукции)

  \section{Используя лишь рекурсивное определение сложения и индукцию, докажите что при всех $n, m \in \mathbb{N}$ верно:}
  \begin{center}
    \begin{varwidth}{\textwidth}
      \begin{enumerate}
        \item $S(m+n) = m+S(n)$
        \item $n+m = m+n$
      \end{enumerate}
    \end{varwidth}
  \end{center}
  вот рекурсивное определение сложения:
  $$ 0+n=n \qquad \textrm{и} \qquad S(m+n) = S(m)+n $$
  \subsection{Докажите, что $S(m+n) = m+S(n)$}
  \label{recsymetry}
  мы делаем индукцию для $m$ \\
  сначала нам надо доказать это для $m=0$, будет $ S(0+n) = 0+S(n) $; $S(n)=S(n)$ \\
  потом нам надо доказать это для $S(m)$, если нам дано это для $m$, \\
  тоесть $ S(m+n) = m+S(n) \implies S(S(m)+n) = S(m)+S(n) $ \\
  $ S(m+n) = m+S(n) $ \\
  $ S(S(m+n)) = S(m+S(n)) $ \\
  $ S(S(m+n)) = S(m)+S(n) $ \\
  $ S(S(m)+n) = S(m)+S(n) $ \\
  $\blacksquare$ (а дальше по индукции)
  \subsection{Докажите, что $n+m = m+n$}
  мы делаем индукцию для $m$ \\
  сначала нам надо доказать это для $m=0$, тоесть $ n+0 = 0+n $, это доказано в \ref{zerosymetry} \\
  потом нам надо доказать это для $S(m)$, если нам дано это для $m$, \\
  тоесть $ n+m = m+n \implies n+S(m) = S(m)+n $ \\ \\
  $ n+m = m+n $ \\
  $ S(n+m) = S(m+n) $ \qquad здесь мы используем то, что мы доказали в \ref{recsymetry} \\
  $ n+S(m) = S(m)+n $ \\
  $\blacksquare$ (а дальше по индукции)
  \subsubsection{Докажите, что $n+0 = n$}
  \label{zerosymetry}
  мы делаем индукцию для $n$ \\
  сначала нам надо доказать это для $n=0$, тоесть $ 0+0 = 0+0 $; $ 0 = 0 $ \\
  потом нам надо доказать это для $S(n)$, если нам дано это для $n$, \\
  тоесть $ n+0 = n \implies S(n)+0 = S(n) $ \\
  $ n+0 = n $ \\
  $ S(n+0) = S(n) $ \\
  $ S(n)+0 = S(n) $ \\
  $\blacksquare$ (а дальше по индукции)

  \section{В некоторой стране лишь конечно много городов, причем любые два различных города соединены дорогой с односторонним движением. Докажите, что есть город, из которого можно добраться в любой другой по имеющимся дорогам}
  назовём город, из которого можно добраться в любой другой по имеющимся дорогам, столицей \\
  мы делаем индукцию для количества городов (назовём это $n$) \\
  стоит заметить, что для $n=0$ это неработает (столица неможет существовать, потомучто она город а городов нет) \\
  сначала нам надо доказать это для $n=1$, но тут всё очевидно, тк других городов нет \\
  потом нам надо доказать это для $n+1$, если нам дано это для $n$, \\
  тоесть если мы добавляем город (назовём его $B$) и столица (назовём её $A$) существует,
  то в получившемся графе тоже будет существовать столица, как бы мы не провели новые дороги (все остальные города мы назовём $C$) \\
  если дорога между $A$ и $B$ идет из $A$ в $B$, то $A$ остаётся столицей, тк из $A$ можно попасть и в $C$ и в $B$, \\
  а если она из $B$ в $A$, то $B$ станет новой столицей, тк из $B$ можно попасть в $A$, а из $A$ в $С$ \\
  $\blacksquare$ (а дальше по индукции)

  \section{Докажите, что $\forall n\in\mathbb{N}:\exists k\in\mathbb{N}: 1+\frac{1}{2}+\frac{1}{3}+\cdots+\frac{1}{k} \geq n$}
  определение безконечного предела последовательности $x_n$:
  \[
    \forall n\in\mathbb{R}:\exists k\in\mathbb{N}:\forall m \geq k: x_m > n
    \qquad\iff\qquad
    \lim_{n \to \infty} x_n = +\infty
  \]
  если мы возьмём $\ds x_n = \sum_{k=1}^n \frac{1}{k}$, тогда из $\ds \lim_{n \to \infty} x_n = +\infty $ будет следовать наше утверждение,
  потомучто утверждение становится менее сильным когда мы убираем квантор "$\forall$"{} и когда уменьшаем область первого квантора из $\mathbb{R}$ в $\mathbb{N}$
  \subsection{Докажите, что гармонический ряд расходится\protect\footnote{Если предел последовательности частичных сумм несуществует или бесконечен, то говорят, что ряд расходится}}
  \begin{align*}
  \sum_{k=1}^\infty \frac{1}{k}
  & = 1 + \left[\frac{1}{2}\right] + \left[\frac{1}{3} + \frac{1}{4}\right]
  + \left[\frac{1}{5} + \frac{1}{6} + \frac{1}{7} + \frac{1}{8}\right] + \left[\frac{1}{9}+\cdots\right] +\cdots \\
  & \geq 1 + \left[\frac{1}{2}\right] + \left[\frac{1}{4} + \frac{1}{4}\right]
  + \left[\frac{1}{8} + \frac{1}{8} + \frac{1}{8} + \frac{1}{8}\right] + \left[\frac{1}{16}+\cdots\right] +\cdots \\
  & = 1 + \frac{1}{2} + \frac{1}{2} + \frac{1}{2} + \frac{1}{2} + \cdots
  \end{align*}
  \vspace{1cm} % footnote spacer

  \section{Найдите две последние цифры десятичной записи числа $99^{1000}$}
  "\underline{две} последние цифры \underline{десятичной} записи"{}
  это насамомделе остаток деления на $10^2$ тоесть нам надо найти $99^{1000} \mod{100}$ \\
  $99 \equiv -1 \Mod{100} \implies 99^{1000} \equiv (-1)^{1000} \equiv \left((-1)^2\right)^{500} \equiv 1^{500} \equiv 1 \Mod{100}$ \\
  ответ \fbox{$01$}

  \section{Докажите, что $a^3$ и $b^3$ сравнимы по модулю $a-b$, если $a>b$}
  если $x$ и $y$ сравнимы по модулю $m$, то $x-y \equiv 0 \Mod{m}$ \\
  $a^3 - b^3 = (a-b)(a^2+ab+b^2)$ \\
  тк у нас все числа целые, то $(a^2+ab+b^2)$ тоже целое и $a^3 - b^3$ делится на $(a-b)$

  \section{Докажите, что если $5m+3n \equiv 0 \Mod{11}$, то $9m+n \equiv 0 \Mod{11}$}
  тк здесь всё происходит $\operatorname{mod}11$, то мы можем рассматривать только $m,n \in [0;10]$ \\
  ещё, тк $11$ это простое чисто, то у нас $\ds \forall k \neq 0: \exists!k^{-1}: k \cdot k^{-1} \equiv 1 \Mod{11}$ \\
  $\ds 5m+3n \equiv 0 \Mod{11} \implies 5m \equiv -3n \equiv 8n \Mod{11} \implies m \equiv 8 \cdot 5^{-1}n \Mod{11}$ \\
  $5^{-1}$ мы можем найти только перебором, $5 \cdot 9 = 45 = 4 \cdot 11 + 1 \implies 5^{-1} \equiv 9 \Mod{11}$ \\
  $\ds m \equiv 8 \cdot 5^{-1}n \Mod{11} \implies m \equiv 8 \cdot 9 n \equiv 6n \Mod{11}$ \\
  мы можем проверить это (то что $m \equiv 6n$), подставив это в $5m+3n \equiv 0 \Mod{11}$, \\
  будет $30n+3n \equiv 33n \equiv 0 \Mod{11}$, что очевидно True \\
  теперь мы можем подставить это во второе выражение и получается $54n+n \equiv 55n \equiv 0 \Mod{11}$

  \section{В некотором языке программирования имеется тип \texttt{Int}, содержащий все целые числа в отрезке $[-M; M-1]$, где $M$ -- большое натуральное число. Если целое число $x$, возникшее в результате вычислений, оказывается вне этого отрезка (т. е. происходит переполнение), то $x$ представляется в \texttt{Int} некоторым числом $I(x) \in [-M; M-1]$. При этом для любого $x \in \mathbb{Z}$ выполнятся равенство: $ I(x) = остаток(x+M, 2M) - M $. Докажите, что для любых целых $x$ и $y$ верно}
  \begin{center}
    \begin{varwidth}{\textwidth}
      \begin{enumerate}
        \item если $x \in [-M; M-1]$, то $I(x) = x$
        \item $I(x+y) = I(I(x)+I(y))$
        \item $I(xy) = I(I(x) \cdot I(y))$
      \end{enumerate}
    \end{varwidth}
  \end{center}
  \subsection{Докажите, что если $x \in [-M; M-1]$, то $I(x) = x$}
  если $x \in [-M; M-1]$, то $x+M \in [0; 2M-1]$, поэтому $ x+M \operatorname{mod} 2M = x+M $, и $ I(x) = x+M-M = x$
  \subsection{Докажите, что $I(x+y) = I(I(x)+I(y))$ и $I(xy) = I(I(x) \cdot I(y))$}
  если мы заменим в определение $I$ остаток на $\operatorname{mod}$, то у нас получается что $I(x) \equiv x \Mod{2M}$,
  и для каждого класса эквивалентности по модулю $2M$ есть одно уникальное значение $I$,
  потомучто $I$ зависит только от $остаток(x+M, 2M)$, а не от $x$ напрямую,
  и $остаток(x+M, 2M)$ это типа порядковый номер класса эквивалентности $x+M \Mod{2M}$,
  поэтому мы можем всё посчитать в кольце $\mathbb{Z}_{2M}$
  и только потом перевести результат в $[-M; M-1]$, вычев из него $M$

  \section{Число $a > 1$ делится на $2$, но не на $4$. Докажите, что у $a$ поровну положительных \textit{четных} и \textit{нечетных} делителей}
  если число делится на $2$, но не на $4$, то $2$ входит в его prime factorization с первой степенью \\
  тоесть $a = 2 \cdot p_1^{k_1}p_2^{k_2} \cdots p_n^{k_n}$, где $\forall n:p_n>2$ и простые \\
  тогда любой положительный делитель $d = 2^{m_0} \cdot p_1^{m_1}p_2^{m_2} \cdots p_n^{m_n}$, где $\forall n:m_n \in [0;k_n]$ \\
  а $m_0 \in \{0,\,1\}$, тоесть половина делителей будет с $2^0$, а половина с $2^1$ $\blacksquare$

  \section{Допустим, что каждая из цифр $0$, $1$ и $2$ входит ровно $100$ раз в десятичную запись некоторого числа $x$. Никаких других цифр в $x$ нет. Докажите, что $x$ не является точным квадратом}
  тк остаток деления на $9$ это сумма 10ичных цифр, то $x \equiv 100+200 \equiv 3 \Mod{9}$, \\
  но все квадратные числа дают остаток $\in \{0,\, 1,\, 4,\, 7\}$ при деление на $9$ (это мы можем проверить табличкой)
  \[
    \begin{array}{c|cccccccccc}
      \texttt{i} & 0 & 1 & 2 & 3 & 4 & 5 & 6 & 7 & 8 \\
      \hline
      \texttt{i}^2 \mod 9 & 0 & 1 & 4 & 0 & 7 & 7 & 0 & 4 & 1 \\
    \end{array}
  \]
  $x$ даёт остаток $3$ и поэтому не может быть квадратом \\
  $\blacksquare$

  \section{Докажите, что существует бесконечно много простых чисел вида $6k + 5$}
  числа вида $6k + 5$ это такие числа, которые $\equiv 5 \Mod{6}$ \\
  допустим что существует конечное количество этих чисел и зделаем множество $S = \{p_1,\,p_2,\, \dots,\, p_n\}$ \\
  мы можем перемножить все эти числа и подставить как $k$, но сначала надо обозначить $\ds P = \prod_{x \in S} x$ \\
  тогда $6P+5$ небудет делится на ниодно из чисел в $S$,
  но чтобы нам доказать что в его prime factorization есть хотябы одно число вида $6k + 5$,
  нам надо нарисовать таблицу умножения в поле $\mathbb{Z}_6$
  $$\begin{array}{c|cccccc}
    \cdot & 0 & 1 & 2 & 3 & 4 & 5 \\
    \hline
    0 & 0 & 0 & 0 & 0 & 0 & 0 \\
    1 & 0 & 1 & 2 & 3 & 4 & 5 \\
    2 & 0 & 2 & 4 & 0 & 2 & 4 \\
    3 & 0 & 3 & 0 & 3 & 0 & 3 \\
    4 & 0 & 4 & 2 & 0 & 4 & 2 \\
    5 & 0 & 5 & 4 & 3 & 2 & 1 \\
  \end{array}$$
  здесь видно что $5$ мы можем получить только если у нас уже есть $5$, тк $5 \cdot 1 \equiv 5 \Mod{6}$ \\
  получается что в prime factorization любого числа вида $6k + 5$ есть хотябы одно простое число вида $6k + 5$ \\
  в $6P+5$ тоже есть такое число, и мы уже знаем что оно $\not\in S$ $\implies$ $S$ не содержит все простые числа вида $6k + 5$ \\
  $\blacksquare$ (противоречие)

  \section{Пусть число $p>3$ простое. Докажите, что $p^2 - 1 \equiv 0 \Mod{24}$}
  $p^2 - 1 = (p-1)(p+1)$, и у нас получилось три последовательных числа ($p-1$, $p$ и $p+1$) \\
  мы знаем что $p$ неделится на $2$ и на $3$, поэтому и $(p-1)$ и $(p+1)$ делятся на $2$, и хотябы одно из них делится на $4$,
  тоесть $(p-1)(p+1)$ делится на $8$, также хотябы одно из $p-1$, $p$ и $p+1$ делится на $3$, и это не $p$ \\
  получилось что $(p-1)(p+1)$ делится на $24$ \\
  $\blacksquare$

  \section{Докажите, что не существует арифметической прогрессии $\left\{a_k\right\}_{k\in\mathbb{N}}$ (с ненулевой разностью), т. ч. числа $\{a_1,\dots, a_n\}$ попарно взаимно просты для всех $n > 0$}
  число $a_{a_1+1} = a_1 + d \cdot a_1 = a_1 \cdot (1+d)$ значит что оно делится на $a_1$ и неявляется взаимно простым с $a_1$, \\
  тоесть для $n = a_1+1$ это неработает $\implies$ что это неработает "для всех $n > 0$"{} \\
  $\blacksquare$

  \section{Докажите, что дробь $\dfrac{n^2-n+1}{n^2+1}$ несократима при любом целом $n>0$}
  тоесть $\gcd(n^2-n+1,\, n^2+1) = 1$, по алгоритму Евклида поучается что $\gcd(a,\, b) = \gcd(b \mod a, a)$ \\
  получается $\gcd(n^2-n+1,\, n^2+1) = \gcd(n^2+1,\, n^2-n+1) = \gcd(-n,\, n^2+1) = \gcd(1,\, -n) = 1$ \\
  $\blacksquare$

  \section{Один из вариантов криптоалгоритма RSA таков}
  \[ p \text{ и } q \text{ простые} \]
  \[ n = pq \]
  \[ m = (p-1)(q-1) \]
  \[ e \in \{2,\dots,m-2\} \qquad \gcd(e,m) = 1 \]
  \[ ed \equiv 1 \Mod{m} \]
  \[ (e,\, n) \text{ это публичный ключ} \]
  \[ (d,\, n) \text{ это серкетный ключ} \]

  \subsection{Докажите, что по данным $e$ и $m$ всегда можно найти число $d \in \{2,\dots,m-2\}$, причем для этого есть алгоритм, лучший полного перебора}
  поскольку $ \gcd(e,m) = 1 $ мы можем использовать $ e^{\varphi(m)} \equiv 1 \Mod{m} $, а сложность $\varphi(n)$ это $\bigO(\sqrt{n})$ \\
  получается что $d \equiv e^{\varphi(m)-1} \Mod{m}$, потомучто $y \cdot y^{x-1} = y^x$, а сложность $x^n$ это $\bigO(\log_2{n})$ \\
  сложность всего алгоритма получается $\bigO(\sqrt{m})$ что меньше чем $\bigO(m)$ (это сложность перебора) \\
  $\blacksquare$

  \subsection{Докажите, что расшифровка корректна, тоесть $P' = P$ для \textit{любого} открытого текста $P$}
  \begin{quote}
    Всякий, зная публичный ключ, может зашифровать некоторое сообщение (отрытый текст представляют в виде числа $P \in \{1,\dots,n-1\}$),
    получая шифротекст $C = P^e \mod n$. Адресат сообщения, знающий секретный ключ, расшифровывает открытый текст $P' = C^d \mod n$.
  \end{quote}
  мы можем убрать $\mod$ в $C$, потомучто $x^y \mod n = (x \mod n)^y \mod n$, тогда $P' = (P^e)^d \mod n = P^{ed} \mod n$ \\
  теперь нам надо доказать что $P^{ed} \equiv P \Mod{n}$, но если мы докажем это для $P^{ed} \equiv P \Mod{p}$ и $P^{ed} \equiv P \Mod{q}$,
  то мы сможем получить версию с $n$ по китайской теоремы об остатках, и для $q$ нам доказать будет ненадо, тк у нас тут всё симметрично
  \subsubsection{Докажите, что $P^{ed} \equiv P \Mod{p}$}
  заметим что, тк $ed \equiv 1 \Mod{m}$, $ed$ можно записать как $k(p-1)(q-1)+1$ \\
  сначала рассмотрим случай где $P \equiv 0 \Mod{p}$, чтобы можно было пользоваться теоремой Ферма \\
  здесь всё очевидно тк $0^x = 0$, идём дальше (теперь $P \not\equiv 0 \Mod{p}$) \\
  $ P^{ed} \equiv P^{k(p-1)(q-1)+1} \equiv P\left(P^{p-1}\right)^{k(q-1)} \equiv P \cdot 1^{k(q-1)} \equiv P \Mod{p} $ \\
  $\blacksquare$
  % тоесть $P^{ed} = P \cdot (P^m)^k$ и если $\gcd(P,\, n) = 1$, то по теореме Эйлера $P^m \equiv 1 \Mod{n}$ и $P'=P$ \\
  % осталось только рассмотреть случай когда $P=p$ (или $q$ тут всё симметрично)

  \section{Докажите, что если число $a^{10} + b^{10} + c^{10} + d^{10} + e^{10} + f^{10} \equiv 0 \Mod{11}$, то $abcdef$ делится на $11^6$}
  $\varphi(11) = 10$, поэтому если $x$ делется на $11$, то $x^{10} \equiv 0 \Mod{11}$, а если нет то $x^{10} \equiv 1 \Mod{11}$ \\
  чтобы $a^{10} + b^{10} + c^{10} + d^{10} + e^{10} + f^{10} \equiv 0 \Mod{11}$ надо чтобы все $a^{10}$ делились на $11$,
  потомучто их остаток неможет быть больше $1$, тоесть все $a$ делятся на $11$ $\implies$ $abcdef$ делится на $11^6$ \\
  $\blacksquare$

  \section{Решите уравнение $19x + 22y = -21$ в целых числах}
  уравнения вида $ax+by=c$ имеют решения тогда и только тогда когда $c \mod \gcd(a,\, b) = 0$ \\
  у нас $\gcd(a,\, b) = 1$ такчто решения есть и их бесконечно много, $\gcd$ мы нашли расширенным алгоритмом Евклида
  $$\begin{array}{c|c|c|c|c}
    \texttt{a} & \texttt{b} & \texttt{b//a} & \texttt{x} & \texttt{y} \\
    \hline
    19 & 22 & 1 &  7 & -6 \\
     3 & 19 & 6 & -6 &  1 \\
     1 &  3 & 1 &  1 &  0 \\
     0 &  1 & \infty &  0 &  1 \\
  \end{array}$$
  из этой таблички мы получаем что $19 \cdot 7 + 22 \cdot (-6) = 1$,
  если мы домножим его на $-21$ то будет $19 \cdot (-7 \cdot 21) + 22 \cdot (6 \cdot 21) = -21$ \\
  мы нашли одно решение нашего уравнения $x,\,y = -147,\,126$ \\
  если у нас есть одно решение $ax_1+by_1 = c$, то
  $$\begin{cases}
    x = x_1 - n\frac{b}{\gcd(a,b)} \\
    y = y_1 + n\frac{a}{\gcd(a,b)}
  \end{cases}$$
  поэтому у нас получается
  $\begin{cases}
    x = -22n - 147 \\
    y = 19n + 126
  \end{cases}$
  и это можно упростить и получается ответ
  $\begin{cases}
    x = -22n + 7 \\
    y = 19n - 7
  \end{cases}$

  \section{Сколько различных решений у сравнения $39x \equiv 104 \Mod{221}$}
  мы можем перебрать все числа которые "делятся"{} на $39$ в $\mathbb{Z}_{221}$
  \[
    \begin{array}{c|cccccccccccccccccc}
      \texttt{i} & 0 & 1 & 2 & 3 & 4 & 5 & 6 & 7 & 8 & 9 & 10 & 11 & 12 & 13 & 14 & 15 & 16 & 17 \\
      \hline
      39\texttt{i} \mod 221 & 0 & 39 & 78 & 117 & 156 & 195 & 13 & 52 & 91 & 130 & 169 & 208 & 26 & 65 & \mathbf{\orange{104}} & 143 & 182 & 0 \\
    \end{array}
  \]
  получилось что у нас существуют решения, тк в табличке есть $104$, и что этих решений $\floor{\dfrac{221}{17}}$, тк $17$ у нас период
  ответ \fbox{$13$}

  \section{Применяя теорему Эйлера и избегая использовать алгоритм Евклида, решите сравнение $24x \equiv 17 \Mod{77}$}
  тк $24$ и $77$ взаимно простые, то мы можем упростить и получить $x \equiv 17 \cdot 24^{\varphi(77)-1} \Mod{77}$, а $\varphi(77)=60$ \\
  чтобы возвести $24$ в $59$ нам сначала надо перевести $59$ в 2ичную систему, получается $59=\texttt{0b111011}$
  $$\begin{array}{c|c|c|c}
    \texttt{i} & 24^{2^\texttt{i}} & \texttt{bit[i]} & \texttt{t} \\
    \hline
    0 & 24 & 1 & 24 \\
    1 & 37 & 1 & 41 \\
    2 & 60 & 0 & 41 \\
    3 & 58 & 1 & 68 \\
    4 & 53 & 1 & 62 \\
    5 & 37 & 1 & 61 \\
  \end{array}$$
  получилось $24^{59} \equiv 61 \Mod{77}$ и $x \equiv 17 \cdot 61 \equiv 36 \Mod{77}$ \\
  ответ \fbox{$36$}

  \section{Решите систему сравнений}
  \[
    \begin{cases}
      x \equiv -14 & \Mod{12} \\
      x \equiv 6 & \Mod{11} \\
      x \equiv 19 & \Mod{5} \\
    \end{cases}
  \]
  мы можем упростить эту систему
  \[
    \begin{cases}
      x \equiv 10 & \Mod{12} \\
      x \equiv 6 & \Mod{11} \\
      x \equiv 4 & \Mod{5} \\
    \end{cases}
  \]
  теперь используем китайскую теорему об остатках \\
  $M = 12 \cdot 11 \cdot 5 = 660$ (в таблице будет $\ds M_i^{-1} = \frac{1}{M_i} \mod a_i$) \\
  потом мы сможем найти $x$ по формуле $\ds x = \sum_{i=1}^n r_i M_i M_i^{-1} \mod M$
  \[
    \begin{array}{c|c|c|c|c|c}
      i & r_i & a_i & M_i & M_i^{-1} & r_i M_i M_i^{-1} \\
      \hline
      1 & 10 & 12 & 55 & 7 & 3850 \\
      2 & 6 & 11 & 60 & 9 & 3240 \\
      3 & 4 & 5 & 132 & 3 & 1584 \\
    \end{array}
  \]
  $x = (3850+3240+1584) \mod M = 8674 \mod M = 94$ \\
  ответ \fbox{$94$}

  \section{Найдите $\gcd(3^{168}-1,\, 3^{140}-1)$}
  есть теорема $ \gcd(n^x-1,\, n^y-1) = n^{\gcd(x,\, y)}-1 $,
  зная эту теорему можно всё решить в одну строчку, \\
  вот так $\gcd(3^{168}-1,\, 3^{140}-1) = 3^{\gcd(168,\, 140)}-1 = 3^{168-140}-1 = 3^{28}-1$ \\
  но я незнаю как её доказать и даже незнаю как она называется, \texttt{>:[} \\
  поэтому я воспользуюсь алгоритмом Евклида с вычитанием, тоесть $\gcd(a,\, b) = \gcd(a-bk,\, b)$ если $a-bk > 0$ \\
  если мы попробуем из $3^{168}-1$ вычесть $(3^{140}-1) \cdot 3^{28}$ то будет $3^{168}-1 - 3^{168}+3^{28} = 3^{28}-1$ \\
  потом мы проделаем тотже манёвр с $3^{140}-1 - 3^{112} \cdot (3^{28}-1) = 3^{140}-1 - 3^{140}+3^{112} = 3^{112}-1$ \\
  здесть мы замечаем что степень каждый раз уменьшается на $28$ и, потомучто $140 = 5 \cdot 28$, у нас получится $0$ \\
  $\gcd(3^{168}-1,\, 3^{140}-1) = \gcd(3^{28}-1,\, 3^{140}-1) = \gcd(3^{28}-1,\, 0) = 3^{28}-1$

  \pagebreak
  \section{Найдите остаток от деления $\underbrace{ 3^{3^{3^{\iddots^{3}}}} }_\text{\normalfont 2020 копий 3}$ на 46}
  $$ f(n) = \underbrace{ 3^{3^{3^{\iddots^{3}}}} }_{n\textnormal{ копий 3}} $$
  нам надо найти $f(2020) \mod m$, но
  сначала нам надо найти значения $\varphi$ функции Эйлера для 46 и $\varphi(46)$ и $\dots$ \\
  \begin{tasks}[label=](5)
    \task $\varphi(46) = 22$
    \task $\varphi(22) = 10$
    \task $\varphi(10) = 4$
    \task $\varphi(4) = 2$
    \task $\varphi(2) = 1$
  \end{tasks}
  все эти числа не делятся на 3, поэтому мы можем использовать формулу $a^k \mod m = a^{k \mod \varphi(m)} \mod m$ \\
  % \footnotetext{\url{https://stackoverflow.com/questions/30713648/how-to-compute-ab-mod-m}}
  \scalebox{2}{
    \parbox{0.4\linewidth}{
      \begin{align*}
        f(2020) \mod 46 &= 3^{f(2019) \mod 22} \mod 46 \\
        &= 3^{3^{f(2018) \mod 10} \mod 22} \mod 46 \\
        &= 3^{3^{3^{f(2017) \mod 4} \mod 10} \mod 22} \mod 46 \\
        &= 3^{3^{3^{3^{f(2016) \mod 2} \mod 4} \mod 10} \mod 22} \mod 46 \\
        &= 3^{3^{3^{3^{3^{f(2015) \mod 1} \mod 2} \mod 4} \mod 10} \mod 22} \mod 46 \\
        &= 3^{3^{3^{3^{3^0 \mod 2} \mod 4} \mod 10} \mod 22} \mod 46 \\
        &= 3^{3^{3^{3^1 \mod 4} \mod 10} \mod 22} \mod 46 \\
        &= 3^{3^{3^3 \mod 10} \mod 22} \mod 46 \\
        &= 3^{3^7 \mod 22} \mod 46 \\
        &= 3^9 \mod 46 \\
        &= 41 \\
      \end{align*}
    }
  }
  % в отличие от $a^k \mod m = a^{k \mod \varphi(m)} \mod m$, которое неработает например для $2^41 \mod 100$, \\
  % $a^k \mod m = a^{k \mod \varphi(m) + \varphi(m)} \mod m$ работает всегда (даже если $m$ и $a$ не взаимно просты)


\end{document}
