\documentclass[aspectratio=169]{beamer}
\beamertemplatenavigationsymbolsempty
\usetheme{Boadilla}
\usepackage{textpos} % package for the positioning
\usepackage{enumitem}
\usepackage{amsmath,amsfonts,amssymb}

\usepackage{listings}
\usepackage{xcolor}
\usepackage{tasks}
\usepackage{braket}

% Define color values for the theme
\definecolor{backcolour}{rgb}{0.97, 0.97, 1.0}    % Light purple background
\definecolor{codegreen}{rgb}{0.0, 0.6, 0.0}       % Green for comments
\definecolor{codegray}{rgb}{0.5, 0.5, 0.5}        % Gray for line numbers
\definecolor{codepurple}{rgb}{0.58, 0.0, 0.83}    % Purple for strings
\definecolor{keywordcolor}{rgb}{0.5, 0.0, 0.5}    % Dark purple for keywords

% Define the style for the listings
\lstdefinestyle{mystyle}{
    backgroundcolor=\color{backcolour},
    commentstyle=\color{codegreen},
    keywordstyle=\color{keywordcolor},
    numberstyle=\tiny\color{codegray},
    stringstyle=\color{codepurple},
    basicstyle=\ttfamily\footnotesize,
    breakatwhitespace=false,
    breaklines=true,
    captionpos=b,
    keepspaces=true,
    numbers=left,
    numbersep=5pt,
    showspaces=false,
    showstringspaces=false,
    showtabs=false,
    tabsize=2
}

\lstset{style=mystyle}

\definecolor{uwopurple}{RGB}{79,38,131} % official purple color for uwo


\title{Foundations of Quantum Computing}
\author[]{Zyad Altahan (50238152) \and \AA{AAAAAAAAAA AAAAAAA}{6} (\AA{AAAAAAAA}{10}) \and Tobias Mette (3105540) \and Aksa Aksa (50146305) \and Mohamed Sonbol (50262724)}
\institute[]{Department of Computer Science \\ University of Bonn}
\date{\today}

\usepackage[listings,many]{tcolorbox}
\makeatletter
\newtcblisting{mylisting}{
  listing only,
  breakable, enhanced jigsaw,
  listing engine=listings,
  colback=gray!20,
  listing options={
    language=python,
    keywordstyle=\color{blue},
    basicstyle=\ttfamily,
    stringstyle=\color{ForestGreen},
    commentstyle=\color{gray},
    ndkeywordstyle={\color{orange}},
    identifierstyle=\color{black},
    numbers=none,
    showstringspaces=false,
    aboveskip={0\p@ \@plus 6\p@}, belowskip={0\p@ \@plus 6\p@},
    columns=fullflexible, keepspaces=true,
    breaklines=true, breakatwhitespace=true,
    extendedchars=false,
    inputencoding=utf8,
    upquote=true,
    xleftmargin=-25pt,
  }
}
\makeatother

% set color
\setbeamercolor{title in head/foot}{bg=white}
\setbeamercolor{author in head/foot}{bg=white}
\setbeamercolor{date in head/foot}{fg=uwopurple}
\setbeamercolor{date in head/foot}{bg=white}
\setbeamercolor{title}{fg=uwopurple}
\setbeamerfont{title}{series=\bfseries}
\setbeamercolor{frametitle}{fg=uwopurple}
\setbeamerfont{frametitle}{series=\bfseries}
\setbeamercolor{block title}{bg=uwopurple!30,fg=black}
\setbeamercolor{item}{fg=uwopurple}
\setbeamercolor{caption name}{fg=uwopurple!70!}
\usefonttheme[onlymath]{serif}


\begin{document}


\begin{frame}
    \titlepage
    \begin{textblock*}{8cm}(5.0cm,-7.0cm)
        {\large \color{uwopurple}\hspace{0.66cm} \textbf{Exercise Sheet 04}} % Change the lecture # right here!
    \end{textblock*}
\end{frame}


\begin{frame}[fragile]{Task 4.2}

{\footnotesize
Given:
\[
i^x = (e^{\ln i})^x = e^{x \ln i}, \quad x \in \mathbb{R}.
\]
We need to compute \( \ln(i) \). The natural logarithm of a complex number \( z \) can be calculated using the formula:
\[
\ln(z) = \ln|z| + i\arg(z).
\]

Here, \( z = i \).
\[
\ln(i) = \ln|i| + i\arg(i).
\]

\[
|i| = \sqrt{\text{Re}(i)^2 + \text{Im}(i)^2} = \sqrt{0^2 + 1^2} = 1.
\]

\[
\arg(i) = \tan^{-1}\left(\frac{\text{Im}(i)}{\text{Re}(i)}\right) = \tan^{-1}\left(\frac{1}{0}\right) = \frac{\pi}{2}.
\hfill \text{(because \( i \) points straight up on the complex plane.)}
\]

Putting these in the above equation, we have:
\[
\ln(i) = \ln(1) + i\frac{\pi}{2}.
\]
Since \( \ln(1) = 0 \), this simplifies to:
\[
\ln(i) = i\frac{\pi}{2}.
\]

Finally, substituting this back into the expression for \( i^x \):
\[
i^x = e^{x \ln i} = e^{x \cdot i\frac{\pi}{2}}.
\]
Finally, substituting this back into the expression for \( i^x \), we find:
\[
i^x = e^{x \ln i} = e^{x \cdot i\frac{\pi}{2}}.
\]
}
\end{frame}

\begin{frame}[fragile]{Task 4.3: Even More Cyclic Groups}

{\footnotesize
\textbf{Part (a): Is \( S \) Unitary?}

To check if \( S \) is unitary, we calculate \( S^\dagger S \), where \( S^\dagger \) is the Hermitian conjugate of \( S \). Given:
\[
S = \begin{bmatrix}
1 & 0 \\
0 & i
\end{bmatrix}, \quad
S^\dagger = \begin{bmatrix}
1 & 0 \\
0 & -i
\end{bmatrix}.
\]

Now, compute \( S^\dagger S \):
\[
S^\dagger S = \begin{bmatrix}
1 & 0 \\
0 & -i
\end{bmatrix}
\begin{bmatrix}
1 & 0 \\
0 & i
\end{bmatrix} = \begin{bmatrix}
1 & 0 \\
0 & (-i)(i)
\end{bmatrix} = \begin{bmatrix}
1 & 0 \\
0 & 1
\end{bmatrix}.
\]

Since \( S^\dagger S = I \), \( S \) is unitary.

\vspace{0.5em}

\textbf{Part (b): Compute \( S^k \lvert 1 \rangle \) for \( k \in \{0, 1, 2, 3, 4\} \)}

Given \( \lvert 1 \rangle = \begin{bmatrix} 0 \\ 1 \end{bmatrix} \), we compute \( S^k \lvert 1 \rangle \). Since \( S \) is diagonal:
\[
S^k = \begin{bmatrix}
1 & 0 \\
0 & i^k
\end{bmatrix}.
\]

At this point, we can observe that \( \lvert 1 \rangle \) will select the second column of \( S^k \), specifically the term \( i^k \) (a fourth root of unity).

}
\end{frame}

\begin{frame}[fragile]{Task 4.3: Continue}

{\footnotesize

For each \( k \), we calculate:
\[
S^k \lvert 1 \rangle = \begin{bmatrix}
1 & 0 \\
0 & i^k
\end{bmatrix} \begin{bmatrix} 0 \\ 1 \end{bmatrix} = \begin{bmatrix}
0 \\
i^k
\end{bmatrix}.
\]

The results are:
\[
\begin{aligned}
S^0 \lvert 1 \rangle &= \begin{bmatrix} 0 \\ 1 \end{bmatrix}, \quad
S^1 \lvert 1 \rangle = \begin{bmatrix} 0 \\ i \end{bmatrix}, \quad
S^2 \lvert 1 \rangle = \begin{bmatrix} 0 \\ -1 \end{bmatrix}, \\
S^3 \lvert 1 \rangle &= \begin{bmatrix} 0 \\ -i \end{bmatrix}, \quad
S^4 \lvert 1 \rangle = \begin{bmatrix} 0 \\ 1 \end{bmatrix}.
\end{aligned}
\]

\textbf{Observation:}
The action of \( S^k \) on \( \lvert 1 \rangle \) reveals a cyclic pattern tied to \( i^k \), which cycles through the fourth roots of unity: \( i^0 = 1 \), \( i^1 = i \), \( i^2 = -1 \), and \( i^3 = -i \). Each power of \( S \) applies a successive \( 90^\circ \) rotation in the complex plane, repeating every 4 steps. This cyclic behavior reflects the periodicity of the fourth roots of unity.
}

\end{frame}


\begin{frame}[fragile]{Task 4.4: Complex numbers as matrices}
a) We were asked to compute the values of the following expressions.
a) We were asked to compute the values of the following expressions.

\[
\begin{array}{llll}
c_1 + c_2 = 5+2i & c_1 . c_2 = 14+2i &
c_1^* = 3-4i & |c_1| = 5
\end{array}
\]

b) We were asked to consider them as matrices and compute the values of the following expressions.
\[
\begin{array}{ll}
c_1 = \begin{bmatrix}3 & -4 \\ 4 & 3\end{bmatrix} & c_2 = \begin{bmatrix}2 & 2 \\ -2 & 2\end{bmatrix}
\end{array}
\]
\[
\begin{array}{llll}
c_1 + c_2 = \begin{bmatrix}5 & -2 \\ 2 & 5\end{bmatrix} & c_1 . c_2 = \begin{bmatrix}14 & -2 \\ 2 & 14\end{bmatrix} &
c_1^\top = \begin{bmatrix}3 & 4 \\ -4 & 3\end{bmatrix} & \sqrt{\det c_1} = 5
\end{array}
\]
\end{frame}

% \begin{frame}[fragile]{Task 4.5}
% task 4.3
% \end{frame}


\begin{frame}[fragile]{Task 4.6: Quaternions as matrices}
 a) We were asked to construct a set of 4x4 real valued matrices $\textbf{1}$, $\textbf{i}$, $\textbf{j}$, and $\textbf{k}$ which represent the unit quaternions $1$, $i$, $j$, and $k$.

 \[
\begin{array}{llll}
1 =
\begin{bmatrix}
1 & 0 & 0 & 0 \\
0 & 1 & 0 & 0 \\
0 & 0 & 1 & 0 \\
0 & 0 & 0 & 1
\end{bmatrix} &
i =
\begin{bmatrix}
0 & -1 & 0 & 0 \\
1 & 0 & 0 & 0 \\
0 & 0 & 0 & -1 \\
0 & 0 & 1 & 0
\end{bmatrix} &
j =
\begin{bmatrix}
0 & 0 & -1 & 0 \\
0 & 0 & 0 & 1 \\
1 & 0 & 0 & 0 \\
0 & -1 & 0 & 0
\end{bmatrix} &
k = \begin{bmatrix}
0 & 0 & 0 & -1 \\
0 & 0 & -1 & 0 \\
0 & 1 & 0 & 0 \\
1 & 0 & 0 & 0
\end{bmatrix}
\end{array}
\]

b) We were asked to construct a set of 2x2 real valued matrices $\textbf{1}$, $\textbf{i}$, $\textbf{j}$, and $\textbf{k}$ which represent the unit quaternions $1$, $i$, $j$, and $k$.

 \[
\begin{array}{llll}
1 =
\begin{bmatrix}
1 & 0 \\
0 & 1
\end{bmatrix} &
i =
\begin{bmatrix}
i & 0 \\
0 & -i
\end{bmatrix} &
j =
\begin{bmatrix}
0 & 1 \\
-1 & 0
\end{bmatrix} &
k = \begin{bmatrix}
0 & i \\
i & 0
\end{bmatrix}
\end{array}
\]
\end{frame}

\begin{frame}[fragile]{Task 4.6: cont.}
c) We were asked to fill the table from task 4.5.
\[
\begin{array}{|c|c c c c c c c c|}
\hline
\cdot & 1 & -1 & i & -i & j & -j & k & -k \\
\hline
1 & 1 & -1 & i & -i & j & -j & k & -k \\

-1 & -1 & 1 & -i & i & -j & j & -k & k \\

i & i & -i & -1 & 1 & k & -k & -j & j \\

-i & -i & i & 1 & -1 & -k & k & j & -j \\

j & j & -j & -k & k & -1 & 1 & i & -i \\

-j & -j & j & k & -k & 1 & -1 & -i & i \\

k & k & -k & j & -j & -i & i & -1 & 1 \\

-k & -k & k & -j & j & i & -i & 1 & -1 \\
\hline
\end{array}
\]
\end{frame}

\begin{frame}[fragile]{Task 4.7 Vector Logic}
\textbf{a)} By substituting the values of $n =  \begin{bmatrix} 1 \\ 0 \end{bmatrix}$   and $s = \begin{bmatrix} 0 \\ 1 \end{bmatrix}$  in the C and D we get:\\
$C = \begin{bmatrix} 1 & 1 & 1& 0\\ 0& 0& 0& 1 \end{bmatrix}$,
$D = \begin{bmatrix} 1 & 0 & 0& 0\\ 0& 1& 1& 1 \end{bmatrix}$

% \newline

\textbf{b)}\\ By calculating the truth tables for $x_1, x_2$, and then feeding the result of $x_1 \otimes x_2$ into C and D we get:\\

\begin{tabular}{ |c|c|c|c| }
\hline
$x_1$ & $x_2$ & $C[x_1 \otimes x_2]$ & $D[x_1 \otimes x_2]$ \\
\hline

n & n & n & n \\
n & s & n & s \\
s & n & n & s \\
s & s & s & s \\
\hline
\end{tabular}
\newline
\\By Observing the result, we can clearly see that the $f_C(x1, x2)$ is the Boolean AND function, and $f_D(x1, x2)$ is the Boolean OR function.\\
\end{frame}

\begin{frame}{Task 4.7 contd. }
\textbf{c)} By recalculating for new n = $\begin{bmatrix} 1 \\ 0 \\ 0 \end{bmatrix}$   and $s = \begin{bmatrix} 0 \\0 \\ 1 \end{bmatrix}$
We get different C and D.\\
$C = \begin{bmatrix} 1 & 0 & 1& 0 & 0& 0& 1& 0& 0\\ 0& 0& 0& 0& 0& 0& 0& 0& 0\\ 0& 0& 0& 0& 0& 0& 0& 0& 1 \end{bmatrix}$,
$D = \begin{bmatrix} 1 & 0 & 0& 0 & 0& 0& 0& 0& 0\\ 0& 0& 0& 0& 0& 0& 0& 0& 0\\ 0& 0& 1& 0& 0& 0& 1& 0& 1 \end{bmatrix}$
\newline
\\ By redoing the calculation step b with the new results, we find the same truth table that we found in b), and the functions are still Boolean AND and OR.\\

Additionally, we can see from recalculating the different values of n and s, that as long as the values of n and s are orthonormal. Then, this method of calculating the AND and OR Boolean functions holds, otherwise, it is not guaranteed.
\end{frame}


% \begin{frame}[fragile]{Task 4.8}
% task 4.8
% \end{frame}

\begin{frame}[fragile]{Task 4.9: More vector logic}
% \[
% \begin{array}{ll}
% I = nn^T + ss^T = \begin{bmatrix} 1 & 0 \\ 0 & 1 \end{bmatrix} &
% N = ns^T + sn^T = \begin{bmatrix} 0 & 1 \\ 1 & 0 \end{bmatrix} \\
% \\
% F = nn^T + ns^T = \begin{bmatrix} 1 & 1 \\ 0 & 0 \end{bmatrix} &
% T = sn^T + ss^T = \begin{bmatrix} 0 & 0 \\ 1 & 1 \end{bmatrix}
% \end{array}
% \]
a) Given matrices $I$, $N$, $F$, and $T$ we were asked to consider a vector $x \in [n,s]$
and feed it into the following functions state which Boolean functions they actually compute.
\[
\begin{array}{ll}
f_0(x) = Fx & f_1(x) = Ix \\
f_2(x) = Nx & f_3(x) = Tx
\end{array}
\]

\[
\begin{array}{llll}
f_0(n) = \begin{bmatrix} 1 \\ 0 \end{bmatrix} & f_0(s) = \begin{bmatrix} 1 \\ 0 \end{bmatrix} &
f_1(n) = \begin{bmatrix} 1 \\ 0 \end{bmatrix} & f_1(s) = \begin{bmatrix} 0 \\ 1 \end{bmatrix} \\
f_2(n) = \begin{bmatrix} 0 \\ 1 \end{bmatrix} & f_2(s) = \begin{bmatrix} 1 \\ 0 \end{bmatrix} &
f_3(n) = \begin{bmatrix} 0 \\ 1 \end{bmatrix} & f_3(s) = \begin{bmatrix} 0 \\ 1 \end{bmatrix} \\
\end{array}
\]
From our results we can deduce that $f_0(x)$ is the FALSE function, $f_1(x)$ is the Identity function, $f_2(x)$ is the NOT function, and $f_3(x)$ is the TRUE function.
\end{frame}

\begin{frame}[fragile]{Task 4.9: cont.}
b) Given the rules in task 4.8, we were asked to show that matrices C and D from task 4.7 can also be computed as:

\[
\begin{array}{ll}
C = F \otimes n^\top + I \otimes s^\top & D = I \otimes n^\top + T \otimes s^\top
\end{array}
\]

For \(C\):
\begin{align*}
   C = n[n \otimes n]^\top + n[n \otimes s]^\top + s[s \otimes n]^\top + s[s \otimes s]^\top \\
   C = nn^\top \otimes n^\top + ns^\top \otimes s^\top + sn^\top \otimes n^\top + ss^\top \otimes s^\top \\
   C = [nn^\top + ns^\top] \otimes n^\top + [nn^\top + ss^\top] \otimes s^\top = F \otimes n^\top + I \otimes s^\top
\end{align*}

For \(D\):
\begin{align*}
   D = n[n \otimes n]^\top + s[n \otimes s]^\top + s[s \otimes n]^\top + s[s \otimes s]^\top \\
   D = nn^\top \otimes n^\top + sn^\top \otimes s^\top + ss^\top \otimes n^\top + ss^\top \otimes s^\top \\
   D = [nn^\top + ss^\top] \otimes n^\top + [sn^\top + ss^\top] \otimes s^\top = I \otimes n^\top + T \otimes s^\top
\end{align*}
\end{frame}

\begin{frame}[fragile]{Task 4.9: cont.}
c) We need to prove the following relationships symbolically or computationally. Since symbolically would take too much sapce i did them computationally.

\[
C = ND [N \otimes N], \quad D = NC [N \otimes N]
\]

\[
\begin{array}{ll}
C = \begin{bmatrix}
1 & 1 & 1 & 0 \\
0 & 0 & 0 & 1
\end{bmatrix} &
D = \begin{bmatrix}
1 & 0 & 0 & 0 \\
0 & 1 & 1 & 1
\end{bmatrix}
\end{array}
\]

\[
\begin{array}{ll}
ND [N \otimes N] = \begin{bmatrix}
1 & 1 & 1 & 0 \\
0 & 0 & 0 & 1
\end{bmatrix} &
NC [N \otimes N] = \begin{bmatrix}
1 & 0 & 0 & 0 \\
0 & 1 & 1 & 1
\end{bmatrix}
\end{array}
\]
\end{frame}


\begin{frame}[fragile]{Task 4.10: First steps towards quantum computing}
Given Matrices M, $\Pi_2$, and $\Pi_2$ we were asked to determine what the following functions do.

\[
\begin{array}{ll}
f_{\Pi_2 M}(x_1, x_2) = \Pi_2 M [x_1 \otimes x_2] &
f_{\Pi_1 M}(x_1, x_2) = \Pi_1 M [x_1 \otimes x_2]
\end{array}
\]

\begin{center}
\begin{tabular}{ c c c c c c }
 \hline
 $x_1$ & $x_2$ & $x_1 \otimes x_2$ & $M[x_1 \otimes x_2]$ & $\Pi_2 M[x_1 \otimes x_2]$ & $\Pi_1 M[x_1 \otimes x_2]$ \\
 \hline
    n & n & \(\begin{bmatrix} 1 & 0 & 0 & 0\end{bmatrix}\) & \(\begin{bmatrix} 1 & 0 & 0 & 0\end{bmatrix}\) & n & n \\
    n & s & \(\begin{bmatrix} 0 & 1 & 0 & 0\end{bmatrix}\) & \(\begin{bmatrix} 0 & 1 & 0 & 0\end{bmatrix}\) & s & n \\
    s & n & \(\begin{bmatrix} 0 & 0 & 1 & 0\end{bmatrix}\) & \(\begin{bmatrix} 0 & 0 & 0 & 1\end{bmatrix}\) & s & s \\
    s & s & \(\begin{bmatrix} 0 & 0 & 0 & 1\end{bmatrix}\) & \(\begin{bmatrix} 0 & 0 & 1 & 0\end{bmatrix}\) & n & s \\
 \hline
\end{tabular}
\end{center}

From the table above we can see that $f_{\Pi_2 M}(x_1, x_2)$ is the XOR function and $f_{\Pi_1 M}(x_1, x_2)$ is just $x_1$.

\end{frame}

\end{document}
