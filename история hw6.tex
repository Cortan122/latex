\documentclass[12pt]{article}

\usepackage{cmap} % поиск в pdf
\usepackage[english,russian]{babel} % локализация и переносы

\usepackage[hidelinks]{hyperref}
\usepackage{xurl}
\usepackage{enumitem}

\usepackage[margin=1in]{geometry}

\usepackage{xcolor}
\newcommand{\red}[1]{{\color{red}{#1}}}
\newcommand{\orange}[1]{{\color{orange}{#1}}}
\newcommand{\brown}[1]{{\color{brown}{#1}}}
\newcommand{\teal}[1]{{\color{teal}{#1}}}

% \pagecolor{black}
% \color{white}

\title{История \\ Домашнее задание №6}
\author{\AA{AAAAA AAAAAAA}{4} \\ AAAAAA}

\begin{document}
  \maketitle

  \section{Противоречия царствования Екатерины II. Почему ей не удалось довести свои реформы до конца?}
  Начнем с того, что политику правления Екатерины II характеризуют термином <<Просвещенный абсолютизм>>.
  Эта политика, которая складывается под влиянием идей Просвещения,
  которые осуждали сословное неравенство, произвол власти,
  засилье церкви в идейной сфере и обосновывали <<естественные права>> человека -- свободу,
  равенство, право на частную собственность, а также принцип разделения властей, верховенство закона и др.

  Сама Екатерина утверждала, что она является приверженцем идей таких философов, как Дидро и Вольтер, которые выступали за Просвещение.
  Так что и сама Екатерина выступала за Просвещение.

  Вот основные и самые противоричивые её реформы:
  \begin{enumerate}
    \item
    Ликвидации остатков прав крепостных крестьян:
    помещикам было разрешено без суда отдавать их на каторжные работы (1765 г.), сажать в тюрьму (1775 г.),
    крепостным крестьянам запретили жаловаться государыне на помещиков, принимать присягу в суде, брать откупа и подряды.

    \item
    Увеличение зоны крепостничества (до Левобережной Украины).

    \item
    Расширении привилегий дворян, которые получили право не служить, если этого не желали (Манифест о вольности дворянства 1762 г.),
    а также монопольное право на владение землей, недрами и крепостными крестьянами (по Жалованной грамоте дворянству 1785 г.).

    \item
    Предоставление в 1785 г. Жалованной грамоты городам обеспечивало некоторыми привилегиями <<городских обывателей>>,
    разделенных на 6 разрядов в зависимости от рода занятий и материального положения, а также регулировало работу органов самоуправления.
    К тому же Екатерина II преследовала цель укрепить позиции <<третьего сословия>> в России,
    надеясь увидеть в нем как силу, обеспечивающую динамичное экономическое развитие страны, так и опору прогрессивных преобразований.
  \end{enumerate}

  Эти реформы показывают, в чем заключаются противоречия.
  Екатерина, являясь сторонником Просвещения, усиливает крепостничество,
  которое ужё является основой дворянского благосостояния,
  a дворянство ужё привилегированный класс.

  Почему так? Она понимает, что на данном этапе отсутствует национальная буржуазия, основные массы дворянства остаются непросвещенными,
  а городского население и крестьянство -- патриархальным.
  Все это говорит ей о том, что проведение основательных либеральных реформ не найдет поддержки
  и приведет лишь к новому дворцовому перевороту и еще большему укреплению позиций дворянства.
  К тому же крестьянское восстание показало что дворяне -- главная опора монархии.
  Поэтому либеральные реформы прекратились, не успев начаться.

  \setcounter{section}{3}
  \section{Роль личности Александра I в реформировании \\ страны <<сверху>>}
  Характер Александа формировался в детстве, когда бабушка отобрала его у матери и определила жить в Царском Селе,
  подле себя, вдалеке от родителей, которые проживали в своих дворцах (в Павловске и Гатчине) и редко появлялись при <<большом дворе>>.
  Впрочем, ребёнок, как это видно из всех отзывов о нём, был мальчиком ласковым и нежным,
  так что возиться с ним для царственной бабушки было огромным удовольствием.
  Юный Александр обладал умом и дарованиями, разделял либеральные идеи, но был ленивым,
  самолюбивым и поверхностным в усвоении знаний, не умея сосредоточиться на длительной и серьёзной работе.
  Его учитель, Фредерик Сезар Лагарп, которого ему нашла бабушка,
  был гражданином Швейцарии, республиканцем, и ознакомил Александра с философией Просвещения.

  Поэтому начало царствования Александра I характеризовалось некоторым стремлением к либеральному реформаторству.
  Однако, эти начинания ни в чём не коснулись устоев государства -- самодержавия и крепостного права.
  В 1803 году он издал указ <<О вольных хлебопашцах>>, который разрешал помещикам отпускать на волю крепостных с наделением их землёй за выкуп.
  Это вызвало недовольство дворян, указ не получил широкого применения,
  хотя правительство признало им принципиальную возможность освобождения крестьян,
  законодательно определило условия этого освобождения и права освобождённых.

  После Отечественной войны 1812 года внутренняя политика Александра I потеряла прежний либеральный налёт.
  По его инициативе создаётся <<Священный союз>>, объединивший европейских монархов для борьбы с революционным движением в Европе%
  \footnote{чтобы не было ещё одного Наполеона}.

  Внутренняя политика Александра I сначала либеральная, затем реакционная,
  направленная на укрепление самодержавия и крепостного права,
  объективно способствовала активизации дворянского революционного движения в России -- декабризма.

  % \vfill
  % \begin{thebibliography}{9}
  %   \bibitem{катя1} \url{https://xn--2-7sbasbsl1azs.xn--p1ai/%D1%80%D0%B5%D1%84%D0%BE%D1%80%D0%BC%D1%8B/}
  %   \bibitem{катя2} \url{https://istoriarusi.ru/imper/reformy-ekaterini-2.html}
  %   \bibitem{катя3} \url{https://ekaterina-ii.ru/vnutrenyaya-politika/reformy-ekateriny/}
  %   \bibitem{катя4} \url{http://www.soloby.ru/398976/%D0%B5%D0%BA%D0%B0%D1%82%D0%B5%D1%80%D0%B8%D0%BD%D0%B5-%D1%86%D0%B0%D1%80%D1%81%D1%82%D0%B2%D0%BE%D0%B2%D0%B0%D0%BD%D0%B8%D1%8F-%D1%81%D0%BE%D1%81%D1%82%D0%B0%D0%B2%D0%B8%D1%82%D1%8C-%D1%81%D1%82%D1%80%D0%B5%D0%BC%D0%B8%D0%BB%D0%B0%D1%81%D1%8C-%D0%BF%D1%80%D0%B5%D0%BE%D0%B1%D1%80%D0%B0%D0%B7%D0%BE%D0%B2%D0%B0%D1%82%D1%8C}
  %   \bibitem{катя5} \url{https://znanija.com/task/28546359}
  %   \bibitem{катя6} \url{https://studopedia.ru/4_76804_preobrazovaniya-ekaterini--i-itogi-ros-oy-modernizatsii-strani.html}
  %   \bibitem{саша1} \url{https://ru.wikipedia.org/wiki/%D0%90%D0%BB%D0%B5%D0%BA%D1%81%D0%B0%D0%BD%D0%B4%D1%80_I}
  %   \bibitem{саша2} \url{https://vuzlit.ru/445930/aleksandr_popytka_reform_sverhu}
  %   \bibitem{саша3} \url{https://ru.wikipedia.org/wiki/%D0%A0%D0%B5%D1%84%D0%BE%D1%80%D0%BC%D1%8B_%D0%90%D0%BB%D0%B5%D0%BA%D1%81%D0%B0%D0%BD%D0%B4%D1%80%D0%B0_I}
  %   \bibitem{саша4} \url{https://studwood.ru/946720/istoriya/rossiya_nachale_aleksandr_popytka_reform_sverhu}
  % \end{thebibliography}

\end{document}
