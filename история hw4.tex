\documentclass[12pt]{article}

\usepackage{cmap} % поиск в pdf
\usepackage[english,russian]{babel} % локализация и переносы

\usepackage[hidelinks]{hyperref}
\usepackage{xurl}
\usepackage{enumitem}

\usepackage[margin=1in]{geometry}

\usepackage[usenames,dvipsnames,svgnames,table,rgb]{xcolor}
\newcommand{\red}[1]{{\color{red}{#1}}}

\title{История \\ Домашнее задание №4}
\author{AAAAA AAAAAAA \\ AAAAAA}

\begin{document}
  \maketitle
  \setcounter{section}{1}

  \section{Понимание Иваном IV Грозным царской власти в контексте русского средневекового мировоззрения}

  Основные представления о царской власти в России, изложенные Иваном Васильевичем, сводились к следующему.
  Царская власть являлась наследственной, и передавалась по мужской линии от отца к его первому сыну.
  Царская власть российского монарха воспринималась как не зависимая от человеческой воли, и имеющая божественное происхождение.
  % Рожденная по воле Божией, которая являлась ее единственным неземным источником,
  % Отсюда происходили присущие только ей \red{четыре} свойства.

  В своих посланиях Курбскому Иван IV Грозный формулирует и развивает основные принципы власти русских государей, из которых выделяются:

  \begin{enumerate}[label=\textbf{\large\arabic*}]
    \item {\bf\large Самодержавие} \\
    Важнейшее место среди них отводилось самодержавию, которое означало,
    что царская власть была неделима и не делегируема, ни от кого не зависела и принадлежала лишь царствующему монарху,
    безраздельному обладателю государственной территории,
    по праву исторически сложившегося династического наследования -- по праву <<отчины и дедины>>.

    \item {\bf\large Юридическая неограниченность} \\
    Царская власть, по мысли Ивана Васильевича отличалась юридической неограниченностью,
    которая устанавливалась в силу ее божественного происхождения и была полностью сосредоточена в руках правящего монарха,
    чего требовали государственные интересы.
    Мнение Ивана IV на этот счет всегда было бескомпромиссно.

    \item {\bf\large Мессианство} \\
    Мессианство это идея высокой ответственности государя за исполнение своего предназначения --
    просвещения всего мира светом евангельской истины, сохранение чистоты веры и превращение православия во вселенскую религию.
    В России мессианство в основном опиралось на учение <<Москва -- третий Рим>>, которое заключалось в том, что
    Россия это последнее оставшееся православное государство, которое является преемником Византии,
    а Русский царь -- наследником Византийских императоров.
  \end{enumerate}

  Важной частью духовно-политической концепции Ивана Грозного стали методы воплощения теории в реальную политическую практику,
  главным и единственным среди которых им был объявлен <<страх>>, принуждение.
  У  Ивана Грозного этот страх ассоциируется со страхом Божиим, который, согласно Ветхому Завету, -- начало премудрости.
  % Будучи подготовленным всей предшествующей религиозно-политической мыслью к идее высокого призвания правителя Русского государства,
  Иван Грозный выбирает жесткие методы для выполнения своих задач и отказывается от помощи политических советников.

  \section{Причины появления и смыслы опричнины\protect\footnotemark}
  \footnotetext{
    С одной стороны опричнина -- это земля, принадлежащая напрямую царю,
    а с другой стороны это некий <<спецназ>>, совершающий набеги на неугодных.
  }
  Причины появления:
  \begin{enumerate}
    \item Остаточное явления феодальной раздробленности, то есть были некоторые феодальные роды, которые не хотели слушаться царя.
    \item Ухудшение обстановки в стране в связи с Ливонской войной, необходимость мобилизации ресурсов тыла.
    \item Иван IV все более верил в свою богоизбранность, а к населению относился как к холопам, которых он <<волен жаловать или казнить>>.
      (см. предыдущий вопрос)
  \end{enumerate}

  Иван действительно преисполнился идеей божественного характера своей власти.
  В книгe Курукина И.В. <<Повседневная жизнь опричников Ивана Грозного>>,
  в главе <<Символика опричнины>> описываются навевающие ужас параллели <<декора и эстетики>> опричников
  с описанием ангелов смерти в различных средневековых религиозных сочинениях как правило на апокалиптическую тему.
  То есть сопоставляются опричники, архангел Михаил и ангелы,
  мучающие грешников в аду\footnote{по одному из источников это делают не демоны, а именно суровые ангелы, не знающие пощады}.
  Таким образом в том числе по задумке Ивана Грозного одной из функций опричнины является нагнетание страха среди людей%
  \footnote{с этим они замечательно справлялись}.

  Основное содержание опричнины -- насилие, с помощью которого опричники во главе со своим <<игуменом>> надеялись искоренить грех непослушания.
  Социально-полити\-ческий смысл опричнины тесно переплетался с религиозными представлениями людей той эпохи.
  Власть с помощью жестокого террора стремилась компенсировать свою слабость и неэффективность и парализовать волю населения к сопротивлению,
  вселить ужас в души людей, заставить их безропотно ей подчиняться, обеспечить материальные и людские ресурсы для ведения войны.

  В результате опричнина вела к истреблению неугодных власти людей и грабежу в целях пополнения княжеской казны и обогащения худородных дворян.


  % \begin{thebibliography}{9}
  %   \bibitem{2} \url{https://ruskline.ru/analitika/2017/03/06/pravovaya_priroda_nasledstvennoj_carskoj_vlasti_po_vzglyadam_ivana_vasilevicha_groznogo}
  %   \bibitem{3} \url{https://studopedia.ru/3_146000_kontseptsiya-samoderzhavnoy-vlasti-ivana-IV-i-ee-realizatsiya.html}
  %   \bibitem{4} \url{https://studopedia.su/18_171635_kontseptsiya-tsarskoy-vlasti-v-trudah-ivana-groznogo.html}
  %   \bibitem{5} \url{https://studfile.net/preview/6006594/page:11/}
  %   \bibitem{wiki} \url{https://ru.wikipedia.org/wiki/%D0%9C%D0%BE%D1%81%D0%BA%D0%B2%D0%B0_%E2%80%94_%D1%82%D1%80%D0%B5%D1%82%D0%B8%D0%B9_%D0%A0%D0%B8%D0%BC}
  %   \bibitem{dialog} \url{http://lib.pushkinskijdom.ru/Default.aspx?tabid=9106}
  % \end{thebibliography}

\end{document}
