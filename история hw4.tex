\documentclass[12pt]{article}

\usepackage{cmap} % поиск в pdf
\usepackage[english,russian]{babel} % локализация и переносы

\usepackage[hidelinks]{hyperref}
\usepackage{xurl}
\usepackage{enumitem}

\usepackage[margin=1in]{geometry}

\usepackage[usenames,dvipsnames,svgnames,table,rgb]{xcolor}
\newcommand{\red}[1]{{\color{red}{#1}}}

\title{История \\ Домашнее задание №4}
\author{AAAAA AAAAAAA \\ AAAAAA}

\begin{document}
  \maketitle
  \setcounter{section}{1}

  \section{Понимание Иваном IV Грозным царской власти в контексте русского средневекового мировоззрения}

  Основные положения правовой идеи царской власти в России, изложенные Иваном Васильевичем, сводились к следующему.
  Царская власть русского государя являлась наследственной по праву мужского первородства по нисходящей мужской линии от отца к его первому сыну.
  Рожденная по воле Божией, которая являлась ее единственным неземным источником,
  царская власть российского монарха мыслилась как не зависимая от человеческой воли.
  % Отсюда происходили присущие только ей \red{четыре} свойства.

  В своих посланиях Курбскому Иван IV Грозный формулирует и развивает основные принципы власти русских государей, из которых выделяются:

  \begin{enumerate}[label=\textbf{\large\arabic*}]
    \item {\bf\large Самодержавие} \\
    Важнейшее место среди них отводилось самодержавию, которое означало,
    что царская власть была неделима и не делегируема, ни от кого не зависела и принадлежала лишь царствующему монарху,
    безраздельному обладателю государственной территории,
    по праву исторически сложившегося династического наследования -- по праву <<отчины и дедины>>.

    \item {\bf\large Юридическая неограниченность} \\
    Царская власть, по мысли Ивана Васильевича отличалась юридической неограниченностью,
    которая устанавливалась в силу ее Божественного происхождения и была полностью сосредоточена в руках правящего монарха,
    чего требовали государственные интересы.
    Мнение Ивана IV на этот счет всегда было бескомпромиссно.

    \item \red{\bf\large Чтото ещё} \\
  \end{enumerate}

  Важной стороной духовно-политической концепции Ивана Грозного стало понимание методов воплощения теоретических постулатов
  в реальную политическую практику, главным и единственным среди которых им был объявлен <<страх>>, принуждение.
  Будучи подготовленным всей предшествующей религиозно-политической мыслью к идее высокого призвания правителя Русского государства,
  Иван Грозный выбирает жесткие методы для выполнения возложенных на него задач,
  и с тем же намерением гарантировать результат взятых на себя обязательств отказывается от помощи политических советников.


  \begin{thebibliography}{9}
    \bibitem{2} \url{https://ruskline.ru/analitika/2017/03/06/pravovaya_priroda_nasledstvennoj_carskoj_vlasti_po_vzglyadam_ivana_vasilevicha_groznogo}
    \bibitem{3} \url{https://studopedia.ru/3_146000_kontseptsiya-samoderzhavnoy-vlasti-ivana-IV-i-ee-realizatsiya.html}
    \bibitem{4} \url{https://studopedia.su/18_171635_kontseptsiya-tsarskoy-vlasti-v-trudah-ivana-groznogo.html}
    \bibitem{5} \url{https://studfile.net/preview/6006594/page:11/}
    \bibitem{1} \url{http://lib.pushkinskijdom.ru/Default.aspx?tabid=9106}
  \end{thebibliography}

\end{document}
