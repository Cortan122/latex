\documentclass{article}

\usepackage{enumitem}
\usepackage{amsmath}
\usepackage{amsfonts}
\usepackage{xcolor}
\usepackage{amssymb}
\usepackage[margin=0.5in]{geometry}
\usepackage[hidelinks, bookmarks=false]{hyperref}

\let\oldemptyset\emptyset
\let\emptyset\varnothing
\let\oldepsilon\epsilon
\let\epsilon\varepsilon
\newcommand{\N}{\mathbb{N}}
\newcommand{\R}{\mathbb{R}}
\newcommand{\Q}{\mathbb{Q}}

\usepackage{environ}
\NewEnviron{centerframebox}{\begin{center}\fbox{\parbox{0.92\textwidth}{\BODY}}\end{center}}

\title{Combinatorial Optimization \\ Exercise Set 6 \\ Tuesday class}
\author{
  \AA{AAAAAAAAAA AAAAAAA}{6} \\
  \href{mailto:\AA{AAAAAAAAAAAAAAAAAAAA}{7}}{\AA{AAAAAAAAAAAAAAAAAAAA}{7}}
  \and
  Carola Ley \\
  \href{mailto:s6caleyy@uni-bonn.de}{s6caleyy@uni-bonn.de}
  \and
  Bailee Zacovic \\
  \href{mailto:s38bzaco@uni-bonn.de}{s38bzaco@uni-bonn.de}
}

\begin{document}
  \maketitle

  \setcounter{section}{6}
  \subsection{Undirected Minimum Mean-Weight Cycle Algorithm}
  \begin{centerframebox}
    The \textsc{Undirected Minimum Mean-Weight Cycle Problem}
    is the following: Given an undirected graph $G$ with edge-weights $c : E(G) \to \R$,
    find a cycle $C$ whose mean-weight $c(E(C))/|E(C)|$ is minimum, or determine that
    $G$ is acyclic.

    Consider the following algorithm for the \textsc{Undirected Minimum Mean-Weight Cycle Problem}:
    First determine with a linear search whether $G$ has cycles or not, and if not return with this information.
    Let $\gamma := \max\{c(e) : e \in E(G)\}$ and define a new edge-weight function via $c'(e) := c(e) - \gamma$.
    Let $T := \emptyset$.
    Now iterate the following: Find a minimum $c'$-weight $T$-join $J$ with a polynomial (black-box) algorithm.
    If $c'(J) = 0$, return any zero-$c'$-weight cycle.
    Otherwise, let $\gamma':= c'(J)/|J|$, reset $c'$ via $c'(e) \rightarrow c'(e) - \gamma'$, and continue.

    Show that this algorithm works correctly and runs in polynomial time.
    Also, explain how to get the cycle to be returned in the case $c'(J) = 0$.
  \end{centerframebox}
  A subgraph $G[J]$ induced by a $T$-join with $T = \emptyset$ will have all of its vertices have an even degree.
  This also means that $G[J]$ will be Eulerian and decomposable into a set edge-disjoint cycles.
  As the first step of our algorithm we convert all the weights to being negative,
  so all of our cycles can be considered as part of the minimum $T$-join.
  The $T$-join will always have either a negative or zero weight, because picking the empty set is always an option.

  At least one of thr desired minimum mean weight cycles $C$ must be in every one of our $T$-joins,
  because all other cycles have a higher mean weight, and the $T$-joins is minimum and wants to gobble up as many cycles as possible.
  The mean weight of $J$ will then be somewhere between $\frac{c'(C)}{|C|} \leq \frac{c'(J)}{|J|} < \frac{c'(A)}{|A|}$,
  where $A$ is the maximum cycle with negative $c'$-weight.
  Consider two cases: $A = C$ and $A \neq C$. If $A = C$ that means that $C$ is the only cycle in $J$,
  and after the subtracting $\gamma'$ from the weights, it will end up having weight exactly zero.
  In the second case, the inequality becomes strict $\frac{c'(C)}{|C|} < \frac{c'(J)}{|J|} < \frac{c'(A)}{|A|}$,
  and, as a result, the cycle $A$ will end up having a positive mean weight after subtracting $\gamma'$.

  This way at least one cycle is eliminated from being considered by the $T$-join in the next iteration,
  because it will have non-negative weight.
  The only cycles left at the end will be the minimum mean weight ones, and all of them will have exactly zero weight.
  But we also have to prove that it will be done in polynomial time,
  because what we have now will run in $O(|\textrm{number of cycles}| \cdot \textrm{complexity of T-join})$,
  and the number of cycles is exponential.
  On average, about half of the cycles in $J$ will have a mean weight above $\frac{c'(J)}{|J|}$,
  so on each iteration the number of available cycles will be reduced by a factor of two, amortized.
  Thus it will be a $\log_2$ of the number of cycles, which is exponential, so it will be linear.
  % But we also know that on every iteration at least one edge becomes positive,
  % because we cannot have a positive mean weight cycle without at lest one positive edge. (this is not true)

  In our final state, where $c'(J) = 0$, our $T$-join must be either empty, or a union of zero-$c'$-weight cycles,
  because a minimum $T$-join cannot contain a positive weight cycle.
  And to get our cycle, we just have to pick one of those zero weight ones.
  To achieve a zero mean weight, the cycle will have either all zero weight edges, or at least one negative weight edge.
  In the first case we can just find it by a greedy depth first search in the subgraph of zero-$c'$-weight edges.

  In the other case, where we have at least one negative edge,
  we have to slightly modify the $T$-join finding algorithm to bias it towards always selecting the non-empty $T$-join.
  As we recall from Theorem 2.4, the algorithm starts by adding all the negative weight edges,
  but then removes them by finding a perfect matching in the complete graph of shortest paths.
  In our case there will be two equals weight candidates for this shortest path:
  going all the way around the cycle, or just through this one edge.
  We can bias Dijkstra's algorithm to always prefer a path with a greater amount of edges,
  in cases where there are two equal weight alternatives.
  After such a modification our minimum mean weight cycle will always be included in the $T$-join,
  and we can find it with a simple linear search.

  \subsection{Minimum cardinality of a $T$-cut}
  \begin{centerframebox}
    Let $G$ be an undirected graph and $T \subseteq V (G)$ with $|T| = 2k$ even.
    Prove that the minimum cardinality of a $T$-cut in $G$ equals the maximum
    of $\min^k_{i=1} \lambda_{s_i, t_i}$ over all pairings $T = \{s_1,\, t_1,\, \dots,\, s_k,\, t_k\}$,
    where $\lambda_{s,t}$ denotes the maximum number of pairwise edge-disjoint $s$-$t$-paths.
  \end{centerframebox}
  Let $m$ denote the minimum cardinality of a $T$-cut. Fix a Gomory-Hu tree $H$ of $G,$ where each edge $\{s,t\}\in E(H)$ may be labelled by $\lambda_{s,t}$ by Menger's theorem, and let $u:E(G)\rightarrow \mathbb{R}$ be given by $u(e)=1$ for all $e\in E(G).$ It follows from Theorem 2.25 that a minimum capacity (in our case, cardinality) $T$-cut can be found among all the fundamental cuts of $H.$

  % First, we claim that $m\geq \underset{\text{pairings}}{\text{max}}\text{min}_{i=1}^k \lambda_{s_i,t_i}$. If not, then there would exist some pairing $\{s_1,t_1,\dots,s_k,t_k\}$ such that $\lambda_{s_i,t_i}>m$ for all $i=1,\dots,k,$ i.e. at least one $s_i$-$t_i$-path ``survives'' any minimum $T$-cut $\delta(X)$, hence $s_i,t_i$ appear in pairs in the connected components of $G-\delta(X).$ This contradicts $|X\cap T|$ odd.

  Now, since $H$ is a tree (in which $s$-$t$-paths are unique), there exists a unique $T$-join $J\subset E(H)$ given by partitioning $T$ into pairs $(s,t)$ and taking the symmetric difference of the $s$-$t$-paths. Since $J$ is unique and therefore optimum, it follows from Lemma 12.9 in the textbook that $J$ is given by a disjoint union of paths whose endpoints lie in $T$. Thus, $J$ induces a pairing $\{s_1,t_1,\dots ,s_k,t_k\}$ given by taking the endpoints of each path. Moreover, $J$ intersects a minimum cardinality $T$-cut realized by a fundamental cut $\delta(C_{\{s,t\}})$ associated to $\{s,t\}\in E(H).$
  % \textcolor{red}{Did not complete final details.}

  \subsection{Odd and Even joins}
  \begin{centerframebox}
    Given an undirected graph $G$ and disjoint sets $S_e, S_o \subseteq V (G)$,
    a partial $(S_e,\, S_o)$-join is a set $J \subseteq E(G)$ such that $|\delta(v) \cap J|$ is even for every
    $v \in S_e$ and odd for every $v \in S_o$.
    (In particular, a $T$-join is the same as a partial $(V (G) \setminus T,\, T)$-join.)
    Consider the \textsc{Minimum Weight Partial $(S_e,\, S_o)$-Join Problem}:
    Given an undirected graph $G$ with edge-weights $c : E(G) \to \R_{\geq 0}$
    and disjoint sets $S_e,\, S_o \subseteq V (G)$, find a partial $(S_e,\, S_o)$-join of minimum weight, or
    determine that none exists.
    Show that this problem can be linearly reduced to the \textsc{Minimum Weight $T$-Join Problem}.
  \end{centerframebox}
  Given an undirected graph $G$ with edge-weights $c : E(G) \to \R_{\geq 0}$ and disjoint sets $S_e,\, S_o \subseteq V (G)$. Define a graph $G'=(V(G'),E(G'))$ where $V(G')=\{(v,1):v\in V(G)\}\cup \{(v,2):v\in V(G)\setminus S_o\}$ and$$E(G')=\{\{(v,1),(w,1)\}:\{v,w\}\in E(G)\}\cup \{\{(v,1),(v,2)\}:v\in V(G)\setminus S_o\}\cup \{\{(v,2),(v,2)\}:v\in V(G)\setminus (S_e\cup S_o)\}.$$We define a weight function $c':E(G')\rightarrow \mathbb{R}_{\geq 0}$ given by $c'(\{(v,1),(w,1)\})=c(\{v,w\})$ and zero otherwise.
  Set $$T:=\{(v,1):v\in S_e\cup S_o\}\cup \{(v,2):v\in S_e\}.$$We claim that if a minimum weight $T$-join in $G'$ exists, it corresponds to a minimum weight partial $(S_e,S_o)$-join in $G$, otherwise no minimum weight partial $(S_e,S_o)$-join exists in $G.$ (Note that we attach zero-weight loops to all vertices $(v,2)$ for $v\in V(G)\setminus (S_e\cup S_o)$; this is a correction factor of ``dummy edges'' which may be used to ensure such $(v,2)$ are incident to an even number of edges in a minimum $T$-join $J$, in the case that edges of the form $\{(v,1),(v,2)\}$ for $v\in V(G)\setminus (S_e\cup S_o)$ appear in $J.$)

  First, observe that since $\{(v,2):v\in S_e\}\subset T,$ and since each such $(v,2)$ is only adjacent to $(v,1)$, a minimum $T$-join $J$ of $G'$ will always contain the edge $\{(v,1),(v,2)\}$ for each $v\in S_e$. Any remaining edges in $J$ incident to such $(v,1)$ correspond to edges incident to $v\in S_e$ in $G$, and there will be an even number of such edges, by previous remark and $|J\cap \delta_{G'}((v,1))|\equiv 1 \pmod{2}$.
  Additionally, since the only neighbors of $(v,1)$ for $v\in S_o$ are of the form $(w,1)$ by construction of $G'$, any edges in $J$ incident to such $(v,1)$ correspond to those edges in $G$ incident to $v\in S_o$. Thus, restricting any minimum $T$-join of $G'$ to edges of the form $\{(v,1),(w,1)\}$ gives rise to a partial $(S_e,S_o)$-join, which is necessarily minimum in $G$ since $c'$ assigns zero weight to those edges not of the form $\{(v,1),(w,1)\}$. Hence the \textsc{Minimum Weight Partial $(S_e,\, S_o)$-Join Problem} reduces linearly to the \textsc{Minimum Weight $T$-Join Problem}. $\blacksquare$

  %   Define a graph $G'=(V(G'),E(G'))$ where $$V(G')=\{(v,i):v\in V(G), i=1,2\}$$and$$E(G')=\{\{(v,i),(w,i)\}:(v,w)\in E(G)\}\cup \{\{(v,1),(v,2)\}:v\in V(G)\}\cup \{\{(v,2),(v,2)\}:v\in V(G)\setminus S_o\}.$$We define a weight function $c':E(G')\rightarrow \mathbb{R}$ given by $$c'(\{(v,i),(w,i)\})=\begin{cases}c(\{v,w\}) & \text{ for }\{v,w\}\in E(G), \; i=1,\\ c(\{v,w\})+1 & \text{ for }\{v,w\}\in E(G), \; i=2\end{cases}$$
  %  $$c'(\{(v,1),(v,2)\})=\begin{cases}
  %      -1 & \text{ for }v\in S_e,\\
  %      K & \text{ for }v\in S_o, \\
  %      0 & \text { else }
  %  \end{cases}$$
  % We claim that if a minimum weight $T$-join in $G'$ exists, it corresponds to a minimum weight partial $(S_e,S_o)$-join in $G$, otherwise no minimum weight partial $(S_e,S_o)$-join exists in $G.$ First, observe that since $c$ is non-negative, a $T$-join $J$ of $G'$ will always contain $\{\{(v,1),(v,2)\}:v\in S_e\}$, if such a $T$-join exists.



  \subsection{Minimum weight $T^*$-joins}
  \begin{centerframebox}
    Let $G$ be a connected graph, $c : E(G) \to R$ and $T \subseteq V (G)$.
    Note that $|T|$ may be either even or odd.
    Consider the problem of finding a $T^*$-join $J^*$ with $|T^* \Delta T| \leq 2$ such that $c(J^*)$ is minimum among all such $J^*$.
    Show that this problem can be linearly reduced to the \textsc{Minimum Weight $T$-Join Problem}.

    Remark: This problem arises when adapting Christofides' algorithm to the problem
    of finding a shortest Hamiltonian path with up to one endpoint of the path given
    in the input.
  \end{centerframebox}
  If $|T|$ is odd, then $T^*$ with $|T^*\Delta T|\leq 2$ and $|T^*|$ even has either one vertex more or less than $T$. We can model this two cases by two copies of $G$, after computing a minimum $T'$-join in the new graph consisting of these two copies, we can compare which of the two computed $T$-joins has less weight and take it as our solution. For the case that $T^*$ has one vertex less than $T$, we add a new vertex $v$ to our graph and connect it to all vertices of $T$ with edge weight $m:=\max_{e\in E(G)}c(e)+1$ higher than the maximum edge weight in $G$ and define $T':=T\cup\{v\}$. Now every minimal $T'$-join uses one of these two edges, thus when removing the endpoint of this edge which is not $v$ from $T$ we get a set $T^*$ and a minimal $T^*$-join. In the second copy of our graph we have to model the case that $T^*$ has one vertex more than $T$. We add a new vertex $v'$ to our graph that is connected to all vertices that are not in $T$ in this copy of $G$ with edges of weight $m$ and define $T'':=T\cup\{v'\}$. In a minimum weight $T''$-join there is one edge incident to $v$, when we add the other endpoint $u$ of this edge to $T$ we get a minimum $T\cup\{u\}$-join in the original graph. Thus we reduced the problem if $|T|$ is odd linearly to computing a Minimum weight $T'\cup T''$-join.

  If $|T|$ is even, there are four cases: $T=T^*$, $T^*=T\setminus\{v,w\}$, $T^*=T\cup\{v,w\}$ and $T^*=T\setminus\{v\}\cup\{w\}$, so analogue to the odd case we use four copies of our original graph G. The first copy for the case $T=T^*$ not needs to be modified. For the case $T^*=T\setminus\{v,w\}$ we add two vertices to the copy of $G$, which are connected to every vertex from $T$ with edge weight $m$ and add these two vertices to $T$.
  For the case $T^*=T\cup\{v,w\}$ we add two vertices to the copy of $G$, which are connected to every vertex from $V(G)\setminus T$ with edge weight $m$ and add these two vertices to $T$. For the case $T^*=T\setminus\{v\}\cup\{w\}$ we add one vertex which is connected to every vertex in $T$ and one which is connected to every vertex in $V(G)\setminus T$ with edge weight $m$ and add both new vertices to $T$.
  A minimum weight $T$-join in the new graph consisting of these four modified copies of $G$ gives us 4 joins, by comparing this 4 joins we find a minimum $T^*$-join $J^*$. Thus we  also reduced the problem if $|T|$ is even linearly to computing a Minimum weight $T$-join.

\end{document}
