\documentclass{article}

\usepackage{amsmath}
\usepackage{amsfonts}
\usepackage{amssymb}
\usepackage[dvipsnames]{xcolor}
\usepackage[margin=0.5in]{geometry}
\usepackage[hidelinks, bookmarks=false]{hyperref}
\usepackage{enumitem}

\let\oldemptyset\emptyset
\let\emptyset\varnothing
\let\oldphi\phi
\let\phi\varphi
\let\oldepsilon\epsilon
\let\epsilon\varepsilon

\usepackage{environ}
\NewEnviron{centerframebox}{\begin{center}\fbox{\parbox{0.92\textwidth}{\BODY}}\end{center}}

\title{Algorithms for Data Science \\ Exercise Sheet 7}
\author{
  Vladislav Imashev \\ \href{mailto:s05vimas@uni-bonn.de}{s05vimas@uni-bonn.de} \and
  AAAAAAAAAA AAAAAAA \\ \href{mailto:AAAAAAAAAAAAAAAAAAAA}{AAAAAAAAAAAAAAAAAAAA} \and
  German Mikhelson \\ \href{mailto:s17gmikh@uni-bonn.de}{s17gmikh@uni-bonn.de} \and
  Aleksandra Volynets \\ \href{mailto:s02avoly@uni-bonn.de}{s02avoly@uni-bonn.de} \and
  Nikita Morev \\ \href{mailto:s99nmore@uni-bonn.de}{s99nmore@uni-bonn.de}
}

\begin{document}
  \maketitle

  \setcounter{section}{7}
  \subsection{Merging Misra-Gries Synopses}
  \begin{centerframebox}
    Prove the theorem on Slide 6 of 2023-12-13.

    \textbf{Theorem}: the previous algorithm calculates an estimation $\hat{f_i}$ of $f_i$ for all $i \in [n]$ such that
    \[ f_i - \frac{m-m'}{k} \leq \hat{f_i} \leq f_i \]
    where
    \begin{itemize}
      \item $m = |\sigma_1| + |\sigma_2|$ and
      \item $m' = \sum_{l=1}^{k-1} A[l]$ after termination
    \end{itemize}
  \end{centerframebox}

  \textbf{Proof:}

  1) $\hat{f_i} \leq f_i$. \\
  $f_i$ - number of repeating $i$ in $\sigma_1 + \sigma_2$. \\
  $f^1_i$ ($f^2_i$) - number of repeating $i$ in $\sigma_1$ ($\sigma_2$). \\
  $\hat{f_i}$ - number of repeating $i$ after running the Misra-Gries Synopses Merger algorithm. \\
  $\hat{f^1_i}$ ($\hat{f^2_i}$) - number of repeating $i$ from $\sigma_1$ ($\sigma_2$) after running the Misra-Gries algorithm on $\sigma_1$ (or $\sigma_2$). \\

  From the Misra-Gries Algorithm: \\
  $A_1[i] = \hat{f^1_i} \leq f^1_i$ and $A_2[i] = \hat{f^2_i} \leq f^2_i$, hence $A_1[i] + A_2[i] = \hat{f^1_i} + \hat{f^2_i} = \hat{f_i} \leq f^1_i + f^2_i = f_i$, hence $\hat{f_i} \leq f_i$. \\

  2) $f_i - \frac{m-m'}{k} \leq \hat{f_i}$.

  We know that $m'$ is the sum of all frequencies after termination and in step 7 we decrement each counter by $c_k$, which means:
  \[m' \leq m\]
  So,
  \[f_i - \frac{m-m'}{k} \leq f_i - \frac{m-m}{k} \leq f_i\]

  \subsection{The Lossy Counting Algorithm}
  \begin{centerframebox}
    Prove Lemmata 1--4 on Slide 16 of 2023-12-13.

    \textbf{Lemma 1}: whenever deletions occur, $b_\mathrm{current} \leq \epsilon m$ \\
    \textbf{Lemma 2}: whenever an entry $(e, f, \Delta)$ gets deleted, $f_e \leq b_\mathrm{current}$ \\
    \textbf{Lemma 3}: if there is no entry for $e$ in $\mathcal{D}$, then $f_e \leq \epsilon m$ \\
    \textbf{Lemma 4}: if $(e, f, \Delta) \in \mathcal{D}$ then $f \leq f_e \leq f + \epsilon m$
  \end{centerframebox}
    \textbf{Lemma 1}:
    Whenever deletions occurs, it is true that
    $$ m = 0 \text{ mod } w $$
    and this is equivalent to
    $$ m = k \cdot w \text{ for some } k \in \mathbb N$$
    This actually means that $b_{current}$ has been incremented $k-1$ times by the time a deletion occur, hence:
    $$b_{current} = k - 1 + 1 = k$$
    And thus we have that
    $$b_{current} = k = \frac{m}{w} \leq \epsilon w$$

    \textbf{Lemma 2}:
    Since $(e, f, \Delta)$ gets deleted, we have incremented $b_{current}$ at least $f_{e} - 1$ times (counting the number of all increments from the very beginning), because we had to have decremented $f$ this many times, for it to be $f=1$ at the end.
    Hence
    $$b_{current} \geq f_{e} - 1 + 1 = f_{e}$$
    and
    $$f_{e} \leq b_{current}$$

    \textbf{Lemma 3}: Prove this by considering two cases. First case: there is no entry for $e$ in $\mathcal{D}$ since it is got deleted. In this case, according to the Lemma 2 we have:
    $$f_e \leq b_{current}$$
    And applying then Lemma 1 we conclude:
    $$f_e \leq b_{current} \leq \epsilon m$$

    \textbf{Lemma 4}:
    Note that $f$ cannot be greater than $f_e$ since we increment $f$ only when we process $e$ (i.e. whenever we came across $e$ in data stream). Hence,
    $$f \leq f_e$$
    Prove the second part of the inequality.
    Before we add $(e, 1, b_{current} - 1)$ in $\mathcal{D}$ we have, according to the Lemma 3 (because the element was absent from $\mathcal{D}$), that
    $$f'_{e} \leq \epsilon m$$
    where $f'_{e}$ is the frequency at the moment before the element was added to $\mathcal{D}$.
    Then, after we process $e$ for $f$ times:
    $$f'_{e} + f \leq f + \epsilon m$$
    Note that $f_e = f'_{e} + f$, therefore:
    $$f_{e} \leq f + \epsilon m$$


  \subsection{The Lossy Counting Algorithm}
  \begin{centerframebox}
    The following fact is stated on Slide 21 of 2023-12-13:

    \textbf{Fact}: for every $j = 1,\dots,B: \quad \sum_{i=0}^j i \cdot d_i \leq jw$

    Why is this fact true?
    % кажется на том-же слайде есть примерное доказательство
    % d_i всегда меньше w, и i всегда меньше j
    % w это размер окна
    % d_i определяется как-то через жопу...
    % но это количество элементов которые появились i окн назад, и всё ещё тут
  \end{centerframebox}
  $jw$ is the total number of elements the algorithm has encountered so far.
  To be counted among $d_i$ an element has to occur at least $i$ times,
  because otherwise it would've been deleted from that bucket.
  So when computing $\sum_{i=0}^j i \cdot d_i$
  we are summing up the number of times those elements would've had to appear to show up in their corresponding buckets.
  But that obviously cannot exceed the total number of elements read by the algorithm.
  % TODO (Kostya; Sasha she/her)

\end{document}
