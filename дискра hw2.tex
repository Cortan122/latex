\documentclass{article}

\usepackage{cmap} % поиск в pdf
\usepackage{mathtext} % русские буквы в формулах
\usepackage[english,russian]{babel} % локализация и переносы
\usepackage[T2A]{fontenc} % кодировка в pdf (магия)
\usepackage[utf8]{inputenc} % кодировка исходного текста
\usepackage{amsmath}
\usepackage{amsfonts}
\usepackage{amssymb}
\usepackage{gensymb}
\usepackage{headertable}

\usepackage[hidelinks,bookmarks=false]{hyperref}
\usepackage{xurl}
\usepackage[margin=0.5in]{geometry}

\newcommand{\notimplies}{\mathrel{{\ooalign{\hidewidth$\not\phantom{=}$\hidewidth\cr$\implies$}}}}

\newcommand{\ds}{\displaystyle}
\newcommand{\DS}{\phantom{$0.5$}}
\newcommand{\N}{\mathbb{N}}
\newcommand{\Z}{\mathbb{Z}}
\newcommand{\Q}{\mathbb{Q}}
\newcommand{\R}{\mathbb{R}}
\newcommand{\range}{\underline}
\newcommand{\Aleph}{2^{\aleph_0}}
\renewcommand{\f}{\frac}
\renewcommand{\l}{\left}
\renewcommand{\r}{\right}
\renewcommand{\P}[1]{\mathcal{P}\l(#1\r)}
\renewcommand{\emptyset}{\varnothing}

\usepackage{xcolor}
\definecolor{red}{HTML}{d62728}
\definecolor{orange}{HTML}{ff7f0e}
\definecolor{green}{HTML}{2ca02c}
% \definecolor{blue}{HTML}{1f77b4}
\newcommand{\red}[1]{{\color{red}{#1}}}
\newcommand{\orange}[1]{{\color{orange}{#1}}}
\newcommand{\green}[1]{{\color{green}{#1}}}
\newcommand{\blue}[1]{{\color{blue}{#1}}}

% \pagecolor{black}
% \color{white}

\title{Дискретная математика \\ Домашнее задание №2}
\author{\AA{AAAAA AAAAAAA}{4} \\ \AA{AAAAAA}{11}}

\begin{document}
  \maketitle
  \HeaderTable{12}

  \section{Приведите пример бинарного отношения $P \subseteq \R \times \R$, являющегося}
  Бинарное отношение R называется: % \cite{дашков}
  \begin{enumerate}
    \item функциональным, если $\forall x\forall y\forall z\ ((xRy \land xRz) \implies y = z)$;
    \item инъективным, если $\forall x\forall y\forall z\ ((yRx \land zRx) \implies y = z)$;
    \item тотальным для множества $Z$, если $\forall x \in Z\ \exists y\ xRy$;
    \item сюръективным для множества $Z$, если $\forall y \in Z\ \exists x\ xRy$.
  \end{enumerate}
  \subsection{не функциональным, инъективным, не тотальным, сюръективным (для $\R$)}
  $\begin{cases}
    f(y) = y^2 \\
    y \in \R \\
  \end{cases}$ или $ \l\{(y^2, y) \mid y \in \R\r\}$
  \subsection{функциональным, не инъективным, тотальным (для $\R$), не сюръективным}
  $\begin{cases}
    f(x) = x^2 \\
    x \in \R \\
  \end{cases}$ или $ \l\{(x, x^2) \mid x \in \R\r\}$

  \section{Пусть $f : A \to B$ и $g : B \to C$. Докажите, что если $g \circ f$ инъекция, то $f$ тоже инъекция}
  мы можем доказать от обратного \\
  допустим $f$ не инъекция, это значит что $\exists a_1,a_2\in A:\ a_1\neq a_2 \land f(a_1)=f(a_2)=b $ \\
  потом в $g \circ f$ когда мы будем в $g$ подставлять $b$ у нас будет $g\l(f\l(a_1\r)\r) = g\l(f\l(a_2\r)\r)$
  и $g \circ f$ не будет инъекцией. \\
  $\blacksquare$

  \section{Докажите, что функция $f: A \to B$ инъективна тогда и только тогда, когда для любого $C$ и любых $g, h: C \to A$ из $f \circ g = f \circ h$ следует $g = h$}
  инвертировав (преобразовав) вторую часть ны получим $g \neq h \implies f \circ g \neq f \circ h$.
  $g \neq h$ означает то что $\exists x \in C: g(x) \neq h(x)$, а
  $f \circ g \neq f \circ h$ означает то что $\exists y \in C: f(g(y)) \neq f(h(y))$.
  теперь, чтобы доказать прямую импликацию, нам надо просто подставить $x$ в $y$
  и заметить что инъективная функция от разных значений никогда недаст нам одинаковое.
  доказать обратную импликацию немного сложнее. \\
  мы сначала заменяем $\l(\exists x \in C: g(x) \neq h(x) \implies \exists y \in C: f(g(y)) \neq f(h(y)) \r) \implies f \text{ инъективна}$ \\
  на $f \text{ неинъективна} \implies \l(\exists x \in C: g(x) \neq h(x) \notimplies \exists y \in C: f(g(y)) \neq f(h(y)) \r)$ \\
  тоесть $\exists \text{ неинъективная } f: \l(\exists x \in C: g(x) \neq h(x)\r) \land \neg\l(\exists y \in C: f(g(y)) \neq f(h(y))\r)$ \\
  если мы тут снова подставим $x$ в $y$ то будет $\exists \text{ неинъективная } f: \forall z_1,z_2 \in A: f(z_1) = f(z_2)$ \\
  очевидно такая функция существует (например она может всегда возвращать одно значение). \\
  $\blacksquare$
  % тут надо заменить то, что если $x \neq y$, то
  % $f(g(y))$ всегда будет равно $f(h(y))$, тк функция, вызванная от одинаковых значений, всегда вернет одинаковые ответы.
  % поэтому вторая часть становится $  $ $\forall z \in A: f(z) = f(z)$

  \section{Приведите примеры элементов следующих множеств}
  $$ S^N = \{f \colon N \to S\} $$ % \cite{setexpo}
  $$ \range{n} = \{0,1,\dots,n-1\} $$ % cite ¯\_(ツ)_/¯
  \subsection{$\Q^{\range{3}}$}
  $\ds\begin{cases}
    f(0) = \f{1}{2} \\
    f(1) = \f{1}{3} \\
    f(2) = \f{1}{5} \\
  \end{cases}$ или $\ds \l\{\l(0,\f{1}{2}\r), \l(1,\f{1}{3}\r), \l(2,\f{1}{5}\r)\r\}$
  \subsection{$\R^{\Q}$}
  $\ds\begin{cases}
    f(x) = x\pi \\
    x \in \Q \\
  \end{cases}$ или $\ds \l\{(x,x\pi) \mid x \in \Q\r\}$
  \subsection{$\R^{\R\times\Z}$}
  $\ds\begin{cases}
    f(x,y) = x\pi^y \\
    x \in \R \\
    y \in \Z \\
  \end{cases}$ или $\ds \l\{((x,y),x\pi^y) \mid x \in \R \land y \in \Z \r\}$

  \section{Применяя, если нужно, теорему Кантора-Бернштейна-Шрёдера}
  $$ \Q \sim \Z \sim \N \sim \N\times\N $$
  $$ \R \sim \R\times\R \sim \range{2}^\N \sim \N^\N \sim \R^\N $$
  \subsection{Докажите, что $ \N^{\N\times\Q} \times \N \sim \R^\Q $}
  тут мы всё можем упростить заменив на равномощные множества \\
  $ \N^{\N\times\Q} \times \N \sim \R^\Q $ \\
  $ \N^{\N\times\N} \times \N \sim \R^\N $ \\
  $ \N^\N \times \N \sim \R $ \\
  $ \R\times\N \sim \R $ \\
  тут заметим что $ \R\times\N \subseteq \R\times\R $ и $ \R \sim \R\times\{0\} \subseteq \R\times\N $ \\
  получается $ \R \sim \R\times\{0\} \lesssim \R\times\N \lesssim \R\times\R \sim \R $, тоесть $ \R \sim \R\times\N $ \\
  $\blacksquare$

  \subsection{Докажите, что $ \range{5}^\N \sim \range{3}^\N $}
  можем заметить что каждый элемент из $ \range{n}^\N $ это бесконечная последовательность циферок в $n$-ричной системе счисления,
  тоесть это число в $[0;1)$ записано в $n$-ричной системе счисления.
  получается что и $ \range{5}^\N $ и $ \range{3}^\N $ это все числа в $[0;1)$,
  только одни записаны в 5-иричной а другие в 3-ичной системе счисления. \\
  $\blacksquare$

  \subsection{Докажите, что трехмерный куб равномощен любому своему сечению, содержащему более одной точки}
  сечение куба, содержащие более одной точки, -- это либо отрезок либо многоугольник.
  тк наше сечение является подмножеством куба, его мощность будет $\leq$ мощности самого куба.
  в любом многоугольнике мы можем найти отрезок, а любой отрезок можно разделить на его длину у получить $[0;1]$.
  тотже трюк с делением на длину можно проделать со всем кубом, и у нас будет $[0;1]^3$.
  ещё мы можем убрать одну точку из наших отрезков, тк $\infty-1 = \infty$.
  \\ $\ds [0;1) \sim [0;1] \sim \text{отрезок} \lesssim \text{многоугольник} \lesssim \text{куб} \sim [0;1]^3 \sim [0;1)^3 $ \\
  теперь нам осталось только доказать, что $\ds [0;1) \sim [0;1)^3 $.
  для этого надо построить биекцию.
  каждое число в $[0;1)$ мы можем записать как бесконечную последовательность циферок,
  но есть некоторые числа, как например $0.1$, которые можно так записать двумя разными способами:
  $0.1\overline{0}$ и $0.0\overline{9}$.
  в таких случаях мы всегда будем брать ту последовательность, в которой в периоде $0$.
  теперь мы для любого числа из $[0;1)$ разбиваем её запись на три последовательности,
  основываясь на остатке деления индекса на $3$, тоесть из
  $0.\red{3}\green{1}\blue{4}\red{1}\green{5}\blue{9}\red{2}\green{6}\blue{5}\red{3}\green{5}\blue{8}\red{.}\green{.}\blue{.}$ $\l(\f{\pi}{10}\r)$
  получится $0.\red{3123...}$, $0.\green{1565...}$ и $0.\blue{4958...}$.
  это можно проделать и в обратную строну и у нас никогда не будет конфликтов, тоесть мы построили биекцию. \\
  $\blacksquare$

  \section{Докажите, что $\R^\R \sim \P{\R}$}
  тут мы всё можем упростить заменив на равномощные множества \\
  $\R^\R \sim \P{\R}$ \\
  $\l(\range{2}^\N\r)^\R \sim \range{2}^\R$ \\
  $\range{2}^{\N\times\R} \sim \range{2}^\R$ \\
  $\range{2}^\R \sim \range{2}^\R$ \\
  $\blacksquare$

  \section{Пусть $X$ -- некоторое множество попарно непересекающихся восьмерок на плоскости (восьмерка состоит из двух касающихся внешним образом окружностей). Докажите, что $X \lesssim N$, т. е. что мощность $X$ не более счетной}
  внутри каждой окружности можно найти рациональную точку.
  мы для каждой восьмерки выбираем одну рандомную рациональную точку в каждой из её окружностей.
  у нас получается функция от восьмерок в $\Q^4$. теперь надо доказать что это инъекция.
  тоесть нам надо доказать, что если у двух восьмерок совпадают эти пары рациональных точек, то они пересекаются.
  это легко доказывается из геометрии,
  потомучто точка касания новых окружностей либо совпадает со старой,
  либо находится в одной из старых окружностей,
  либо старая точка находится в одной из новых окружностей.
  мы, построив инъекцию, получили $X \lesssim \Q^4$ и тк $\Q^4 \sim \N$, в итоге $X \lesssim \N$ \\
  $\blacksquare$

  \section{Докажите, что существует множество $S \subseteq \P{\R}$, такое что выполнены все следующие условия: (а) $S \sim R$; (б) если $X, Y \in S$ и $X \neq Y$, то $X \cap Y = \emptyset$; (в) если $X \in S$, то $X \sim R$}
  тоесть нам надо разделить $\R$ на $\Aleph$ непересекающихся кусочков, длинна каждого из которых тоже $\Aleph$.
  тут нам понадобится тотже трюк с циферками что был в вопросе \textbf{5.3}, только тут нам надо это будет сделать наоборот.
  мы обозначим $n$-ую циферку числа $a$ как $a\l[n\r]$, тоесть например $\l(\f{\pi}{10}\r)\l[3\r] = 4$. \\
  $\ds S = \l\{ \l\{ y \in [0;1) \mid y[2] = x[1] \land y[4] = x[2] \cdots\land\cdots y[2n] = x[n] \r\} \mid x \in [0;1) \r\}$ \\
  вот мы построили такое множество, и, очевидно, оно существует. \\
  $\blacksquare$
  % \subsection{Постройте аналогичное множество $S' \subseteq \P{\R^2}$}

  \section{Пусть $C = \l\{f: \R \to \R \mid f \ \text{непрерывна} \r\}$. Докажите, что $C \sim R$}
  любая непрерывная функция однозначно определена её рядом Тейлора, а ряд Тейлора это бесконечная последовательность чисел,
  тоесть существует биекция между $C$ и $\R^\N$, и как мы помним из пятого вопроса $\R^\N \sim \R$.
  получается $C \sim \R^\N \sim \R$ \\
  $\blacksquare$

  \section{Пусть множества $A$ и $B$ таковы, что $A - B$ бесконечно, а $B$ конечно или счетно. Докажите, что $A - B \sim A$}
  здесь мы можем добавить условие $B - A = \emptyset$. это ничего неменяет, тк $A - B$ останется такимже а мощность $B$ может только уменьшиться.
  изза этого изменения мы можем сказать что $|A-B| = |A|-|B|$.
  сначала рассмотрим случай, когда у нас $|B| < |A|$.
  тогда у нас всё очевидно, тк $\forall \infty \geq \aleph_0\ \forall n < \infty: \infty-n=\infty$.
  туда надо просто подставить $\infty=|A|$ и $n=|B|$.
  теперь осталось только рассмотреть случай когда $A \sim B$.
  тут $B$ неможет быть конечным, чтобы $A-B$ небыло пустым.
  тоесть получается что $A \sim B \sim \N$.
  при вычитание множеств их мощность может только уменьшатся, поэтому $A-B \lesssim A$,
  но $\N \lesssim A-B$ и $A \sim \N$. тоесть $A-B \sim \N$ \\
  $\blacksquare$

  \section{Рассмотрим группу из $n > 0$ человек. Какое наименьшее значение $n$ гарантирует, что в группе найдутся два разных человека с одинаковыми днями рождения?}
  принцип Дирихле говорит, что для любых натуральных $n$ и $m$, если $n$ объектов распределить на $m$ множеств, то
  хотябы одно из этих множеств будет содержать хотябы $\ds\l\lceil\f{n}{m}\r\rceil$ элементов.
  у нас тут $m=366$, тк у когото может быть др 29 февраля.
  нам надо найти наименьшее такое $n$, чтобы $\ds 2=\l\lceil\f{n}{366}\r\rceil $, тоесть $\ds \f{n}{366} > 1$.
  получается что $n > 366$ и тк оно натуральное $n=367$ \\
  ответ: \fbox{$367$}

  \section{Докажите, что есть число с десятичной записью вида $11\dots1$, кратное $2017$}
  число $m$ с записью такого вида длинны $n$ будет равно $\ds m=\sum_{i=0}^{n-1} 10^i $.
  и тут тк $2017$ простое число то $\Z_{2017}$ это полноценное поле а не какоето там кольцо.
  поэтому у нас степени $10$ покрывают всё множество,
  потомучто у нас порядок любого элемента в мультипликативной полугруппе равен максимально возможному.
  и если мы возьмём такое $n$, чтобы у нас в сумме были все числа из $\Z_{2017}$, то
  $m$ будет суммой арифметической прогрессии, тоесть $\ds m=\f{0+2016}{2}2017 $, что очевидно делится на $2017$ \\
  $\blacksquare$

  \vfill
  \begin{thebibliography}{9}
    \bibitem{дашков} Дашков Е. В. Введение в математическую логику // URL: \url{https://drive.google.com/file/d/1wLW9t2UFJk9tTu8nM_YAYYr9zzZwKD1W/view}
    \bibitem{setexpo} \url{https://en.wikipedia.org/wiki/Exponentiation#Over_sets}
    \bibitem{powerset} \url{https://en.wikipedia.org/wiki/Power_set}
    \bibitem{pigeon} \url{https://en.wikipedia.org/wiki/Pigeonhole_principle}
  \end{thebibliography}

\end{document}
