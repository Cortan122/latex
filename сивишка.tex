\documentclass[12pt]{article}
\usepackage[colorlinks=true,bookmarks=false]{hyperref}
\usepackage{122}

\title{Curriculum Vitae}
\author{AAAAA AAAAAAA \\ \href{mailto:AAAAAAAAAAAAAAAAAAAA}{AAAAAAAAAAAAAAAAAAAA}}

\begin{document}
  \maketitle

  \section{Опыт работы}
  \begin{itemize}
    % https://hpc.hse.ru/news/397844217.html
    \item цифровой ассистент по ОС Linux \\
      Высшая Школа Экономики, отдел суперкомпьютерного моделирования \\
      июль 2020 -- август 2020

  \end{itemize}

  \section{Образование}
  \begin{itemize}
    \item Высшая Школа Экономики \\
      факультет компьютерных наук, программная инженерия \\
      сентябрь 2019 -- настоящее время

    \item University of Groningen \\
      международная академическая мобильность, exchange student \\
      февраль 2022 -- июнь 2022

    \item Курс Mail.ru x HSE <<Разработка приложений на Android. Весна 2021>> \\
      март 2021 -- июнь 2021

  \end{itemize}

  \section{Волонтёрство}
  \begin{itemize}
    \item Московская командная олимпиада школьников по программированию \\
      24 октября 2021

  \end{itemize}

  \section{Проекты}
  \begin{itemize}
    \item Командный проект «Unicorug» \\
    Курс Software Engineering, University of Groningen,
    \href{https://www.rug.nl/ocasys/fwn/vak/show?code=WBCS017-10}{WBCS017-10} \\
    апрель 2022 -- июнь 2022 \\
    \url{https://unico.web.rug.nl/}

    % https://github.com/scanhex/map_simulating
    \item Командный проект «COVID-19 Pandemic Simulation Map» \\
    апрель 2020 \\
    \href{http://covid-simulation.tk/}{\texttt{https://covid-simulation.com/}}

    \item Командный проект по теме «Умный будильник» \\
    Курс Internet of things Ecosystems ПИ ФКН \\
    ноябрь 2021 -- декабрь 2021 \\
    \url{https://github.com/HowToCodeWithPaws/WakeMeUpInside}

    \item Командный проект по теме «Android-приложение для построения шаблонов для вышивания» \\
    Курс Mail.ru x HSE <<Разработка приложений на Android. Весна 2021>> \\
    март 2021 -- июнь 2021 \\
    \url{https://github.com/Cortan122/FancyWork}

    \item Курсовой проект по теме «Кроссплатформенное десктоп приложение для редактирования скриншотов» \\
    Высшая Школа Экономики, факультет компьютерных наук, 2 курс \\
    \url{https://github.com/Cortan122/ImageEditor}

    \item Курсовой проект по теме «Приложение для визуализации метода рекурсивного спуска» \\
    Высшая Школа Экономики, факультет компьютерных наук, 1 курс \\
    \url{https://github.com/Cortan122/ParserVisualizer}

  \end{itemize}

  \section{Дополнительная информация}
  \begin{itemize}
    \item Уверенное владение языками С, C++, C\#, JavaScript, Java, Python, R, Bash, SQL, \LaTeX, Intel x86 assembly, HTML/CSS
    \item Процентиль по программе 22.32, GPA 8.65
  \end{itemize}
\end{document}
