\documentclass{article}

\usepackage{enumitem}
\usepackage{amsmath}
\usepackage{amsfonts}
\usepackage{amssymb}
\usepackage[margin=0.5in]{geometry}
\usepackage[hidelinks, bookmarks=false]{hyperref}

\usepackage[dvipsnames]{xcolor}
\usepackage{algorithm}
\usepackage{algpseudocode}
\usepackage{algorithmicx}

\let\oldemptyset\emptyset
\let\emptyset\varnothing
\newcommand{\R}{\mathbb{R}}
\newcommand{\Rd}{\R^d}

\usepackage{environ}
\NewEnviron{centerframebox}{\begin{center}\fbox{\parbox{0.92\textwidth}{\BODY}}\end{center}}

\title{Discrete and Computational Geometry \\ Assignment 3}
\author{
  \AA{AAAAAAAAAA AAAAAAA}{6} \\
  \href{mailto:\AA{AAAAAAAAAAAAAAAAAAAA}{7}}{\AA{AAAAAAAAAAAAAAAAAAAA}{7}}
  \and
  Emilia Groß-Hardt \\
  \href{mailto:emilia.ghdt@uni-bonn.de}{emilia.ghdt@uni-bonn.de}
}

\usepackage{titlesec}
\titleformat{\section}
  {\normalfont\Large\bfseries}{Problem \thesection : }
  {0em}{\mdseries}

\newcommand{\algorithmRuledHeader}[2]{\hrule height.8pt \vspace{2pt}\textbf{Algorithm #1:} #2\vspace{2pt}\hrule}

\begin{document}
  \maketitle
  \begin{center}
    { \bfseries Deadline: 3 Nov 2024, 23:55 }
  \end{center}

  \section{Modified Graham-Scan}
  \begin{centerframebox}
    Modify the Graham-Scan (Lecture 5, Algorithm 1) such that it also computes the convex
    hull of points that are not in \textit{general position}, i.e.:
    \begin{itemize}
      \item For a given set of points in the plane we may assume that no three points lie on a
      common line.
      \item For points in the plane we may assume that no two points have the same x-coordinate
      or y-coordinate.
    \end{itemize}
    The modified algorithm should still run in $O(n \log(n))$ time and $O(n)$ space.

    \begin{center}
      \begin{minipage}{.75\linewidth}
        \renewcommand{\thealgorithm}{5.1}
        \begin{algorithm}[H]
          \caption{Graham-Scan}
          \begin{algorithmic}[1]
          \Procedure{Convex-Hull}{Set of points $P$ in $\R^2$}
            \State Let $p_1$ be the point in $P$ with minimal $y$-coordinate
            \State Express the remaining points in $P$ by their polar coordinates with origin $p_1$
            \State Let $p_2, \ldots, p_n$ denote the points sorted by increasing polar angle (w.r.t. $p_1$)
            \State Let $L$ be a list containing $p_1, p_2$ in this order
            \For{$i \gets 3$ \textbf{to} $n$}
              \State Append $p_i$ to $L$
              \While{the last three points in $L$ do not make a left turn}
                \State Remove the middle point of the last three points from $L$
              \EndWhile
            \EndFor
            \State \Return $L$
          \EndProcedure
          \end{algorithmic}
        \end{algorithm}
        \renewcommand{\thealgorithm}{\arabic{algorithm}}
      \end{minipage}
    \end{center}
  \end{centerframebox}
  We have to address two places where we relied on that assumption.
  \begin{enumerate}
    \item
    Firstly, we needed to pick the point with the lowest $y$, which now may not be unique.
    This can be addressed by always picking the point with the lowest $x$ among them.
    \textcolor{ForestGreen}{Can we still assume there are no duplicate points?}

    \item Secondly, we used the fact that each point will have a unique polar coordinate angle.
    In this case, it is advantageous to simply pick the point with the farthest polar distance,
    because all others are guaranteed not to be part of the convex hull.
  \end{enumerate}

  \algorithmRuledHeader{1}{Modified Graham-Scan}
  \begin{algorithmic}[1]
    \Procedure{Convex-Hull}{Set of points $P$ in $\R^2$}
      \State Let $P_1'$ be the set of points in $P$ with minimal $y$-coordinate
      \State Let $p_1$ be the point in $P_1'$ with minimal $x$-coordinate
      \State Express the remaining points in $P$ by their polar coordinates with origin $p_1$
      \State Sort the points by increasing polar angle, and separate them into groups when their angles are equal
      \State In each group, keep only the point with the maximal polar radius and discard the rest
      \State Let $p_2, \ldots, p_n$ denote the points in this sorted order
      \Comment{\textcolor{ForestGreen}{After this line the algorithm remains unaltered}}
      \State Let $L$ be a list containing $p_1, p_2$ in this order
      \For{$i \gets 3$ \textbf{to} $n$}
        \State Append $p_i$ to $L$
        \While{the last three points in $L$ do not make a left turn}
          \State Remove the middle point of the last three points from $L$
        \EndWhile
      \EndFor
      \State \Return $L$
    \EndProcedure
  \end{algorithmic}
  \hrule
  % (c) Clover

  \section{Worst case}
  \begin{centerframebox}
    For $n \geq 8$, describe a construction of an input instance of $n$ points in $\R^3$ and a permutation
    of this input instance, where Algorithm 7.1 takes $\Omega(n^2)$ steps. Describe for which operations
    the algorithm needs to take this many steps.
  \end{centerframebox}
  We can take basically the same worst case, as for the previous 2d version of this algorithm, and adapt it to 3d.
  Create a triangle with the first 3 points, and then pick the other points on planes parallel to that triangle an equal distance apart,
  such that they are all sorted in order of increasing distance and the projection of then to the original plane lies within the triangle.

  This way each face (expect for the first triangle) of our convex hull, as we create it,
  will be in conflict with all other points we haven't considered yet.
  But the algorithm has to populate those conflict lists,
  whenever it creates new faces, and that process is linear in the number of conflicts.
  This gives us $O(n^2)$ because $O(n)$ conflicts have to be updated for each of the $n-4$ points.
  % (c) Clover

  \section{Analyzing the Algorithm}
  \begin{centerframebox}
    Consider Algorithm 7.1 from Lecture 7. The algorithm needs to find the cycle of horizon
    edges $L$ that lie on the boundary of $\operatorname{Conflicts}(i)$.
    \begin{enumerate}[label=(\alph*)]
      \item Design an algorithm for this step that takes $O(|\operatorname{Conflicts}(i)|)$ time in the worst-case.
      Use the pointers of the DCEL. Show correctness and running time of your algorithm.
      \item Show that the expected total time spent on computations for finding horizon edges during
      the course of the algorithm is in $O(n)$
    \end{enumerate}
  \end{centerframebox}

  \subsection{Traversing the conflict faces}
  \begin{enumerate}
    \item Start with a point on one of the conflict faces.
    \item Consider all the outgoing edges in clockwise order.
    \item If the face associated with that edge is also a conflict face, traverse the edge,
          and store that traversal in a linked list. (go to 2)
    \item Otherwise look at the next edge. (go to 3)
    \item If we arrived at an already visited point, return the cycle formed by the edges in our stored linked list.
          That might necessitate the removal of some initial part of said linked list.
  \end{enumerate}
  % (very sloppy)
  % (c) Clover

  \subsection{Expected runtime}
  Here the expected running time is determined by which random permutation of the datapoints was selected.
  Since every possible permutation is still equally likely,
  or each point has an equal probability to end up in each possible position,
  we can use the same tricks used in the last lecture to prove the same for the 2d case of the algorithm.

  The expected total time spent on computations for finding horizon edges can be expressed
  as the sum of the number of conflict faces for each point:
  \[ \sum_{i=5}^{n} |\operatorname{Conflicts}(i)| \]

  Using a random permutation, where a lot of the points are gonna be already inside the convex hull,
  causes this sum to be reduced to $O(n)$ in expectation.

  % (just plain bullshit)
  % (c) Clover

\end{document}
