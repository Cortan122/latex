\documentclass[12pt]{article}

\usepackage{cmap} % поиск в pdf
\usepackage[english,russian]{babel} % локализация и переносы

\usepackage[hidelinks]{hyperref}
\usepackage{xurl}
\usepackage{enumitem}

\usepackage[margin=.5in]{geometry}

\usepackage{xcolor}
\definecolor{purple}{HTML}{800080}
\newcommand{\blue}[1]{\textcolor{blue}{#1}}
\newcommand{\red}[1]{\textcolor{red}{#1}}
\newcommand{\orange}[1]{\textcolor{orange}{#1}}
\newcommand{\Orange}[1]{{\color{orange}{#1}}}
\newcommand{\brown}[1]{\textcolor{brown}{#1}}
\newcommand{\teal}[1]{\textcolor{teal}{#1}}
\newcommand{\purple}[1]{\textcolor{purple}{#1}}

\title{История \\ Домашнее задание №2}
\author{\AA{AAAAA AAAAAAA}{4} \\ AAAAAA}
% \date{30 мая 2020 г.}

\begin{document}
  \maketitle

  \setcounter{section}{2}
  \section{<<Троица>> Андрея Рублева как выражение идей преп. Сергия Радонежского}
  % \subsection{Про икону}
  Сюжет иконы <<Троица>> преп. Андрея Рублева является модификацией канонического сюжета на тему Священной истории Ветхого Завета,
  называемой <<Гостеприимство Авраама>>.

  На таких иконах обычно изображаются три ангела-путника,
  которых встретил Авраам у Мамврийского дуба, сам Авраам, его жена Сарра
  и работники, готовящие трапезу для ангелов.

  Многочисленные византийские, латинские, сирийские и коптские комментаторы первого тысячелетия христианской эры,
  трактовали этот ветхозаветный сюжет следующим образом:
  три мужа-странника, пришедших к Аврааму, символизировали Святую Троицу.

  В связи с этим именно <<Гостеприимство Авраама>>, с точки зрения строгого догматического богословия,
  является единственной попыткой корректного изображения Троицы,
  не прибегая сомнительным с точки зрения строгой догматики изображениям Бога Отца в виде старца.

  Тем не менее, преп. Андрей Рублев, существенно урезает обычный сюжет <<Гостеприимства Авраама>>:
  ни Авраама, ни Сарры, ни тельца, ни даже яств у Рублева нет -- только чаша стоит на трапезе -- как образ таинства Евхаристии.
  Вместо пира -- безмолвная беседа.

  Почему преподобному понадобилось сокращение сюжета иконы до трех фигур мужей-ангелов -- трех Лиц Святой Троицы?

  Троица Рублева изображает единство ипостасей Святой Троицы, мир, любовь и согласие между ними,
  а Сергий Радонежский всегда был поборником мира и согласия, так среди манахов, так и на Руси.
  Не даром, по преданию, икона была написана Андреем Рублевом по поручению ученика и приемника Сергия Радонежского -- Никона Радонежского,
  для восхваления Сергия Радонежского.

  Деятельность Сергия Радонежского проявилась в двух основных аспектах: церковном и политическом.

  \subsection{Церковный аспект деятельности Сергия Радонежского}
  Дело в том, кто к моменту монгольского нашествия на Русь, и в первое время после него традиция общежительских монастырей (киновий) принесенная на Русь преподобным Феодосием Печерским в XI веке, стала забываться, и большинство монастырей стало скитским (особножительским).

  Оба типа монастырей (киновиальный и скитский) появились в Египте, на Синае, в Палестине, в других частях христианского Востока и Запада еще в IV веке, оба были принесены на Русь сразу после ее крещения.

  Но к началу XIV века основным типом монастырей на Руси становится скит.

  Тем не менее, преп. Сергий, следуя пришедшему из Византии XIV века течению исихазма, возродил на Руси общежитейский уклад монастырей.

  Отражением общежитейского уклада является сконцентрированность сюжета <<Троицы>> Рублева на молитвенном соработничестве персонажей.
  Для выражения этой идеи молитвенного соработничества иконописец отсекает из сюжета иконы все второстепенные,
  <<рассеивающие внимание>> детали.

  \subsection{Политический аспект деятельности Сергия Радонежского}

  Сергий Радонежский сделал многое для того чтобы поднять Русь на борьбу с татаро-монгольскими захватчиками.
  Он благословил князя Дмитрия Донского\footnote{не все историки признают это историческим фактом},
  который, по преданию, приезжал к нему перед битвой.
  Кроме того Сергий Радонежский отправил на битву иноков\footnote{по другой версии послушников, так как инокам нельзя носить оружие}
  Троицы-Сегриевой-Обители Пересвета и Ослябю.
  С этой точки зрения можно считать символом мира и единства на Руси, за который так ратовал Сергий Радонежский.

  Д. С. Лихачёв писал о преподобном: <<%
  Сергий Радонежский был проводником определённых идей и традиций: с Церковью связывалось единство Руси.
  Князья ссорились, наводили татар на Русскую землю, как когда-то половцев.
  Шло постоянное соперничество за великое княжение, за титул великого князя.
  А Церковь-то была едина.
  И поэтому главная идея рублёвской Троицы -- идея единства, столь важная во тьме разделения нашего.
  Дмитрий Донской начинает не с объединения территории, а с объединения духовно-нравственного.>>

  % \blue{Возможно, это было отражением той организаторской деятельности, которую вел игумен будущей Троицкой лавры, преп. Сергий Радонежский, под началом которого подвизался преп. Андрей Рублев.}

  \setcounter{section}{4}
  \section{Место личности в духовных течениях на Руси XIV-XV вв.: исихазм, ереси стригольников и жидовствующих}

  \subsection{Введение}
  Позднее Средневековье и в Западной Европе, и на Руси характеризуется активизацией духовных поисков,
  причем этот процесс проходил как внутри как ортодоксальных религиозных течений, так и среди еретических,
  неортодоксальных учений.

  Рост городов, развитие ремесла и торговли, потребность в грамотных ремесленниках и торговцах, наконец,
  последствия великой эпидемии чумы, в ходе которой, по разным оценкам, погибло от четверти до трети населения христианского мира --
  все это привлекало внимание к роли личности в обществе и личной ответственности за свои добродетели или грехи.

  Особое возмущение вызывала симония (практика поставления в духовный сан за деньги) среди духовенства,
  а также алчность и разврат его представителей.
  Большинство адептов новой религиозности считало что в духовный сан нужно поставлять не за деньги,
  а за моральные качества, в чём также проявлялся интерес к личности человека.

  Кроме того, борьба с заразными болезнями и с бедностью, концентрировавшейся в городах и вокруг них,
  актуализировала проблему социального служения Церкви.

  Все это находит отражение в таких смелых и новаторских церковных течениях, как францисканство на Западе, и исихазм в Византии и на Руси.

  Любопытно, что оба этих течения подозревали в ереси, но противоречий ортодоксально-хри\-сти\-ан\-ской доктрине в них не нашли:
  францисканство было одобрено папами, а учение Григория Паламы признано догматически верным в Православной Церкви с XIV века.

  С другой стороны, различные неортодоксальные учения (на Руси это были стригольники и жидовствующие,
  унаслидовавшие много учений более ранних сект богумлилов, катаров и др.) тоже критиковали церковную симонию,
  требовали возращения к евангельской простоте и бедности --
  но при этом отвергали авторитет церковной иерархии и Священного Предания.

  Таким образом, возросшее влияние городского населения,
  рост его грамотности стимулировали интерес к личной ответственности человека за свои грехи и добрые дела.
  Простой заказ молитв у духовенства за спасение души стал восприниматься как недостаточный для такого спасения,
  ибо моральный облик духовенства не всегда соответствовал чаяниям новых городских слоев.

  Реакцией на такие запросы стало в ортодоксальных течениях внимание к нравственности духовного сословия,
  стимулирование социально-благотворительной и школьно-педагогической роли Церкви и оживление монашества.
  А в еретических течениях -- попытки радикального отвержения авторитета духовенства и Священного Предания.

  \subsection{Исихазм}
  Исихасты -- дословно переводится как <<молчальники>>, <<безмолвники>> или <<пребывающие в покое>>.

  Ключевой мыслью исихазма является утверждение, что существует путь к сопричастности с Богом,
  который открывается в проявляемых Им Энергиях своей непознаваемой природы.
  Этот путь богопознания связан с тишиной и уединением, не случайно само название духовно-мистического учения,
  которое происходит от слова молчание.
  Представитель школы исихазма занимается безмолвием и молитвой.
  Молитва -- одно из средств познания Бога.
  Кроме того, его отличает аскетический образ жизни, стремление к морально-нравственному идеалу христианства.

  Исихасты стремились сделать для себя молитву настолько же естественной как дыхание.
  Как и все мистики, они считали что с Богом можно говорить напрямую.
  % \teal{без посредников (то есть без церкви), и поэтому фокусировались о личности}

  \subsection{Стригольники}
  Скорее всего, название происходит от профессии одного из первых вождей ереси -- Карпа, который и был собственно стригольником,
  то есть цирюльником, то есть парикмахером.

  Стригольники не были тайной сектой, как, например, <<жидовствующие>> -- другая ересь,
  появившаяся несколько позже.
  Скорее это было направление, объединенное непризнанием существующей иерархии и видимой Церкви.
  % Но его участники делали из этой общей посылки самые разнообразные выводы.

  Надо сказать, что церковная власть, хоть и напуганная появлением стригольников, действовала ещё довольно мягко.
  Казнь Карпа и Никиты была скорее расправой толпы.
  Митрополит Фотий направил четыре послания против ереси, в которых, хоть и призывает к <<силовым методам>>, постоянно подчеркивает,
  что ни в коем случае нельзя допускать смертную казнь -- а лишь заключение, призванное привести еретиков к покаянию.
  К 15 веку не осталось открытых стригольников.


  \subsection{Жидовствующие}
  У этой секты был тайный характер, прекрасно поставленная конспирация.
  Жидовствующие таились много лет, пуская корни в среде высшего духовенства и в окружении государя.

  Сам основатель ереси, киевский еврей, врач, астролог и чернокнижник%
  \footnote{Здесь надо заметить, что все сведенья о ересях этого периода мы можем черпать только из трудов их критиков}
  Схария, жил на территории Крымского ханства.
  Новгородский протопоп Алексий, ещё один представитель жидовствующих, смог покорить своим обаянием и добиться влияния на великого князя Ивана III.

  В 1487 году несколько священников, посвященных в секту, напились и начали публично обвинять друг друга в ереси,
  попутно хуля православные обычаи и святыни.
  Об этом моментально стало известно новгородскому архиепископу Геннадию, которого совершенно потрясла открывшая крамола.
  С этого момента Геннадий Новгородский, будущий святой Русской Церкви,
  становится одним из самых суровых и непримиримых борцов с ересью, также получившей название новгородской.

  Жидовствующие не признавали Иисуса Христа Мессией, и тем более Богом.
  Мессия только должен был прийти; предписания Ветхого Завета сохраняли свой обязательный характер;
  все церковные таинства теряли свой смысл.
  Мощи, иконы, иные святыни прямо отвергались.

  С ересью активно боролись в Новгороде,
  но в Москве, столице нового растущего государства, не только не сопротивлялись ереси, но и прямо ей покровительствовали.

  В 1490 году после смерти престарелого митрополита Московского Геронтия
  жидовствующим удалось возвести на кафедру московских святителей своего ставленника,
  архимандрита Симонова монастыря Зосиму.
  %  -- марионетку Федора Курицына, <<министра иностранных дел>> Московского государства.

  В 1499 году жидовствующие попали в немилость, и вскоре ересь была побеждена.
  % Архиепископ Геннадий и Иосиф Волоцкий -- главные борцы с ересью.
  % \teal{Иосиф -- (вроде) первый русский инквизитор (спорно)}.
  % Иосиф рассуждал иначе чем европейские инквизиторы: если неверные еретики не прельщают никого из православных,
  % то не следует делать им зла и ненавидеть.
  %
  % Так или иначе, ересь оказалась побеждена.
  % Основные ее руководители не дожили до костров 1503-1504 годов, на которые отправились многие их ученики.
  % Но число казненных не исчислялось даже десятками -- еще одно явное отличие от ситуации в просвещенной Европе.
  % А волоколамский игумен Иосиф и его оппонент Нил Сорский были причислены к лику святых -- таково было соломоново решение Церкви,
  % предоставляющей каждому человеку свободу выбора между позицией двух выдающихся деятелей того трагичного и грозного времени.



\end{document}
