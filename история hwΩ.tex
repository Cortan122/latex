\documentclass[12pt]{article}

\usepackage{cmap} % поиск в pdf
\usepackage[english,russian]{babel} % локализация и переносы

\usepackage[hidelinks]{hyperref}
\usepackage{xurl}
\usepackage{enumitem}

\usepackage[margin=.5in]{geometry}

\usepackage{xcolor}
\definecolor{purple}{HTML}{800080}
\newcommand{\blue}[1]{\textcolor{blue}{#1}}
\newcommand{\red}[1]{\textcolor{red}{#1}}
\newcommand{\orange}[1]{\textcolor{orange}{#1}}
\newcommand{\brown}[1]{\textcolor{brown}{#1}}
\newcommand{\teal}[1]{\textcolor{teal}{#1}}
\newcommand{\purple}[1]{\textcolor{purple}{#1}}

\title{История \\ Домашнее задание №$\Omega$}
\author{AAAAA AAAAAAA \\ AAAAAA}
% \date{30 мая 2020 г.}

\begin{document}
  \maketitle

  \section{При каком лидере СССР была принята Третья программа партии (строительства коммунизма)?}
  3) Н. С. Хрущев

  \section{Какой период советской истории получил определение <<застоя>>?}
  3) 1964-1982 гг.

  \section{Кто из ниже перечисленных лидеров был президентом СССР?}
  3) М. С. Горбачев

  \section{Какой из приведенных ниже органов власти советского времени избирался на основе конкурентных выборов?}
  3) Съезд народных депутатов

  \newpage
  \section{Раскройте значение партийных, комсомольских и профсоюзных институций в построении советского общества (1920-1930-е гг.)}
  Сначало надо разобрать каким было советское общество, роль различных организация в построение которого нам надо осветить.

  \subsection{Про идеократию}
  Тот тип социально-политической системы, которая установилась в советском обществе 1920-1930-ых годов,
  называли по-разному, кто -- тоталитаризмом, кто -- идеократией.

  Идеократия -- это общественный строй, основанный на идеях.
  Идеология нацелена на всеобъемлемость своего влияния, и на тотальную всеохватность умов.

  От религиозно-фундаменталистских обществ домодерного типа идеократия отличается тем,
  что общеобязательная в таких государствах идеология формально ищет опору не в сверхъестественном начале,
  что, однако, не противоречит божественному почитанию в идеократиях вождей-основателей, фактически приравненных к пророкам или богам.

  Несмотря на декларируемый атеизм советской коммунистической идеократии, это была в действительности форма новой религии с новыми святыми,
  писанием, преданием, ритуалами и даже мощами -- отсюда и борьба с традиционными религиями, носившая очевидный конкурентный характер.

  \subsection{Про партию}
  Утверждение тоталитарной идеократии в советском обществе 1920-1930ых годов было бы невозможно без учреждения массовых организаций квазирелигиозного типа.
  Недаром Сталин называл партию <<орденом меченосцев>> в одной из своих работ.
  Избавленная в условиях жесткой однопартийной системы от необходимости политической и выборной конкуренции,
  партия большевиков была превращена в <<ядро политической системы>>, механизм привода государственной политики в массы.
  В отличие от авторитарных форм правления, при которых участие масс в политике либо запрещается,
  либо в лучшем случае не рекомендуется, тоталитарные системы требуют под страхом если не репрессий,
  то социальной маргинализации, активного участия в политике именно массовых слоев народа.

  % \subsection{Про всё}
  В качестве школы подготовки таких кадров для участия масс в политике были учреждены молодежные
  и детские политические организации (комсомол, пионеры), а также переучреждены профсоюзные организации.

  Отныне все политическая жизнь страны стала определяться указаниями ЦК и его Политюбро,
  де-факто выполнвяшего функции правительства вплоть до распада СССР.

  \subsection{Про комсомол}
  Комсомол активно участвовал в культурной революции.
  Были созданы <<ударные отряды по ликбезу>>, тысячи комсомольцев влились в ряды <<культармейцев>>.
  Они обучали неграмотных, создавали новые школы ликбеза, открывали читальни и библиотеки.
  В 1930 комсомол взял шефство над всеобучем, выступил инициатором создания двухгодичных вечерних школ для малограмотных.
  В ходе социалистического строительства возникли неотложные проблемы подготовки квалифицированных кадров,
  создания новой, социалистической интеллигенции.
  Комсомол объявил поход молодёжи в науку.

  Очень скоро комсомол остался единственной в РСФСР, а затем и в СССР политической молодёжной организацией.
  Через структуру этой организации осуществлялось идеологическое воспитание молодёжи, и реализовывались политические и социальные проекты.
  Под руководством комсомола в 1922 году была создана детская политическая организация:
  Всероссийская, а позже -- Всесоюзная пионерская организация имени В. И. Ленина.

  Комсомол рассматривается как своеобразное <<министерство молодежи>>, связующее звено между государством и молодежью.
  Он также активно участвовал в борьбе с религией.

  \subsection{Про профсоюзы}
  7-14 (20-27 января) 1918 года на I Всероссийском съезде профсоюзов был избран постоянный руководящий центр профдвижения -- Всероссийский центральный Совет Профессиональных Союзов (ВЦСПС).

  Уже в 1918 году ВЦСПС оказался в фактическом подчинении у органов советской власти.
  Профсоюзные структуры стали помощниками советских органов в изъятии продовольствия.
  При Наркомате продовольствия (Наркомпроде) РСФСР работало Военно-продовольственное бюро ВЦСПС,
  которое совместно с ним руководило рабочими продовольственными отрядами.

  Также как комсомол был <<министерством молодежи>> ВЦСПС был <<министерством трудящихся>>.

  ВЦСПС выполнял следующие функции:
  \begin{enumerate}
    \item участвовал в разработке народно-хозяйственных планов;
    \item руководил социалистическим соревнованием и движением за коммунистический труд;
    \item руководил Всесоюзным обществом изобретателей и рационализаторов, научно-техническими обществами,
    добровольными спортивными обществами профсоюзов, развитием туризма;
    \item создавал профсоюзные школы и курсы;
    \item утверждал профсоюзный бюджет и бюджет государственного социального страхования;
    \item представлял советские профсоюзы в международном профсоюзном движении
    и от их имени входил в международные профсоюзные объединения.
  \end{enumerate}

  \newpage
  \section{Перечислите и кратко охарактеризуйте факторы Победы СССР в Великой Отечественной войне}
  \begin{enumerate}[label=\textbf{\large\arabic*}]
    \item \textbf{Месть за родственников} \\
    Крайняя жестокость фашистов на оккупированных территориях,
    вдохновила многих жителей СССР на месть за своих погибших родственников,
    и стала основной причиной начала партизанской войны.
    Народ понял, что немецкий тоталитаризм ещё хуже советского.

    \item \textbf{Мобилизационные возможности тоталитарной экономики} \\
    % \blue{благодаря тому что экономика плановая все силы были переброшены на снабжение фронта}
    % \teal{а тех ресурсов, которых небыло, (продовольствие) поставлялись ленд-лизом}
    %
    % \teal{Страна была превращена в единый боевой лагерь, подчинённый единственной цели -- победе в Великой Отечественной войне.}
    %
    Лозунг <<Все для фронта, все для победы!>> Выражал сущность программных действий,
    разработанных Советским правительством для превращения страны в единый боевой лагерь,
    подчинённый единственной цели -- победе в Великой Отечественной войне.

    \item \textbf{Географический фактор: суровые зимы и огромные территории} \\
    % Влияние географического фактора (огромная территория) и климатического (суровые зимы).
    Немецкие войска дошли до Москвы только в октябре, это дало время Красной Армии на мобилизацию.
    Зимой 1941-1942 года снег выпал очень рано, и зима была суровой, к этому немецкие войска также не были готовы.
    Огромные территории также позволили провести эвакуацию.

    \item \textbf{Эвакуация в восточные регионы промышленных предприятий и населения} \\
    По опубликованным данным, в 1941-1942 годах на Восток
    были перемещены различными видами транспорта около 17 миллионов человек, более 2600 предприятий.
    Эвакуация позволила спасти огромные производственные и людские ресурсы,
    сыгравшие важнейшую роль в налаживании работы тыла по обеспечению фронтовых нужд.
    Многие эвакуированные заводы дали основу для послевоенного и современного промышленного потенциала ранее слаборазвитых,
    тыловых районов страны.

    \item \textbf{Мощная пропагандистская кампания} \\
    % , развернутая советским тоталитарным государством.
    % \teal{
    %   она настолько мощная что досихпор очень сложно найти описание Победы СССР, где нет ничего про
    %   "{}Духовно-нравственные факторы"{}, "{}Патриотизм Советского народа"{} и "{}Веру в победу Красной Армии"{}
    % }
    Этим занималось Управление пропаганды и агитации ЦК ВКП(б) и
    Отделение по работе с войсками противника Главного политического управления РККА.
    Пропаганда была оборотной стороной силового воздействия воюющего государства,
    занятого одновременно вооружённым противоборством на фронте и обеспечением эффективной работы тыла.
    С началом войны жители втянутых в неё стран столкнулись с опасностью потери здоровья и жизни,
    резким снижением уровня потребления, разрушением привычной социальной среды.
    В пропагандистских целях широко использовались военные успехи.

    \item \textbf{Блокировка фашистской пропаганды} \\
    %\teal{Ярким прмеров блокировки фашистской пропаганды была конфискация радиоприёмников:}
    До Великой Отечественной войны радиолюбительство было одним из популярных увлечений в СССР,
    хотя каждый приёмник в стране подлежал регистрации и за него нужно было платить абонентскую плату.
    Однако после 22 июня 1941 года пользование радиоаппаратурой оказалось под запретом.

    К тому времени в пользовании жителей Советского Союза находилось больше 1 млн радиоприёмников.
    Владельцам было предписано в пятидневный срок принести их <<на временное хранение>> в местные органы Наркомата связи.
    Некоторые изъятые аппараты, возможно, были переустроены для нужд армии и партизанского движения.

    Вместе с тем боязнь вражеской пропаганды оказалась преувеличенной.
    В Третьем рейхе совершенно не представляли себе реального уровня развития радиолюбительства в СССР.
    Нацисты полагали, что кроме проводного радио у населения практически ничего нет.

    \item \textbf{Экономическая и военно-техническая помощь союзников\footnote{lend-lease или ленд-лиз}} \\
    Размер помощи Советскому Союзу со стороны Соединённых Штатов в рамках ленд-лиза по годам достиг колоссальных размеров.
    Всего Советский Союз получил помощи на 9.4 миллиардов долларов, из них 41.15\% военного снаряжения.
    С расходами на перевозку помощь Соединённых Штатов достигла 11.3 миллиардов долларов.
    СССР выиграл необходимое время для перебазирования военной и прочей промышленности вглубь страны
    и закрывал жизненно важные для ведения войны на то время <<узкие места>> в снабжении армии и промышленности,
    для чего советским правительством самим определялась номенклатура желаемых поставок по ленд-лизу.

    Стоит отметить, что в советской историограифи роль помощи США значительно преуменьшалась, так
    в 1947 году председатель Госплана СССР Николай Вознесенский в книге <<Военная экономика СССР в период Отечественной войны>> сообщил,
    что поставки союзников составляли 4\% от советского производства военного времени.
    Это была <<лукавая цифра>>.
    И не только потому что объем поставок был преуменьшен почти в два раза.
    Замалчивалось ключевое значение заграничного снабжения для отдельных видов производства военной продукции.

    \item \textbf{Заградительные отряды} \\
    Одной из малоизученных страниц истории Великой Отечественной войны является деятельность заградительных отрядов.
    В советское время этот вопрос был покрыт завесой секретности.
    Подобный порядок сохранялся и после окончания войны.
    % Неудивительно, что с началом перестроечных <<разоблачений>> в общественном мнении был сформирован некий зловещий образ
    % <<палачей из НКВД>>, расстреливавших отступающих красноармейцев из пулемётов.

    Заградительные отряды -- это отряды, предотвращавшие отступление войск.
    Они размещались в ближайшем прифронтовом тылу для поддержания дисциплины.
    Эти отряды были обязаны расстреливать шпионов, дезертиров и солдат, бежавших с поля боя.

    Сталин заимствовал идею заградительных отрядов у Троцкого, который использовал их во время гражданской войны
    и обосновал их тем, что <<красноармеец должен быть поставлен в условия выбора между возможной почетной смертью в бою...
    и неизбежной позорной смертью расстрела, если бросит позицию и побежит назад>>.

    \item \textbf{Крайняя жестокость советского руководства к собственным солдатам} \\
    Войска шли в атаку, движимые ужасом.
    Ужасна была встреча с немцами, с их пулеметами и танками, огненной мясорубкой бомбежки и артиллерийского обстрела.
    Не меньший ужас вызывала неумолимая угроза расстрела.
    Чтобы держать в повиновении аморфную массу плохо обученных солдат, расстрелы проводились перед боем.

    Ещё, Например, солдат, попавших в плен, объявляли врагами народа.
    Это делалось чтобы они боялись попасть в плен больше смерти.
    С самого начала Великой Отечественной войны под подозрение в предательстве попали все военнослужащие и гражданские лица,
    оказавшиеся даже на непродолжительное время за линией фронта.
    Расширялась практика заочного осуждения военнослужащих, находившихся за линией фронта, как изменников Родины.
    Достаточным основанием для такого решения были полученные оперативным путём сведения об их якобы антисоветской деятельности.
    Вердикт выносился без всякой проверки, иногда лишь по одному заявлению.

    \item \textbf{Зацикленность Гитлера на собственной нацистской идеологии} \\
    Например, он отказался распустить колхозы, хотя ему советовали сделать это для того, чтобы получить поддержку населения.
    Но он считал, что колхозы идеальны как способ порабощения русского крестьянства.
    Очень напоминает позицию Наполеона в войне 1812 года.
    Когда ему посоветовали освободить крестьян, он ответил: <<Я не хочу быть королем Жакерии>>.
    Жакерия -- крестьянское восстание в средневековой Франции.
    Гитлер открыл некоторые церкви на оккупированных территориях, но Сталин сделал тоже самое,
    чем полностью нивелировал пропагандистский эффект от открытия церквей Гитлером.
  \end{enumerate}

\end{document}
