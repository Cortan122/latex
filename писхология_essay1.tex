\documentclass[a4paper,11pt]{article}
% LaTeX template for the essays for Cognitive Psychology at AI at the RUG.
% Initially made by Ren\'e Mellema and Jacob Schoemaker.
% V1.1 (05/03/2019) Fixed some small errors in spacing and added more
% explanation as to how spacing around floats works
% V1 (04/03/2019) Initial version
% Set the margins
\usepackage[margin=2.54cm, rmargin=4.9cm]{geometry}
\usepackage{graphicx}
% For setting the page number
\usepackage{fancyhdr}
\usepackage{helvet}
% For setting the line spacing
\usepackage{setspace}
% For changing the headers
\usepackage{titlesec}
% For changing the figure environment
\usepackage[position=below, % Captions go below
            skip=0pt, % Controls the space between figures and captions
            labelsep=period, % Use a . as the seperator
            labelfont=bf, % Put the label in bold
            font={small,stretch=1.2}, % Change fontsize and line spacing
            ]{caption}
% For getting rid of the hypenation
\usepackage[none]{hyphenat}
% Use the APA style
\usepackage[hidelinks,bookmarks=false]{hyperref}
\usepackage{apacite}
% If you want to use hyperref, include it before APAcite

% Set the font to Helvetica
\renewcommand\familydefault{\sfdefault}
\usepackage[T1]{fontenc}
\usepackage[document]{ragged2e}

\pagestyle{fancy}
% Clear the header and footer
\fancyhead{}
\renewcommand{\headrulewidth}{0pt}
\fancyfoot{}
% Set the right side of the footer to be the page number
\fancyfoot[R]{\thepage}
\fancyfootoffset[R]{3cm}

% Set up the paragraphs correctly.
\setlength\parindent{0pt} % Turn of indenting
\setlength\parskip{6pt}   % Change the space between paragraphs
% Set the line spacing
\setstretch{1.5}
% Turns off numbering for sections.
\setcounter{secnumdepth}{0}

% Changing the section headers
\titlespacing*{\section}{0pt}{6pt}{0pt}
% Less space around floats
\setlength{\textfloatsep}{6pt plus 1pt minus 1pt}
% space between float on top of page and text
\setlength\floatsep{6pt plus 1pt minus 1pt}
% space between two floats
\setlength\intextsep{6pt plus 1pt minus 6pt}
% space between a float and surrounding text (using h)

\begin{document}
  \begin{singlespace}
    \large
    {\bf \AA{AAAAAAAAAA AAAAAAA}{6} \AA{AAAAAAAA}{9}} \\[12pt]
    {\huge\bf ADHD in the 21st century: the role of the environment in ADHD} \\[6pt]
    \textbf{Chapter:} 5 \\
    \textbf{Date:} March 2022
  \end{singlespace}
  \vspace{-\parskip}

  \section{Introduction}
  ADHD, which stands for ``attention-deficit/hyperactivity disorder'',
  is a neurological developmental disability \cite{adolescents},
  that effects about 5\% of the world's population \cite{underdiagnosis}.
  The disorder is expressed differently in adults and in children:
  during childhood the hyperactivity related symptoms are more pronounced,
  while in adulthood problems related to inattention become more noticeable \cite{underdiagnosis}.

  ADHD can vary wildly in severity, and, especially in children, be misused as a catch-all label for active kids.
  In particularly severe cases it can be treated with Ritalin, a powerful stimulant,
  which may be unintuitive, considering ADHD is characterized by hyperactivity.
  One possible explanation is that Ritalin stimulates the neural pathways responsible for
  inhibiting our reactions to distracting stimuli \cite{thebook}.

  From the patient's perspective, getting an official ADHD diagnosis can be difficult \cite{diagnosisdifficulty}.
  As a result, many adolescents, as well as parents of children, suspecting they might have ADHD,
  seek out information online \cite{informationseeking}.
  There is no scientific consensus on whether ADHD is overdiagnosed or underdiagnosed,
  but most agree that it is often misdiagnosed \cite{overorunderdiagnosed}.

  The prevalence of mild cases and potential overdiagnosis can make Ritalin an unappealing treatment option.
  Prescribing such a powerful drug can often do more harm then good \cite{thebook}.
  Despite the drug's track record of relative long term safety, proven by 70 years of research,
  it remains widely controversial \cite{controversy}.
  Since 90\% of ADHD patients currently treated with stimulants,
  would instead benefit more form physiological therapy,
  the current use of Ritalin as a first-line treatment is expectedly met with objections \cite{controversy}.

  % With official recommendations recommending Ritalin as a first-line treatment,
  % it is no wonder that there are objections, % fixme
  % as up to 90\% of ADHD patients currently treated with stimulants,
  % would instead benefit more form physiological therapy \cite{controversy}.

  \section{Findings}
  Many people effected by ADHD describe their disorder as an
  ``interaction between themselves and their environment''.
  Whether certain symptomatic behaviors are considered dysfunctional
  is highly dependent on the expectations placed upon the subject by their environment and culture \cite{adhdcontext}.
  For example, distractibility diminishes significantly when tasks are altered to make them more stimulating \cite{delisle2011context}.

  This gives credence to an alternative model of ADHD.
  Rather then an inherent ``attention deficit'' in all situations,
  it emphasizes a lack of long term motivation, refereed to as ``motivational salience'',
  becoming bored easily when performing uninteresting or underwhelming tasks \cite{adhdcontext}.
  Thus, in contexts providing frequent motivational feedback and requiring no long term planning,
  subjects were no longer impaired by ADHD \cite{delisle2011context}.

  In the 21st century such highly motivational environments are provided by video games and social media apps.
  As our society becomes more integrated with technology,
  life without a smartphone becomes increasingly difficult.
  We are forced to carry a constant source of distractions everywhere we go.
  This disproportionately effects ADHD patients,
  and, as a result, they are more likely to develop video game addiction \cite{gameaddiction}.

  The cost-optimized American education system,
  to increase classroom size to 40 children per class,
  places heavy expectations on the behavior of its students,
  which patients with ADHD have a hard time meeting.
  The classroom curriculum is also designed to match the learning ability of the weakest students,
  which creates a severely under-stimulating environment \cite{controversy}.
  In children with ADHD, this leeds to a cascade effect:
  due to poor academic performance they are 4 times more likely to get held back,
  causing their environment to become even less stimulating \cite{heldback}.

  Compared to adults, children have very little control over their environment \cite{adhdcontext}.
  Even simple alterations to a child's physical surroundings,
  like, for example, replacing distractions with helpful reminders,
  seem to be helpful \cite{thebook}.
  Waiving arbitrary classroom rules, providing alternative means of accomplishing different kinds of homework tasks,
  and, most importantly, providing more attention to each individual student
  can significantly reduce the effect of ADHD on academic performance \cite{ervin1998classroom}.

  Insuring a welcoming home environment is very important, as children spend most of their time at home.
  Addressing poor parenting practices, which may arise as
  low academic performance leads parents to mischaracterize their child as ``lazy'' or ``stupid'',
  is possible with behavioral parent training (BPT).
  BPT involves a sequence of weakly therapy sections visited by the parents and optionally their child,
  over the course of which the therapist teaches a series of different parenting techniques appropriate for children with ADHD.
  Examples of such techniques may include
  ignoring minor rule violations,
  explicitly declaring unwritten rules, and
  setting up a point-based system to track good and bad behavior \cite{parenttraining}.

  Another non-pharmaceutical treatment option is cognitive behavioral therapy (CBT).
  CBT has been demonstrated as an effective treatment not only for ADHD but also for many of its comorbidities.
  It can help patients establish coping strategies to deal with changes in their environment.
  Group CBT can also create a sense of shared identity, through the sharing of personal experiences with other people hwo have ADHD
  \cite{bramham2009evaluation}.

  In some contexts ADHD symptoms can ``become strengths rather than liabilities'' \cite{adhdcontext}.
  This, along with its abnormally high frequency, suggests that the ``disorder''
  may be evolutionarily advantageous \cite{jensen1997evolution}.

  \section{Discussion \& Conclusion}
  The environment plays a major role in how patients with ADHD experience their symptoms.
  As a result, it can be an enticing target for possible treatment.
  Unlike medication, treatments that effect the environment and how it is perceived by the patient, can have long lasting effects.
  ADHD is an incurable chronic condition; condemning a patient to take pills for the rest of their life seem cruel.

  The treatment landscape of ADHD has improved considerably over the corse of the 21st century,
  but there is still a lot of room for improvement.
  Further research is required into the efficacy of combined medical and psychological therapy,
  and America needs to stop using stimulants with harmful side effects as a first-line treatment option.
  The situation is complicated by the fact that the exact cause of ADHD remains unknown.

  % In this section you state the implications of the findings for what was
  % described in the book. This should take about 250 words (but there is \textbf{no
  % word limit nor minimum number of words} for this section).

  % For this course, you are required to use APA style. Using hanging
  % indentation (1 cm), as shown below. The references do not add to the word
  % count.

  \nocite{ref2}
  \bibliographystyle{apacite}
  \bibliography{references}

\end{document}
