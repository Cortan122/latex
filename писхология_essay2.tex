\documentclass[a4paper,11pt]{article}
% LaTeX template for the essays for Cognitive Psychology at AI at the RUG.
% Initially made by Ren\'e Mellema and Jacob Schoemaker.
% V1.1 (05/03/2019) Fixed some small errors in spacing and added more
% explanation as to how spacing around floats works
% V1 (04/03/2019) Initial version
% Set the margins
\usepackage[margin=2.54cm, rmargin=4.9cm]{geometry}
\usepackage{graphicx}
% For setting the page number
\usepackage{fancyhdr}
\usepackage{helvet}
% For setting the line spacing
\usepackage{setspace}
% For changing the headers
\usepackage{titlesec}
% For changing the figure environment
\usepackage[position=below, % Captions go below
            skip=0pt, % Controls the space between figures and captions
            labelsep=period, % Use a . as the seperator
            labelfont=bf, % Put the label in bold
            font={small,stretch=1.2}, % Change fontsize and line spacing
            ]{caption}
% For getting rid of the hypenation
\usepackage[none]{hyphenat}
% Use the APA style
\usepackage[hidelinks,bookmarks=false]{hyperref}
\usepackage{apacite}
% If you want to use hyperref, include it before APAcite

% Set the font to Helvetica
\renewcommand\familydefault{\sfdefault}
\usepackage[T1]{fontenc}
\usepackage[document]{ragged2e}

\pagestyle{fancy}
% Clear the header and footer
\fancyhead{}
\renewcommand{\headrulewidth}{0pt}
\fancyfoot{}
% Set the right side of the footer to be the page number
\fancyfoot[R]{\thepage}
\fancyfootoffset[R]{3cm}

% Set up the paragraphs correctly.
\setlength\parindent{0pt} % Turn of indenting
\setlength\parskip{6pt}   % Change the space between paragraphs
% Set the line spacing
\setstretch{1.5}
% Turns off numbering for sections.
\setcounter{secnumdepth}{0}

% Changing the section headers
\titlespacing*{\section}{0pt}{6pt}{0pt}
% Less space around floats
\setlength{\textfloatsep}{6pt plus 1pt minus 1pt}
% space between float on top of page and text
\setlength\floatsep{6pt plus 1pt minus 1pt}
% space between two floats
\setlength\intextsep{6pt plus 1pt minus 6pt}
% space between a float and surrounding text (using h)

% \hfuzz=10pt
% \overfullrule=10pt

\begin{document}
  \begin{singlespace}
    \large
    {\bf \AA{AAAAAAAAAA AAAAAAA}{6} \AA{AAAAAAAA}{9}} \\[12pt]
    {\huge\bf Animal Language} \\[6pt]
    \textbf{Chapter:} 10 \\
    \textbf{Date:} March 2022
  \end{singlespace}
  \vspace{-\parskip}

  \section{Introduction}
  It is widely hypothesized that humans have an innate propensity for spoken language,
  however this hypothesis is controversial.
  Many researchers claim that the ``FOXP2'' gene,
  which in vertebrates plays a crucial role in complex sound processing,
  like song mimicry in birds and echolocation in bats \cite{campbell2009conservation},
  is responsible for our unique capacity for language learning,
  as mutations in this gene cause a severe speech disorder \cite{thebook2}.
  But recent research suggests that this gene experienced no recent evolutionary pressure,
  so it can't possibly be the only gene necessary for language development \cite{atkinson2018no}.

  Many non-human animals have sophisticated communication systems,
  but they are not general and flexible enough to be considered ``languages'' \cite{thebook2}.
  For the most part, they are limited to a small vocabulary of fitness-relevant messages,
  like alarm calls, which are produced as a direct response to external stimuli \cite{fitch2020animalcognition}.
  These systems also lack syntax and grammar,
  or some other way of combining multiple messages into more complex ones \cite{thebook2}.
  In contrast, human languages are inherently recursive:
  phonemes are combined to form words, words are combined to form sentences,
  and sentences are combined to form narratives \cite{bowling2015animal}.

  %fixme
  You might also think that human language is unique in its diversity and the fact that it needs to be learned.
  There are over 7151 distinct human languages discovered so far, and this number keeps growing each year \cite{ethnologue}.
  But a recent study in vervet monkeys shows that this is not the case:
  their alarm calls for different predictors are non-innate and have to be learned through experience \cite{liska1993bee}.

  Over the years, many researchers tried to teach human language to non-human animals,
  but those experiments are widely regarded as controversial \cite{thebook2}.
  Can this really be considered ``language''?
  And which experiment achieved the best results?

  \section{Findings}
  Many human dog owners swear that their dog understands them,
  as the dogs respond to their names, or words like ``walk'' or ``outside'' \cite{jordania2006asked}.
  But, just because dogs can learn to recognize certain words or commands, does not necessarily mean they can understand them.
  These words are learned in the exact same way as commands: through operant conditioning \cite{mckinley2003efficacy}.
  Dogs learn that if they react to the word ``outside'', the owner is more likely to actually take them outside. %fixme
  A dog, who knows the command ``sit'', can definitely associate hearing the word ``sit'' with the action of sitting,
  but whether or not it ``thinks'' of the word when it sits of its own accord is impossible to know.

  In humans language is closely connected to cognition:
  we use it, not only to communicate, but also to combine and manipulate concepts
  in a phenomenon known as inner speech \cite{fitch2020animalcognition}.
  In fact most of our thoughts are associated with either words or images,
  and for some people having a wordless thought is impossible \cite{vicente2016unsymbolized}.
  Thus it makes sense to think about animal language not just as animal communication systems, which are very limited,
  but more as an equal combination of cognition and communication.
  We know for a fact that animals have a significantly higher amount of concepts then what they are able to communicate \cite{fitch2020animalcognition}.

  When evaluating the success of ``language training'' experiments we have to consider whether or not the animals
  understand what they are saying,
  register those human-to-animal interactions as communication,
  and are able to connect words with their internal conceptual representation of the world. %\cite{myass}
  Language training experiments are most often attempted on apes, as they are closely related to humans \cite{thebook2}.
  Researchers can reliably train apes to mimic around 100 signs of the American Sign Language (ASL),
  but the animals are unable to form coherent sentences \cite{hungape}.
  Some particularly prominent critics claim that,
  due to over-interpretation and confirmation bias on the side of the trainers, who did not speak ASL,
  the apes did not learn to mimic specific signs,
  but learned how to flail in a way that resembles ASL in general \cite{wynne2007aping}.

  Another promising example is the case of Alex, the African Grey Parrot.
  Unlike the aforementioned ape experiments with ASL, Alex was taught English words \cite{wilson2021clever}.
  This reduced the effects of over-interpretation and confirmation bias,
  as humans are very good at auditorily distinguishing words from non-words \cite{connine1990word}.
  Other measures were taken to further reduce bias:
  different people were used for testing and training,
  which reduced the prevalence of the Clever Hans effect,
  an effect where instead of solving a problem on its own,
  the animal might be picking up on unintentional clues from its trainers \cite{cleverhanseffect}.
  Additionally only intrinsic motivators were used
  to avoid confusing the targeted concepts with a food reward \cite{wilson2021clever}.

  Alex's accomplishments include being able to count to six, recognize seven colors, and distinguish 50 different objects.
  Alex is also the first and only non-human animal to have ever asked a question \cite{jordania2006asked}.
  He made requests spontaneously, not only in response to external stimuli,
  and showed frustration when his requests were followed incorrectly,
  suggesting a thorough understanding of their content.
  Despite all these seemingly remarkable achievements,
  Pepperberg, the researcher responsible for these experiments,
  never claimed that Alex has learned language, instead referring to it as a ``two-way communication'' system \cite{wilson2021clever}.

  \section{Discussion \& Conclusion}
  In his book, \textit{Cognition: Exploring the Science of the Mind}, Reisberg \citeyear{thebook2}
  claims that the success of language training studies greatly correlates with brain size, and genetic similarity to humans.
  But a more critical view of the experiments suggests otherwise.
  It is true that the closer an animal is related to humans, the stronger its cognitive abilities,
  sometimes exceeding humans in specific tasks \cite{fitch2020animalcognition}.
  However, in language training experiments, Alex showed accomplishments, comparable to those of apes,
  despite his significantly smaller brain size.
  Terrace \citeyear{terrace1983apes}, the head of the Nim Chimpsky experiment,
  argued that the same ``linguistic'' skills can be taught to a pigeon through simple operant conditioning.

  We can conclude that animal communication systems,
  whether natural or imposed by humans in language training experiments,
  can not be considered language, as they lack close integration with cognitive processes,
  which is a crucial characteristic of human language \cite{fitch2020animalcognition}.

  \nocite{ref2}
  \bibliographystyle{apacite}
  \bibliography{references}

\end{document}
