\documentclass{article}

\usepackage{cmap} % поиск в pdf
\usepackage{mathtext} % русские буквы в формулах
\usepackage[english,russian]{babel} % локализация и переносы
\usepackage[T2A]{fontenc} % кодировка в pdf (магия)
\usepackage[utf8]{inputenc} % кодировка исходного текста
\usepackage{amsmath}
\usepackage{amsfonts}
\usepackage{amssymb}
\usepackage{gensymb}
\usepackage{headertable}

\usepackage[hidelinks,bookmarks=false]{hyperref}
\usepackage{xurl}
\usepackage[margin=0.5in]{geometry}
\usepackage{graphicx}
\usepackage{tikz}
\usepackage{wrapfig}
\usepackage{textcomp}

\newcommand{\notimplies}{\mathrel{{\ooalign{\hidewidth$\not\phantom{=}$\hidewidth\cr$\implies$}}}}
\newcommand{\VCenter}[2]{\vcenter{\hbox{\scalebox{#1}{$#2$}}}}

\newcommand{\ds}{\displaystyle}
\newcommand{\DS}{\phantom{$0.5$}}
\newcommand{\N}{\mathbb{N}}
\newcommand{\Z}{\mathbb{Z}}
\newcommand{\Q}{\mathbb{Q}}
\newcommand{\R}{\mathbb{R}}
\newcommand{\range}{\underline}
\newcommand{\Aleph}{2^{\aleph_0}}
\newcommand{\Cnk}[2]{C_{#1}^{#2}}
\renewcommand{\f}{\frac}
\renewcommand{\l}{\left}
\renewcommand{\r}{\right}
\renewcommand{\P}[1]{\mathcal{P}\l(#1\r)}
\renewcommand{\emptyset}{\varnothing}

\usepackage{xcolor}
\definecolor{red}{HTML}{d62728}
\definecolor{orange}{HTML}{ff7f0e}
\definecolor{green}{HTML}{2ca02c}
% \definecolor{blue}{HTML}{1f77b4}
\newcommand{\red}[1]{{\color{red}{#1}}}
\newcommand{\orange}[1]{{\color{orange}{#1}}}
\newcommand{\green}[1]{{\color{green}{#1}}}
\newcommand{\blue}[1]{{\color{blue}{#1}}}

% \pagecolor{black}
% \color{white}

\title{Дискретная математика \\ Домашнее задание №3}
\author{AAAAA AAAAAAA \\ AAAAAA}

\begin{document}
  \maketitle
  \HeaderTable{13}

  \section{Сколько различных натуральных делителей у числа $1224$?}
  $1224 = 2^3 \cdot 3^2 \cdot 17$ \\
  а количество различных натуральных делителей это всегда\cite{divisor} произведение всех степеней из prime factorization,
  к которым мы прибавили один. \\
  ответ: $(3+1)\cdot(2+1)\cdot(1+1) = 4\cdot3\cdot2 = \fbox{24}$

  \section{Сколько есть способов разложить $7$ различных монет по $3$ различным карманам?}
  каждая из семи монеток может быть в ровно одном кармане. получается это типа как семизначное число в троичной системе.
  если формально, это будет количество функций (отображений) из $\range{7}$ в $\range{3}$. \\
  ответ: $3^7 = \fbox{2187}$

  \section{Рассмотрим слово $\alpha = \texttt{cbcad}$. Сколько слов длины от $1$ до $5$, составленных их букв латинского алфавита (и необязательно осмысленных), стояли бы прежде $\alpha$ в словаре?}
  нам сначала надо найти количество слов длины $5$ стоящих до $\alpha$.
  для этого нам надо $\texttt{cbcad}$ представить как число в 26ричной системе, тк в латинском алфавите $26$ буков.
  количество чисел меньше когото числа равно его значению,
  а $\texttt{cbcad}_{26} = \texttt{21203}_{26} = 3+26^2\cdot2+26^3+26^4\cdot2 = 932883$.
  теперь нам надо найти все слова длины $4$ стоящие до $\alpha$. это будут все числа меньше $\texttt{cbca}_{26}+1$.
  и также будет для других длин.
  в ответе у нас будет $ \texttt{21203}_{26} + \texttt{2120}_{26}+1 + \texttt{212}_{26}+1 + \texttt{21}_{26}+1 + \texttt{2}_{26}+1$.
  если мы это всё посчитаем \underline{\textit{\textbf{\Large на питоне}}} то будет $\fbox{970202}$.

  \section{Пусть $n \in \N$. Найдите выражение в замкнутом виде для суммы:}
  \subsection{$\ds \sum_{k=0}^n\l(-1\r)^k\Cnk{n}{k}$}
  тут надо сначала рассмотреть случай с $n=0$, потомучто в условие так написано.
  там будет $1$ тк $\Cnk{x}{0}$ всегда $1$. это будет также во всех пунктах этого номера.
  нулевой элемент у нас всегда $1$.
  потом нам надо рассмотреть случаи с $n \operatorname{mod} 2 = 1$. там получается всё симметрично,
  потомучто $\ds \Cnk{n}{k}=\Cnk{n}{n-k}$ и там получается что знаки у этих равных элементов разные.
  тут вообще надо нарисовать треугольник паскаля, ато ничего непонятно. \\
  \setlength\tabcolsep{0pt}
  \begin{tabular}{lccccccccccccc}
    $n=0$\hspace{12pt} &\hphantom{-10}&\hphantom{-10}&\hphantom{-10}&\hphantom{-10}&\hphantom{-10}&\hphantom{-10}&1&\hphantom{-10}&\hphantom{-10}&\hphantom{-10}&\hphantom{-10}&\hphantom{-10}&\hphantom{-10} \\
    $n=1$\hspace{12pt} &&&&&&1&\hphantom{-10}&-1&&&&& \\
    $n=2$\hspace{12pt} &&&&&1&&-2&&1&&&& \\
    $n=3$\hspace{12pt} &&&&1&&-3&&3&&-1&&& \\
    $n=4$\hspace{12pt} &&&1&&-4&&6&&-4&&1&& \\
    $n=5$\hspace{12pt} &&1&&-5&&10&&-10&&5&&-1& \\
    $n=6$\hspace{12pt} &1&&-6&&15&&-20&&15&&-6&&1 \\
  \end{tabular}
  \setlength\tabcolsep{6pt} \\
  мы тут видим что везде кроме $n=0$ у нас получается $0$, но непонятно почему это происходит в чётных $n$.
  теперь нам осталось доказать, что у нас в чётных всё работает выразив их через нечётные.
  мы для этого выражаем каждый элемент через сумму двух элементов над ним. \\
  $ \red{\Cnk{2n}{0}} - \green{\Cnk{2n}{1}} + \blue{\Cnk{2n}{2}} -\cdots+ \red{\Cnk{2n}{2n}}
  = \red{0+\Cnk{2n-1}{0}} - \green{\Cnk{2n-1}{0}-\Cnk{2n-1}{1}} + \blue{\Cnk{2n-1}{1}+\Cnk{2n-1}{2}} -\cdots+ \red{\Cnk{2n-1}{2n-1}+0} $ \\
  теперь если мы перерасставим скобки (сгруппируем слагаемые так чтобы они взаимно уничтожались),
  то у нас тут будет много пар где везде нули (и да рандомных нуля с краю), и сумма тоже будет ноль. \\
  $0+\l(\red{\Cnk{2n-1}{0}} \green{-\Cnk{2n-1}{0}}\r)+\l(\green{-\Cnk{2n-1}{1}} \blue{+\Cnk{2n-1}{1}}\r)+\cdots+\l(\orange{-\Cnk{2n-1}{2n-1}} \red{+\Cnk{2n-1}{2n-1}}\r)+0$
  \\ ответ: \fbox{$\begin{cases}
    1 & \text{если } n=0 \\
    0 & \text{если } n \neq 0 \\
  \end{cases}$}

  \subsection{$\ds \sum_{2|k}\Cnk{n}{k}$}
  тут, чтобы избавится от непонятной нотации, мы запишем это как $\ds \sum_{k=0}^{2\l\lfloor n/2 \r\rfloor}\Cnk{n}{k}$.
  теперь можно рисовать треугольник. \\
  \setlength\tabcolsep{0pt}
  \begin{tabular}{lccccccccccccc}
    $n=0$\hspace{12pt} &\hphantom{10}&\hphantom{10}&\hphantom{10}&\hphantom{10}&\hphantom{10}&\hphantom{10}&1&\hphantom{10}&\hphantom{10}&\hphantom{10}&\hphantom{10}&\hphantom{10}&\hphantom{10} \\
    $n=1$\hspace{12pt} &&&&&&1&\hphantom{10}&0&&&&& \\
    $n=2$\hspace{12pt} &&&&&1&&0&&1&&&& \\
    $n=3$\hspace{12pt} &&&&1&&0&&3&&0&&& \\
    $n=4$\hspace{12pt} &&&1&&0&&6&&0&&1&& \\
    $n=5$\hspace{12pt} &&1&&0&&10&&0&&5&&0& \\
    $n=6$\hspace{12pt} &1&&0&&15&&0&&15&&0&&1 \\
  \end{tabular}
  \setlength\tabcolsep{6pt} \\
  из того что каждый элемент треугольника паскаля это сумма двух стоящих над ним,
  мы получаем что сумма каждого второго элемента это сумма всех элементов предыдущей строки.
  мы также знаем что сумма всех элементов $n$-ой строки это $2^n$.
  тоесть у нас тут $2^{n-1}$. \\
  ответ: \fbox{$\begin{cases}
    1 & \text{если } n=0 \\
    2^{n-1} & \text{если } n \neq 0 \\
  \end{cases}$}

  \subsection{$\ds \sum_{k=0}^n\f{\Cnk{n}{k}}{k+1}$}
  % \setlength\tabcolsep{0pt}
  % \begin{tabular}{lccccccccccccc}
  %   $n=0$\hspace{12pt} &\hphantom{$\f{10}{6}$}&\hphantom{$\f{10}{6}$}&\hphantom{$\f{10}{6}$}&\hphantom{$\f{10}{6}$}&\hphantom{$\f{10}{6}$}&\hphantom{$\f{10}{6}$}&1&\hphantom{$\f{10}{6}$}&\hphantom{$\f{10}{6}$}&\hphantom{$\f{10}{6}$}&\hphantom{$\f{10}{6}$}&\hphantom{$\f{10}{6}$}&\hphantom{$\f{10}{6}$} \\
  %   $n=1$\hspace{12pt} &&&&&&1&\hphantom{$\f{10}{6}$}&$\f{1}{2}$&&&&& \\
  %   $n=2$\hspace{12pt} &&&&&1&&$\f{2}{2}$&&$\f{1}{3}$&&&& \\
  %   $n=3$\hspace{12pt} &&&&1&&$\f{3}{2}$&&$\f{3}{3}$&&$\f{1}{4}$&&& \\
  %   $n=4$\hspace{12pt} &&&1&&$\f{4}{2}$&&$\f{6}{3}$&&$\f{4}{4}$&&$\f{1}{5}$&& \\
  %   $n=5$\hspace{12pt} &&1&&$\f{5}{2}$&&$\f{10}{3}$&&$\f{10}{4}$&&$\f{5}{5}$&&$\f{1}{6}$& \\
  %   $n=6$\hspace{12pt} &1&&$\f{6}{2}$&&$\f{15}{3}$&&$\f{20}{4}$&&$\f{15}{5}$&&$\f{6}{6}$&&$\f{1}{7}$ \\
  % \end{tabular} \hfill
  % \begin{tabular}{lccccccccccccc}
  %   $n=0$\hspace{12pt} &\hphantom{$\f{10}{6}$}&\hphantom{$\f{10}{6}$}&\hphantom{$\f{10}{6}$}&\hphantom{$\f{10}{6}$}&\hphantom{$\f{10}{6}$}&\hphantom{$\f{10}{6}$}&1&\hphantom{$\f{10}{6}$}&\hphantom{$\f{10}{6}$}&\hphantom{$\f{10}{6}$}&\hphantom{$\f{10}{6}$}&\hphantom{$\f{10}{6}$}&\hphantom{$\f{10}{6}$} \\
  %   $n=1$\hspace{12pt} &&&&&&1&\hphantom{$\f{10}{6}$}&$\f{1}{2}$&&&&& \\
  %   $n=2$\hspace{12pt} &&&&&1&&1&&$\f{1}{3}$&&&& \\
  %   $n=3$\hspace{12pt} &&&&1&&$\f{3}{2}$&&1&&$\f{1}{4}$&&& \\
  %   $n=4$\hspace{12pt} &&&1&&2&&2&&1&&$\f{1}{5}$&& \\
  %   $n=5$\hspace{12pt} &&1&&$\f{5}{2}$&&$\f{10}{3}$&&$\f{5}{2}$&&1&&$\f{1}{6}$& \\
  %   $n=6$\hspace{12pt} &1&&3&&5&&5&&3&&1&&$\f{1}{7}$ \\
  % \end{tabular}
  % \setlength\tabcolsep{6pt} \\
  $\ds \sum_{k=0}^n\f{\Cnk{n}{k}}{k+1} = \sum_{k=0}^n\f{n!}{k!(n-k)!(k+1)} = \sum_{k=0}^n\f{n!}{(k+1)!\l((n+1)-(k+1)\r)!}
  = \l(\sum_{k=0}^n \Cnk{n+1}{k+1}\r)\f{1}{n+1} = \l(-1+\sum_{k=0}^{n+1} \Cnk{n}{k}\r)\f{1}{n+1} = $
  \fbox{$\ds \f{2^{n+1}-1}{n+1}$}

  \section{Каждую секунду робот делает один шаг направо или один шаг вверх. Сколькими различными путями он может попасть в точку $(m, n)$ из точки $(0, 0)$ при $m,n \in \N$?}
  чтобы робот нашёл нашу точку надо чтобы он сделал $m$ шагов направо и $n$ шагов вверх.
  тоесть он всего должен сделать $n+m$ шагов и $n$ из них должны быть вверх (ну а все остальные направо).
  а сколько у нас есть способов \textit{выбрать} $n$ из $n+m$¿? \\
  ответ: \fbox{$\Cnk{n+m}{n}$}

  \section{Рассмотрим $n > 1$ идущих подряд целых чисел: $a, a+1, \dots, a+(n-1)$. Докажите, что их произведение всегда кратно $n!$}
  это произведение будет равно $\ds x = \f{\l(a+n-1\r)!}{\l(a-1\r)!}$, а формула для $\ds \Cnk{c}{b} = \f{c!}{b!\l(c-b\r)!}$.
  если мы туда подставим $c=a+n-1$ и $b=n$ будет $\ds \Cnk{a+n-1}{n} = \f{\l(a+n-1\r)!}{n!\l(a-1\r)!}$.
  получается что $\ds \f{x}{n!} = \Cnk{a+n-1}{n}$, а все $\Cnk{z}{y}$ у нас целые и поэтому $\ds \f{x}{n!}$ тоже целое и $x$ всегда кратно $n!$. \\
  $\blacksquare$

  \section{Сколько существует шестизначных чисел, в записи которых поровну четных и нечетных цифр}
  мы сначала можем решить аналогичную задачу, но с двоичными числами.
  количество чисел $< 2^6$, в записи которых поровну четных и нечетных цифр, - это $\Cnk{6}{3}$,
  потомучто нам надо выбрать какие $3$ цифры из $6$ будут единичками.
  в десятичной системе нам просто надо домножить на $5^6$, тк у нас $5$ четных и $5$ нечетных цифр.
  будет $5^6\cdot\Cnk{6}{3} = 5^6 \cdot 20 = 312500$, но нам ещё надо вычесть те числа, которые начинаются с нуля,
  тоесть те числа $< 10^5$, в записи которых $3$ и нечетные цифры.
  их будет $5^5\cdot\Cnk{5}{3}$ (логика тут такая же, это типа почти как рекурсия). \\
  ответ: $5^6\cdot\Cnk{6}{3} - 5^5\cdot\Cnk{5}{3} = 5^6 \cdot 20 - 5^5 \cdot 10 = \fbox{281250}$

  \section{Сколькими способами можно разделить $6$ яблок, $3$ груши и $2$ сливы между Иваном, Петром, Анной и Дарьей, если фрукты одного вида не различаются?}
  чтобы разложить $n$ предметов по $k$ ящикам,
  надо записать $n+(k-1)$ объектов и выбрать какие $k-1$ из них будит разделителями (а оставшиеся будут предметами).
  тоесть получается формула для размещения $n$ предметов по $k$ ящикам: $\Cnk{n+k-1}{k-1}$ \cite{box}.
  нам тут надо перемножить эту формулу для $k=4$ и $n=\{6,\, 3,\, 2\}$. \\
  ответ: $ \Cnk{9}{3} \cdot \Cnk{6}{3} \cdot \Cnk{5}{3} = 84 \cdot 20 \cdot 10 = \fbox{16800} $

  \section{Найдите коэффициент при $x^{57}$ в многочлене $(x^2 + x^7 + x^9)^{20}$}
  оказывается для трёхчленов тоже есть формула \cite{trinom}, также как и у двучленов.
  $$ (a+b+c)^n = \sum_{\stackrel{i,j,k}{i+j+k=n}}  {n \choose i,j,k}\, a^i \, b^j \, c^k $$
  $$ {n \choose i,j,k} = \frac{n!}{i!\,j!\,k!} $$
  тут нам на сумму надо наложить ещё одно ограничение, потомучто мы хотим коэффициент при $x^{57}$.
  надо чтобы $a^i b^j c^k = x^{2i} x^{7j} x^{9k}$ было равно $x^{57}$, тоесть $2i+7j+9k = 57$.
  получаем такую систему: \\
  $\ds \begin{cases}
    i+j+k = 20 \\
    2i+7j+9k = 57 \\
    i,j,k \in \N \\
  \end{cases}$ \hfill $\begin{cases}
    i = 20-j-k \\
    5j+7k = 17 \\
    i,j,k \in \N \\
  \end{cases}$ \hfill $\begin{cases}
    k = 1 \\
    j = 2 \\
    i = 17 \\
  \end{cases}$ \\
  у нас нашлось только одно решение, и поэтому нам ненужна сумма. \\
  ответ: $\ds \f{20!}{1!\,2!\,17!} = \fbox{3420}$

  \section{Сколькими способами можно разложить семь различных монет по трем различным карманам так, чтобы каждый карман был не пуст?}
  из задания \textbf{2} мы знаем что всего у нас тут $3^7$ способа, и теперь нам надо убрать все случаи, где есть пустые карманы.
  у нас есть $3$ случая, в которых у нас два пустых кармана, но нам ещё надо посчитать в скольких случаях будет ровно один пустой карман.
  му можем посчитать количество способов засунуть эти монетки в два кармана. будет $2^7$. но это не совсем то что нам надо.
  там в двух из $2^7$ случаев будет два пустых кармана (все монетки в одном кармане), а это мы уже посчитали.
  потом нам надо $2^7-2$ домножить на $3$, потомучто у нас есть три разных способа выкинуть один карман. \\
  ответ: $3^7 - 3 \cdot (2^7-2) - 3 = \fbox{1806}$

  \section{Скольким способами можно переставить $10$ стоящих в ряд книг, чтобы ровно $4$ из них остались на месте?}
  у нас есть $\Cnk{10}{4}$ способов выбрать какие книги останутся на месте, а какие будут переставлены.
  теперь нам надо найти сколькими способами можно расставить $6$ книг так, чтобы они все переместились.
  такая перестановка на самом деле очень смешно называется: "{}беспорядок"{}, а на английском ещё смешнее: "{}derangement"{} \cite{der}.
  количество беспорядков размера $n$ называется субфакториалом и обозначается $!n$ \cite{sub}.
  посчитать его можно рекурсивно, а можно просто поделить нормальный факториал на $e$ \cite{sub}.
  тоесть $!6 = \l[\f{6!}{e}\r] = 265$, а $\Cnk{10}{4} = 210$. \\
  ответ: $\Cnk{10}{4} \cdot !6 = 210 \cdot 265 = \fbox{55650}$

  \section{Сколько чисел из $\range{2020}$ не взаимно просты с $2020$?}
  $\varphi(2020)$ это количество чисел из $\range{2020}$ взаимно простых c $2020$.
  тогда нам просто надо $2020 - \varphi(2020)$, а $\varphi(2020) = 800$. \\
  ответ: $ 2020 - 800 = \fbox{1220} $

  \section{Сколько есть семизначных чисел $\overline{a_7a_6 \dots a_1}$, где $a_i \leq a_{i+1}$ при всех $i$?}
  \begin{wrapfigure}{r}{4.3cm}
    \vspace{-.5cm}
    \hfill
    \begin{tikzpicture}[x=.5cm,y=.5cm]
      \draw[step=1, gray, very thin] (0,9) grid (7,0);
      \draw[-stealth, thick] (0,9)--
        (0,8)--(1,8)--
        (1,7)--(2,7)--
        (2,5)--(3,5)--
        (3,4)--(4,4)--
        (4,3)--(5,3)--
        (5,2)--(6,2)--
        (6,2)--(7,2)--
      (7,0);

      \node[anchor=north] at (0.5,0) {$8$};
      \node[anchor=north] at (1.5,0) {$7$};
      \node[anchor=north] at (2.5,0) {$5$};
      \node[anchor=north] at (3.5,0) {$4$};
      \node[anchor=north] at (4.5,0) {$3$};
      \node[anchor=north] at (5.5,0) {$2$};
      \node[anchor=north] at (6.5,0) {$2$};

      \node[anchor=east] at (0,0) {\texttt{\textquotesingle 0\textquotesingle}};
      \node[anchor=east] at (0,1) {\texttt{\textquotesingle 1\textquotesingle}};
      \node[anchor=east] at (0,2) {\texttt{\textquotesingle 2\textquotesingle}};
      \node[anchor=east] at (0,3) {\texttt{\textquotesingle 3\textquotesingle}};
      \node[anchor=east] at (0,4) {\texttt{\textquotesingle 4\textquotesingle}};
      \node[anchor=east] at (0,5) {\texttt{\textquotesingle 5\textquotesingle}};
      \node[anchor=east] at (0,6) {\texttt{\textquotesingle 6\textquotesingle}};
      \node[anchor=east] at (0,7) {\texttt{\textquotesingle 7\textquotesingle}};
      \node[anchor=east] at (0,8) {\texttt{\textquotesingle 8\textquotesingle}};
      \node[anchor=east] at (0,9) {\texttt{\textquotesingle 9\textquotesingle}};
    \end{tikzpicture}
  \end{wrapfigure}
  тут нам надо вспомнить задание \textbf{5}. мы представим наше число как некую гистограмму.
  каждой циферке будет соответствовать столбик, и его высота будет равна значению этой циферки, а $x$ координата будет её индексом в числе.
  тогда каждый путь нашего робота будет соответствовать такому числу. вот например я тут нарисовал для числа $8754322$.
  тоесть мы можем не считать количество таких семизначных чисел, мы просто посчитаем количество различных путей нашего робота.
  мы уже знаем что их будет $\Cnk{7+9}{7} = 11440$.
  но мы пропустили одну очень важную деталь: у нас $a_7 \neq 0$, потомучто тогда число не будет семизначным.
  хотя это на самом деле не так страшно. изза ограничения на циферки (они всегда должны идти вниз), у нас только одно такое число: $0000000$.
  нам ничего не мешает просто вычесть его из ответа. \\
  ответ: $11440 - 1 = \fbox{11439}$

  \vfill
  \begin{thebibliography}{9}
    % \bibitem{дашков} Дашков Е. В. Введение в математическую логику // URL: \url{https://drive.google.com/file/d/1wLW9t2UFJk9tTu8nM_YAYYr9zzZwKD1W/view}
    \bibitem{divisor} \url{https://en.wikipedia.org/wiki/Divisor#Further_notions_and_facts}
    \bibitem{trinom} \url{https://en.wikipedia.org/wiki/Trinomial_expansion}
    \bibitem{box} \url{http://math-hse.info/a/2014-15/soc-dm/lectures/lecture2_20150205.pdf}
    \bibitem{der} \url{https://en.wikipedia.org/wiki/Derangement}
    \bibitem{sub} \href{https://ru.wikipedia.org/wiki/%D0%A1%D1%83%D0%B1%D1%84%D0%B0%D0%BA%D1%82%D0%BE%D1%80%D0%B8%D0%B0%D0%BB}{\textbf{\texttt{https://ru.wikipedia.org/wiki/Субфакториал}}}
  \end{thebibliography}
\end{document}
