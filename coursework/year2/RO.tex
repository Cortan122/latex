\documentclass[a4paper,12pt]{article}

\newcommand{\CRTdocFormat}{Руководство оператора}
\usepackage{styledoc19}

\newcommand{\CRTdocnumMID}{РО 01-1}
\usepackage{CRTconfig}

\begin{document} % конец преамбулы, начало документа
  \CRTpreamble

  \section{Назначение программы}
  \subsection{Функциональное назначение}
  Разрабатываемое приложение <<\CRTname>> предназначено для создания снимка экрана рабочей зоны компьютера и для дальнейшей работы и обработки таких снимков с возможностью добавления текста и графических набросков на основе алгоритмов математических кривых Безье.

  \subsection{Эксплуатационное назначение}
  Программа предназначена для бытового пользователя на компьютерах с операционной системой Windows и Linux.

  \newpage
  \section{Условия выполнения программы}
  \subsection{Минимальный состав аппаратных средств}
  \begin{enumerate}
    \item Процессор архитектуры x86 или x64 с частотой не менее 1 ГГц;
    \item Не менее 1 ГБ ОЗУ;
    \item Не менее 5 МБ свободного места на жестком диске;
    \item Поддержка графических интерфейсов OpenGL или Vulkan;
  \end{enumerate}

  \subsection{Минимальный состав программных средств}
  Одна из нижеперечисленных операционных систем:
  \begin{itemize}
    \item Unix-подобная операционная система с оконной системой X11;
    \item Windows XP или более поздняя версия операционной системы (32-разрядные или 64-разрядные).
  \end{itemize}

  \subsection{Требования к оператору} % (пользователю)
  % Образование не ниже среднего школьного.
  % Практические навыки работы с пользовательским интерфейсом операционной системы Windows.
  % Оператор должен быть способен механически взаимодействовать с компьютером и запускать программу.
  \begin{enumerate}
    \item Среднее школьное образование.
    \item Практические навыки работы с пользовательским интерфейсом операционной системы.
    \item Способность механически взаимодействовать с персональным компьютером и запускать программу.
  \end{enumerate}


  \newpage
  \section{Выполнение программы}
  \subsection{Установка программы}
  Для установки программы необходимо загрузить на компьютер архив \CRTzip{} и распаковать его.

  \subsection{Запуск программы и работа с приложением}
  Для начала работы программы необходимо запустить исполняемый файл \CRTexe{}.

  Интерфейс приложения разбит на 3 области: \CRTfigref{basicui}{Области интерфейса}
  \begin{enumerate}
    \item Верхняя панель содержит общую информацию об окне:
    \begin{itemize}
      \item Кадровая частота окна с учётом того, что при бездействие пользователя кадры не обновляются.
      \item Размер окна в пикселя.
      \item Индикатор выбранного цвета добавляемых элементов.
      При нажатие открывает цветовую палитру (см. \autoref{sec:colorpalette}).
      При отсутствие выбранного цвета индикатор приобретает неопределённый бесцветный вид \CRTfigref{colorblank}{Индикатор отсутствия цвета}. В противном случае он залит сплошной заливкой выбранного цвета \CRTfigref{colorred}{Индикатор выбранного цвета}.
    \end{itemize}
    \item Центральная область содержит редактируемое изображение.
    \item Нижняя панель содержит информацию о редактируемом изображении:
    \begin{itemize}
      \item Имя открытого файла. Если редактируемое изображение находится в буфере обмена, то отображается сообщение <<No open file>>.
      \item Количество элементов, добавленных пользователем на редактируемое изображение, или имя типа активного элемента, если таковой имеется.
      \item Размер редактируемого изображения в пикселях.
      \item Масштаб редактируемого изображения в процентах.
    \end{itemize}
  \end{enumerate}

  \subsubsection{Экран цветовой палитры}
  \label{sec:colorpalette}
  Этот экран предназначен для выбора цвета добавляемых элементов.
  Он содержит таблицу доступных цветов \CRTfigref{colorpalette}{Таблица цветов},
  каждый из которых можно выбрать либо левой кнопкой мыши, либо нажатием первой буквы в его имени.
  В зависимости от размера окна таблица меняет размер и скрывает схожие оттенки цветов \CRTfigref{colorpalette4x4}{Маленькая таблица цветов}.
  Каждый цвет подписан шестнадцатеричным кодом и представлен прямоугольником соответствующей заливки.
  % Каждый цвет представлен соответствующим прямоугольником и подписан своим шестнадцатеричным кодом.
  % Каждый цвет представлен прямоугольником этого цвета, в котором содержится имя и шестнадцатеричное представление цвета.
  % Обозначение шестнадцатеричной системы счисления, который представляет из себя информацию о цвете.

  \subsubsection{Экран выбора файла}
  Этот экран открываться при нажатие пользователем комбинаций клавиш Ctrl+S или Ctrl+O.
  Он может использоваться как и для сохранения \CRTfigref{saveas}{Выбор файла для сохранения}, так и для открытия \CRTfigref{open}{Выбор файла для открытия} файлов.
  Он содержит текстовое поле для ввода пути к файлу и кнопку для завершения процесса.
  При сохранении файла начальное значение текстового поля указывает в папку загрузок текущего пользователя,
  а при открытие в рабочую директорию процесса.
  Если полный путь к фалу слишком длинный и полностью не помещается в текстовое поле, показывается только та часть, в которой расположен текстовый курсор.

  \subsubsection{Редактирование изображения}
  \begin{enumerate}
    \item Левой кнопкой мыши на изображение добавляется нарисованная от руки линия.
    \item Правой кнопкой мыши на изображение добавляется текстовое поле \CRTfigref{bluetext}{Текстовое поле}.
    Пока оно активно правой кнопка мыши будет его перемещать и увеличивать размер его шрифта.
    Чтобы завершить набор текста надо нажать Enter.
    \item Используя комбинацию клавиш Ctrl+Z пользователь может удалить последний добавленный элемент.
    \item Нажав клавиатурную клавишу B пользователь получает возможность добавления на изображение закрашенного прямоугольника.
    Позиция точки первого угла прямоугольника определяется при начале нажатия левой кнопки мыши,
    а позиция точки второго угла прямоугольника определяется при завершение её нажатия.
    Зажав клавишу Shift во время этого процесса можно выключить заливку прямоугольника \CRTfigref{bluebox}{Прямоугольник без заливки}.
    \item Нажав клавиатурную клавишу X пользователь получает возможность обрезать изображение \CRTfigref{bluecrop}{Обрезание изображения}. Прямоугольник определяется также.
    \item Используя комбинацию клавиш Ctrl+V пользователь может подгрузить новое изображение из буфера обмена.
    \item Используя комбинацию клавиш Ctrl+С пользователь может сохранить отредактированное изображение в буфера обмена.
    \item Используя комбинацию клавиш Ctrl+P пользователь может сделать полный снимок экрана.
  \end{enumerate}


  \begin{CRTbibliography}
  \end{CRTbibliography}

  \CRTterminology

  \CRTlistRegistration
\end{document} % конец документа
