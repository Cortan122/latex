\documentclass[a4paper,12pt]{article}

\newcommand{\CRTdocFormat}{Программа и методика испытаний}
\usepackage{styledoc19}

\newcommand{\CRTdocnumMID}{ПМИ 01-1}
\usepackage{CRTconfig}

\begin{document} % конец преамбулы, начало документа
  \CRTpreamble

  \section{Объекты испытаний}
  \subsection{Наименование программы}
  <<\CRTname>>

  \subsection{Область применения программы}
  Программа используется в повседневной жизни, когда требуется создание и сохранение снимка экрана с возможностью дальнейшего редактирования, включая обрезание, создание набросков и добавление текста.

  \newpage
  \section{Цель испытаний}
  Целью проведения описанных далее испытаний является проверка корректности работы программы,
  а также её соответствия требованиям, изложенным в документе <<Техническое задание>>.

  \newpage
  \section{Требования к программе}
  Программа должна соответствовать следующим функциональным требованиям, изложенным в документе <<Технической задание>>.
  \subsection{Требования к функциональным характеристикам}
  \subsubsection{Требования к составу выполняемых функций}
  \label{sec:funcs}
  \begin{enumerate}
    \item Создание снимка экрана на Windows и X11.
    \item Возможность перемещать изображение по экрану.
    \item Возможность увеличивать и уменьшать масштаб.
    \item Добавление текста на изображение встроенным растровым шрифтом.
    \item Добавление графических набросков на изображение мышкой.
    \item Сохранение полученного рисунка в файл.
    \item Сохранение полученного рисунка в буфер обмена.
    \item Функция добавления картинок на изображение с помощью перетаскивания.
  \end{enumerate}

  \subsubsection{Требования к интерфейсу}
  \begin{enumerate}
    \item Поле для отображения редактируемого изображения.
    \item Индикатор текущего увеличения и размера изображения.
    \item По клавише быстрого вызова для каждой функции, описанной в \autoref{sec:funcs}.
    \item Обучающая инструкция, в которой написан список всех функций, описанных в \autoref{sec:funcs}.
  \end{enumerate}

  \subsubsection{Требования к организации входных данных}
  На вход должен подаваться снимок экрана из буфера обмена или любое другое изображение из файла.
  \subsubsection{Требования к организации выходных данных}
  На выход программа должна записывать полученный рисунок в файл в формате PNG или в буфер обмена.
  \subsection{Требования к надежности}
  При любом вводе пользователя программа не должна завершаться аварийно.
  При неправильном формате вводимых данных программа должна выводить сообщение с предупреждением о неправильном формате данных.

  \newpage
  \section{Требования к программной документации}
  \subsection{Предварительный состав программной документации}
  \label{sec:doclist}
  В рамках данной работы должна быть разработана следующая программная документация в соответствии и ГОСТ ЕСПД:
  \begin{itemize}
    \item <<\CRTname>>. Техническое задание \cite{gostTZ};
    \item <<\CRTname>>. Программа и методика испытаний \cite{gostPMI};
    \item <<\CRTname>>. Текст программы \cite{gostTP};
    \item <<\CRTname>>. Пояснительная записка \cite{gostPZ};
    \item <<\CRTname>>. Руководство оператора \cite{gostRO};
  \end{itemize}

  \subsection{Специальные требования к программной документации}
  Документы к программе должны быть выполнены в соответствии с ГОСТ 19.106-78 и ГОСТами к каждому виду документа
  (см. п. \ref{sec:doclist});

  Пояснительная записка должна быть загружена в систему Антиплагиат через LMS <<НИУ ВШЭ>>.

  Документация и программа сдаются в электронном виде в формате .pdf или .docx в архиве формата .zip или .rar;

  За один день до защиты комиссии все материалы курсового проекта:
  \begin{itemize}
    \item техническая документация,
    \item программный проект,
    \item исполняемый файл,
    \item отзыв руководителя,
    \item лист Антиплагиата
  \end{itemize}
  должны быть загружены одним или несколькими архивами в проект дисциплины <<Курсовой проект 2020-2021>> в личном кабинете в информационной образовательной среде LMS (Learning Management System) НИУ ВШЭ.

  \newpage
  \section{Средства и порядок испытаний}
  \subsection{Технические средства, используемые во время испытаний}
  \begin{enumerate}
    \item Процессор архитектуры x86 или x64 с частотой не менее 1 ГГц;
    \item Не менее 2 ГБ ОЗУ;
    \item Не менее 5 МБ свободного места на жестком диске;
    \item Графическое устройство поддерживавшие OpenGL.
  \end{enumerate}
  \subsection{Программные средства, используемые во время испытаний}
  \begin{enumerate}
    \item Поддержка графических интерфейсов OpenGL или Vulkan;
    \item Одна из нижеперечисленных операционных систем:
    \begin{itemize}
      \item Unix-подобная операционная система с оконной системой X11;
      \item Windows XP или более поздняя версия операционной системы (32-разрядные или 64-разрядные).
    \end{itemize}
  \end{enumerate}

  \subsection{Порядок проведения испытаний}
  Испытания должны проводиться в следующем порядке:
  \begin{enumerate}
    \item Выполнить начальную загрузку операционной системы.
    \item Скачать на компьютер архив \CRTzip{} и распаковать его.
    \item Для начала работы программы запустить исполняемый файл \CRTexe{}.
    \item Провести необходимые испытания, описанные ниже в разделе <<Методы испытаний>>.
  \end{enumerate}

  \newpage
  \section{Методы испытаний}
  \subsection{Проверка требований к функциональным характеристикам}
  \subsubsection{Требования к интерфейсу}
  \begin{enumerate}
    \item Поле для отображения редактируемого изображения \CRTfigref{image20210516094820}{Поле для редактирования}.
    \item Индикатор текущего увеличения и размера изображения \CRTfigref{scale}{Индикатор размера}.
    \item По клавише быстрого вызова для каждой функции, описанной в \autoref{sec:funcs}.
    \begin{itemize}
      \item Создание снимка экрана на Windows и X11 -- Ctrl+P.
      \item Возможность перемещать изображение по экрану -- WASD.
      \item Возможность увеличивать и уменьшать масштаб -- +/-.
      \item Добавление текста на изображение встроенным растровым шрифтом -- правая кнопка мыши.
      \item Добавление графических набросков на изображение мышкой -- левая кнопка мыши.
      \item Сохранение полученного рисунка в файл -- Ctrl+S.
      \item Сохранение полученного рисунка в буфер обмена -- Ctrl+C.
    \end{itemize}
    \item Обучающая инструкция, в которой написан список всех функций, описанных в \autoref{sec:funcs} \CRTfigref{help}{Обучающая инструкция}.
  \end{enumerate}

  \subsubsection{Требования к организации входных данных}
  Снимок экрана из буфера обмена подгружается при запуске программы \CRTfigref{screenshot}{Снимок экрана из буфера обмена},
  а другие файлы открываются через комбинацию клавиш Ctrl+O \CRTfigref{open}{Выбор файла для открытия}.

  \subsubsection{Требования к организации выходных данных}
  Полученный рисунок сохраняется в буфер обмена при нажатие на Ctrl+C и при закрытии программы,
  а, чтобы сохранить в файл, пользователю стоит воспользоваться комбинацией клавиш Ctrl+Shift+S \CRTfigref{saveas}{Выбор файла для сохранения}.

  \subsection{Проверка требований к надежности}
  При введение некорректного пути в текстового поле на экране выбора файла, внизу появляется соответствующие сообщение \CRTfigref{cantopen}{Сообщение о неудачном открытие}.

  \begin{CRTbibliography}
  \end{CRTbibliography}

  \CRTterminology

  \CRTlistRegistration
\end{document} % конец документа
