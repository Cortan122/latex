\documentclass[a4paper,12pt]{article}

\newcommand{\CRTdocFormat}{Руководство оператора}
\usepackage{styledoc19}

\newcommand{\CRTdocnumMID}{РО 01-1}
\usepackage{CRTconfig}

\begin{document} % конец преамбулы, начало документа
  \CRTpreamble

  \section{Назначение программы}
  \subsection{Функциональное назначение}
  Разрабатываемое приложение <<\CRTname>> предназначено для визуализации синтаксического анализа методом рекурсивного спуска.

  \subsection{Эксплуатационное назначение}
  Программа наглядно демонстрирует работу метода рекурсивного спуска и помогает лучше понять процесс его работы,
  может использоваться как для изучения непосредственно методов синтаксического анализа,
  так и для симуляции условий в задачах связанных с работой таких методов.

  \newpage
  \section{Условия выполнения программы}
  \subsection{Минимальный состав аппаратных средств}
  \begin{enumerate}
    \item Процессор архитектуры x86 или x64 с частотой не менее 1 ГГц;
    \item Не менее 2 ГБ ОЗУ;
    \item Не менее 5 МБ свободного места на жестком диске;
    \item Графическое устройство DirectX 9 с драйвером WDDM 1.0 или более поздней версии.
  \end{enumerate}

  \subsection{Минимальный состав программных средств}
  \begin{enumerate}
    \item Windows 7 или более поздняя версия операционной системы (32-разрядные или 64-разрядные);
    \item Установленный .NET Framework версии 4.5 и выше;
    % \item Программа должна быть написана на языке программирования C\#.
  \end{enumerate}

  \subsection{Требования к оператору} % (пользователю)
  % Образование не ниже среднего школьного.
  % Практические навыки работы с пользовательским интерфейсом операционной системы Windows.
  % Оператор должен быть способен механически взаимодействовать с компьютером и запускать программу.
  \begin{enumerate}
    \item Среднее школьное образование.
    \item Практические навыки работы с пользовательским интерфейсом операционной системы Windows.
    \item Способность механически взаимодействовать с персональным компьютером и запускать программу.
  \end{enumerate}


  \newpage
  \section{Выполнение программы}
  \subsection{Установка программы}
  Для установки программы необходимо загрузить на компьютер архив \CRTzip{} и распаковать его.

  \subsection{Запуск программы и работа с приложением}
  Для начала работы программы необходимо запустить исполняемый файл \CRTexe{}.

  Интерфейс приложения разбит на 4 области: \CRTfigref{main}{Области интерфейса}
  \begin{enumerate}
    \item \hyperref[sec:a1]{\textbf{Левая нижняя}}: в этой области располагается панель управления процессом визуализации.
    \item \hyperref[sec:a2]{\textbf{Правая нижняя}}: эта область содержит текстовый блок с объяснениями алгоритма и кнопки, отвечающие за внешний вид синтаксического дерева.
    \item \hyperref[sec:a3]{\textbf{Правая верхняя}}: эта область содержит синтаксическое дерево и строку, синтаксический анализ которой визуализируется.
    \item \hyperref[sec:a4]{\textbf{Левая верхняя}}: эта область содержит цветовую палитру синтаксического дерева.
  \end{enumerate}

  \subsubsection{Левая нижняя область интерфейса}
  \label{sec:a1}
  Здесь расположены:
  \begin{itemize}
    \item два бегунка: один для скорости автоматической прокрутки,
    а другой для процесса визуализации. \CRTfigref{controlpanelSliders}{Бегунки}
    \item поле ввода скорости автоматической прокрутки. \CRTfigref{controlpanelTextbox}{Поле ввода скорости}
    \item 8 кнопок для управления процессом визуализации: \CRTfigref{controlpanelButtons}{Кнопки}
    \begin{enumerate}
      \item \textbf{Загрузить сохранение}: кнопка для загрузки сохраненного состояния визуализатора,
      при нажатии появляется диалоговое окно для выбора файла в формате JSON.
      \item \textbf{Сохранить}: кнопка для сохранения текущего состояния визуализатора в файл в формате JSON
      или экспорта рисунка синтаксического дерева в файл в формате PNG или XAML.
      При нажатии появляется диалоговое окно для выбора файла, и если пользователь выбирает формат JSON,
      то в выбранный файл сохраняется текущее состояние визуализатора, а если он выбирает формат PNG или XAML,
      то в выбранный файл сохраняется рисунок синтаксического дерева. \CRTfigref{savedialog}{Выбор формата файла}
      \item \textbf{Прокрутка назад}: при нажатии на кнопку запускается автоматическая прокрутка назад,
      но если она уже запущена, то запускается автоматическая прокрутка вперед.
      \item \textbf{Первый кадр}: кнопка показывает первый кадр визуализации синтаксического анализа.
      \item \textbf{Предыдущий кадр}: кнопка показывает предыдущий кадр визуализации синтаксического анализа.
      \item \textbf{Следующий кадр}: кнопка показывает следующий кадр визуализации синтаксического анализа.
      \item \textbf{Последний кадр}: кнопка показывает последний кадр визуализации синтаксического анализа.
      \item \textbf{Воспроизведение/Пауза}: при нажатии на кнопку запускается автоматическая прокрутка вперед,
      но, если автоматическая прокрутка уже запущена, она останавливается.
    \end{enumerate}
  \end{itemize}

  \subsubsection{Правая нижняя область интерфейса}
  \label{sec:a2}
  Здесь расположены текстовый блок с объяснениями алгоритма и 6 кнопок,
  отвечающих за его содержание и за внешний вид синтаксического дерева: \CRTfigref{buttons}{Кнопки}
  \begin{enumerate}
    \item \textbf{Предыдущий туториал}: кнопка меняет содержимое текстового блока с объяснениями алгоритма, выбирая предыдущий файл.
    Кнопка становится неактивной, если выбран первый такой файл.
    \item \textbf{Следующий туториал}: кнопка меняет содержимое текстового блока с объяснениями алгоритма, выбирая следующий файл.
    Кнопка становится неактивной, если выбран последний такой файл.
    \item \textbf{Подровнять дерево}: кнопка скрывает те узлы синтаксического дерева, у которых только один ребёнок.
    \CRTfigref{trimtree}{Обрезка дерева}
    \item \textbf{Поменять ориентацию дерева}: при смене ориентации дерева,
    его листья и корень меняются положениями относительно пользователя,
    но гравитация не меняется. \CRTfigref{treeori}{Ориентация дерева}
    \item \textbf{Поменять гравитацию дерева}:
    Гравитация вверх делает так, чтобы пустые пространства скапливались внизу относительно пользователя,
    а гравитация вниз делает так, чтобы пустые пространства скапливались вверху относительно пользователя.
    По умолчанию гравитация вверх. \CRTfigref{treegrav}{Гравитация дерева}
    \item \textbf{Режим новичка}: кнопка прорисовывает линиями ветки синтаксического дерева поверх обычного изображения.
    \CRTfigref{treehelp}{Режим новичка}
  \end{enumerate}

  \subsubsection{Правая верхняя область интерфейса}
  \label{sec:a3}
  Здесь расположены:
  \begin{itemize}
    \item Синтаксическое дерево, состоящие из узлов, цвет каждого из которых совпадает с цветом одного из кругов в \autoref{sec:a4}.
    При наведении курсора на любой из этих узлов появляется подсказка о том, к какому именно кругу он относится.
    \CRTfigref{tooltip}{Подсказка}
    \item Поле ввода, где пользователь вводит строку, синтаксический анализ которой будет визуализироваться.
    Оно появляется после нажатие на эту самую строку.
    \CRTfigref{inputbox}{Поле ввода строки}
  \end{itemize}

  \subsubsection{Левая верхняя область интерфейса}
  \label{sec:a4}
  Здесь расположены 4 круга разных цветов.
  Их цвета совпадают с цветами соответствующих узлов синтаксического дерева.
  При нажатие на любой из этих кругов всплывает диалоговое окно,
  позволяющие пользователю определять собственные цвета узлов синтаксического дерева. \CRTfigref{colorbox}{Диалоговое окно}

  \begin{CRTbibliography}
  \end{CRTbibliography}

  \CRTterminology

  \CRTlistRegistration
\end{document} % конец документа
