\documentclass[a4paper,12pt,reqno]{article}

\newcommand{\CRTdocFormat}{Техническое задание}
\usepackage{styledoc19}

\newcommand{\CRTdocnumMID}{ТЗ 01-1}
\usepackage{CRTconfig}

\begin{document} % конец преамбулы, начало документа
  \CRTpreamble

  \section{Введение}
  \subsection{Наименование программы}
  \subsubsection{Наименование программы на русском языке}
  <<\CRTname>>
  \subsubsection{Наименование программы на английском языке}
  <<\CRTnameeng>>

  \subsection{Краткая характеристика области применения}
  Программа используется в сфере образования для демонстрации работы
  методов синтаксического анализа, в том числе и метода рекурсивного спуска.
  % \par \todo

  \newpage
  \section{Основания для разработки}
  \subsection{Документы, на основании которых ведется разработка}
  Приказ № 2.3-02/2004-04 от 20.04.2020 <<Об изменении тем, руководителей курсовых работ студентов образовательной программы
  <<Программная инженерия>> факультета компьютерных наук>>.

  \subsection{Наименование темы разработки}
  Наименование темы разработки -- <<\CRTname>>

  Программа выполняется в рамках темы курсовой работы в соответствии с учебным планом подготовки бакалавров по направлению 09.03.04 <<Программная инженерия>> Национального исследовательского университета <<Высшая школа экономики>>, факультет компьютерных наук.

  \newpage
  \section{Назначение разработки}
  \subsection{Функциональное назначение}
  Разрабатываемое приложение <<\CRTname>> предназначено для визуализации синтаксического анализа методом рекурсивного спуска.
  % \par \todo
  \subsection{Эксплуатационное назначение}
  Программа наглядно демонстрирует работу метода рекурсивного спуска и помогает лучше понять процесс его работы, может использоваться как для изучения непосредственно методов синтаксического анализа, так и для симуляции условий в задачах связанных с работой таких методов.
  % \par \todo

  \newpage
  \section{Требования к программе}
  \subsection{Требования к функциональным характеристикам}
  \subsubsection{Требования к составу выполняемых функций}
  \label{sec:funcs}
  \begin{enumerate}
    \item Шаг вперед.
    \item Шаг назад.
    \item Автоматическая прокрутка вперед (с выбором скорости шагов).
    \item Автоматическая прокрутка назад (с выбором скорости шагов).
    \item Сохранение текущего состояния в файл и загрузка состояния из файла.
  \end{enumerate}

  \subsubsection{Требования к интерфейсу}
  \begin{enumerate}
    \item Поле ввода, где пользователь вводит строку, синтаксический анализ которой будет визуализироваться.
    \item Поле ввода и бегунок, для скорости автоматической прокрутки.
    \item Кнопка и клавиша быстрого вызова для каждой функции, описанной в \autoref{sec:funcs}.
    \item Область отображения формальной грамматики, описанной на yacc-подобном языке или на БНФ.
    \item Подсветка того кода, который сейчас выполняются.
  \end{enumerate}

  \subsubsection{Требования к организации входных данных}
  На вход должна подаваться строка символов.
  \subsubsection{Требования к организации выходных данных}
  На выход программа должна записывать рисунок синтаксического дерева в файл в формате PNG.
  \subsection{Требования к надежности}
  При любом вводе пользователя программа не должна завершаться аварийно.
  При неправильном формате вводимых данных программа должна выводить сообщение с предупреждением о неправильном формате данных
  и запрашивать их ещё раз.
  % \clearpage

  \subsection{Условия эксплуатации}
  \subsubsection{Климатические условия}
  Климатические условия эксплуатации, при которых должны обеспечиваться заданные характеристики, должны удовлетворять требованиям, предъявляемым к персональным компьютерам \cite{gostclimate}.
  Персональный компьютер предназначен для работы в закрытом отапливаемом помещении со стабильными климатическими условиями.
  \begin{enumerate}
    \item влажность от 20\% до 70\%;
    \item температура от 5\degree C до 30\degree C;
    \item атмосферное давление --- от 84 до 106,7 кПа (от 630 до 800 мм рт. ст.)
  \end{enumerate}

  \subsubsection{Требования к пользователю}
  \begin{enumerate}
    \item Среднее школьное образование.
    \item Практические навыки работы с пользовательским интерфейсом операционной системы Windows.
    \item Способность механически взаимодействовать с персональным компьютером и запускать программу.
  \end{enumerate}


  \subsection{Требования к составу и параметру технических средств}
  Для корректной работы приложения необходимо:
  \begin{enumerate}
    \item Процессор архитектуры x86 или x64 с частотой не менее 1 ГГц;
    \item Не менее 2 ГБ ОЗУ;
    \item Не менее 5 МБ свободного места на жестком диске;
    \item Графическое устройство DirectX 9 с драйвером WDDM 1.0 или более поздней версии.
  \end{enumerate}

  \subsection{Требования к информационной и программной совместимости}
  Для корректной работы приложения необходимо:
  \begin{enumerate}
    \item Windows 7 или более поздняя версия операционной системы (32-разрядные или 64-разрядные);
    \item Установленный .NET Framework версии 4.5 и выше;
    \item Программа должна быть написана на языке программирования C\#.
  \end{enumerate}

  % \subsection{Требования к маркировке и упаковке}
  % Программа поставляется в виде программного изделия на внешнем носителе информации --
  % компакт диске (CD), на котором должны содержаться программная документация, приложение
  % (исполняемые файлы и прочие необходимые для работы программы файлы) и
  % презентация проекта.
  % Программное изделие должно иметь маркировку с обозначением наименования изделия,
  % темы разработки, фамилии, имени и отчества исполнителя и руководителя разработки, учебной
  % группы и года выпуска изделия.

  \newpage
  \section{Требования к программной документации}
  \subsection{Предварительный состав программной документации}
  \label{sec:doclist}
  В рамках данной работы должна быть разработана следующая программная документация в соответствии и ГОСТ ЕСПД:
  \begin{itemize}
    \item <<\CRTname>>. Техническое задание \cite{gostTZ};
    \item <<\CRTname>>. Программа и методика испытаний \cite{gostPMI};
    \item <<\CRTname>>. Текст программы \cite{gostTP};
    \item <<\CRTname>>. Пояснительная записка \cite{gostPZ};
    \item <<\CRTname>>. Руководство оператора \cite{gostRO};
  \end{itemize}

  \subsection{Специальные требования к программной документации}
  Документы к программе должны быть выполнены в соответствии с ГОСТ 19.106-78 и ГОСТами к каждому виду документа (см. п. \ref{sec:doclist});

  Пояснительная записка должна быть загружена в систему Антиплагиат через LMS <<НИУ ВШЭ>>.

  Документация и программа сдаются в электронном виде в формате .pdf или .docx в архиве формата .zip или .rar;

  За один день до защиты комиссии все материалы курсового проекта:
  \begin{itemize}
    \item техническая документация,
    \item программный проект,
    \item исполняемый файл,
    \item отзыв руководителя,
    \item лист Антиплагиата
  \end{itemize}
  должны быть загружены одним или несколькими архивами в проект дисциплины <<Курсовой проект 2019-2020>> в личном кабинете в информационной образовательной среде LMS (Learning Management System) НИУ ВШЭ.


  \newpage
  \section{Технико-экономические показатели}
  \subsection{Предполагаемая потребность}
  Данная программа предназначена для обучения работе методов синтаксического анализа.
  Может быть использована любыми образовательными программами, а также любым пользователем для самообразования.

  \subsection{Ориентировочная экономическая эффективность}
  В рамках данной работы расчет экономической эффективности не предусмотрен.

  \newpage
  \section{Стадии и этапы разработки}
    \subsection{Техническое задание}
      \subsubsection*{Обоснование необходимости разработки}
      \begin{enumerate}
        \item Постановка задачи;
        \item Сбор теоретического материала;
        \item Выбор и обоснование критериев эффективности и качества разрабатываемого продукта;
      \end{enumerate}
      \subsubsection*{Научно-исследовательские работы}
      \begin{enumerate}
        \item Определение структуры входных и выходных данных;
        \item Предварительный выбор методов решения поставленной задачи;
        \item Определение требований к техническим средствам;
        \item Обоснование возможности решения поставленной задачи.
      \end{enumerate}
      \subsubsection*{Разработка и утверждение технического задания}
      \begin{enumerate}
        \item Определение требований к программе;
        \item Определение стадий, этапов и сроков разработки программы и документации на неё;
        \item Выбор языка программирования;
        \item Согласование и утверждение технического задания.
      \end{enumerate}
      \subsubsection*{Подготовка и передача программы}
      \begin{enumerate}
        \item утверждение даты защиты программного продукта;
        \item подготовка программы и программной документации для презентации и защиты;
        \item представление разработанного программного продукта руководителю и получение отзыва;
        \item загрузка Пояснительной записки в систему Антиплагиат через ЛМС НИУ ВШЭ;
        \item загрузка материалов курсового проекта (курсовой работы) в ЛМС, проект дисциплины <<Курсовая работа 2019>> (п. 5.2);
        \item Защита программного продукта (курсового проекта) комиссии.
      \end{enumerate}
    \subsection{Рабочий проект}
      \subsubsection*{Разработка программы}
      \begin{enumerate}
        \item Реализация алгоритма синтаксического анализа;
        % \item \todo
        \item Реализация программного интерфейса;
        \item Отладка программы.
      \end{enumerate}
      \subsubsection*{Разработка программной документации}
      \begin{enumerate}
        \item Разработка программных документов в соответствии с требованиями ЕСПД.
      \end{enumerate}
      \subsubsection*{Испытания программы}
      \begin{enumerate}
        \item Разработка, согласование и утверждение программы и методики испытаний;
        \item Проведение предварительных приемо-сдаточных испытаний;
        \item Корректировка программы и программной документации по результатам испытаний.
      \end{enumerate}
      \subsubsection*{Сроки разработки и исполнители}
      Разработка должна закончиться к 24 мая 2020 года.

      Исполнитель: \AA{AAAAAAA AAAAAAAAAA AAAAAAAAAA}{1}, студент группы \AA{AAAAAA}{11} факультета компьютерных наук НИУ ВШЭ.
    \subsection{Внедрение}
      \subsubsection*{Подготовка и защита программного продукта}
      \begin{enumerate}
        \item Подготовка программы и документации для защиты;
        \item Утверждение дня защиты программы;
        \item Презентация разработанного программного продукта;
        \item Передача программы и программной документации в архив НИУ ВШЭ.
      \end{enumerate}

  % приложения нумеруются отдельно и надо выровнять по правому краю

  % \newpage
  % \addition{Используемые понятия и определения}
  % \todo
  % \newpage

  % \addition{Иллюстрации интерфейса} \label{interface}
  % \todo.

  \section{Порядок контроля и приемки}
  Проверка программного продукта, в том числе и на соответствие техническому заданию,
  осуществляется исполнителем вместе с заказчиком согласно <<Программе и методике испытаний>>, а также пункту 5.2.
  Защита выполненного проекта осуществляется комиссии, состоящей из преподавателей департамента программной инженерии,
  в утверждённые приказом декана ФКН сроки.

  \begin{CRTbibliography}
  \end{CRTbibliography}

  \CRTlistRegistration
\end{document} % конец документа
