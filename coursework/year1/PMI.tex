\documentclass[a4paper,12pt]{article}

\newcommand{\CRTdocFormat}{Программа и методика испытаний}
\usepackage{styledoc19}

\newcommand{\CRTdocnumMID}{ПМИ 01-1}
\usepackage{CRTconfig}

\begin{document} % конец преамбулы, начало документа
  \CRTpreamble

  \section{Объекты испытаний}
  \subsection{Наименование программы}
  <<\CRTname>>

  \subsection{Область применения программы}
  Программа используется в сфере образования для демонстрации работы
  методов синтаксического анализа, в том числе и метода рекурсивного спуска.

  \newpage
  \section{Цель испытаний}
  Целью проведения описанных далее испытаний является проверка корректности работы программы,
  а также её соответствия требованиям, изложенным в документе <<Техническое задание>>.

  \newpage
  \section{Требования к программе}
  Программа должна соответствовать следующим функциональным требованиям, изложенным в документе <<Технической задание>>.
  \subsection{Требования к функциональным характеристикам}
  \subsubsection{Требования к составу выполняемых функций}
  \label{sec:funcs}
  \begin{enumerate}
    \item Шаг вперед.
    \item Шаг назад.
    \item Автоматическая прокрутка вперед (с выбором скорости шагов).
    \item Автоматическая прокрутка назад (с выбором скорости шагов).
    \item Сохранение текущего состояния в файл и загрузка состояния из файла.
    % \item \todo
  \end{enumerate}

  \subsubsection{Требования к интерфейсу}
  \begin{enumerate}
    \item Поле ввода, где пользователь вводит строку, синтаксический анализ которой будет визуализироваться.
    \item Поле ввода и бегунок, для скорости автоматической прокрутки.
    \item Кнопка и клавиша быстрого вызова для каждой функции, описанной в \autoref{sec:funcs}.
    \item Область отображения формальной грамматики, описанной на yacc-подобном языке или на БНФ.
    \item Подсветка того кода, который сейчас выполняются.
    % \item \todo
  \end{enumerate}

  \subsubsection{Требования к организации входных данных}
  На вход должна подаваться строка символов.
  \subsubsection{Требования к организации выходных данных}
  На выход программа должна записывать рисунок синтаксического дерева в файл в формате PNG.
  \subsection{Требования к надежности}
  При любом вводе пользователя программа не должна завершаться аварийно.
  При неправильном формате вводимых данных программа должна выводить сообщение с предупреждением о неправильном формате данных
  и запрашивать их ещё раз.

  \newpage
  \section{Требования к программной документации}
  \subsection{Предварительный состав программной документации}
  \label{sec:doclist}
  В рамках данной работы должна быть разработана следующая программная документация в соответствии и ГОСТ ЕСПД:
  \begin{itemize}
    \item <<\CRTname>>. Техническое задание \cite{gostTZ};
    \item <<\CRTname>>. Программа и методика испытаний \cite{gostPMI};
    \item <<\CRTname>>. Текст программы \cite{gostTP};
    \item <<\CRTname>>. Пояснительная записка \cite{gostPZ};
    \item <<\CRTname>>. Руководство оператора \cite{gostRO};
  \end{itemize}

  \subsection{Специальные требования к программной документации}
  Документы к программе должны быть выполнены в соответствии с ГОСТ 19.106-78 и ГОСТами к каждому виду документа
  (см. п. \ref{sec:doclist});

  Пояснительная записка должна быть загружена в систему Антиплагиат через LMS <<НИУ ВШЭ>>.

  Документация и программа сдаются в электронном виде в формате .pdf или .docx в архиве формата .zip или .rar;

  За один день до защиты комиссии все материалы курсового проекта:
  \begin{itemize}
    \item техническая документация,
    \item программный проект,
    \item исполняемый файл,
    \item отзыв руководителя,
    \item лист Антиплагиата
  \end{itemize}
  должны быть загружены одним или несколькими архивами в проект дисциплины <<Курсовой проект 2019-2020>> в личном кабинете в информационной образовательной среде LMS (Learning Management System) НИУ ВШЭ.

  \newpage
  \section{Средства и порядок испытаний}
  \subsection{Технические средства, используемые во время испытаний}
  \begin{enumerate}
    \item Процессор архитектуры x86 или x64 с частотой не менее 1 ГГц;
    \item Не менее 2 ГБ ОЗУ;
    \item Не менее 5 МБ свободного места на жестком диске;
    \item Графическое устройство DirectX 9 с драйвером WDDM 1.0 или более поздней версии.
  \end{enumerate}
  \subsection{Программные средства, используемые во время испытаний}
  \begin{enumerate}
    \item Windows 7 или более поздняя версия операционной системы (32-разрядные или 64-разрядные);
    \item Установленный .NET Framework версии 4.5 и выше;
    % \item Программа должна быть написана на языке программирования C\#.
  \end{enumerate}

  \subsection{Порядок проведения испытаний}
  Испытания должны проводиться в следующем порядке:
  \begin{enumerate}
    \item Выполнить начальную загрузку операционной системы.
    \item Скачать на компьютер архив \CRTzip{} и распаковать его.
    \item Для начала работы программы запустить исполняемый файл \CRTexe{}.
    \item Провести необходимые испытания, описанные ниже в разделе <<Методы испытаний>>.
  \end{enumerate}

  \newpage
  \section{Методы испытаний}
  \subsection{Проверка требований к функциональным характеристикам}
  \subsubsection{Требования к интерфейсу}
  % \todo:
  \begin{itemize}
    \item Поле ввода, где пользователь вводит строку, синтаксический анализ которой будет визуализироваться.
    \CRTfigref{inputbox}{Поле ввода строки}
    \item Поле ввода \CRTfigref{controlpanelTextbox}{Поле ввода скорости}
    и бегунок \CRTfigref{controlpanelSliders}{Бегунки}, для скорости автоматической прокрутки.
    \item Область отображения формальной грамматики, описанной на yacc-подобном языке или на БНФ. \CRTfigref{bnf}{Формальная грамматика}
    \item Подсветка того кода, который сейчас выполняются. \CRTfigref{bnfbold}{Подсветка кода}
    \item Кнопка для каждой функции, описанной в \autoref{sec:funcs}. \CRTfigref{funcButtons}{Кнопки}
    \begin{enumerate}
      \item Шаг вперед.
      \item Шаг назад.
      \item Автоматическая прокрутка вперед.
      \item Автоматическая прокрутка назад.
      \item Сохранение текущего состояния в файл.
      \item Загрузка сохранённого состояния из файла.
    \end{enumerate}
  \end{itemize}

  \subsubsection{Требования к организации входных данных}
  Ввод строки символов, синтаксический анализ которой будет визуализироваться, осуществляется пользователем через поле ввода,
  которое появляется после нажатие на текст, расположенный над синтаксическим деревом. (см. \autoref{fig:inputbox})

  \subsubsection{Требования к организации выходных данных}
  Программа будет записывать рисунок синтаксического дерева в файл в формате PNG,
  если нажать на кнопку сохранения текущего состояния в файл (см. \autoref{fig:funcButtons}) и выбрать
  <<Растровый рисунок дерева (*.png)>>. \CRTfigref{savepng}{Запись рисунка в файл}

  \subsection{Проверка требований к надежности}
  На различных этапах работы пользователь может случайно или намеренно вводить противоречивые данные, не отвечающие требованиям программы.
  В таких ситуациях программа остается работоспособной и не завершается аварийно.
  Среди таких сценариев, например, попытка ввести неправильное значение в текстовое поле для строки,
  синтаксический анализ которой будет визуализироваться.
  В таком случае, будет визуализирована попытка синтаксического анализа этой строки.
  В итоге получится дерево самой длинной корректной подстроки, начинающийся с первого символа. \CRTfigref{wrongVal}{Дерево подстроки}

  Но при вводе пустой строки нельзя построить такое дерево,
  и поэтому выводится сообщение об ошибке и восстанавливается старое значение входной строки. \CRTfigref{inputerror}{Сообщение об ошибке}

  При попытке загрузить сохранённое состояние из файла в формате JSON также возможны ошибки,
  если файл не соответствует требуемому формату.
  Если в выбранный файл не является корректным JOSNом, выводится сообщение об ошибке. \CRTfigref{loaderror}{Некорректный JSON}
  Если в файле есть лишние поля, они просто будут игнорироваться, а если каких-то полей нет, то они сохраняют свои старые значения.
  Это сообщение об ошибке также выводится когда тип одного из поле не соответствует требуемому.

  Если в поле ввода скорости вести число, непопадающие в интервал разрешенных значений $[0; 60]$,
  оно заменяется на ближайшее корректное число.
  То же самое происходит если в JSON файле будет некорректное значение скорости.

  Ошибки при сохранении не возникают, потому что их обрабатывает диалоговое окно.
  Например оно не даст попробовать записать в файл только для чтения. \CRTfigref{readonly}{Обрабатанная ошибка}

  Можно сделать вывод, что программа работает корректно, не завершаясь аварийно, и спроектирована таким образом,
  чтобы при возникновении ошибок была возможность провести процесс заново, то есть соответствует предъявляемым к ней требованиям.

  \begin{CRTbibliography}
  \end{CRTbibliography}

  \CRTterminology

  \CRTlistRegistration
\end{document} % конец документа
