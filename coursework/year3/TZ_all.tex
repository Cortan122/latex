\documentclass[a4paper,12pt,reqno]{article}

\newcommand{\CRTdocFormat}{Техническое задание}
\usepackage{styledoc19}

\newcommand{\CRTdocnumMID}{ТЗ 01-1}
\usepackage{CRTconfig}

\usepackage{xltabular}

% https://docs.google.com/document/d/1IgRpeFcbCxL-1lOXPQ0pwlTcjY3MxvBn22CY4djOJ1A
\begin{document}
  \CRTpreamble

  \section{Введение}
  \subsection{Наименование программы}
  \subsubsection{Наименование программы на русском языке}
  Разработка приложения <<\CRTname>>
  \subsubsection{Наименование программы на английском языке}
  Development of the application <<\CRTnameeng>>

  % 1) Просматривать каталоги товаров в различных магазинах
  % 2) Сформировать заказ желаемой продукции
  % 3) Минимизировать стоимость заказа желаемых товаров
  % 4) Оформить доставку продукции

  \subsection{Краткая характеристика области применения}
  <<\CRTname>> -- программа, позволяющая осуществлять покупку товаров по наиболее оптимальной стоимости,
  путем формирования заказа с помощью перечня продукции различных магазинов.
  Кроме того, программа может служить каталогом товаров в различных магазинах.
  Область применения программы -- сфера услуг.

  \newpage
  \section{Основания для разработки}
  \subsection{Документы, на основании которых ведется разработка}
  Основанием для разработки является учебный план подготовки бакалавров по направлению 09.03.04 <<Программная инженерия>> и утвержденная академическим руководителем тема курсового проекта.

  \subsection{Наименование темы разработки}
  Наименование темы разработки -- приложение <<\CRTname>>

  Программа выполняется в рамках темы курсовой работы в соответствии с учебным планом подготовки бакалавров по направлению 09.03.04 <<Программная инженерия>> Национального исследовательского университета <<Высшая школа экономики>>, факультет компьютерных наук, департамент программной инженерии.

  \newpage
  \section{Назначение разработки}
  \subsection{Функциональное назначение}
  Программа предназначена для помощи в формировании наиболее выгодного по стоимости заказа желаемой продукции,
  а также осуществления его доставки.
  Кроме того, программа может использоваться как каталог товаров в различных магазинах.

  \subsection{Эксплуатационное назначение}
  Данное приложение может применяться для просмотра каталогов товаров в различных магазинах,
  минимизации стоимости заказа желаемой продукции, формирования и доставки заказа.
  Программа может быть полезна для экономии бюджета.
  Кроме того, она будет полезна для осведомления о стоимости товаров в различных магазинах.

  \newpage
  \section{Требования к программе}
  \subsection{Требования к функциональным характеристикам}
  \subsubsection{Требования к составу выполняемых функций}
  \label{sec:funcs}
  Программа должна обеспечивать возможность выполнения следующих функций.

  \noindent\textbf{Android приложение}:
  \begin{enumerate}
    \item Меню:
    \begin{itemize}
      \item возможность перейти в раздел поиска товара по названию;
      \item возможность перейти в раздел визуального поиска необходимого товара по картинке из списка;
      \item возможность перейти в раздел выбора товара по параметрам;
      \item возможность перейти в раздел <<Корзина>>;
      \item возможность перейти в раздел авторизации;
      \item возможность перейти в раздел выбора магазинов;
      \item возможность перейти в раздел мои заказы;
      \item возможность перейти в раздел информации о приложении;
    \end{itemize}
    \item Раздел <<Поиск товаров по названию>>:
    \begin{itemize}
      \item возможность ввести в поле поиска название товара и часть его параметров, перейдя впоследствии в раздел выбора товара по параметрам с уже частично заполненными характеристикам (используется сервер, написанный в рамках проекта \AA{AAAA AAAAAAAAA}{3} и В.А. Федченко);
    \end{itemize}
    \item Раздел <<Визуальный поиск товара>>:
    \begin{itemize}
      \item возможность пролистывать списки с ассортиментом магазинов, разбитым на категории товаров (используется сервер, написанный в рамках проекта \AA{AAAA AAAAAAAAA}{3} и В.А. Федченко);
    \end{itemize}
    \item Раздел <<Выбор товара по параметрам>>:
    \begin{itemize}
      \item возможность выбрать желаемые параметра для товара;
      \item возможность добавить в корзину самый дешевый товар, подходящий под выбранные параметры (используется алгоритм минимизации стоимости, написанный в рамках проекта \AA{AAAA AAAAAAAAA}{3} и В.А. Федченко);
    \end{itemize}
    \item Раздел <<Корзина>>:
    \begin{itemize}
      \item возможность увидеть список товаров, добавленных в корзину ранее, отсортированный по магазинам и стоимости заказов в каждом из них;
      \item возможность выбрать конкретный товар из списка добавленных для просмотра полной информации о нём;
      \item возможность просмотра подытогов по стоимости заказов в магазинах по-отдельности;
      \item возможность перейти на страницы магазинов для оплаты;
    \end{itemize}
    \item Раздел <<Авторизация>>:
    \begin{itemize}
      \item возможность зарегистрироваться в программе после введения ФИО, адреса электронной почты и номера телефона;
      \item возможность авторизоваться по ранее зарегистрированным данным (используется сервер, написанный в рамках проекта \AA{AAAA AAAAAAAAA}{3} и В.А. Федченко);
      \item возможность сменить аккаунт;
      \item возможность восстановить пароль;
    \end{itemize}
    \item Раздел <<Выбор магазинов>>:
    \begin{itemize}
      \item возможность выбора магазинов, продукция которых будет использоваться при выборе товаров;
    \end{itemize}
    \item Раздел <<Мои заказы>>:
    \begin{itemize}
      \item возможность просмотра истории заказов, сделанных ранее;
      \item возможность повторения заказа из истории;
    \end{itemize}
    \item Раздел <<Информация о приложении>>:
    \begin{itemize}
      \item возможность просмотреть гайд по приложению;
      \item возможность просмотреть авторов приложения;
    \end{itemize}
  \end{enumerate}

  \noindent\textbf{Сервер}:
  \begin{enumerate}
    \item Программный интерфейс приложения (API) имеющий возможность:
    \begin{enumerate}
      \item регистрации пользователя;
      \item аутентификации пользователя; % тут Аутентификация
      \item получения информации о пользователе;
      \item добавления пользователю аккаунт внешнего магазина; % тут Авторизация
      \item восстановления пароля пользователя;
      \item получения полного списка товаров из всех доступных магазинов;
      \item поиска товаров по имени;
      \item минимизации стоимости заказа желаемой продукции;
      \item хранить историю покупок пользователя;
    \end{enumerate}
    \item База данных, содержащая:
    \begin{enumerate}
      \item Структуру пользователя, в которой содержится:
      \begin{enumerate}
        \item идентификатор пользователя;
        \item данные для аутентификации;
        \item номер телефона;
        \item ФИО;
        \item адрес электронной почты;
        \item адрес проживания; % todo: disambiguation
        \item данные авторизации, например OAuth-токен, одного или более интернет магазинов.
      \end{enumerate}
      \item Структуру товара, в которой содержится:
      \begin{enumerate}
        \item идентификатор товара;
        \item наименование интернет магазина, в котором продаётся данный товар;
        \item общая категория товара, например <<Кисломолочные продукты>>;
        \item тип товара, например <<Кефир>>;
        \item бренд товара, например <<Домик в деревне>>;
        \item вес или объём товара, например <<1 литр>>;
        \item полное наименование товара, включающие в себя его тип, бренд и объём;
        \item цена товара без скидки;
        \item изображение, показывающие внешний вид товара;
        \item срок годности товара;
      \end{enumerate}
    \end{enumerate}
    \item Программа, регулярно обновляющая базу данных товаров, новой информацией.
  \end{enumerate}

  \subsubsection{Требования к организации входных данных}
  Для корректной работы программе необходим доступ к аккаунтам пользователя в магазинах, продукцию которых он желает заказывать.
  Также необходимо стабильное подключение к сети Интернет.
  \subsubsection{Требования к организации выходных данных}
  Программа должна отображать каталог продукции магазинов и сформированного заказа в режиме реального времени на экране пользователя.
  \subsubsection{Требования к временным характеристикам}
  При стабильном высокоскоростном подключении к сети Интернет (не менее 10 Мбит/с) время ожидания не должно превышать 10 секунд.
  % \subsection{Требования к надежности}
  % При любом вводе пользователя программа не должна завершаться аварийно.
  % При неправильном формате вводимых данных программа должна выводить сообщение с предупреждением о неправильном формате данных.

  \subsection{Требования к интерфейсу}
  Интерфейс должен соответствовать нижеперечисленным требованиям:
  \begin{itemize}
    \item интерфейс должен быть реализован на русском языке;
    \item интерфейс должен воспроизводить функциональность из раздела 4.1;
    \item вверху экрана должно быть окно поиска товаров, где справа будет кнопка <<Автоподбор>>.
    \item внизу экрана должна находиться кнопка <<Корзина>>.
  \end{itemize}

  \subsection{Условия эксплуатации}
  \subsubsection{Климатические условия}
  Климатические условия эксплуатации, при которых должны обеспечиваться заданные характеристики, должны удовлетворять требованиям, предъявляемым к персональным компьютерам \cite{gostclimate}.
  Персональный компьютер предназначен для работы в закрытом отапливаемом помещении со стабильными климатическими условиями.
  \begin{enumerate}
    \item влажность от 20\% до 70\%;
    \item температура от 5\degree C до 30\degree C;
    \item атмосферное давление -- от 84 до 106,7 кПа (от 630 до 800 мм рт. ст.)
  \end{enumerate}


  \subsubsection{Требования к видам обслуживания}
  На персональном компьютере или смартфоне, где производится эксплуатация программы, необходимо обеспечить регулярные проверки оборудования и программного обеспечения на наличие сбоев и неполадок. Следует также обеспечить защиту персонального компьютера или смартфона от воздействия шпионских программ, троянских программ и других видов вирусов.
  Если произошел какой-либо непредвиденный сбой в программе, то пользователю для устранения текущих неполадок рекомендуется написать разработчику. Разработчик в свою очередь обязан принять меры по устранению неполадок и выслать пользователю исправленную версию программного продукта.

  \subsubsection{Требования к численности и квалификации персонала}
  Минимальное количество персонала, требуемого для работы программы, должно составлять не менее 2 штатных единиц:
  \begin{enumerate}
    \item Конечный пользователь -- пользователь смартфона.
    \item Администратор сервера.
  \end{enumerate}
  Пользователь смартфона должен обладать практическими навыками работы с пользовательским интерфейсом операционной системы Android.
  Администратор сервера должен обладать практическими навыками запуска сервера под управлением ОС Ubuntu 18.04.6 или аналогичной ей.
  Инструкция по настройке сервера содержится в руководстве программиста данного проекта.

  \subsection{Требования к составу и параметру технических средств}
  Для оптимальной работы приложения необходимо учесть следующие системные требования:

  \begin{enumerate}
    \item Мобильный телефон со следующими минимальными требованиями:
    \begin{enumerate}
      \item операционная система Android версии 7.0 Marshmallow и выше (API level 24+);
      \item 64-разрядный (x64) процессор;
      \item 2 ГБ оперативной памяти (ОЗУ);
      \item 250 МБ свободного места на внутреннем накопителе.
    \end{enumerate}
    \item Сервер со следующими минимальными требованиями:
    \begin{enumerate}
      \item операционная система Ubuntu 18.04.6 или аналогичная ей;
      \item 64-разрядный (x64) 1 ядерный процессор;
      \item 2 ГБ оперативной памяти (ОЗУ);
      \item ГБ свободного места на внутреннем накопителе для программы и ее зависимостей;
      \item 30 ГБ свободного места на внутреннем накопителе для хранения файлов пользователей.
    \end{enumerate}
  \end{enumerate}

  \subsection{Требования к информационной и программной совместимости}
  \subsubsection{Требования к информационным структурам и методам решения}
  Требования к методам решения не предъявляются.
  \subsubsection{Требования к программным средствам, используемым программой}
  Для работы программы необходим следующий состав программных средств:
  \begin{itemize}
    \item ОС Android 7.0 или выше на мобильном устройстве;
    \item ОС Ubuntu 18.04.6 или аналогичная ей на сервере.
  \end{itemize}
  \subsubsection{Требования к исходным кодам и языкам программирования}
  Требования к исходным кодам и языкам программирования не предъявляются.
  \subsubsection{Требования к защите информации и программы}
  Требования к защите информации и программы не предъявляются.

  \subsection{Требования к маркировке и упаковке}
  Программа поставляется в виде программного изделия на внешнем носителе информации -- внешнем флеш-носителе, на котором должны содержаться программная документация, приложение (исполняемые файлы и прочие необходимые для работы программы файлы) и презентация проекта.

  \subsection{Требования к транспортировке и хранению}
  \subsubsection{Требования к хранению и транспортировке программных документов, предоставляемых в печатном виде}
  Требования к транспортировке и хранению программных документов являются стандартными и должны соответствовать общим требованиям хранения и транспортировки печатной продукции.
  \begin{enumerate}
    \item В помещении для хранения печатной продукции допустимы температура воздуха от 10\degree С до 30\degree С и относительная влажность воздуха от 30\% до 60\%.
    \item Документацию хранят и используют на расстоянии не менее 0,5 м от источников тепла и влаги. Не допускается хранение печатной продукции в помещениях, где находятся агрессивные агенты -- растворители, спирт, бензин.
    \item Не допускается попадание на документацию агрессивных агентов.
    \item Транспортировка производится в специальных контейнерах с применением мер по предотвращению деформации документов внутри контейнеров, а также проникновения влаги, вредных газов, пыли, солнечных лучей и образования конденсата внутри контейнеров.
    \item Программные документы, предоставляемые в печатном виде, должны соответствовать общим правилам учета и хранения программных документов, предусмотренных стандартами Единой системы программной документации и соответствовать требованиям ГОСТ 19.602-78.
  \end{enumerate}

  \newpage
  \section{Требования к программной документации}
  \subsection{Предварительный состав программной документации}
  \label{sec:doclist}
  В рамках данной работы должна быть разработана следующая программная документация в соответствии и ГОСТ ЕСПД:
  \begin{itemize}
    \item <<\CRTname>>. Техническое задание \cite{gostTZ};
    \item <<\CRTname>>. Программа и методика испытаний \cite{gostPMI};
    % \item <<\CRTname>>. Текст программы \cite{gostTP};
    \item <<\CRTname>>. Пояснительная записка \cite{gostPZ};
    \item <<\CRTname>>. Руководство оператора \cite{gostRO};
    \item <<\CRTname>>. Руководство программиста \cite{gostRP};
  \end{itemize}

  \subsection{Специальные требования к программной документации}
  \label{sec:smartlms}
  Документы к программе должны быть выполнены в соответствии с ГОСТ 19.106-78 и ГОСТами к каждому виду документа (см. п. \ref{sec:doclist});

  Пояснительная записка должна быть загружена в систему Антиплагиат через Smart LMS <<НИУ ВШЭ>>.

  Документация и программа сдаются в электронном виде в формате .pdf или .docx в архиве формата .zip или .rar;

  За один день до защиты комиссии все материалы курсового проекта:
  \begin{itemize}
    \item техническая документация,
    \item программный проект,
    \item исполняемый файл,
    \item отзыв руководителя,
    \item лист Антиплагиата
  \end{itemize}
  должны быть загружены одним или несколькими архивами в проект дисциплины <<Курсовой проект, 3 курс ПИ>> в личном кабинете в информационной образовательной среде Smart LMS (Learning Management System) НИУ ВШЭ.


  \newpage
  \section{Технико-экономические показатели}
  \subsection{Предполагаемая потребность}
  В данной таблице приведен список прямых конкурентов данного приложения.

  \begin{table}[h!]
    \caption{Прямые конкуренты}
    \begin{tabularx}{\textwidth}{|p{3.5cm}|p{3cm}|X|}
      \hline
      Конкурент & Разработчик & Ссылка \\\hline
      Яндекс.Маркет & Yandex Apps & \url{https://play.google.com/store/apps/details?id=ru.beru.android} \\\hline
      Яндекс.Лавка & Yandex Apps & \url{https://play.google.com/store/apps/details?id=com.yandex.lavka} \\\hline
      Самокат & Smart Space LLC & \url{https://play.google.com/store/apps/details?id=ru.sbcs.store} \\\hline
      Перекрёсток доставка продуктов & АО «ТД «Перекрёсток» & \url{https://play.google.com/store/apps/details?id=ru.perekrestok.app} \\\hline
      ЛЕНТА -- каталог продуктов & LENTA & \url{https://play.google.com/store/apps/details?id=com.icemobile.lenta.prod} \\\hline
      Яндекс Еда -- заказ продуктов & Yandex Apps & \url{https://play.google.com/store/apps/details?id=ru.foodfox.client} \\\hline
    \end{tabularx}
  \end{table}

  Большинство конкурентов приложения реализует только одну из востребованных пользователями функций, тогда как в данный проект заложены самые удачные идеи конкурентов, наиболее востребованные среди пользователей.
  Кроме того, ни одно приложение не дает столь быстрого поиска самого дешевого товара, удовлетворяющего запросам пользователя, как discount delivery.
  По указанным выше причинам приложение будет конкурентно на отечественном рынке.


  Ниже приведен итоговый анализ конкурентов по выделенному набору критериев.
  По каждому критерию выставляется оценка от 0 до 10, после чего вычисляется средневзвешенная оценка:
  \begin{enumerate}
    \item Регистрация -- ее наличие и полезность для пользователя, вес 0.1;
    \item Сортировка товаров по стоимости -- ее наличие и удобство использования, вес 0.2;
    \item Удобство нахождения оптимального по стоимости товара -- количество и простота действий, которые надо совершить для нахождения товара с желаемыми характеристиками, вес 0.5;
    \item Просмотр каталога продукции -- удобство просмотра всего ассортимента продукции (разбиение на категории, легкость использования), вес 0.2;
    \item Заказ продукции -- возможность и удобство заказа продукции прямо из приложения, вес 0.3;
    \item Число агрегируемых магазинов -- количество магазинов, продукцию которых можно просмотреть и приобрести, вес 0.4;
    \item Поиск товаров по названию, вес 0.3;
    \item История заказов -- возможность и удобство просмотра осуществленных ранее заказы, вес 0.4.
  \end{enumerate}
  Максимальный возможный балл 24.

  \begin{table}[h!]
    \caption{Детальный анализ конкурентов}
    \begin{tabularx}{\textwidth}{|p{4cm}|X|X|X|X|X|X|}
      \hline
      & Яндекс Маркет & Яндекс Лавка & Самокат & Пе\-ре\-крёс\-ток & Лента & Яндекс Еда \\\hline
      Регистрация & 9 & 10 & 9 & 8 & 7 & 7 \\\hline
      Сортировка товаров по стоимости & 10 & 2 & 9 & 10 & 10 & 1 \\\hline
      Удобство нахождения оптимального по стоимости товара & 5 & 4 & 6 & 7 & 6 & 1 \\\hline
      Просмотр каталога продукции & 2 & 10 & 10 & 10 & 10 & 7 \\\hline
      Заказ продукции & 4 & 10 & 10 & 8 & 6 & 5 \\\hline
      Число агрегируемых магазинов & 10 & 3 & 3 & 1 & 1 & 10 \\\hline
      Поиск товаров по названию & 10 & 8 & 8 & 10 & 7 & 10 \\\hline
      История заказов & 10 & 9 & 10 & 10 & 6 & 7 \\\hline
      \textbf{Итого} & \textbf{7,5} & \textbf{6,5} & \textbf{7,625} & \textbf{7,542} & \textbf{6,0} & \textbf{5,875} \\\hline
    \end{tabularx}
  \end{table}

  \subsection{Предполагаемая потребность}
  Приложение <<\CRTname>> будет востребовано для заказа товаров.
  Программа будет востребована среди тех, кто хочет сэкономить время и деньги при осуществлении покупок.

  \subsection{Экономические преимущества разработки по сравнению с отечественными и зарубежными образцами или аналогами}
  В сравнении с конкурентами данное приложение имеет следующие преимущества:

  \begin{enumerate}
    \item дает возможность пользоваться практическим всем функционалом без регистрации;
    \item не требует вложения денежных средств во время использования;
    \item задействована продукция нескольких магазинов;
    \item имеет механизм регистрации для синхронизации при использовании разных устройств пользователем;
    \item позволяет найти самую дешевую вариацию товара, совершая минимум действий.
  \end{enumerate}

  \newpage
  \section{Стадии и этапы разработки}
  Стадии и этапы разработки были выявлены с учетом ГОСТ 19.102-77:

  \noindent
  \begin{xltabular}{\textwidth}{|l|p{6cm}|X|}
    \hline\endhead
    \hline\endfoot
    \textbf{Стадии разработки} & \textbf{Этапы работ} & \textbf{Содержание работ} \\\hline
    Техническое задание
    & \multirow{2}{6cm}{Обоснование необходимости разработки программы}
     &Постановка задачи \\\cline{3-3}
    &&Сбор исходных материалов \\\cline{2-3}
    & Научно-исследовательские работы
     &Определение структуры входных и выходных данных \\\cline{3-3}
    &&Определение требований к техническим средствам \\\cline{3-3}
    &&Обоснование принципиальной возможности решения поставленной задачи \\\cline{3-3}
    &&Разработка и утверждение технического задания \\\cline{2-3}
    & Разработка и утверждение технического задания
     &Определение требований к программе \\\cline{3-3}
    &&Определение стадий, этапов и сроков разработки программы и документации на неё \\\cline{3-3}
    &&Согласование и утверждение технического задания \\\hline

    Технический проект
    & Разработка технического проекта
     &Разработка алгоритма решения задачи \\\cline{3-3}
    &&Окончательное определение конфигурации технических средств \\\cline{2-3}
    & Утверждение технического проекта
     &Разработка плана мероприятий по разработке программы \\\hline

    Рабочий проект
    & Разработка приложения
     &Программирование и отладка Android приложения, выполняет М.И. Шкляр \\\cline{2-3}
    & Разработка серверной части
     &Программирование и отладка серверной части, выполняют AAAA AAAAAAA, В.А. Федченко \\\cline{2-3}
    & Разработка программной документации
     &Разработка программных документов в соответствии с требованиями ГОСТ 19.101-77 \cite{gostAll} \\\cline{2-3}
    & Испытания программы
     &Разработка, согласование и утверждение порядка и методики испытаний \\\hline

    Внедрение
    & Подготовка и защита программного продукта
     &Разработка алгоритма решения задачи \\\cline{3-3}
    &&Подготовка программы и программной документации для презентации и защиты \\\cline{3-3}
    &&Утверждение дня защиты программы \\\cline{3-3}
    &&Презентация программного продукта \\\cline{3-3}
    &&Передача программы и программной документации в архив НИУ ВШЭ \\\cline{3-3}
  \end{xltabular}

  Подготовка и передача программы:
  \begin{itemize}
    \item утверждение даты защиты программного продукта;
    \item подготовка программы и программной документации для презентации и защиты;
    \item представление разработанного программного продукта руководителю и получение отзыва;
    \item загрузка Пояснительной записки в систему Антиплагиат через ЛМС НИУ ВШЭ;
    \item загрузка материалов курсового проекта (курсовой работы) в ЛМС, проект дисциплины <<Курсовой проект, 3 курс ПИ>> (п. \ref{sec:smartlms});
    \item защита программного продукта (курсового проекта) комиссии.
  \end{itemize}
  Разработка должна закончиться к 19 апреля 2022 года.

  Исполнители:
  \begin{enumerate}
    \item Шкляр Михаил Игоревич, студент группы БПИ198 факультета компьютерных наук НИУ ВШЭ.
    \item \AA{AAAAAAA AAAAAAAAAA AAAAAAAAAA}{1}, студент группы AAAAAA факультета компьютерных наук НИУ ВШЭ.
    \item Федченко Всеволод Александрович, студент группы AAAAAA факультета компьютерных наук НИУ ВШЭ.
  \end{enumerate}

  \section{Порядок контроля и приемки}
  Проверка программного продукта, в том числе и на соответствие техническому заданию,
  осуществляется исполнителем вместе с заказчиком согласно <<Программе и методике испытаний>>, а также пункту 5.2.
  Защита выполненного проекта осуществляется комиссии, состоящей из преподавателей департамента программной инженерии,
  в утверждённые приказом декана ФКН сроки.

  \begin{CRTbibliography}
  \end{CRTbibliography}

  \CRTlistRegistration
\end{document} % конец документа
