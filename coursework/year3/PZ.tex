\documentclass[a4paper,12pt]{article}

\newcommand{\CRTdocFormat}{Пояснительная записка}
\usepackage{styledoc19}

\newcommand{\CRTdocnumMID}{ПЗ 01-1}
\usepackage{CRTconfig}

\usepackage{environ}
\usepackage{ltablex}
\keepXColumns

\NewEnviron{CRTmethodtableC}[3]{
  \noindent\begin{tabularx}{\textwidth}{|l|l|p{\widthof{#2}}|X|}
    \caption{Описание функций файла #1}\label{tabmethod#3}
    \\\hline \multicolumn{4}{|l|}{\textbf{Методы}}
    \\\hline \textbf{Имя} & \textbf{Тип} & \textbf{Аргументы} & \textbf{Назначение} \\\hline
    \endfirsthead
    \caption*{Продолжение таблицы \ref{tabmethod#3}}
    \\\hline \multicolumn{4}{|l|}{\textbf{Методы}}
    \\\hline \textbf{Имя} & \textbf{Тип} & \textbf{Аргументы} & \textbf{Назначение}
    \endhead
    \BODY
  \end{tabularx}
}

% \includegraphics[width=\textwidth, align=t, smash=br]{example-image}
% AAAAAAA AAAAAAAAAA AAAAAAAAAA. Курсовая работа 3. Discount Delivery

\begin{document}
  \CRTsign
  \CRTpreamble

  \section{Введение}
  \subsection{Наименование программы}
  Разработка сервера сбор данных для приложения <<\CRTname>>

  \subsection{Документы, на основании которых ведется разработка}
  Основанием для разработки является учебный план подготовки бакалавров по направлению 09.03.04 <<Программная инженерия>> и утвержденная академическим руководителем тема курсового проекта.

  \newpage
  \section{Назначение и область применения}
  \subsection{Назначение программы}
  \subsubsection{Функциональное назначение}
  Разрабатываемое приложение <<\CRTname>> предназначено для помощи в формировании заказа желаемой продукции, а также осуществления его доставки.
  Кроме того, приложение может использоваться как каталог товаров в различных магазинах.
  \subsubsection{Эксплуатационное назначение}
  Приложение <<\CRTname>> может применяться для просмотра каталогов товаров в различных магазинах,
  формирования агрегированного заказа желаемой продукции, формирования и доставки заказа.
  Программа может быть полезна для экономии бюджета.
  Кроме того, она будет полезна для осведомления о стоимости товаров в различных магазинах.
  \subsection{Область применения}
  Программа используется в повседневной жизни, когда требуется заказать продукцию из нескольких интернет магазинов сразу.

  \newpage
  \section{Технические характеристики}
  \subsection{Постановка задачи на разработку программы}
  Программа разработана в соответствии с техническим заданием <<\CRTname>>.

  Для разработки данного проекта были выделены следующий задачи:
  \begin{enumerate}
    \item Поиск потенциальных приложений интернет магазинов, внутреннии структуры программных интерфейсов которых совместимы со стандартными методами обратной разработки.
    \item Изучение программных интерфейсов данных приложений и извлечение методов, необходимых для доступа к полному списку доступных товаров. Это сложный процесс, который необходимо проделывать заново для каждого магазина.
    \item Сбор всех предоставляемых продуктов в единую локальную базу данных. Эта база данных будет в дальнейшем использоваться приложением.
    \item Получение всей доступной информации о категориях и других метаданных продуктов, таких как бренд производителя, цена, объём, изображение описывающее внешний вид, срок годности и вес. Данная информация играет важнейшую роль в поиске нужных товаров пользователями приложения.
    \item Объединение данных, полученных из разных интернет магазинов в единый формат, а также заполнение пропущенных значений.
    \item Разработка скрипта, который бы обновлял базу данных каждый заданный промежуток времени. Таким образом у приложения всегда будут актуальные данные о ценах и скидках на продукты.
    \item Интеграция с REST API сервером, разработанным в рамках этого проекта, и следовательно с самим мобильным приложением.
  \end{enumerate}

  \subsection{Описание применяемых технологических методов}
  Программа представляет собой серверную утилиту, состоящею из некоторого количества независимых скриптов,
  написанных на языках программирования Python \cite{python} и Bash.
  Для анализа программных интерфейсов мобильных приложений использовалась программа <<HttpToolkit>> \cite{httptoolkit}.
  Также использовались библиотеки языка Python, BeautifulSoup \cite{beautifulsoup} и Psycopg \cite{psycopg2}
  для анализа HTML кода и записи в базу данных соответственно.

  \subsubsection{Описание общей схемы работы программы}
  \begin{enumerate}
    \item Программа запускается по расписанию каждый день, используя классический демон cron, стандартный механизм регулярного запуска программ, встроенный в операционную систему Linux.
    \item Скрипт, расположенный в файле \texttt{get\_products.py} и использующий библиотеку Psycopg,
      подключается к локальной базе данных PostgreSQL, расположенной на той-же физической машине, что и он сам,
      и создаёт SQL транзакцию, которая выполняет следующие действия:
    \begin{itemize}
      \item Создание всех необходимых таблиц, если они не присутствовали в базе данных.
      \item Удаление устаревших данных из таблиц.
      \item Скачивание всех продуктов магазина <<Дикси>>.
      \item Скачивание всех категорий и подкатегорий магазина <<Пятёрочка>>.
      \item Скачивание всех продуктов магазина <<Пятёрочка>>.
      \item Сопоставление продуктов с соответствущими им категориями и подкатегориями.
      \item Скачивание скидочного каталога магазина <<Магнит>> в формате HTML.
      \item Извлечения из него списка продуктов.
    \end{itemize}
    \item Если все операции были проведены успешно, транзакция завершается и результат сохраняется в базу данных.
    \item При запросах клиента, сервер может обратится к этой базе данных.
  \end{enumerate}

  \subsubsection{Протокол HTTPS}
  \label{sec:https}
  Исторически, большинство веб-сайтов и приложений использовали незашифрованный протокол HTTP.
  Это привило к большим проблемам безопасности, так как любой компьютер или сетевой узел, расположенный на пути между клиентом и сервером,
  сможет прочитать контент всех сообщений \cite{https}.
  Такое отсутствие безопасности приемлемо пока не передаётся никакой конфиденциальной информации,
  и всё, что делают сайты, это показывают статический не персонализированный контент.
  Но как только приложению становится необходимо передавать пароли, или другие персональные данные,
  использование HTTP начинает предоставлять большую угрозу безопасности.

  Несмотря на то что протокол HTTPS существовал ещё в 2000 году \cite{https_rfc},
  большинство сообщений всё ещё использовало небезопасный протокол HTTP.
  Только в 2010ых годах, когда интернет начал рутинно использоваться для важных взаимодействий,
  он начал набирать существенную популярность, и с 2018 все мобильные приложения обязаны использовать протокол HTTPS \cite{android2018}.

  Благодаря криптографическим свойствам протокола HTTPS, любое приложение, которое его использует,
  может передавать любые данные в сообщениях своему серверу,
  и никто, даже сам пользователь, не сможет подделать или прочитать их.
  Под эту защиту попадают не только сами данные, но и их назначение и точное значение URL,
  то есть тот API метод, который вызывает приложение.
  Поэтому понять внутреннею структуру серверного интерфейса приложения довольно сложно.

  Но защиты протокола HTTPS не непробиваемы.
  Для верификации личности сервера, он использует предопределённый список центров сертификации, хранящийся на уровне операционной системы.
  Добавив свой публичный ключ в этот список на совей локальной машине мы можем выдавать себя за любой сервер,
  и перехватывать API запросы всех приложений, установленных на нашем устройстве.
  Такая манипуляция называется атакой посредника \cite{middleman}.

  Однако не все приложения уязвимы такой атаке.
  Некоторые используют закрепление сертификатов, храня свой список центров сертификации, отдельный от системного,
  а некоторые вообще отказываются работать на rooted устройствах.
  Но большинство приложений не защищены от такой атаки, так как она не предоставляет угрозу безопасности данных,
  поскольку её должен выполнять сам пользователь устройства, а не удалённый злоумышленник.

  Таким образом мы можем видеть все запросы приложения конкретного магазина и, воспроизведя их с сервера,
  получить доступ к базе данных продуктов.

  \subsubsection{Описание алгоритма взаимодействия с API мобильного приложения магазина <<Пятёрочка>>}
  Из всех методов REST API интересующего нас приложения, к работе данный программы релевантно только два.
  \begin{enumerate}
    \item Первый метод используется для получения списка доступных продуктов, по одной странице, то есть по 200 экземпляров, за раз.
      Для успешного его вызова необходимо совершить HTTP POST запрос на URL \url{https://products.5-delivery.ru/api/products/list}
      со следующими параметрами:
        $$\begin{cases}
          \texttt{page} &= n, \\
          \texttt{lpage} &= 200, \\
          \texttt{stock} &= 4, \\
          \texttt{region} &= 1 .
        \end{cases}$$
      В результате запрос нам возвращает структуру данных в формате JSON,
      из которых нам важно поле \texttt{products} в объекте \texttt{ANS}.

      В этом запросе параметр $n$ отвечает за номер страницы, которую мы получаем в ответ.
      Поэтому, чтобы получить полный список продуктов, необходимо выполнить данный запрос несколько раз повторно.
      Таким образом, алгоритм совершает запросы с постоянно увеличивающимся значением $n$,
      до тех пор пока они не станут возвращать неполные страницы, то есть страницы с длинной меньше 200.

    \item Второй метод используется для восстановления имён категорий, и соответствущих им подкатегорий.
      Аналогично первому методу, для успешного использования необходим HTTP POST запрос на URL \url{https://products.5-delivery.ru/shop/1/cats_by_stock} с параметром $\texttt{stock} = 4$.
      Результирующая структура данных в объекте \texttt{ANS} будет содержать поля \texttt{cats} и \texttt{tree}
      для хранения метаданных категорий и структуры дерева подкатегорий соответственно.
  \end{enumerate}

  \subsubsection{Описание алгоритма взаимодействия с API мобильного приложения магазина <<Дикси>>}
  Из всех доступных методов нам интересен только один.
  В отличаи от API магазина <<Пятёрочка>>, этот метод менее строгий:
  для его использование не необходим POST запрос, а достаточно просто GET запроса.
  Однако он требует уникальный API токен, который необходимо извлекать из трафика мобильного приложения.
  Таким образом, для успешного выполнения данного метода необходимо совершить HTTP GET запрос на URL
  \url{https://loyalty-api.dixy.ru//api/v1/online_catalog/products?shop_id=882&count=5000}
  с заголовками <<\texttt{User-Agent: Dart/1.20}>> и <<\texttt{Dixy-Api-Token}>>.
  В результате мы получаем структуру данных в формате JSON, в которой нас интересует поле \texttt{products}.
  Для каждого продукта все его нужные метаданные хранятся в поле \texttt{specs}.

  \subsubsection{Описание алгоритма взаимодействия с сайтом магазина <<Магнит>>}
  У магазина <<Дикси>> нет подходящего API мобильного приложение.
  Поэтому, чтобы получить список продуктов, необходимо совершить обычный HTTP запрос на сайт \url{https://magnit.ru/promo/}
  и скачать исходный HTML код данной веб страницы.
  Для синтаксического анализа скаченного кода используется библиотека BeautifulSoup
  и CSS селектор <<\texttt{a.card-sale\_catalogue}>>, который находит все карточки продуктов на странице каталога.
  Далее из каждой найденной карточки определяется имя продукта, ссылка на изображение и цена.

  \subsubsection{Описание алгоритма поиска продуктов одинакового типа}
  \label{sec:search}
  Автоматическое определение продуктов одинакового типа,
  а именно таких продуктов, которые были бы взаимозаменяемыми для среднестатистического пользователя,
  является ключевой функцией данной программы.
  Это необходимо для формирования заказов, которые включают несколько интернет магазинов сразу,
  чтобы пользователь знал какие продукты можно заменять, а какие нет.

  В этом алгоритме важную роль играют категории продуктов.
  Мы делаем предположение, что продукты одной подкатегории, одинакового объёма, веса и бренда, будут взаимозаменяемыми.
  При сравнение названий категорий игнорируется регистр букв и порядок слов.
  Такие образом, например, категории <<Батончик шоколадный>> и <<Шоколадный батончик>> будут считаться эквивалентными.

  Все измерения количества, веса и объёма конвертируются в килограммы и сравниваются с погрешностью $\pm 10\%$.
  У некоторых продуктов нет информации о производителе в их метаданных, но его бренд упоминается в название.
  В таких случаях бренд определяется путём сравнения названия продукта cо всеми уникальными брендами,
  которые уже встречались в метаданных.
  Таким образом удаётся удовлетворить даже тех покупателей, у которых есть сильная лояльность к брендам.

  В случаях, когда всех этих параметров не достаточно, также предусматривается опция поика по имени.
  Аналогично методу сравнения названия категорий, при поиске игнорируется порядок слов и регистр букв.
  Вдобавок на каждом слове используется некий алгоритм стемминга (от английского stem),
  такой специальный лингвистический алгоритм способен автоматически извлекать корень или основу слова,
  избавляясь от суффиксов и окончаний, что особенно важно в русском языке.
  Кроме того, используется фильтр шумовых слов (stopwords), который убирает предлоги и другие бесполезные для поиска слова.
  Все эти алгоритмы компьютерной лингвистики предоставляются библиотекой Natural Language Toolkit,
  которая реализует их не только для русского языка, но и для многих других \cite{nltk}.

  В результате комбинирования этих методов, поиск будет считать запросы
  <<Стиральный порошок 3кг>> и <<Порошок для стирки 2.9кг>> эквивалентными.

  % \subsubsection{Описание алгоритма минимизации цены заказа}
  % Для каждого продукта в заказе пользователя запускается алгоритм поиска, описанный в разделе \ref{sec:search},
  % и сохраняются самые дешёвые варианты для каждого из трёх доступных интернет магазинов: <<Пятёрочки>>, <<Дикси>> и <<Магнита>>.
  % В случае присутствия одинаковых товаров разного объёма, они сравниваются по ценам за килограмм.

  % После чего алгоритм перебирает все $2^3 - 1 = 7$ возможные комбинации магазинов,
  % которые мы собираемся использовать в заказе, и выбирает самую дешёвую из них, учитывая тот факт,
  % что разные магазины имеют разную стоимость доставки и разные минимальные суммы заказов.

  % Такой простой метод полного перебора в нашем случае приемлемый, так как у нас только три магазина,
  % и перебор 7 возможных комбинаций не займёт большое количество времени.
  % В перспективе данный алгоритм можно переработать на использования эвристик и перебор не только комбинаций используемых магазинов,
  % но и распределений каждого продукта на каждый магазин.
  % Но тогда полный перебор станет невозможным, так как количество комбинаций будет равно $3^n$,
  % где $n$ это количество различных продуктов в заказе.
  % Так количество вычислений начинает становиться неуправляемым когда количество продуктов превышает $7$.

  \subsection{Обоснование выбора алгоритма решения задачи}
  Алгоритм поиска, описанный в разделе \ref{sec:search} и включающий в себя стемминг и фильтрацию ненужных слов,
  был выбран так как это стандартный алгоритм, использующийся в большинстве поисковых систем~\cite{googlestemming}.

  Метод обхода защит протокола HTTPS, описанный в разделе \ref{sec:https},
  был выбран так как его использование не требует полного понимание и декомпилирования внутреннего машинного кода исследуемых приложений.

  \subsection{Описание и обоснование выбора состава технических и программных средств}
  \subsubsection{Состав технических и программных средств}
  Состав технических средств, необходимых для работы системы:
  \begin{enumerate}
    \item операционная система Ubuntu 18.04.6 или аналогичная ей;
    \item интерпретатор языка программирования Python версии 3.5 или больше;
    \item система управления базами данных (СУБД) PostgreSQL версии 10.19 или больше;
    \item 64-разрядный (x64) 1 ядерный процессор;
    \item 2 ГБ оперативной памяти (ОЗУ);
    \item 1 ГБ свободного места на внутреннем накопителе для программы и ее зависимостей;
    \item 30 ГБ свободного места на внутреннем накопителе для хранения файлов пользователей.
  \end{enumerate}

  \subsubsection{Обоснование выбора состава технических средств}
  Состав технических средств был выбран согласно составу технических средств для библиотеки Psycopg.
  Также версия языка программирования Python была выбрана, так как она поддерживает аннотации типов \cite{pythontype}.

  Версия операционной системы Ubuntu была выбрана, так как, на момент написания программы,
  она была последняя LTS версия c долгосрочная поддержка \cite{releases}.

  \newpage
  \section{Технико-экономические показатели}
  \subsection{Предполагаемая потребность}
  Предполагается, что разработанным приложением <<\CRTname>> будет пользоваться среднестатистический пользователь мобильных телефонов,
  для повседневных заказов продуктов из интернет магазинов.

  \subsection{Экономические преимущества разработки по сравнению с отечественными и зарубежными образцами или аналогами}
  В сравнении с конкурентами данное приложение имеет следующие преимущества:
  \begin{enumerate}
    \item дает возможность пользоваться практическим всем функционалом без регистрации и без разглашение своего номера телефона;
    \item задействована продукция нескольких магазинов;
    \item не требует вложения денежных средств во время использования;
    \item не требует использование VPN для пользователей расположенный вне территории Российской Федерации;
    \item позволяет найти самую дешевую вариацию товара, совершая минимальное количество действий.
  \end{enumerate}


  \begin{CRTbibliography}
    \bibitem{crontab}
    crontab(5) Linux User's Manual. 22 Oct. 2012.
    //URL: \url{https://man7.org/linux/man-pages/man5/crontab.5.html}
    (Дата обращения: 27.08.2021, режим доступа: свободный)

    \bibitem{https_rfc}
    E. Rescorla, <<HTTP Over TLS>>, RFC Editor, RFC 2818, May 2000. [Online].
    //URL: \url{https://datatracker.ietf.org/doc/html/rfc2818}

    \bibitem{httptoolkit}
    Intercept, debug \& mock HTTP with HTTP Toolkit [Электронный ресурс]
    //URL: \url{https://httptoolkit.tech/}
    (Дата обращения: 10.01.2022, режим доступа: свободный)

    \bibitem{pythontype}
    <<Support for type hints>>. Документация Python [Электронный ресурс]
    //URL: \url{https://docs.python.org/3/library/typing.html}
    (Дата обращения: 11.05.2021, режим доступа: свободный)

    \bibitem{googlestemming}
    Uyar, Ahmet. <<Google Stemming Mechanisms>>.
    Journal of Information Science, vol. 35, no. 5, Oct. 2009, pp. 499--514, doi:10.1177/1363459309336801.

    \bibitem{python}
    Документация Python [Электронный ресурс]
    //URL: \url{https://docs.python.org/3/}
    (Дата обращения: 11.05.2021, режим доступа: свободный)

    \bibitem{beautifulsoup}
    Документация библиотеки <<Beautiful Soup>> [Электронный ресурс]
    //URL: \url{https://www.crummy.com/software/BeautifulSoup/bs4/doc/}
    (Дата обращения: 22.03.2022, режим доступа: свободный)

    \bibitem{psycopg2}
    Документация библиотеки <<Psycopg>> [Электронный ресурс]
    //URL: \url{https://www.psycopg.org/docs/}
    (Дата обращения: 17.01.2022, режим доступа: свободный)

    \bibitem{android2018}
    Статья <<Android Developers Blog: Previewing Android P>> [Электронный ресурс]
    //URL: \url{https://android-developers.googleblog.com/2018/03/previewing-android-p.html}
    (Дата обращения: 17.01.2022, режим доступа: свободны

    \bibitem{https}
    Статья <<HTTPS>> Wikipedia.org
    //URL: \url{https://ru.wikipedia.org/wiki/HTTPS}
    (Дата обращения: 08.04.2022, режим доступа: свободный)

    \bibitem{middleman}
    Статья <<Man-in-the-middle attack>> Wikipedia.org
    //URL: \url{https://en.wikipedia.org/wiki/Man-in-the-middle_attack}
    (Дата обращения: 08.04.2022, режим доступа: свободный)

    \bibitem{nltk}
    Статья <<Natural Language Toolkit>> Wikipedia.org
    //URL: \url{https://en.wikipedia.org/wiki/Natural_Language_Toolkit}
    (Дата обращения: 08.04.2022, режим доступа: свободный)

    \bibitem{releases}
    Статья <<Releases>> Ubuntu Wiki
    //URL: \url{https://wiki.ubuntu.com/Releases}
    (Дата обращения: 17.01.2022, режим доступа: свободный)

    https://wiki.ubuntu.com/Releases
  \end{CRTbibliography}

  \CRTterminology
  \addition{Описание и функциональное назначение методов и полей}
  \begin{CRTmethodtablePython}{get\_products.py}{psycopg2.cursor}{getproducts}
    try\_parse\_float & float & str & Пытается сконвертировать строку в вещественное число \\\hline
    guess\_category\_name & str & dict | str & Угадывает категорию продукта по его имени \\\hline
    get\_5ka\_page & list & int & Скачивает одну страницу продуктов онлайн магазина <<Пятёрочка>> \\\hline
    get\_5ka\_all & list & void & Скачивает полный список продуктов онлайн магазина <<Пятёрочка>> \\\hline
    get\_5ka\_categoires & list & void & Скачивает полный список категорий онлайн магазина <<Пятёрочка>>, а также дерево подкатегорий \\\hline
    create\_table & void & psycopg2.cursor & Создаёт таблицу в базе данных \\\hline
    add\_5ka\_products & void & psycopg2.cursor & Добавляет продукты онлайн магазина <<Пятёрочка>> в базу данных \\\hline
    get\_dixy\_products & list & int & Скачивает полный список продуктов онлайн магазина <<Дикси>>, включающий в себя все их метаданные, до некоторого максимального количества продуктов, которое по умолчанию выставляет как 5000 \\\hline
    add\_dixy\_products & void & psycopg2.cursor & Добавляет продукты онлайн магазина <<Дикси>> в базу данных \\\hline
    get\_magnit\_promo & ResultSet[Tag] & void & Скачивает скидочный каталог онлайн магазина <<Магнит>> \\\hline
    add\_magnit\_products & void & psycopg2.cursor & Добавляет продукты онлайн магазина <<Магнит>> в базу данных \\\hline
    write\_to\_db & void & void & Открывает соединение с базой данных, и записывает в неё все результаты. Является точкой входа в программу \\\hline
  \end{CRTmethodtablePython}

  \CRTlistRegistration
\end{document} % конец документа
