\documentclass[12pt]{article}

\usepackage{cmap} % поиск в pdf
\usepackage[english,russian]{babel} % локализация и переносы

\usepackage[hidelinks]{hyperref}
\usepackage{xurl}
\usepackage{enumitem}

\usepackage[margin=1in]{geometry}

\usepackage{xcolor}
\newcommand{\red}[1]{{\color{red}{#1}}}
\newcommand{\orange}[1]{{\color{orange}{#1}}}
\newcommand{\brown}[1]{{\color{brown}{#1}}}
\newcommand{\teal}[1]{{\color{teal}{#1}}}

% \pagecolor{black}
% \color{white}

\title{История \\ Домашнее задание №5}
\author{\AA{AAAAA AAAAAAA}{4} \\ AAAAAA}

\begin{document}
  \maketitle

  \section{Церковная реформа патриарха Никона -- изоляционизм\protect\footnotemark{} или первая попытка вестернизации\protect\footnotemark{}?}
  \addtocounter{footnote}{-2}
  \stepcounter{footnote}\footnotetext{политика государственной замкнутости, в основе которой лежит идея невовлечения в дела иных государств и наций}
  \stepcounter{footnote}\footnotetext{заимствование западноевропейского или англо-американского образа жизни в области экономики, политики, образования и культуры}

  Реформу Никона на самом деле нельзя считать вестернизацией, потому что обряды,
  заимствованные из современной Никону Греции\footnote{
    Здесь правильнее будет сказать что практика богослужений была заимствованна не из Греции,
    а от Греков, потому что Греции в то время де-факто не существовало,
    и непонятно, какой именно регион имеется ввиду.
  },
  очень сильно отошли от оригинальных церковных обычаев Византии.
  Получилось так, что западные (католические) традиции были больше похожи на старо\-московские, чем на новые.
  Большинство этих изменений в Греции произошли в связи с тем что,
  последние 200 лет\footnote{c 1453 по 1830 год} она была захвачена Османской Империей.
  % были менее чем старые похожи на западные (католические)
  % A было похоже на С, меньше чем B было похоже на C
  % A было похоже меньше чем B на C

  Эту реформу изоляционной считать тоже нельзя.
  Здесь сам Никон говорил что он <<%
    вывел Московскую Русь из позиции изоляционизма среди Православных Церквей и обрядовой реформой приблизил ее к другим Поместным Церквам,
    напомнил о единстве Церкви при поместном разделении,
    подготовил каноническое объединение Великороссии и Малороссии,
    оживил жизнь Церкви, сделав доступным народу творения ее отцов и объяснив ее чины,
    трудился над изменением нравов духовенства...%
  >>.

  Но если это не изоляционизм и не вестернизация, то тогда что?
  Эта реформа была проведена с целью сделать из Москвы <<вселенское православное государство>>,
  то есть единое государство объединяющие всех православных.
  Она оказалась успешной, так как в 1654 году Киев перешел <<под руку московского царя>>.

  \setcounter{section}{4}
  \section{Реформы Петра I -- это полноценная вестернизация или механическое заимствование новшеств для укрепления самодержавия?}
  Реформы Петра можно разделить на 4 основные категории:
  \begin{enumerate}[label=\textbf{\large\arabic*}]
    \item {\bf\large Культурные} (бритые бороды и европейские платья) \\
    Хотя такие внешние изменения могут поначалу показаться бессмысленными,
    они на самом деле не являлись <<механическим заимствованием новшеств>>.
    Пётр проводил культурные реформы чтобы Россия не выделялась на фоне европейских государств,
    чтобы русских воспринимали как своих, а не как восточных варваров.
    Кроме того, всегда легче завоевать народ близкой культуры.

    \item {\bf\large Административные} (сенат и коллегии вместо приказов) \\
    Эти реформы в основном делались для укрепления самодержавия,
    Петру нужна была административная система, которая бы позволила ему более эффективно управлять страной.

    \item {\bf\large Военные} (новая система чинов, призыв рекрутов и полки нового строя) \\
    Формирование армии нового типа -- регулярной, формирующейся на основе набора рекрутов из податных сословий и <<охочих>> людей.
    Увеличение мобильности командного состава различных типов войск.
    В том числе строительство военно-морского флота --
    без него эффективные боевые действия против морских держав организовать было слишком сложно.

    \item {\bf\large Экономические} \\
    Государству не хватало денежных средств для ведения войн и развития промышленности.
    Развитие промышленности требовало дешевой, а в идеале -- бесплатной рабочей силы.
    Поэтому личным указом Пётр I приписывал крестьян к заводам и фабрикам,
    на которых они обязывались работать взамен получая возможность не платить налог государству.
    Эта реформа\footnote{Указ о приписных крестьянах 1703 года} была не совсем вестернизаторской,
    так как в передовых европейских странах\footnote{Англия и Голландия} экономика основывалась на вольнонаёмном труде,
    а эти изменения наоборот укрепляли крепостничество.
    Это можно назвать вестернизацией методами восточной деспотии.
    % выделяется как некая \yellow{<<анти-вестернизация>>} \red{[вестернизация через жопу]},
    % то есть это была вестернизация методами восточной деспотии.

  \end{enumerate}

  Все эти реформы делались с одной масштабной целью --
  повысить престиж России на международный арене и превратить её в Империю.
  Образцом для них в основном послужила Пруссия, несмотря на то, что
  Пётр посещал многие страны Европы, в том числе и Англию и Голландию.

  Это полноценная вестернизация для укрепления самодержавия!

  % \vfill
  % \begin{thebibliography}{9}
  %   \bibitem{никонwiki} \url{https://ru.wikipedia.org/wiki/%D0%A0%D0%B0%D1%81%D0%BA%D0%BE%D0%BB_%D0%A0%D1%83%D1%81%D1%81%D0%BA%D0%BE%D0%B9_%D1%86%D0%B5%D1%80%D0%BA%D0%B2%D0%B8}
  %   \bibitem{никон1} \url{https://histrf.ru/lenta-vremeni/event/view/tsierkovnaia-rieforma-nikona}
  %   \bibitem{никон2} \url{http://molitva-info.ru/duhovnaya-zhizn/tserkovnaya-reforma-nikona.html}
  %   \bibitem{петя1} \url{https://topwar.ru/67698-pri-petre-vesternizaciya-stala-neobratimoy.html}
  %   \bibitem{петя2} \url{https://yandex.ru/turbo?text=https%3A%2F%2Fspravochnick.ru%2Fpolitologiya%2Fsamoderzhavie_v_rossii_i_ego_prichiny%2Freforma_samoderzhaviya_petra_i%2F}
  %   \bibitem{петя3} \url{https://trojden.com/students/russian-history/samigin-history-mid-prof-education-2007/22}
  %   \bibitem{петяреформы} \url{https://xn--1-itb3afj.xn--p1acf/%D1%80%D0%B5%D1%84%D0%BE%D1%80%D0%BC%D1%8B}
  % \end{thebibliography}

\end{document}
