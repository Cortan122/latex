\documentclass[12pt]{article}

\usepackage{cmap} % поиск в pdf
\usepackage[english,russian]{babel} % локализация и переносы

\usepackage[hidelinks]{hyperref}
\usepackage{xurl}
\usepackage{enumitem}

\usepackage[margin=.5in]{geometry}

\usepackage{xcolor}
\definecolor{purple}{HTML}{800080}
\newcommand{\red}[1]{{\color{red}{#1}}}
\newcommand{\orange}[1]{{\color{orange}{#1}}}
\newcommand{\brown}[1]{{\color{brown}{#1}}}
\newcommand{\teal}[1]{{\color{teal}{#1}}}
\newcommand{\purple}[1]{{\color{purple}{#1}}}

\title{История \\ Домашнее задание №11}
\author{AAAAA AAAAAAA \\ AAAAAA}
\date{22 мая 2020 г.}

\begin{document}
  \maketitle

  \section{Противоречия политики Н. С. Хрущева: десталинизация и волюнтаризм}
  Период правления Хрущёва часто называют <<оттепелью>>: были выпущены на свободу многие политические заключённые,
  по сравнению с периодом правления Сталина активность репрессий значительно снизилась.

  Кроме того, Советский Союз достиг больших успехов в покорении космоса.
  Было развёрнуто активное жилищное строительство.
  На железных дорогах была прекращена эксплуатация паровозов\footnote{их заменили тепловозы и электровозы}.
  % Вместе с тем с именем Хрущёва связаны и организация самой жёсткой в послевоенный период антирелигиозной кампании,
  % и расстрел рабочих в Новочеркасске, и судебные процессы с вынесением смертных приговоров против валютчиков и цеховиков,
  % которых советская пропаганда называла <<расхитителями социалистической собственности>>,
  % и принятие ошибочных решений в сельском хозяйстве (в частности, просчёты при освоении целины, укрупнение сёл),
  % внешней политике (Карибский кризис), и подавление Венгерского восстания 1956 года, и травля Бориса Пастернака,
  % и травля художников-авангардистов.

  \subsection{Волюнтаризм}
  Назначение Н. С. Хрущева в 1953 году первым секретарём ЦК КПСС ознаменовало новую эпоху в экономике СССР.
  Стиль руководства Хрущева стали официально называть волюнтаризмом.
  Это связано с тем, что он проводил экономические реформы административными методами,
  пренебрегая всеми рекомендациями опытных экономистов и ученых страны.

  Последствия произвольных решений Хрущева были плачевными для экономики СССР и состояли в следующем:
  \begin{itemize}
    \item безосновательное стремление <<догнать и перегнать Америку>>, связанное с постоянными не подтвержденными экономическими расчетами вложениями инвестиций в промышленность, привело к дефициту бюджета страны;
    \item возникновение <<перегрева>> экономики из-за постоянного расширения отрасли по изготовлению средств производства;
    \item повышение темпов урбанизации привело к повышению уровня безработицы в городах и снижению общей квалификации работников из-за перенаселения и необразованности жителей глубинки;
    \item снижение производства сельскохозяйственной продукции и падение уровня развития аграрного сектора.
  \end{itemize}

  Результатом волюнтаризма Хрущева стало его отстранение от должности в 1964 году путем заговора среди политических деятелей в высшем эталоне власти страны.
  Позже в Большой Советской Энциклопедии была дана следующая характеристика личности Н. С. Хрущева:
  <<В его деятельности имелись элементы субъективизма и волюнтаризма>>.

  Во всех учебниках по истории СССР написано о том, что все решения и действия Хрущева,
  которые касались внешней и внутренней политики страны, были довольно агрессивными и непредсказуемыми.
  Во всех словарях понятие <<волюнтаризм>> объясняется на примере деятельности Хрущева.

  \subsection{Десталинизация}
  С укреплением у власти Хрущёва <<оттепель>> стала ассоциироваться с развенчанием культа личности Сталина.
  Вместе с тем в 1953-1956 годах Сталин ещё продолжал официально почитаться в СССР как великий лидер;
  в тот период на портретах он часто изображался вдвоём с Лениным.
  На 20ом съезде КПСС в 1956 году Хрущёв сделал доклад <<О культе личности и его последствиях>>,
  в котором были подвергнуты критике культ личности Сталина и сталинские репрессии,
  а во внешней политике СССР был провозглашён курс на <<мирное сосуществование>> с капиталистическим миром.

  Десталинизация чрезвычайно негативно повлияла на отношения с Китаем.

  В ночь с 31 октября на 1 ноября 1961 года тело Сталина было вынесено из Мавзолея и перезахоронено у Кремлевской стены.

  При Хрущёве к Сталину относились нейтрально-положительно.
  Во всех советских изданиях хрущёвской оттепели Сталина называли видным деятелем партии,
  стойким революционером и крупным теоретиком партии, сплотившим партию в период тяжёлых испытаний.
  Но в то же время во всех изданиях того времени писали, что Сталин имел свои недостатки и,
  что в последние годы своей жизни он совершил крупные ошибки и перегибы.

  \subsection{Антирелигиозная кампания}
  Ключевым в новой Программе КПСС стало слово <<воспитание>>.
  Оно означало, что теперь партия намерена осуществлять не диктатуру пролетариата,
  а воспитывать достойных членов социалистического (а в перспективе -- коммунистического) общества.
  Прописанный в этой же Программе КПСС <<моральный кодекс строителя коммунизма>> предусматривал,
  что члены социалистического общества должны сочетать в себе <<духовное богатство, моральную чистоту и физическое совершенство>>.

  Борьба с религией велась силами не только правоохранительной системы, но и партийных и советских органов власти,
  руководства и коллективов предприятий, профсоюзов, комсомола, общественных организаций.
  Эта тотальность гонений должна была создать для верующих атмосферу отвержения, культурной изоляции,
  в которой они чувствовали бы себя гражданами второго сорта, изгоями общества,
  недостойными вместе со всем народом войти в светлое будущее.

  Эта кампания велась против всех религий и конфессий СССР, однако в то же время она носила ярко выраженный антисектантский характер.
  Именно им было посвящено значительное количество антирелигиозных публикаций в СМИ, литературных произведений и кинофильмов.

  \subsection{Итоги и противоречия}
  С одной стороны была десталинизация, которая казалось бы некоторой демократизацией или <<оттепелью>>, то есть неким смягчением режима.
  Наряду с этим, например, были:
  \begin{enumerate}
    \item усиленны гонения на религию;
    \item усилилось напряжение в отношениях между СССР и США;
    \item разрыв отношений с коммунистическими режимами в Албании и в Китае;
    \item абсолютно директивная экономическая политика;
    \item борьба с подсобными хозяйствами;
    \item подавления народных протестов, например новочеркасский расстрел и венгерские протесты 1956 года;
    % \item преследования некоторых художников абстракционистов и разгром их выставок;
    \item начало компании против формализма и абстракционизма в живописи;
    % \item насаждение везде кукурузы;%
    % \footnote{Хрущёв съездил в америку, увидел что везде растёт кукуруза и решил что у нас тоже она будет расти}
  \end{enumerate}

  Можно немного ослабить противоречие, сказав что, одной из причин десталинизации могла быть борьба Хрущёва за власть.
  На фоне провозглашённого после смерти Сталина коллективного руководства страной,
  Хрущёв постепенно отодвинул от власти своих конкурентов,
  перешёл к единоличному правлению и начал создавать собственный культ личности.

  \section{Было ли тоталитарным советское общество периода <<застоя>>? Почему?}
  % \subsection{v1}
  % В своей работе <<Тоталитарная диктатура и автократия>>, Карл Фридрих
  % сформулировал ряд определяющих признаков тоталитарного общества.
  % Исходный перечень состоял из шести признаков, но во втором издании книги авторы добавили ещё два,
  % а впоследствии другие исследователи также вносили уточнения:
  % \begin{enumerate}[label=\textbf{\large\arabic*}]
  %   \item \textbf{Наличие одной всеобъемлющей идеологии, на которой построена политическая система общества} \\
  %   Это, очевидно, коммунизм.
  %   \item \textbf{Наличие единственной партии, -- как правило, руководимой диктатором, -- которая сливается с государственным аппаратом и тайной полицией} \\
  %   Единственная партия -- КПСС, диктатор -- Брежнев, тайная полиция -- КГБ.
  %   Это даже было прописано в законе:
  %   Брежневская конституция 1977 года закрепила однопартийную политическую систему.
  %   (Статья 6 <<О руководящей и направляющей роли КПСС>>)
  %   \item \textbf{Крайне высокая роль государственного аппарата; проникновение государства практически во все сферы жизни общества} \\
  %   TODO
  %   \item \textbf{Отсутствие плюрализма в средствах массовой информации} \\
  %   TODO

  % \end{enumerate}
  % \subsection{v2}
  Систему тоталитаризма отличают следующие основные признаки:
  \begin{enumerate}[label=\textbf{\large\arabic*}]
    \item \textbf{Идеологический абсолютизм} \\
    Конституцией 1977 года марксизм-ленинизм был закреплён официальной идеологией Советского Союза.

    Тома полных собраний сочинений основоположников стояли на почётном месте во всех советских библиотеках.
    Существовала также официально одобренная интерпретация трудов классиков, которая менялась с ходом времени.

    Конечной целью марксизма-ленинизма провозглашалось установление коммунистического строя во всём мире;
    при этом СССР и другие социалистические страны должны были служить исходной базой для распространения коммунизма на другие страны.
    СССР также претендовал на роль руководителя всего мирового коммунистического движения,
    что создало основу для конфликта с Югославией и позже с Китаем.
    \item \textbf{Единовластие одной партии} \\
    В годы правления Брежнева слова <<государство>> и <<партия>> практически неразличимы.
    Политический курс страны определяется на съездах ЦК КПСС, партийная структура сливается с государственной.

    Действительно, не замечать дряхлеющее с каждым днем состояние власти было просто невозможно.
    В 1965—1984 гг. состав Политбюро практически не обновлялся,
    большинство его членов пребывало в высшем органе партии около 15 лет, в составе ЦК около 10—12 лет.
    В 1970-е годы средний возраст членов Политбюро достиг 70 лет, состояние их здоровья было зачастую крайне тяжелым.

    Это даже было прописано в законе:
    Брежневская конституция 1977 года закрепила однопартийную политическую систему.
    (Статья 6 <<О руководящей и направляющей роли КПСС>>)
    \item \textbf{Организованный террор и репрессии} \\
    В Брежневский период <<застоя>> репрессий было меньше чем при Сталине, но они всё равно были.

    В 1972-1973 гг. на правозащитные организации обрушилась волна арестов, началась кампания против А. Сахарова,
    который постоянно обращался к властям с требованием защиты политзаключенных.
    В 1974 г. А. Сахарову была присуждена Нобелевская премия мира,
    но разрешение на поездку в Швецию для получения премии ему не дали,
    мотивируя отказ причастностью академика к секретной научной информации.

    После того как в 1975 г. СССР подписал Заключительный акт Совещания по безопасности и сотрудничеству в Европе,
    ситуация с соблюдением прав человека и политических свобод из внутреннего дела страны превратилась в международную.
    После этого советские правозащитные организации оказались под защитой международных норм,
    что крайне раздражало брежневское руководство.

    Во второй половине 1970-х годов Советскому Союзу постоянно предъявляются обвинения
    на официальном международном уровне в несоблюдении прав человека.
    Ответом властей становится усиление репрессий против хельсинкских групп.
    \item \textbf{Монополия власти на информацию} \\
    Функции цензурного контроля были возложены на специальные государственные учреждения.
    Цензура контролировала все внутренние официальные каналы распространения информации:
    книги, периодические издания, радио, телевидение, кино, театр.
    Также широко была распространена и самоцензура.

    Основными объектами цензуры были так называемая <<антисоветская пропаганда>>,
    военные и экономические секреты, негативная информация о состоянии дел в стране,
    любая информация, которая потенциально могла стать поводом для волнений и неудобных аллюзий.
    \item \textbf{Централизованный контроль над экономикой} \\
    Одной из главных проблем экономики СССР был товарный дефицит в стране.
    Товарный дефицит в тех или иных сферах был характерен для определённых периодов в истории существования СССР и сформировал
    <<экономику продавца>> -- производители и система торговли в условиях планового хозяйствования
    не были заинтересованы в качественном сервисе, своевременных поставках,
    привлекательном дизайне и поддержании высокого качества товаров.
    К тому же из-за проблем, характерных для плановой экономики,
    периодически исчезали из продажи самые обычные товары первой необходимости.

    Политический консерватизм, парализуя жизнь общества,
    привел к постепенному свертыванию каких-либо экономических рычагов и замещению их административными методами хозяйствования.
    Государственное планирование доводилось до абсурда, что нашло отражение в постановлении ЦК КПСС от 12 июля 1979 г.
    <<О дальнейшем совершенствовании хозяйственного механизма и задачах партийных и государственных органов>>.
    В нем декларировалось дальнейшее <<повышение роли государственного плана>> как <<важнейшего инструмента государственной политики>>.
    Предлагалось также существенно улучшить систему плановых показателей с тем,
    <<чтобы они всемерно побуждали трудовые коллективы на борьбу за повышение производительности труда,
    максимальное использование основных фондов, за экономию материальных ресурсов>>.
    Число обязательных плановых показателей, которые должны были <<всемерно побуждать на борьбу>>, было увеличено в сотни раз,
    а их содержание уточнялось во втором постановлении, составленном в таком же казенно-бюрократическом духе:
    <<Об улучшении планирования и усилении воздействия хозяйственного механизма на повышение эффективности производства и качества работы>>.

    Плановые показатели охватывали теперь все сферы народного хозяйства, имели свою иерархию -- их могли устанавливать на предприятии,
    в министерстве и Госплане СССР, корректировать в зависимости от вида плана -- годового, пятилетнего, перспективного,
    а также целевой, комплексной или программы регионального развития.
    Плановые показатели начинали жить своей собственной виртуальной жизнью, не согласуясь с ее реалиями.
    % \item \textbf{Милитаризация страны} \\
    % TODO холодная война
  \end{enumerate}

  Из этого всего мы можем сделать вывод что период <<застоя>> действительно был тоталитарным,
  несмотря на меньший накал репрессий по сравнению со Сталинским режимом,
  например при Сталине за стихи расстреливали, при Брежневе -- отправляли в лагеря.
  Из-за этого некоторые историки называют Брежневский режим посттоталитаризмом.

  % \vfill
  % \begin{thebibliography}{9}
  %   \bibitem{хрущев} \url{https://ru.wikipedia.org/wiki/%D0%A5%D1%80%D1%83%D1%89%D1%91%D0%B2,_%D0%9D%D0%B8%D0%BA%D0%B8%D1%82%D0%B0_%D0%A1%D0%B5%D1%80%D0%B3%D0%B5%D0%B5%D0%B2%D0%B8%D1%87}
  %   \bibitem{оттепель} \url{https://ru.wikipedia.org/wiki/%D0%A5%D1%80%D1%83%D1%89%D1%91%D0%B2%D1%81%D0%BA%D0%B0%D1%8F_%D0%BE%D1%82%D1%82%D0%B5%D0%BF%D0%B5%D0%BB%D1%8C}
  %   \bibitem{семилетка} \url{https://ru.wikipedia.org/wiki/%D0%A1%D0%B5%D0%BC%D0%B8%D0%BB%D0%B5%D1%82%D0%BA%D0%B0}
  %   \bibitem{религия} \url{https://ru.wikipedia.org/wiki/%D0%A5%D1%80%D1%83%D1%89%D1%91%D0%B2%D1%81%D0%BA%D0%B0%D1%8F_%D0%B0%D0%BD%D1%82%D0%B8%D1%80%D0%B5%D0%BB%D0%B8%D0%B3%D0%B8%D0%BE%D0%B7%D0%BD%D0%B0%D1%8F_%D0%BA%D0%B0%D0%BC%D0%BF%D0%B0%D0%BD%D0%B8%D1%8F0}
  %   \bibitem{волюнтаризм} \url{https://advi.club/psihologiya-i-obshhestvo/131-chto-oznachaet-volyuntarizm.html}
  % \end{thebibliography}

\end{document}
