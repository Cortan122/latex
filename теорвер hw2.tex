\documentclass{article}
\usepackage{122}

\title{Теория вероятности \\ Домашнее задание №2}
\author{\AA{AAAAA AAAAAAA}{4} \\ \AA{AAAAAA}{11} \\ Вариант 3}

\begin{document}
  \maketitle

  \setcounter{section}{2}
  \section{Случайная величина $X$ имеет нормальное распределение. Найти плотность распределения вероятностей $f(y)$, если $y=x^3$}
  $$ f(x) = \f{1}{\sqrt{2\pi}}\exp\l(-\f{x^2}{2}\r)$$
  $$ X = \mathcal{N}\l(0,\, 1\r)$$
  $$ f_Y(y) = f_X(\psi(y))\l|\psi'(y)\r|$$
  тут нам надо сначала выразить $y$ через $x$, тоесть взять обратную функцию.
  и тк у нас $x^3$ то обратной будет $\psi(y) = \sqrt[3]{y}$.
  нам ещё надо взять производную $\ds \psi'(y) = \f{1}{3\sqrt[3]{y^2}}$. \\
  $\ds f(y) = f_X\l(\sqrt[3]{y}\r) \l|\f{1}{3\sqrt[3]{y^2}}\r| = \f{f_X\l(\sqrt[3]{y}\r)}{3\sqrt[3]{y^2}}$ \\
  $\ds f(y) = \f{1}{3\sqrt[3]{y^2}\sqrt{2\pi}}\exp\l(-\f{\sqrt[3]{y}^2}{2}\r)$
  % тоесть нам надо найти плотность вероятности $\ds \f{1}{\sqrt{2\pi}} \exp\l(-\f{x^6}{2}\r)$

  \section{Ежегодная потребность в электроэнергии для НИИ составляет в среднем $500$ кВт.ч. Какой расход электроэнергии можно наблюдать в любой день недели с вероятностью не менее $0.85$? (Институт потребляет энергию $365$ дней в году).}
  $$ P(\xi < a) > 1 - \f{E(\xi)}{a} $$
  $\ds 1 - \f{500}{365 \cdot a} = 0.85$ \\
  $\ds 1 - 0.85 = \f{500}{365 \cdot a}$ \\
  $\ds a = \f{500}{365 \cdot \l(1 - 0.85\r)} \approx 9.13242$ \\
  ответ: от $0$ до $\fbox{9.13242}$

  \subsection{Как изменится ответ задачи, если будет известно, что значение среднего квадратичного отклонения ежегодного расхода электроэнергии составит $50$ кВт.ч?}
  $$ E(\xi) = \f{500}{365} $$
  $$ \sigma = \f{50}{365} $$
  $$ P(\l|\xi - E(\xi)\r| < a) > 1 - \f{\sigma^2}{a^2} $$
  $\ds 1 - \f{\sigma^2}{a^2} = 0.85$ \\
  $\ds 1 - 0.85 = \f{\sigma^2}{a^2}$ \\
  $\ds \sqrt{1 - 0.85} = \f{\sigma}{a}$ \\
  $\ds a = \f{50}{365 \cdot \sqrt{1 - 0.85}} \approx 6.08828$ \\
  ответ: от $-4.7184$ до $7.45814$ (тоесть от $0$ до $\fbox{7.45814}$)

\end{document}
