\documentclass[aspectratio=169]{beamer}
\beamertemplatenavigationsymbolsempty
\usetheme{Boadilla}
\usepackage{textpos} % package for the positioning
\usepackage{enumitem}
\usepackage{amsmath,amsfonts,amssymb}

\usepackage{listings}
\usepackage{xcolor}
\usepackage{tasks}
\usepackage{braket}

% Define color values for the theme
\definecolor{backcolour}{rgb}{0.97, 0.97, 1.0}    % Light purple background
\definecolor{codegreen}{rgb}{0.0, 0.6, 0.0}       % Green for comments
\definecolor{codegray}{rgb}{0.5, 0.5, 0.5}        % Gray for line numbers
\definecolor{codepurple}{rgb}{0.58, 0.0, 0.83}    % Purple for strings
\definecolor{keywordcolor}{rgb}{0.5, 0.0, 0.5}    % Dark purple for keywords

% Define the style for the listings
\lstdefinestyle{mystyle}{
    backgroundcolor=\color{backcolour},   
    commentstyle=\color{codegreen},
    keywordstyle=\color{keywordcolor},
    numberstyle=\tiny\color{codegray},
    stringstyle=\color{codepurple},
    basicstyle=\ttfamily\footnotesize,
    breakatwhitespace=false,         
    breaklines=true,                 
    captionpos=b,                    
    keepspaces=true,                 
    numbers=left,                    
    numbersep=5pt,                  
    showspaces=false,                
    showstringspaces=false,
    showtabs=false,                  
    tabsize=2
}

\lstset{style=mystyle}

\definecolor{uwopurple}{RGB}{79,38,131} % official purple color for uwo


\title{Foundations of Quantum Computing}
\author[]{Zyad Altahan (50238152) \and \AA{AAAAAAAAAA AAAAAAA}{6} (\AA{AAAAAAAA}{10}) \and Tobias Mette (3105540) \and Aksa Aksa (50146305) \and Mohamed Sonbol (50262724)}
\institute[]{Department of Computer Science \\ University of Bonn}
\date{\today}

\usepackage[listings,many]{tcolorbox}
\makeatletter
\newtcblisting{mylisting}{
  listing only,
  breakable, enhanced jigsaw,
  listing engine=listings,
  colback=gray!20,
  listing options={
    language=python,
    keywordstyle=\color{blue},
    basicstyle=\ttfamily,
    stringstyle=\color{ForestGreen},
    commentstyle=\color{gray},
    ndkeywordstyle={\color{orange}},
    identifierstyle=\color{black},
    numbers=none,
    showstringspaces=false,
    aboveskip={0\p@ \@plus 6\p@}, belowskip={0\p@ \@plus 6\p@},
    columns=fullflexible, keepspaces=true,
    breaklines=true, breakatwhitespace=true,
    extendedchars=false,
    inputencoding=utf8,
    upquote=true,
    xleftmargin=-25pt,
  }
}
\makeatother

% set color
\setbeamercolor{title in head/foot}{bg=white}
\setbeamercolor{author in head/foot}{bg=white}
\setbeamercolor{date in head/foot}{fg=uwopurple}
\setbeamercolor{date in head/foot}{bg=white}
\setbeamercolor{title}{fg=uwopurple}
\setbeamerfont{title}{series=\bfseries}
\setbeamercolor{frametitle}{fg=uwopurple}
\setbeamerfont{frametitle}{series=\bfseries}
\setbeamercolor{block title}{bg=uwopurple!30,fg=black}
\setbeamercolor{item}{fg=uwopurple}
\setbeamercolor{caption name}{fg=uwopurple!70!}
\usefonttheme[onlymath]{serif}


\begin{document}


\begin{frame}
    \titlepage
    \begin{textblock*}{8cm}(5.0cm,-7.0cm)
        {\large \color{uwopurple}\hspace{0.66cm} \textbf{Exercise Sheet 03}} % Change the lecture # right here! 
    \end{textblock*}
\end{frame}


\begin{frame}[fragile]{Task 3.1: subtasks (a) and (b)}
Because the Pauli matrices are hermitian, $I^\dagger I = I^2$ (and also for $X$, $Y$ and $Z$). So subtasks (a) and (b) can be computed at the same time.

$I^2 = \begin{bmatrix}+1 & 0 \\ 0 & +1 \end{bmatrix}^2 = \begin{bmatrix}1 & 0 \\ 0 & 1 \end{bmatrix} = I$
\hfill
$X^2 = \begin{bmatrix}0 & +1 \\ +1 & 0 \end{bmatrix}^2 = \begin{bmatrix}1 & 0 \\ 0 & 1 \end{bmatrix} = I$

$Z^2 = \begin{bmatrix}+1 & 0 \\ 0 & -1 \end{bmatrix}^2 = \begin{bmatrix}1 & 0 \\ 0 & 1 \end{bmatrix} = I$
\hfill
$Y^2 = \begin{bmatrix}0 & -i \\ +i & 0 \end{bmatrix}^2 = \begin{bmatrix}1 & 0 \\ 0 & 1 \end{bmatrix} = I$

As we can see, the all square to the identity matrix.
\end{frame}


\begin{frame}[fragile]{Task 3.1: subtasks (c), (d) and (e)}
Outer products of $\ket{0}$ and $\ket{1}$ have just one non-zero value each, so we can simply combine then with the coefficients taken from the matrix representations.

$I = \ket{0}\bra{0} + \ket{1}\bra{1}$ \hfill
$X = \ket{1}\bra{0} + \ket{0}\bra{1}$ \hfill
$Y = i\ket{1}\bra{0} - i\ket{0}\bra{1}$ \hfill
$Z = \ket{0}\bra{0} - \ket{1}\bra{1}$

TODO


\end{frame}

\begin{frame}[fragile]{Task 3.2}
Because for all our matrices, squaring them gives us the identity, $X^{2n} = I$ and $X^{2n+1} = X$ for all even and odd numbers.
Adding a factor of $-i$ in front makes it more interesting.
We can consider the series for each value in the final matrix separately. The final result for $R_X(\theta)$ can be expressed in terms of $I$ and $X$.

The series for $I$ is: $1 - \left(\frac{\theta}{2}\right)^2\frac{1}{2!} + \left(\frac{\theta}{2}\right)^4\frac{1}{4!} + \cdots$
which converges to 1 \\
The series for $X$ is: $1 - i\left(\frac{\theta}{2}\right)\frac{1}{1!} - \left(\frac{\theta}{2}\right)^2\frac{1}{2!} + i\left(\frac{\theta}{2}\right)^3\frac{1}{3!} + \left(\frac{\theta}{2}\right)^4\frac{1}{4!} + \cdots$ \\
which is the sum of the taylor series for $\cos(\theta/2)$ and $-i\sin(\theta/2)$.
This can be written as $\cos(\theta/2)-i\sin(\theta/2) = e^{-i \theta/2}$
\[ R_X(\theta) = \begin{bmatrix}
    1 & e^{-i \theta/2} \\
    e^{-i \theta/2} & 1 \\
\end{bmatrix} \]


TODO
\end{frame}


\begin{frame}[fragile]{Task 3.3}
\section*{Solution for Exercise 3.3}

{\tiny
\textbf{Using Taylor Series Expansion for \(e^z\):}

\[
e^z = \sum_{n=0}^\infty \frac{z^n}{n!}
\]

Expanding it:

\[
e^{i\theta} = 1 + i\theta + \frac{(i\theta)^2}{2!} + \frac{(i\theta)^3}{3!} + \frac{(i\theta)^4}{4!} + \dots
\]

For our case: \(\theta = vA\), so:

\[
e^{i v A} = 1 + i v A + \frac{(vA)^2}{2!} + \frac{(vA)^3}{3!} + \dots
\]

Simplifying, since \(i^2 = -1\), we get:

\[
e^{i v A} = 1 + i v A - \frac{(vA)^2}{2!} - i \frac{(vA)^3}{3!} + \frac{(vA)^4}{4!} + \dots
\]

\textbf{Now, using Taylor Series Expansion for \(\cos(\theta)\):}

\[
\cos(\theta) = 1 - \frac{\theta^2}{2!} + \frac{\theta^4}{4!} - \frac{\theta^6}{6!} + \dots
\]

Substituting \(\theta = vA\):

\[
\cos(vA) = 1 - \frac{v^2 A^2}{2!} + \frac{v^4 A^4}{4!} - \dots
\]
}
\end{frame}
\begin{frame}

{\tiny
Since \(A\) is involutory (\(A^2 = I\)):

\[
A^{2n} = I \quad \text{and} \quad A^{2n+1} = A
\]

Thus:

\[
\cos(vA) = I \left(1 - \frac{v^2}{2!} + \frac{v^4}{4!} - \dots\right) = I \cos(v)
\]

\textbf{Similarly, for \(\sin(\theta)\):}

\[
\sin(\theta) = \theta - \frac{\theta^3}{3!} + \frac{\theta^5}{5!} - \dots
\]

Substituting \(\theta = vA\):

\[
i\sin(vA) = i\left(vA - \frac{(vA)^3}{3!} + \frac{(vA)^5}{5!} - \dots\right)
\]

Using \(A^2 = I\), this simplifies to:

\[
i \sin(vA) = i A \left(v - \frac{v^3}{3!} + \frac{v^5}{5!} - \dots\right)
\]

\[
i \sin(vA) = i A \sin(v)
\]

\textbf{Adding these up in the equation for \(e^{i v A}\):}

\[
e^{i v A} = I \cos(v) + i A \sin(v)
\]
}
\end{frame}

\begin{frame}[fragile]{Task 3.4}
Task 3.4
\end{frame}

\begin{frame}[fragile]{Task 3.5}
Task 3.5
\end{frame}


\begin{frame}[fragile]{Task 3.6}
Task 3.6
\end{frame}

\begin{frame}[fragile]{Task 3.7}
Task 3.7
\end{frame}


\begin{frame}[fragile]{Task 3.8}
Task 3.8
\end{frame}


\end{document}