\documentclass{article}

\usepackage{enumitem}
\usepackage{amsmath}
\usepackage{amsfonts}
\usepackage{xcolor}
\usepackage{amssymb}
\usepackage[margin=0.5in]{geometry}
\usepackage[hidelinks]{hyperref}

\let\oldemptyset\emptyset
\let\emptyset\varnothing
\let\oldepsilon\epsilon
\let\epsilon\varepsilon
\newcommand{\N}{\mathbb{N}}
\newcommand{\R}{\mathbb{R}}
\newcommand{\Q}{\mathbb{Q}}

\usepackage{environ}
\NewEnviron{centerframebox}{\begin{center}\fbox{\parbox{0.92\textwidth}{\BODY}}\end{center}}

\title{Combinatorial Optimization \\ Exercise Set 5 \\ Tuesday class}
\author{
  \AA{AAAAAAAAAA AAAAAAA}{6} \\
  \href{mailto:\AA{AAAAAAAAAAAAAAAAAAAA}{7}}{\AA{AAAAAAAAAAAAAAAAAAAA}{7}}
  \and
  Carola Ley \\
  \href{mailto:s6caleyy@uni-bonn.de}{s6caleyy@uni-bonn.de}
  \and
  Bailee Zacovic \\
  \href{mailto:s38bzaco@uni-bonn.de}{s38bzaco@uni-bonn.de}
}

\begin{document}
  \maketitle

  \setcounter{section}{5}
  \subsection{Vertices of Fractional perfect matching polytope are half-integral}
  \begin{centerframebox}
    Let $G = (V,\, E)$ be an undirected graph and $Q$ its fractional perfect
    matching polytope, which is defined by
    \[ Q = \{x \in \R^E : x_e \geq 0\; (e \in E),\, \sum_{e \in \delta(v)} x_e = 1\; (v \in V)\} \]

    Prove that a vector $x \in Q$ is a vertex of $Q$ if and only if there exist vertex disjoint
    odd circuits $C_1,\, \dots,\, C_k$ and a perfect matching $M$ in $G - (V(C_1) \cup\dots\cup V(C_k))$
    such that
    \[ x_e = \begin{cases}
      \frac{1}{2} & \textrm{if } e \in E(C_1) \cup\dots\cup E(C_k) \\
      1 & \textrm{if } e \in M \\
      0 & \textrm{otherwise}
    \end{cases} \]
  \end{centerframebox}
  $\Leftarrow$: Suppose there exist vertex disjoint
    odd circuits $C_1,\, \dots,\, C_k$ and a perfect matching $M$ in $G - (V(C_1) \cup\dots\cup V(C_k))$ such that
   $  x_e = \begin{cases}
      \frac{1}{2} & \textrm{if } e \in E(C_1) \cup\dots\cup E(C_k) \\
      1 & \textrm{if } e \in M \\
      0 & \textrm{otherwise}
    \end{cases} $
    , then $\forall_{e\in E}~ x_e\geq 0$. For $v\in G - (V(C_1) \cup\dots\cup V(C_k))$ there is exactly one incident matching edge and for $v\in V(C_1) \cup\dots\cup V(C_k)$ there are exactly two incident circle  edges with $x_e=\frac 12$, thus $\forall_{v\in V} \sum_{e \in \delta(v)} x_e = 1$, so $x\in Q$.\\
    $x$ is a vertex if there is a supporting hyperplane $\{y:cy=\delta\}$ with $c\neq 0$, $\delta:= \max \{cy:y\in Q\}<\infty$ and $\{x\}=Q\cap \{y:cy=\delta\}$. Let $c_e= \begin{cases}-1 & \text{if } x_e=0\\ 0& \text{else} \end{cases} $, then $cx=0$. There can't be a $y$ with $cy>0$. Suppose there is a $x\neq x'\in Q$ with $cx'=0$, then $x_e=0 \Rightarrow x'_e=0$. Now the vertices in the perfect matching $M$ have only one edge left, by $\sum_{e \in \delta(v)} x'_e = 1$ therefore $x_e=1 \Rightarrow x'_e=1$. As $x\neq x'$ there must be an odd circuit $C_i$ with an edge $e$ with $x'_e\neq \frac 12$. Following the circuit starting from this edge, we get alternating edge weights of $x'_e$ and $1-x_e'$, which leads to a contradiction when arriving at the first edge again as the circuit has odd length. Thus there is no such $x'\neq x$ and $\{x\}=Q\cap \{y:cy=\delta\}$ is a vertex of $Q$.\\
  $\Rightarrow$:
  Let $G'=(V',E')$ be a graph with $V'=\{(v,i):v\in V, i\in \{1,2\}\}, E'= \{\{(v,1),(w,2)\}: \{v,w\}\in E\}$. $G'$ is bipartite, so by Theorem 1.63 $Q_{G'}$ is integral. Every vector in $Q_{G'}$ corresponds to a vector in $Q$,  by $x_{\{v,w\}}=\frac 12 (x_{\{(v,1),(w,2)\}}+x_{\{(v,2),(w,1)\}})$. Let $x'$ be the values of the corresponding vertex to $x$ in $Q_{G'}$. If $x'_{\{(v,1),(w,2)\}}=x'_{\{(v,2),(w,1)\}}=1$, then $\{v,w\}$ is a matching edge in $x$ in $G$. If $x'_{\{(v,1),(w,2)\}}=x'_{\{(v,2),(w,1)\}}=0$ then $x_{\{v,w\}}=0$. If $x'_{\{(v,1),(w,2)\}}=1$ and $x'_{\{(v,2),(w,1)\}}=0$ then there is a cycle in $G'$ with alternating $x'$-values 1 and 0. By the construction of $G'$, this cycle corresponds to an odd cycle in $G$. On this odd cycle we get $x_e=\frac 12 (1+0)=\frac 12$ for every edge $e$. As $\forall_{v\in V}\sum_{e \in \delta(v)} x'_e = 1$, every vertex is covered by an edge of the matching or two edges of an odd cycle with $x_e=\frac 12$, thus $x$ is of the given form.

  \subsection{Minimum Cost Edge Cover Problem}
  \begin{centerframebox}
    Consider the \textsc{Minimum Cost Edge Cover Problem}: Given a
    graph $G$ with weights $c : E(G) \to \R_{\geq 0}$, find an edge cover $F \subseteq E(G)$ that minimizes $\sum_{e\in F} c(e)$.
    Show that the \textsc{Minimum Cost Edge Cover Problem} can be linearly reduced to the \textsc{Minimum Weight Perfect Matching Problem}.
  \end{centerframebox}
  Let $G$ be a graph with weight $c:E(G)\to \R_{\geq 0},$ and for each $v\in V(G),$ let $c_v:=\underset{\{v,w\}\in E(G)}{\text{min}} c(\{v,w\}),$ and set $K:=\sum_{v\in V(G)}c_v.$ Define a new weight function $c':E(G)\to \R$ given by $$c'(\{v,w\})=c_v+c_w-c(\{v,w\}).$$ We claim that a maximum weight matching in $(G,c')$ yields a minimum cost edge cover. To this end, we note that for $M$ a matching contained in $F,$ we may write

  $$\sum_{e\in F} c(e)=\sum_{v\text{ exposed by }M}c_v+\sum_{\{u,w\}\in M}c(\{u,w\})=\sum_{v\in V(G)}c_v+\sum_{\{u,w\}\in M}\left(c(\{u,w\})-c_v-c_w\right)=K+\sum_{\{u,w\}\in M}\left(c(\{u,w\})-c_v-c_w\right).$$
  The problem of finding a minimum weight edge cover associated to $(G,c)$ reduces linearly to the \textsc{Maximum Weight Matching Problem} associated to $(G,c').$ By the argument in Lemma 1.4, we may linearly reduce this further to the \textsc{Minimum Weight Perfect Matching Problem}. $\blacksquare$
  % To this end, we note that any such edge cover $F$ will contain paths of length at most $2$, by the non-negativity of $c$ (the inner edge of any length-$3$ path may always be removed from the cover).

  \subsection{T-cuts and T-joins}
  \begin{centerframebox}
    Let $G$ be a graph and $T \subseteq V (G)$ with $|T|$ even. Prove:
    \begin{enumerate}[label=(\roman*)]
      \item A set $F \subseteq E(G)$ intersects every $T$-join if and only if it contains a $T$-cut.
      \item A set $F \subseteq E(G)$ intersects every $T$-cut if and only if it contains a $T$-join.
    \end{enumerate}
  \end{centerframebox}
  \subsubsection*{(i)}
  $\Leftarrow$: If $F$ contains a $T$-cut, by proposition 2.13.1. for any $T$-join $J$ and any $T$-cut $C$ $J\cap C\neq\emptyset$, thus $J\cap F\neq \emptyset$, so $F$ intersects every $T$-join.\\
  $\Rightarrow$: Let  $F \subseteq E(G)$ be a set that intersects every $T$-join. Assume $F$ contains no $T$-cut, then by definition for all $X\subseteq V(G)$ with $\delta(X)\subseteq F$, $X\cap T$ is even. Then every connected component of $G-F$ contains an even number of vertices in $T$, thus by proposition 2.3 there exists a $T$-join in $G-F$, which is a contradiction to $F$ intersecting every $T$-join, so $F$ contains a $T$-cut.

  \subsubsection*{(ii)}
  $\Leftarrow$: If $F$ contains a $T$-join, by proposition 2.13.1. for any $T$-join $J$ and any $T$-cut $C$ $J\cap C\neq\emptyset$, thus $C\cap F\neq \emptyset$, so $F$ intersects every $T$-cut.\\
  $\Rightarrow$: Let $F \subseteq E(G)$  be a set that intersects every $T$-cut. Let $G'=(V(G),F)$ be the graph with only the edges in $F$. If $G'$ has a connected component $Z$ with $|V(Z)\cap T|$ odd, then $\delta(V(Z))$ is a $T$-cut not intersecting $F$, which is a contradiction. Thus for every connected component $Z$ of $G'$, $|V(Z)\cap T|$ is even, thus by proposition 2.3 there exists a $T$-join in $F$.

  \subsection{Double weights matching}
  \begin{centerframebox}
    Let $G$ be an undirected graph and $c_1,\, c_2 : E(G) \to \R$ be two weight
    functions. Let $\mathcal{M}$ be the set of all matchings that have maximum weight with
    respect to $c_1$. How can we find, in polynomial time, a matching $M \in \mathcal{M}$ such
    that $c_2(M)$ is maximum among all matchings in $\mathcal{M}$? Can you devise a strongly
    polynomial algorithm?
    (For this, in particular, the algorithm should work for arbitrary real numbers,
    assuming that we can perform addition, subtraction and comparison.)

    Note: You can use the fact that there exists a strongly polynomial algorithm for
    the \textsc{Maximum Weight Matching Problem}.
  \end{centerframebox}
  The first thing that comes to mind is to iterate over the set $\mathcal{M}$ and pick the matching with the smallest $c_2(M)$ value,
  but $\mathcal{M}$ is actually of exponential size, and isn't given to us as an input.

  What we can do, is construct a third weight function $c_3$, such that all maximum $c_3$-weight matchings must be in $\mathcal{M}$,
  and also (with a \textbf{lower priority}) have a maximum $c_2$-weight.
  Define $c_3$ as a function from $E(G)$ to a set of ordered tuples $c_3 : E(G) \to (\R,\, \R)$, where $c_3(e) := (c_1(e)\, c_2(e))$.
  These tuples\footnote{just like tuples in python} are comparable according to this formula:
  \[ (a,\, b) < (c,\, d) \iff \begin{cases}
    b < d & a = c \\
    a < c & \textrm{otherwise}
  \end{cases} \]
  We also have to define how to add them: $(a,\, b) + (c,\, d) = (a+c,\, b+d)$.

  Then we can theoretically run the \textsc{Maximum Weight Matching Problem} algorithm on the weights $c_3$,
  and it \textit{theoretically} shouldn't have any problems with our weights not being $\R$,
  but instead this random tuple group we just defined.
  Because the second element of our tuples is considered only when the first elements are equal,
  this algorithm will only pay attention to the $c_2$ weights after the $c_1$ are already maximal,
  which is exactly what we want for our $\mathcal{M}$ set maximizing $c_2(M)$ afterwards.

  If we aren't happy with these ``non-math-y'' tuples, we can replace them with
  infinite ordinal numbers $c_3(e) := \omega \cdot c_1(e) + c_2(e)$, or try to concatenate them into a single real number.
  Let's first consider the case where all of our weights are integers.
  Then we can define $c_3(e) := c_1 \cdot 10^k + c_2$,
  with a sufficiently large $k$ for those numbers to not ``overlap'' when written in base 10.
  The value $k$ needs to be at least $\log_10 (\max c_2 \cdot |E(G)|) + 2$,
  so the sum of all the biggest $c_2$-weight edges will still not reach the part of the number devoted to storing $c_1$.

  The case where $c_1,\, c_2 : E(G) \to \Q$ can be easily converted into integers
  by multiplying every fraction by the $\operatorname{lcm}$ (least common multiple) of all of them.
  But in the case with irrational weights, we have to round them,
  and calculate the precision required so all the possible sums of weights will retain the same order.
  I had an idea to calculate the minimum difference between such sums, but that seems to be an NP-hard problem...

  % TODO (Kotya)

\end{document}
