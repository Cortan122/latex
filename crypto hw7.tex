\documentclass{article}

\usepackage{enumitem}
\usepackage{amsmath}
\usepackage{amsfonts}
\usepackage{amssymb}
\usepackage{textcomp}
\usepackage{gensymb}
\usepackage[dvipsnames]{xcolor}
\usepackage[margin=0.5in]{geometry}
\usepackage[hidelinks,bookmarks=false]{hyperref}

\usepackage{environ}
\NewEnviron{centerframebox}{\begin{center}\fbox{\parbox{0.92\textwidth}{\BODY}}\end{center}}

\let\oldphi\phi
\let\phi\varphi
\newcommand{\Z}{\mathbb{Z}}

\hypersetup{
  colorlinks=true,
  urlcolor=MidnightBlue,
  linkcolor=WildStrawberry,
  % pdfborderstyle={/S/U/W 2}
}

\title{Cryptography \\ Exercise Sheet 7}
\author{
  AAAAAAAAAA AAAAAAA \\
  \href{mailto:AAAAAAAAAAAAAAAAAAAA}{AAAAAAAAAAAAAAAAAAAA}
}

\usepackage[listings,many]{tcolorbox}
\makeatletter
\newtcblisting{mylisting}{
  listing only,
  breakable, enhanced jigsaw,
  listing engine=listings,
  colback=gray!20,
  listing options={
    language=python,
    keywordstyle=\color{blue},
    basicstyle=\ttfamily,
    stringstyle=\color{ForestGreen},
    commentstyle=\color{gray},
    ndkeywordstyle={\color{orange}},
    identifierstyle=\color{black},
    numbers=none,
    showstringspaces=false,
    aboveskip={0\p@ \@plus 6\p@}, belowskip={0\p@ \@plus 6\p@},
    columns=fullflexible, keepspaces=true,
    breaklines=true, breakatwhitespace=true,
    extendedchars=false,
    inputencoding=utf8,
    upquote=true,
    xleftmargin=-25pt,
  }
}
\makeatother

\begin{document}
  \maketitle

  \setcounter{section}{7}
  \subsection{RSA example}
  \begin{centerframebox}
    Run RSA for $\kappa = 40$. Document your procedure.
  \end{centerframebox}
  The first step in the key generation process for RSA is the generation of two large random primes.
  To that and we generate a cryptographically secure $\frac{\kappa}{2}$-bit random number,
  and check if it's a prime.
  The primality test used here is the probabilistic Miller-Rabin algorithm with $k=32$ iterations.
  I used an implementation from \href{https://stackoverflow.com/a/36525088}{Stack Overflow},
  which also added additional checks for small prime divisors.
  \begin{mylisting}
    from random import SystemRandom
    crypto_rng = SystemRandom()

    # Miller-Rabin primality test
    def is_prime(n, k=32):
        from random import randint
        if n < 2: return False
        for p in [2,3,5,7,11,13,17,19,23,29]:
            if n % p == 0: return n == p
        s, d = 0, n-1
        while d % 2 == 0:
            s, d = s+1, d//2
        for i in range(k):
            x = pow(randint(2, n-1), d, n)
            if x == 1 or x == n-1: continue
            for r in range(1, s):
                x = (x * x) % n
                if x == 1: return False
                if x == n-1: break
            else: return False
        return True

    def random_prime(kappa):
        while True:
            n = crypto_rng.getrandbits(kappa) + 2**kappa
            if is_prime(n):
                return n
  \end{mylisting}

  But we have to have another one of those random loops, because the length $\kappa$ is not for each of the primes,
  but for their product $N = p\cdot q$, and despite us picking $\frac{\kappa}{2}$ as the length for each of them,
  there's still a small chance it might fail the $2^{\kappa-1} \leq p\cdot q < 2^\kappa$ condition.
  \begin{mylisting}
    def generate_pq(kappa):
        while True:
            p = random_prime((kappa - 1) // 2)
            q = random_prime((kappa - 1) // 2)
            if 2 ** (kappa - 1) <= p * q <= 2**kappa:
                return p, q
  \end{mylisting}

  The next few steps are much more straightforward, from a coding perspective.
  We generate our primes $p = 969011,\, q = 895789$,
  multiply them to get $N = p \cdot q = 868029394679$ and $L = \phi(N) = (p-1)\cdot(q-1) = 868027529880$,
  pick $e = 2^{16} + 1 = 65537$, as it is the most commonly used value for this constant,
  and compute the inverse $d = e^{-1} \mod L = 139163606153$.
  Because we are using a modern version of Python 3,
  the inverse con be computed in one line with the built in ``\texttt{pow(e, -1, L)}'' function.
  Otherwise we would've had to use the Extended Euclidean algorithm.
  \begin{mylisting}
    kappa = 40
    p, q = generate_pq(kappa)
    N = p * q
    L = (p - 1) * (q - 1)
    e = 2**16 + 1
    d = pow(e, -1, L)
  \end{mylisting}

  From this the public key becomes $(N,\, e) = (868029394679,\, 65537)$
  and the private key $(N,\, d) = (868029394679,\, 139163606153)$.
  We can try encrypting and decrypting a message.
  It has to me a number from $0$ to $N$ (for larger messages we have to come up with some \textbf{padding scheme}).
  I choose $m = 122$.

  Encrypting gives us the ciphertext $c = m^e \mod N = 423023733710$,
  and it can be successfully decrypted into $m' = c^d \mod N = 122$.
  So we can conclude that our implementation works correctly!
  \begin{mylisting}
    message = 122
    ciphertext = pow(message, e, N)
    decrypted = pow(ciphertext, d, N)
    print(message, ciphertext, decrypted)
  \end{mylisting}

  However, the choice of $\kappa=40$ is way too small. A modern machine can factor the number $N$ in less than a millisecond!!
  But we can simply run the same code with a larger $\kappa$, and it will still be quite fast.
  For example, with $\kappa = 2048$ the code takes about $2$ seconds to run, and most of it is spent on generating primes,
  which isn't much of an issue, because the key only has to be generated once.

  \subsection{Modular Arithmetic}
  \begin{centerframebox}
    Use paper and pencil for this exercise.
  \end{centerframebox}
  This means we can't use a programming language.
  % Excuse me, what do you mean by that??
  % What if I use a pen, instead of a pencil...?
  % I choose to interpret this as ``don't just calculate this in python''!

  \subsubsection{Calculate $2^{7654321} \mod 11$. Explain your reasoning.}
  Here we can use the Euler's $\phi$ Function theorem (my favorite).
  \[ a^{\phi(n)} \equiv 1 \mod n \]
  And it can also be written as:
  \[ a^b \equiv a^{(b \operatorname{mod} \phi(n))} \mod n \]
  Because $11$ is a prime, $\phi(11) = 11 - 1 = 10$.
  And $2^{7654321} \equiv 2^{7654321 \operatorname{mod} 10} \equiv 2^1 \equiv 2 \mod 11$.

  \subsubsection{Compute $9^{(2^{7654321})} \in \Z_{23}$.}
  \begin{centerframebox}
    Hint: You need the Little Theorem of Fermat and $9 = 3^2$.
    For a generalizable solution, you would need the Chinese Remainder Theorem.
  \end{centerframebox}
  I think we don't need the Chinese Remainder Theorem, nor Fermat's Little Theorem here.
  We can just use Euler's $\phi$ Function theorem again!

  Because $23$ is also a prime, $\phi(23) = 22$, and now we have to compute $\phi(22) = \phi(11)\phi(2) = 10 \cdot 1 = 10$.
  Now we can apply those phi function on both levels:
  \begin{align*}
    9^{(2^{7654321})}
    &\equiv 9^{2^{7654321} \operatorname{mod} 22} \equiv 9^{2^{7654321 \operatorname{mod} 10} \operatorname{mod} 22} \\
    &\equiv 9^{2^1 \operatorname{mod} 22} \equiv 9^2 \equiv 81 \equiv 12 \\
    &\mod 23\\
  \end{align*}

  \subsection{EEA examples}
  \begin{centerframebox}
    In each run of the algorithm, use the table to document it. Think of the cross-check. State the result.
  \end{centerframebox}

  \subsubsection{Run the Extended Euclidean Algorithm on $42$, $235$. (Do NOT swap the inputs!)}
  \begin{center}
    \begin{tabular}{c|cccc}
      $i$ & $r_i$ & $q_i$ & $s_i$ & $t_i$ \\
      \hline
      $0$ & $42$ &  & $1$ & $0$ \\
      $1$ & $235$ & $0$ & $0$ & $1$ \\
      $2$ & $42$ & $5$ & $1$ & $0$ \\
      $3$ & $25$ & $1$ & $-5$ & $1$ \\
      $4$ & $17$ & $1$ & $6$ & $-1$ \\
      $5$ & $8$ & $2$ & $-11$ & $2$ \\
      $6$ & $1$ & $8$ & $28$ & $-5$ \\
      $7$ & $0$ &  & $-235$ & $42$ \\
    \end{tabular}
  \end{center}
  Check $r_i = s_ia + t_ib$ for $i = 7$.
  \[ -235 \cdot 42 + 42 \cdot 235 = 0 \]

  \subsubsection{Compute the inverse of $42 \in \Z_{1009}$.}
  Run the Extended Euclidean Algorithm on $42$ and $1009$.
  \begin{center}
    \begin{tabular}{c|cccc}
      $i$ & $r_i$ & $q_i$ & $s_i$ & $t_i$ \\
      \hline
      $0$ & $42$ &  & $1$ & $0$ \\
      $1$ & $1009$ & $0$ & $0$ & $1$ \\
      $2$ & $42$ & $24$ & $1$ & $0$ \\
      $3$ & $1$ & $42$ & $-24$ & $1$ \\
      $4$ & $0$ &  & $1009$ & $-42$ \\
    \end{tabular}
  \end{center}
  Check $r_i = s_ia + t_ib$ for $i = 4$.
  \[ 1009 \cdot 42 + -42 \cdot 1009 = 0 \]

  If we get the last non-zero row with $i = 3$, you get:
  \[ 1 = -24 \cdot 42 + 1 \cdot 1009 \]
  \[ 1 \equiv -24 \cdot 42 \mod 1009 \]
  \[ 42^{-1} \equiv -24 \mod 1009 \]
  We can also compute the proper form for $-24 \equiv 985 \mod 1009$.

  \subsubsection{Say $L = 28 \cdot 30$ and you choose $e = 26$. Is $e$ invertible? If so determine its inverse $d$.}
  Run the Extended Euclidean Algorithm on $28 \cdot 30 = 840$ and $26$.

  \begin{center}
    \begin{tabular}{c|cccc}
      $i$ & $r_i$ & $q_i$ & $s_i$ & $t_i$ \\
      \hline
      $0$ & $840$ &  & $1$ & $0$ \\
      $1$ & $26$ & $32$ & $0$ & $1$ \\
      $2$ & $8$ & $3$ & $1$ & $-32$ \\
      $3$ & $2$ & $4$ & $-3$ & $97$ \\
      $4$ & $0$ &  & $13$ & $-420$ \\
    \end{tabular}
  \end{center}
  Check $r_i = s_ia + t_ib$ for $i = 4$.
  \[ 13 \cdot 840 + -420 \cdot 26 = 0 \]

  Because the value $\gcd(a,\, b) = r_3 = 2$ is not $1$, i.e. the numbers are not coprime, $e$ is not invertible.

  \subsubsection{Say $L = 28 \cdot 30$ and you choose $e = 17$. Is $e$ invertible? If so determine its inverse $d$.}
  Run the Extended Euclidean Algorithm on $28 \cdot 30 = 840$ and $17$.

  \begin{center}
    \begin{tabular}{c|cccc}
      $i$ & $r_i$ & $q_i$ & $s_i$ & $t_i$ \\
      \hline
      $0$ & $840$ &  & $1$ & $0$ \\
      $1$ & $17$ & $49$ & $0$ & $1$ \\
      $2$ & $7$ & $2$ & $1$ & $-49$ \\
      $3$ & $3$ & $2$ & $-2$ & $99$ \\
      $4$ & $1$ & $3$ & $5$ & $-247$ \\
      $5$ & $0$ &  & $-17$ & $840$ \\
    \end{tabular}
  \end{center}
  Check $r_i = s_ia + t_ib$ for $i = 5$.
  \[ -17 \cdot 840 + 840 \cdot 17 = 0 \]

  Here the $\gcd(a,\, b) = r_4 = 1$, so $e$ is invertible, and we can use the same idea as in part 7.3.2 to get the inverse.
  \[ 1 = 5 \cdot 840 - 247 \cdot 17 \]
  \[ 1 \equiv -247 \cdot 17 \mod 840 \]
  \[ 17^{-1} \equiv -247 \mod 840 \]
  We can also compute the proper form for $-247 \equiv 593 \mod 840$.

  \subsubsection{Determine $x \in \Z_{899}$ with $x \mod 29 = 7$ and $x \mod 31 = 13$.}
  Note that $899 = 31 \cdot 29$.
  TODO

  \subsection{Cleptography}
  \begin{centerframebox}
    Note: this term is actually spelt ``Kleptography''.

    Someone has replaced the key generation of an RSA implementation.

    We may assume $q \leq \frac{N_1}{p} + c\kappa$ for some small c with high probability.
    This is reasonable. (Namely, with the help of the Prime Number Theorem.)
  \end{centerframebox}
  \subsubsection{Prove that $0 < N - N_1 < 2^{\frac{\kappa}{2} + \log_2\kappa + \log_2 c + 7}$}
  Let's do some algebraic transformations to $N - N_1$:
  \begin{align*}
    N - N_1 &= \frac{N - N_1}{p} \cdot p \\
    &= \left(\frac{N}{p} - \frac{N_1}{p}\right) \cdot p \\
    &= \left(q - \frac{N_1}{p}\right) \cdot p \\
    &\leq \left(\frac{N_1}{p} + c\kappa - \frac{N_1}{p}\right) \cdot p \\
    &\leq c\kappa p
  \end{align*}
  Now if we compare $c\kappa p$ with that large power of 2 expression:
  \begin{align*}
    c\kappa p &< 2^{\frac{\kappa}{2} + \log_2\kappa + \log_2 c + 7} \\
    &< 2^\frac{\kappa}{2} \cdot 2^{\log_2\kappa} \cdot 2^{\log_2 c} \cdot 2^7 \\
    &< 2^\frac{\kappa}{2} \cdot 2^7 \cdot \kappa \cdot c \\
    p&< 2^\frac{\kappa}{2} \cdot 2^7
  \end{align*}
  The size of $p$ and $q$ is around $\frac{\kappa}{2}$ bits, and they can differ by at most 2 bits,
  so, with the help of the extra $2^7$ factor, we can safely say that $p < 2^\frac{\kappa}{2} \cdot 2^7$.

\end{document}
