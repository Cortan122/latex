\documentclass{article}

\usepackage{enumitem}
\usepackage{amsmath}
\usepackage{amsfonts}
\usepackage{amssymb}
\usepackage{textcomp}
\usepackage{gensymb}
\usepackage[dvipsnames]{xcolor}
\usepackage[margin=0.5in]{geometry}
\usepackage[hidelinks,bookmarks=false]{hyperref}

\usepackage{environ}
\NewEnviron{centerframebox}{\begin{center}\fbox{\parbox{0.92\textwidth}{\BODY}}\end{center}}

\let\oldphi\phi
\let\phi\varphi
\newcommand{\Z}{\mathbb{Z}}

\hypersetup{
  colorlinks=true,
  urlcolor=MidnightBlue,
  linkcolor=WildStrawberry,
  % pdfborderstyle={/S/U/W 2}
}

\title{Cryptography \\ Exercise Sheet 9}
\author{
  \AA{AAAAAAAAAA AAAAAAA}{6} \\
  \href{mailto:\AA{AAAAAAAAAAAAAAAAAAAA}{7}}{\AA{AAAAAAAAAAAAAAAAAAAA}{7}}
}

\usepackage[listings,many]{tcolorbox}
\makeatletter
\newtcblisting{mylisting}{
  listing only,
  breakable, enhanced jigsaw,
  listing engine=listings,
  colback=gray!20,
  listing options={
    language=python,
    keywordstyle=\color{blue},
    basicstyle=\ttfamily,
    stringstyle=\color{ForestGreen},
    commentstyle=\color{gray},
    ndkeywordstyle={\color{orange}},
    identifierstyle=\color{black},
    numbers=none,
    showstringspaces=false,
    aboveskip={0\p@ \@plus 6\p@}, belowskip={0\p@ \@plus 6\p@},
    columns=fullflexible, keepspaces=true,
    breaklines=true, breakatwhitespace=true,
    extendedchars=false,
    inputencoding=utf8,
    upquote=true,
    xleftmargin=-25pt,
  }
}
\makeatother

\newcommand{\rem}{\,\operatorname{rem}\,}

\begin{document}
  \maketitle

  \setcounter{section}{9}
  \subsection{An example of Pollard's $\rho$ method}
  \subsubsection{The table}
  \begin{centerframebox}
    Complete the table below, which represents a run of Pollard's $\rho$ algorithm for
    $N = 100 181$ and the initial value $x_0 = 399$, up to $i = 6$. Use
    the update function $F(x) = x^2 + 1$ in $\Z_N$
  \end{centerframebox}
  To get the next value of $x_i$ we use $F(x_{i-1})$,
  and to get the next value of $x_{2i}$ we use $F(F(x_{2(i-1)}))$.

  \begin{mylisting}
    def pollard_rho(x, N):
      y = x
      d = 1

      def F(x):
        return (x*x + 1) % N

      while d == 1:
        x = F(x)
        y = F(F(y))
        d = gcd(abs(x - y), N)

      return d
  \end{mylisting}

  \begin{center}
    \begin{tabular}{c|rrrrr}
      $i$ & $x_i \rem N$ & $x_i \rem 17$ & $x_{2i} \rem N$ & $x_{2i} \rem 17$ & $\gcd(x_i - x_{2i},\, N)$ \\\hline
      0 & 399 & 8 & 399 & 8 & 1 \\
      1 & 59021 & 14 & 84891 & 10 & 1 \\
      2 & 84891 & 10 & 95168 & 2 & 1 \\
      3 & 61828 & 16 & 77478 & 9 & 1 \\
      4 & 95168 & 2 & 5433 & 10 & 1 \\
      5 & 84920 & 5 & 39918 & 2 & 1 \\
      6 & 77478 & 9 & 91894 & 9 & 17
    \end{tabular}
  \end{center}

  \subsubsection{The idea}
  \begin{centerframebox}
    The smallest prime divisor of $N$ is $17$. Describe the idea of the algorithm by looking at
    $x_i \rem 17$ and $x_{2i} \rem 17$ and in particular, why
    we stopped at $i = 6$
  \end{centerframebox}
  % No Idea....
  We stopped because $\gcd(x_i - x_{2i},\, N)$ stopped being $1$.
  This happens when $x_i \rem 17$ equals $x_{2i} \rem 17$.
  The logic here is that the repeated application of $F(x)$ will get stuck in a loop,
  and the faster $x_{2i}$ value will eventually catch up to $x_i$ form behind.

  \subsubsection{The factorization}
  \begin{centerframebox}
    Complete the factorization of $N$ using Pollard's $\rho$ algorithm.
  \end{centerframebox}
  We already know one of the factors of $N$ is $17$.
  So we can just divide it by $17$ and try again.
  And the new value of $N_1 = 100 181 / 17 = 5893$.

  \begin{center}
    \begin{tabular}{c|rrr}
      $i$ & $x_i \rem N_1$ & $x_{2i} \rem N_1$ & $\gcd(x_i - x_{2i},\, N_1)$ \\\hline
      0 & 399 & 399 & 1 \\
      1 & 91 & 2389 & 1 \\
      2 & 2389 & 880 & 1 \\
      3 & 2898 & 869 & 1 \\
      4 & 880 & 5433 & 1 \\
      5 & 2418 & 4560 & 1 \\
      6 & 869 & 3499 & 1 \\
      7 & 858 & 2756 & 1 \\
      8 & 5433 & 2844 & 1 \\
      9 & 5346 & 1000 & 1 \\
      10 & 4560 & 1867 & 1 \\
      11 & 3097 & 4801 & 71 \\
    \end{tabular}
  \end{center}

  Now we can divide again by $71$.
  And make $N_2 = 100 181 / 17 / 71 = 83$.
  But $83$ is already a prime number, so we're done.
  We can try running the algorithm again for good measure.

  \begin{center}
    \begin{tabular}{c|rrr}
      $i$ & $x_i \rem N_2$ & $x_{2i} \rem N_2$ & $\gcd(x_i - x_{2i},\, N_2)$ \\\hline
      0 & 399 & 399 & 1 \\
      1 & 8 & 65 & 1 \\
      2 & 65 & 50 & 1 \\
      3 & 76 & 39 & 1 \\
      4 & 50 & 38 & 1 \\
      5 & 11 & 78 & 1 \\
      6 & 39 & 13 & 1 \\
      7 & 28 & 17 & 1 \\
      8 & 38 & 22 & 1 \\
      9 & 34 & 4 & 1 \\
      10 & 78 & 41 & 1 \\
      11 & 26 & 70 & 1 \\
      12 & 13 & 17 & 1 \\
      13 & 4 & 22 & 1 \\
      14 & 17 & 4 & 1 \\
      15 & 41 & 41 & 83 \\
    \end{tabular}
  \end{center}

  We got back our $83$, and that means we have to stop for good.
  $N = 83 \cdot 71 \cdot 17$

  \subsection{Discrete logarithm}
  \begin{centerframebox}
    Choose one of the presented discrete logarithm algorithms and determine
    the discrete logarithm of $2$ with basis $g = 13$ in the group $Z^\times_p$ where $p$ is the
    second smallest 42-bit prime (ie. $2^{41} \leq p < 2^{42}$).

    Hint: For your choice the value $p - 1$ may be interesting!

    Hint: Your implementation should also be fast with
    \[p = 171 230 365 655 473 418 066 740 317 377 554 260 841\]
    or
    \[p = 191 131 368 170 409 106 437 186 478 244 810 374 001.\]
  \end{centerframebox}

\end{document}
