\documentclass{article}

\usepackage{amsmath}
\usepackage{amsfonts}
\usepackage{amssymb}
\usepackage[dvipsnames]{xcolor}
\usepackage[margin=0.5in]{geometry}
\usepackage[hidelinks, bookmarks=false]{hyperref}
\usepackage{enumitem}

\let\oldemptyset\emptyset
\let\emptyset\varnothing
\let\oldphi\phi
\let\phi\varphi
\let\oldepsilon\epsilon
\let\epsilon\varepsilon

\newcommand{\R}{\mathbb{R}}

\usepackage{environ}
\NewEnviron{centerframebox}{\begin{center}\fbox{\parbox{0.92\textwidth}{\BODY}}\end{center}}

\title{Algorithms for Data Science \\ Exercise Sheet 9}
\author{
  Vladislav Imashev \\ \href{mailto:s05vimas@uni-bonn.de}{s05vimas@uni-bonn.de} \and
  \AA{AAAAAAAAAA AAAAAAA}{6} \\ \href{mailto:\AA{AAAAAAAAAAAAAAAAAAAA}{7}}{\AA{AAAAAAAAAAAAAAAAAAAA}{7}} \and
  German Mikhelson \\ \href{mailto:s17gmikh@uni-bonn.de}{s17gmikh@uni-bonn.de} \and
  Aleksandra Volynets \\ \href{mailto:s02avoly@uni-bonn.de}{s02avoly@uni-bonn.de} \and
  Nikita Morev \\ \href{mailto:s99nmore@uni-bonn.de}{s99nmore@uni-bonn.de}
}


\begin{document}
  \maketitle

  \setcounter{section}{9}
  \subsection{Distances}
  \begin{centerframebox}
    \begin{enumerate}[label=(\roman*)]
      \item Show that the \textbf{Minkowski} distance $D_p$ violates the triangle inequality for $p < 1$.
      \[ D\left(X,Y\right) = \left(\sum_{i=1}^n |x_i-y_i|^p\right)^{\frac{1}{p}} \]
      \item Show that the \textbf{Jaccard} distance is a metric.
      \[ d_J(A,B) = 1 - J(A,B) = \frac{|A \cup B| - |A \cap B|}{|A \cup B|} \]
      \item Show that the \textbf{edit} distance is a metric.

      \item \textbf{Def.}: For $\vec{x},\vec{y}\in \R^d$ define the cosine distance
      $D_{\cos}(\vec{x}, \vec{y})$ by
      \[ D_{\cos}(\vec{x},\vec{y}) =
      \arccos\left(\frac{\langle\vec{x},\vec{y}\rangle}{\Vert{\vec{x}}\Vert\Vert{\vec{y}}\Vert}\right) =
      \arccos\left(\frac{\sum_{i=1}^{d}x_{i}y_{i}}{\sqrt{\sum_{i=1}^{d}x_{i}^{2}}\sqrt{\sum_{i=1}^{d}y_{i}^{2}}}\right) \]

      \begin{enumerate}
        \item  Is the following claim true? Argue why or why not.

        \textbf{Claim:} For all $r > 0$, $D_{\cos}$ is a metric over the sphere in $\R^d$ of radius $r$ around the origin,
        that is, over the set $\{\vec{x} \in \R^d : \Vert\vec{x}\Vert = r \}$,
        for all $d > 0$ integer.
        \item Is it a metric also over $\R^d$? Argue why or why not.
      \end{enumerate}
    \end{enumerate}
  \end{centerframebox}
  TODO %(Vadim; Sasha she/her; Vlad)

  \subsection{High-Dimensional Spaces}
  \begin{centerframebox}
    Give your answer with a proof (sketch) to the question on Slide 31 of Lecture 2024-01-10.

    \begin{itemize}
      \item divide the $d$-cube $[-1,1]^d$ into $2d$ unit $d$-cubes and draw the largest centered $d$-ball (of radius $1/2$) in each sector
      \item draw the $d$-ball centered at the origin that touches the surface of all other balls
      \item let $r_d$ be the radius of this $d$-ball
    \end{itemize}

    \textbf{Question}: $\displaystyle \lim_{d \to \infty} r_d =\ ?$
  \end{centerframebox}
  The central ball we want is defined by touching the smaller radius $1/2$ balls,
  so we can draw a diagonal line from the origin to the center of one of those balls,
  and its length will be the sum of the radii of the two balls.
  \[ r_d + \frac{1}{2} = \operatorname{dist}[(0,\dots,0); (1/2,\dots,1/2)] \]
  \[ r_d = \sqrt{\sum_{i=1}^d (0 - 1/2)^2} - \frac{1}{2} \]
  \[ r_d = \sqrt{d \cdot (1/2)^2} - \frac{1}{2} \]
  \[ r_d = \sqrt{\frac{d}{4}} - \frac{1}{2} \]
  \[ r_d = \frac{\sqrt{d}}{2} - \frac{1}{2} \]
  \[ r_d = \frac{\sqrt{d} - 1}{2} \]
  And we know that $\sqrt{n}$ is an unbounded function that tends to infinity.
  So $r_d$ must also tend to infinity. And even escape the $d$-cube.
  % TODO (Vlad)

  \subsection{Volume of the Shell of $d$-Balls}
  \begin{centerframebox}
    Give the formal claim and its proof asked on Slide 37 of Lecture 2024-01-10.

    \textbf{shell of $d$-balls}: region between two concentric $d$-balls of radiuses $r$ and $r-\epsilon$

    \textbf{Question}: fraction of the volume of the shell of width $\epsilon$ w.r.t. the total volume of the $d$-ball of radius containing it?

    \textbf{Informal claim}: for any $\epsilon > 0$, if $d$ is sufficiently large then almost all the volume is contained in a \textbf{thin} shell of width $\epsilon$
  \end{centerframebox}
  The volume of a $n$-ball equals:
  \[ V_n(R) = \frac{\pi^{n/2}}{\Gamma\left(\tfrac n2 + 1\right)}R^n. \]
  The formal claim can be expressed as a limit for the ratio of volumes of the full ball, and the smaller ball without the shell:
  \[ \forall \epsilon > 0 : \lim_{d \to \infty} \frac{V_d(r - \epsilon)}{V_d(r)} = 0 \]
  Because both balls have the same dimension $d$, the $\pi^{n/2}$ and $\Gamma$ terms cancel out:
  \[ \forall \epsilon > 0 : \lim_{d \to \infty} \frac{(r - \epsilon)^d}{r^d} = 0 \]
  We can rebracket $\frac{(r - \epsilon)^d}{r^d}$ as $\left(\frac{r - \epsilon}{r}\right)^d$,
  and that will be a constant smaller than $1$ to an ever increasing power, which obviously tends to $0$.
  % TODO (Sasha he/him)

  \subsection{Volume of the Unit $d$-Cube at the Surface}
  \begin{centerframebox}
    Give the formal claim and its proof asked on Slide 38 of Lecture 2024-01-10.

    \textbf{Informal claim}: for any $\epsilon > 0$, if $d$ is sufficiently large then the probability that a point chosen uniformly at random belongs to a thin strip of width $\epsilon$ at the surface is (nearly) $1$.
  \end{centerframebox}
  First lets write this as a formal claim, replacing the probability with a volume ratio:
  \[ \forall \epsilon > 0 : \lim_{d \to \infty} \frac{(a - 2\epsilon)^d}{a^d} = 0 \]
  Because the question only asked about unit cubes, we can replace $a$ with 1.
  \[ \forall \epsilon > 0 : \lim_{d \to \infty} \frac{(1 - 2\epsilon)^d}{1} = 0 \]
  \[ \forall \epsilon > 0 : \lim_{d \to \infty} (1 - 2\epsilon)^d = 0 \]
  We know that $1 - 2\epsilon$ is some number smaller than $1$, and the limit for any such number $b$ is $\lim_{n \to \infty} b^n = 0$.
  % TODO (Kostya)

  \subsection{Generation of Random Unit Vectors in $\R^d$}
  \begin{centerframebox}
    Consider the following method generating a random vector $\vec{y}$:
    \begin{enumerate}
      \item for $i = 1,\dots, d$, generate a Gaussian $\mathcal{N}(0, 1)$ random variable $x_i$
      independently,
      \item return \[ \vec{y} = \frac{\vec{x}}{\Vert\vec{x}\Vert}, \] where $\vec{x} = (x_1,\dots,x_d)$
    \end{enumerate}

    \textbf{Task A (15 points)}: Show that the vector $\vec{y}$ obtained is a point chosen uniformly at random from the surface of the $d$-dimensional unit ball (see also, Claim 1 on Slide 24 of Lecture 2024-01-17).

    \textbf{Task B (10 points)}: For a random vector $\vec{y} = (y_1,\dots,y_d)$ generated in Task A,
    let $\vec{z} = (y_1, \dots, y_k)$ for some $1 \leq k \leq d$ (i.e., $\vec{z}$ is the projection of $\vec{y}$ onto
    its first $k$ coordinates). Prove that
    \[\mathbb{E}\left[\Vert\vec{z}\Vert^2\right] = k/d\]
    (see, also, Claim 2 on Slide 24 of Lecture 2024-01-17).
  \end{centerframebox}
  \subsubsection{Task A}
  TODO %(Veronika)
  \subsubsection{Task B}
    From Task A $\vec{y}$ is uniformly distributed on the surface of the d-dimensional unit ball, \[ \vec{y} = \frac{\vec{x}}{\Vert\vec{x}\Vert} => {\Vert\vec{y}\Vert}^2 = 1 = y^2_1 + ... + y^2_d \]

    Since $\vec{y}$ is uniformly distributed, this means that each of $y^2_i$ contributes equally to ${\Vert\vec{y}\Vert}^2$ is equal to $\frac{1}{d}$. So, $\vec{z} = (y_1, \dots, y_k)$ and ${\Vert\vec{z}\Vert}^2 = y^2_1 + ... + y^2_k = \frac{1}{d} + ... + \frac{1}{d} = k \cdot \frac{1}{d} = \frac{k}{d}$, hence

    \[\mathbb{E}\left[\Vert\vec{z}\Vert^2\right] = \mathbb{E}\left[y^2_1 + ... + y^2_k\right] = \mathbb{E}\left[\frac{k}{d}\right] = \{\frac{k}{d} \text{ is constant}\} = \frac{k}{d}\]
\end{document}
