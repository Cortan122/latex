\documentclass[12pt]{article}

\usepackage{cmap} % поиск в pdf
\usepackage[english,russian]{babel} % локализация и переносы

% \usepackage{hyperref}
\usepackage[margin=1in]{geometry}

\title{История. \\ Домашнее задание №3.}
\author{AAAAA AAAAAAA \\ AAAAAA}

\begin{document}
  \maketitle
  \setcounter{section}{2}

  \section{Можно ли считать преп. Иосифа Волоцкого апологетом\protect\footnotemark{} самодержавия\protect\footnotemark{}? Почему?}
  \addtocounter{footnote}{-2}
  \stepcounter{footnote}\footnotetext{так называли всех активных защитников и пропагандистов идеологических течений, систем, теорий или учений}
  \stepcounter{footnote}\footnotetext{так называли абсолютную монархию в России}

  Отношение преп. Иосифа Волоцкого к самодержавию сложно коротко описать, так как оно были немного противоречивым.
  С одной стороны он, в целях борьбы с ересью <<жидовствующх>>, призывает великого князя искоренять её вплоть до митрополита.
  В данном случае власть великого князя распространяется и над духовной властью тоже.
  Суд царя не <<посужается>> святительством, и он патронирует Церковь.

  Однако в седьмом <<Слове>> своего основного труда <<Просветитель>> он учил,
  что если царь <<царствует над людьми, но над ним самим царствуют скверные страсти и грехи>>,
  то неповиновение такому царю не только возможно, но и необходимо, даже под угрозой расправы.

  % Если в одном случае у Иосифа царь есть божий слуга, то в другом -- он не слуга, но наместник Бога.
  Если в одном случае у Иосифа царь есть божий слуга,
  которому <<следует поклоняться и служить телом, а не душой, и воздавать им честь как царю, а не как Богу>> (акцент на естестве царя),
  то в другом -- он не слуга, но наместник Бога, его волеизъявитель:
  <<Вас бо Бог в себе место избра на земли и на свой престол вознес, посади, милость и живот положи у вас>>
  (акцент на сакральность самой власти царя).
  Здесь же находим идею о том, что царь <<подлежит суду и ответственности не пред людьми, а только пред Лицем Всевышняго>>.
  Именно эта вторая часть учения Волоцкого была усвоена Иваном Грозным и легла в основу его религиозно-политических воззрений.

  Можно предположить что вторая позиция Иосифа существовала только для того,
  чтобы убедить Ивана III принять его сторону в споре с Нилом Сорским и в борьбе с <<жидовствующими>>.
  У Волоцкого в борьбе с этой ересью не было других путей, так как приближался 1492 год, по христианскому исчислению 7000 от сотворения мира.
  Именно в этом году ожидались кончина мира и второе пришествие Христа.
  Этим можно объяснить такую распространенность <<жидовствующих>> (они боялись конца света и не верили в него).

  В любом случае получается, что сам Иосиф Волоцкий не стремился сделать власть царя абсолютной;
  его учения просто были потом использованы Иваном Грозным для обоснования своей власти.

  \section{Спор <<иосифлян>> и <<нестяжателей>> и его последствия для истории России}

  Иосифляне и нестяжатели расходились по многим вопросам,
  но главной была проблема церковной собственности и, в частности, земель, принадлежащих Церкви.
  Нил Сорский\footnotemark{} и его <<заволжские старцы>> считали, что Церкви не должно принадлежать значимое материальное имущество,
  что Церковь должна быть <<нищенствующей>> и жить на милостыню.
  Иосиф Волоцкий, напротив, полагал, что Церковь должна располагать материальными богатствами,
  чтобы употреблять их на дело благотворительности, при этом соглашаясь с идеей личного нестяжательства.
  А потому, кстати, в монастырях иосифлян был весьма строгий устав, и обет личного нестяжательства соблюдался весьма строго.

  \footnotetext{Спор этих двух подходов к монашеству можно рассматривать как спор их лидеров: преп. Иосифа Волоцкого и преп. Нила Сорского.}

  И Нил Сорский, и Иосиф Волоцкий хотели, чтобы Церковь была менее зависимой от государства,
  чтобы она могла быть своеобразным морально-нравственным контролером этого самого государства и политической власти.
  Поначалу Иван III\footnotemark{} приблизил к себе не только нестяжателей, но и некоторых богословов из рядов жидовствующих\footnotemark{},
  а потом резко поменял свою точку зрения и стал куда более лоялен к иосифлянам.
  Можно предположить что это произошло из-за того, что Иосиф поменял взгляды на самодержавию (см. четвёртый абзац ответа на предыдущий вопрос).

  \addtocounter{footnote}{-2}
  \stepcounter{footnote}\footnotetext{
    Можно сказать что этот спор был трёхсторонний: Иван III, который фактически выбирал победителя, имел свой интерес:
    право на полноценное владение церковными землями в интересах государства.
  }
  \stepcounter{footnote}\footnotetext{
    В ближайшем окружении Ивана III были люди придерживающиеся этой ереси,
    именно поэтому он им симпатизировал (это были люди, которых он лично знал)
  }

  Победа Иосифа имела эпохальное значение.
  В результате сформировался баланс светской и религиозной власти с уклоном в пользу царя.
  Ещё, так как Волоцкий выступал за казни еретиков, это дало зелёный свет преследованию, например, старообрядцев.
  Некоторые историки считают, что иосифляне стали одними из идеологов введения опричнины во второй половине XVI века.
  Впоследствии это привело к тому, что репрессии Ивана Грозного развернулись и против самой Церкви.
  Жертвами её стали множество священников и монахов, в том числе и митрополит Филипп,
  % один из наиболее известных иосифлян.
  убитый при Иване Грозном за критику опричнины.

  Одно время Иосифа Волоцкого воспринимали как корыстного и властного церковного политика, тогда как Нил Сорский, рисовался идеалистически.
  Церковь, в общем, практически разрешила этот спор, канонизировав обоих: Иосифа Волоцкого в 1579 году, а Нила Сорского в 1656\footnotemark{} году.

  \footnotetext{
    Мы вообще не знаем, когда именно произошла канонизация, а 1656 это дата первого упоминания его как канонизированного святого.
  }

  % доминиканцы и францисканцы

\end{document}
