\documentclass[12pt]{article}

\usepackage{cmap} % поиск в pdf
\usepackage[english,russian]{babel} % локализация и переносы

\usepackage[hidelinks]{hyperref}
\usepackage{xurl}
\usepackage{enumitem}

\usepackage[margin=.8in]{geometry}

\usepackage{xcolor}
\newcommand{\red}[1]{{\color{red}{#1}}}
\newcommand{\orange}[1]{{\color{orange}{#1}}}
\newcommand{\brown}[1]{{\color{brown}{#1}}}
\newcommand{\teal}[1]{{\color{teal}{#1}}}

% \pagecolor{black}
% \color{white}

\title{История \\ Домашнее задание №8}
\author{AAAAA AAAAAAA \\ AAAAAA}

\begin{document}
  \maketitle

  \setcounter{section}{2}
  \section{Причины Русской революции 1917 г.}

  У этой революции были глубинные причины (системный кризис в стране) и сиюминутные причины,
  то есть некие поводы, из-за которых революция началась именно в 1917 году, а не например в 1910-ом.
  Здесь поводом оказалось ухудшение жизни народа в результате экономической разрухи,
  вызванной Первой мировой войной, которая обострила все противоречия в Российской империи.

  Военная нагрузка для экономики оказалась слишком тяжела.
  Приведём весьма характерные цифры.
  Военный бюджет достиг в 1916 году 25 млрд. рублей и на треть покрывался за счёт печатания денег, что вело к росту цен.
  В 1914-1916 годах они выросли более чем в два раза.
  Началось строительство военных предприятий стоимостью 600 млн рублей.

  Сокращалось производство товаров для населения, так как промышленные мощности были загружены военными заказами.
  Производство предметов первой необходимости упало на 11.2\% по сравнению с 1913 годом.

  Не выдерживал транспорт.
  Нехватка вагонов вела к проблемам с поставками сырья для промышленности и продовольствия в города и на фронт.
  В конце 1916 года подвоз продовольствия для армии составлял 61\% от нормы, а в феврале 1917-го -- 42\%.

  % Эти экономические проблемы, вместе с отездом царя в Ставку, стали осноной причиной начала революции.
  % Но это только причина её начала.
  % Мы должны помнить про неё, потому что иначе будет непонатно почему революция началась именно в 1917 году, а не например в 1910-ом.
  % Это была \orange{искра}, которая подожгла то \orange{топливо}, которое накапливалось уже 100 лет.

  Но революция бы не состоялась если бы долгие годы не накапливались глубинные проблемы.
  Вот список этих глубинных проблем:
  \begin{enumerate}
    \item Аграрный вопрос продолжал оставаться острым и нерешенным: было помещичье землевладение, малоземелье и безземелье значительной части крестьян.
    \item Тяжелые условия жизни и труда рабочих, отсутствие рабочего законодательства.
    \item Противоречия между потребностями социально-экономического развития России и очень устаревшей общественно-политической системой (самодержавием).
    \item Продолжал существовать тяжелый национальный гнет, жестокое эксплуатирование населения национальных окраин.
    \item Ограниченность политических прав и свобод граждан.
    \item Недовольство существующим режимом различных слоев населения (рабочих, крестьян, части буржуазии и дворянства, интеллигенции).
    \item Потеря авторитета и ослабление царизма, недовольство политикой царя Николая II.
  \end{enumerate}
  % \teal{
  %   надо помнить что февраьская революция начиссь с <<хлебныx бунтов>>.
  %   тоесть самая \red{главная} из этих причин это \red{№2}.
  %   но все остальные тут тоже важные, тк без них революция бы провалилась.
  %   (там в списке ещё нет того что царь уехал из столицы в Ставку (за день до начала революции))
  % }

  \section{Какие политические силы стремились одержать верх в 1917 г.?}
  Во время этой революции было двоевластие, то есть было два конкурирующих органа верховной власти.
  С одной стороны временное правительство, а с другой советы.
  % Органы власти:
  \begin{enumerate}[label=\textbf{\large\arabic*}]
    \item \textbf{\large Временное правительство} \\
    Было создано IV думой, и из 11 вошедших в него министров 8 были членами думы.
    Поскольку оно не было выбрано демократически, были устроены выборы в учредительное собрание, чтобы оно смогло окончательно решить судьбу страны.
    % медленно вытекло из 4ой думы, выборы в которою были не самые демократические \teal{(сословия бла бла бла)}.
    Временное правительство было официально признано правительством США, Великобритании и Франции.
    \item \textbf{\large Советы} \\
    До революции были органами самоуправления и не претендовали на верховную власть.
    Но потом начали конкурировать с временным правительством,
    несмотря на то что временное правительство было создано по соглашению между Государственной думой и Петроградским Советом.
    \item \textbf{\large Учредительное собрание} \\
    Было избрано только в ноябре 1917 года, а созвано вообще в январе 1918 года.
    Всего было избрано 715 депутатов, из которых 370 (51.74\%) мандатов получили правые эсеры и центристы.
    Поэтому оно не успело ничего сделать и было сразу разогнано Большевиками.
  \end{enumerate}

  За влияние также боролось несколько партий:
  \begin{enumerate}[label=\textbf{\large\arabic*}]
    \item \textbf{\large Большевики} \\
    Радикальное крыло РСДРП\footnote{Российская социал-демократическая рабочая партия}.
    Они считали что коммунизм можно построить только революцией.
    \item \textbf{\large Меньшевики} \\
    Умеренное крыло РСДРП. Они считали что коммунизм надо строить постепенно маленькими реформами, но для этого сначала нужна буржуазная революция.
    \item \textbf{\large Эсеры} \\
    Партия социалистов-революционеров\footnote{поэтому они СР-ы}.
    Они тоже как и РСДРП были разделены по радикальности, но не так значительно.
    Они от РСДРП отличались тем, что они не были марксистами, то есть они считали что социальную революцию проведут не рабочие а крестьяне.
    \item \textbf{\large Кадеты} \\
    Конституционно-демократическая партия. Либералы.
  \end{enumerate}

  % \begin{thebibliography}{9}
  %   \bibitem{причины1} \url{http://bagazhznaniy.ru/history/rossijskaya-revolyuciya-1917g-prichiny-zadachi-posledstviya}
  %   \bibitem{причины2} \url{https://ru.wikipedia.org/wiki/%D0%A0%D0%B5%D0%B2%D0%BE%D0%BB%D1%8E%D1%86%D0%B8%D1%8F_1917_%D0%B3%D0%BE%D0%B4%D0%B0_%D0%B2_%D0%A0%D0%BE%D1%81%D1%81%D0%B8%D0%B8}
  %   \bibitem{причины3} \url{https://histrf.ru/biblioteka/b/1917-v-chiom-ghlavnaia-prichina-rievoliutsii}
  % \end{thebibliography}

\end{document}
