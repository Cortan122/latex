\documentclass{article}

\usepackage{amsmath}
\usepackage{amsfonts}
\usepackage{amssymb}
\usepackage{graphicx}
\usepackage[dvipsnames]{xcolor}
\usepackage[margin=0.5in]{geometry}
\usepackage[hidelinks]{hyperref}
\usepackage{enumitem}

\usepackage{array}

\graphicspath{{.}{./img/}}

\let\oldemptyset\emptyset
\let\emptyset\varnothing
\newcommand{\inc}{\operatorname{inc}}

\usepackage{environ}
\NewEnviron{centerframebox}{\begin{center}\fbox{\parbox{0.92\textwidth}{\BODY}}\end{center}}

\title{Algorithms for Data Science \\ Exercise Sheet 5}
\author{
  Vladislav Imashev \\ \href{mailto:s05vimas@uni-bonn.de}{s05vimas@uni-bonn.de} \and
  \AA{AAAAAAAAAA AAAAAAA}{6} \\ \href{mailto:\AA{AAAAAAAAAAAAAAAAAAAA}{7}}{\AA{AAAAAAAAAAAAAAAAAAAA}{7}} \and
  German Mikhelson \\ \href{mailto:s17gmikh@uni-bonn.de}{s17gmikh@uni-bonn.de} \and
  Aleksandra Volynets \\ \href{mailto:s02avoly@uni-bonn.de}{s02avoly@uni-bonn.de} \and
  Nikita Morev \\ \href{mailto:s99nmore@uni-bonn.de}{s99nmore@uni-bonn.de}
}

\begin{document}
  \maketitle

  \setcounter{section}{5}
  \subsection{Total order}
  \begin{centerframebox}
    Prove the proposition on Slide 21 of 2023-11-22.

    \textbf{Definition}: for $(I,\, \leq)$ we define the order $\preceq$ on the powerset $2^I$ of $I$ by
    \[ A \preceq B \iff A=B \textrm{ or } \max((A\cup B)\setminus(A\cap B))\in B \]
    for all $A,\, B \subseteq I$

    \textbf{Proposition}: $\preceq$ is a total order on $2^I$
  \end{centerframebox}
  This order is very similar to the Lexicographic Order we are all familiar with.
  Except the sequences are read backwards, starting with the maximum element.
  This arrangement is actually called the Colexicographic Order!

  For the total order property we have to show that the element $x = \max((A\cup B)\setminus(A\cap B))$
  is either in $B$ or in $A$, and can't be in both at the same time.
  More specifically $A \preceq B \lor B \preceq A$.
  Because we subtract $A\cap B$, the element $x$ can't be in both sets at once,
  and because we start with $A\cup B$, it has to come from at lest one of them.
  So $A \preceq B$ is mutually exclusive with $B \preceq A$ and they can't be both false at the same time.
  Thus $\preceq$ is a total order.
  $\blacksquare$

  \subsection{Total order properties}
  \begin{centerframebox}
    Prove the claim on Slide 22 of 2023-11-22.

    \textbf{Claim}: the following facts hold for all $A,B,C \subseteq I$:
    \begin{enumerate}[label=(\roman*)]
      \item $A \subseteq B ~\Longrightarrow~ A \preceq B$
      \item $A \prec B \preceq C ~\Longrightarrow~ \max(B \setminus A) \leq \max(C \setminus A)$
      \item $A \prec \inc(A,\, i) \textrm{ for all } i \in I \setminus A$
      \item $A \prec B \textrm{ and } j = \max(B \setminus A) ~\Longrightarrow~ \inc(A,\, j) \subseteq B$
    \end{enumerate}
  \end{centerframebox}
  \subsubsection*{(i) $A \subseteq B ~\Longrightarrow~ A \preceq B$}
  \textbf{Proof:} \\
  Let us write down the definition of $A\preceq B$
  \[ A \preceq B \iff A = B \textrm{ or } \max((A\cup B) \setminus(A\cap B)) \in B\]
  Since A can be equal to B, this satisfies our definition. \\
  Let us show that if $A \subseteq B$, then $A \preceq B$. So, from the condition $A \subseteq B$ it follows that $A \cup B \in B$ and $A \cap B \in A$, so that
  \[\max((A \cup B) \setminus (A \cap B)) = \max(B \setminus A) \in B,\]
  therefore, this property is correct. $\blacksquare$

  \subsubsection*{(ii) $A \prec B \preceq C ~\Longrightarrow~ \max(B \setminus A) \leq \max(C \setminus A)$}
  \textbf{Proof:} \\
  Let us write in more detail the following expression:

  \begin{equation}
    (X \cup Y) \setminus (X \cap Y) = (Y \setminus X) \cup (X \setminus Y) = \max(\max(Y \setminus X), \max(X \setminus Y))
    \tag{*}
  \end{equation}

  So, let us write our expressions $A \prec B$, $B \preceq C$, $A \prec C$ using \textbf{(*)}:
  \[1.\text{  } A \prec B \iff \max(\max(B \setminus A), \max(A \setminus B)) \in B ~\Longrightarrow~ \max(\max(B \setminus A), \max(A \setminus B)) = \max(B \setminus A) \in B\]
  \[2.\text{  } B \preceq C \iff B = C \text{ or } \max(\max(C \setminus B), \max(B \setminus C)) \in C ~\Longrightarrow~ B = C \text{ or } \max(\max(C \setminus B), \max(B \setminus C)) = \max(C \setminus B) \in C\]
  \[3.\text{  } A \prec C \iff \max(\max(C \setminus A), \max(A \setminus C)) \in C ~\Longrightarrow~ \max(\max(C \setminus A), \max(A \setminus C)) = \max(C \setminus A) \in C\]

  So, since
  \[1. \text{   } \max(B \setminus A) \in B\]
  \[2. \text{   } \max(C \setminus A) \in C \]
  \[3. \text{   from } \max(C \setminus B) \in C \text{ and } B = C ~\Longrightarrow~ B \subseteq C \]

  From 1., 2., 3. hence $B \setminus A \subseteq C \setminus A ~\Longrightarrow~ \max(B \setminus A) \leq \max(C \setminus A)$. $\blacksquare$

  \subsubsection*{(iii) $A \prec \inc(A,\, i) \textrm{ for all } i \in I \setminus A$}
  \textbf{Proof:} \\
  Let us remember the definition $\inc(A, i)$:
  \[\inc(A, i) := (A \cup {i}) \setminus \{j \in I: j < i\} \text{ for } A \subseteq I \text{ and } i \in I\]

  So, let us write the expression $A \prec \inc(A,\, i)$:
  \[A \prec \inc(A,\, i) \iff \max( (\inc(A, i) \cup A) \setminus (\inc(A, i) \cap A) )\]

  Let us write $(\inc(A, i) \cup A)$ and $(\inc(A, i) \cap A)$:
  \[1. \text{  } \inc(A, i) \cup A = (A \cup {i}) \setminus \{j \in I: j < i\} \cup A = A \cup \{i\}\]
  \[2. \text{  } \inc(A, i) \cap A = (A \cup {i}) \setminus \{j \in I: j < i\} \cap A = \{j \in A: j > i\}\]

  So let us paste new expression in our general:
  \[\max( (\inc(A, i) \cup A) \setminus (\inc(A, i) \cap A) ) = \max((A \cup {i}) \setminus \{j \in A: j > i\}) = \max(\{j \in A: j \leq i\}) = \{i\} \in \inc(A, i) \text{  } \blacksquare\]

  \textbf{(iv)} $A \prec B \textrm{ and } j = \max(B \setminus A) ~\Longrightarrow~ \inc(A,\, j) \subseteq B$ \\
  \textbf{Proof:}\\
  Let us use expression \textbf{(*)} from \textbf{(ii)} for $A \prec B$:

  \[A \prec B \iff \max(\max(B \setminus A), \max(A \setminus B)) \in B ~\Longrightarrow~ \max(\max(B \setminus A), \max(A \setminus B)) = \max(B \setminus A) = j \in B\]
  \[~\Longrightarrow~ A \in B \text{ and } j \in B ~\Longrightarrow~ (A \cup j) \subseteq B,\]
  \[(A \cup j) = B\text{ because there may be a case when } B \setminus A = \max(B \setminus A)\]

  Let us write $\inc(A, j)$:
  \[\inc(A, j) = (A \cup j) \setminus \{i \in I: i < j\} ~\Longrightarrow~ \inc(A, j) \subseteq (A \cup j),\]
  \[\inc(A, j) = (A \cup j) \text{ because there may be a case where }j{ is the smallest element}\]
  \[~\Longrightarrow~ \inc(A, j) \subseteq (A \cup j) \subseteq B \text{  }\blacksquare\]

  \subsection{Ganter-Reuter algorithm}
  \begin{centerframebox}
    A database has the following five transactions:

    \begin{center}
      \begin{tabular}{|c|c|}
        \hline
        TID & items bought \\ \hline
        1 & M, O, N, K, E, Y \\ \hline
        2 & D, O, N, K, E, Y \\ \hline
        3 & M, A, K, E \\ \hline
        4 & M, U, C, K, Y \\ \hline
        5 & C, O, K, E \\ \hline
      \end{tabular}
    \end{center}

    List the family of closed frequent itemsets with a (relative) minimum support of $0.6$ by the Ganter-Reuter algorithm.
    Give all computational details similarly to the example used in the lecture.
  \end{centerframebox}
  $support(X) = 0.6 = \frac{|D[x]|}{|D|=5}  ~\Longrightarrow~ |D[x]| = t =0.6 \cdot 5 = 3$

  Firstly we should delete all infrequent items so we have:
  \[ \left\{ {M, O, K, E, Y} \right\} \]
  Let's sort in alphabetical order
  \[ E < K < M < O < Y \]
  Then
  \begin{center}
    \begin{tabular}{|c|c|}
      \hline
      TID & items bought \\ \hline
      1 & M, O, K, E, Y \\ \hline
      2 & O, K, E, Y \\ \hline
      3 & M, K, E \\ \hline
      4 & M, K, Y \\ \hline
      5 & O, K, E \\ \hline
    \end{tabular}
  \end{center}

  \begin{enumerate}
    \item apply the GR Th to $A = \emptyset$
    \begin{center}
      $\max(\rho(\inc(\emptyset, E))\setminus\emptyset) = \max(g(E)\setminus\emptyset) = \max(\rho(E)) = \max(EK) = K \nleqslant E$
      $\max(\rho(\inc(\emptyset, K))\setminus\emptyset) = \max(\rho(K)) = \max(K) = K = K ~\Longrightarrow~$ \\
      $K$ is the next closed frequent set generated \\
      $\textcolor{red}{\emptyset} \prec \textcolor{blue}{E} \prec \textcolor{red}{K}$
    \end{center}

    \item apply the GR Th to $A = K$
    \begin{center}
      $\max(\rho(\inc(K,E))\setminus K) = \max(\rho(EK)\setminus K) = \max(EK\setminus K) = E ~\Longrightarrow~$ \\
      $EK$ is the next closed frequent set generated \\
      $\textcolor{red}{\emptyset} \prec \textcolor{blue}{E} \prec \textcolor{red}{K} \prec \textcolor{red}{EK}$
    \end{center}

    \item apply the GR Th to $A = EK ~\Longrightarrow~  i \in I \setminus A = \left\{ {M, O, Y} \right\}$
    \begin{center}
      $\max(\rho\inc(EK, M))\setminus EK) = \max(\rho(M)\setminus EK) = $ \\
      using $(EK \cup M)\setminus \left\{ {j \in I: j < M}\right\} = \left\{ {E, K} \right\} ~\Longrightarrow~ \left\{ {M}\right\}$
      $= \max(MK\setminus EK) = \max(M\setminus E) = M  \leqslant M ~\Longrightarrow~$ \\
      $M$ is the next closed frequent set generated \\
      $\textcolor{red}{\emptyset} \prec \textcolor{blue}{E} \prec \textcolor{red}{K} \prec \textcolor{red}{EK} \prec \textcolor{blue}{M} \prec \textcolor{blue}{ME} \prec \textcolor{red}{MK}$
    \end{center}

    \item apply the GR Th to $A = MK ~\Longrightarrow~  i \in I \setminus A = \left\{ {E, O, Y} \right\}$
    \begin{center}
      $\max(\rho(\inc(MK, E))\setminus M) = \max(\rho(EKM)\setminus M) = \max(MKE \setminus M) = K \nleqslant E$
      $\max(\rho(\inc(MKO))\setminus M) = \max(\rho(O)\setminus M) = \max(OKE\setminus M) = O \leqslant O ~\Longrightarrow~$ \\
      $OKE$ is the next closed frequent set generated \\
       $\textcolor{red}{\emptyset} \prec \textcolor{blue}{E} \prec \textcolor{red}{K} \prec \textcolor{red}{EK} \prec \textcolor{blue}{M} \prec \textcolor{blue}{EM} \prec \textcolor{red}{KM} \prec \dots \prec \textcolor{red}{OKE}$
    \end{center}

    \item apply the GR Th to $A = OKE ~\Longrightarrow~  i \in I \setminus A = \left\{ {M, Y} \right\}$
    \begin{center}
      $\max(\rho(\inc(OKE, M)) \setminus OKE) \max(\rho(OM)\setminus OKE) =\max(KEOMY\setminus OKE) = \max(MY) = Y \nleqslant M$
      $\max(\rho(\inc(OKE, Y))\setminus OKE) \max(\rho(Y)\setminus OKE) = \max(KY\setminus OKE) = Y \leqslant Y ~\Longrightarrow~$ \\
      $KY$ is the next closed frequent set generated \\
    \end{center}

  \end{enumerate}
  Answer: $\left\{ {K, EK, KM, EKO, KY} \right\}$

  \subsection{Two missing numbers}
  \begin{centerframebox}
    Let $a,\, b \in [n]$ be distinct numbers and suppose somebody
    shows all elements of $[n] \setminus \{a,\, b\}$ in some arbitrary order to you;
    each element exactly once.
    Your task is to find out the missing numbers $a$ and $b$ using $O(\log n)$ space.
    (See also, Slide 8 of 2023-11-29.)
  \end{centerframebox}
  We can keep track of the sum of the numbers being called out by adding each one in turn to our accumulator:
  $\sum_{x_i \in [n] \setminus \{a, b\}}x_i$.
  The total sum of the sequence is defined by the following formula: $1 + 2 + 3 + ... + n = \frac{n \dot (n+1)}{2}$ \\
  One missing number can be found by subtracting the accumulated sum from the total sum calculated with the formula.

  In order to find two missing numbers, we can as well define a second accumulator: $\sum_{x_i \in [n] \setminus{\,} \{a, b\}}x_i^2$. The total sum of squares is denoted by the formula:
  $1^2 + 2^2 + 3^2 + ... + n^2 = \frac{n \cdot (n+1) \cdot (2n+1)}{6}$.

  Therefore, to find $a$ and $b$, we solve the following system of equations:
  \begin{equation*}
    \left\{\begin{array}{@{}l@{}}
      a+b=\frac{n \cdot (n+1)}{2} - \sum_{x_i \in [n] \setminus{\,} \{a, b\}}x_i \\
      a^2 + b^2 = \frac{n \cdot (n+1) \cdot (2n+1)}{6} - \sum_{x_i \in [n] \setminus{\,} \{a, b\}}x_i^2
    \end{array}\right.
  \end{equation*}

  \subsection{Identifying the Majority}
  \begin{centerframebox}
    Solve the puzzle on Slide 9 of 2023-11-29.

    \textbf{Problem}: From a long list of names read out, some of them many times, return the name which was called the majority of the time, if there is such a name.

    \textbf{Resources available}: one counter + ability to keep just one name at a time in your mind
  \end{centerframebox}
  To solve this puzzle, we can use The Misra-Gries Algorithm with $k = 2$.
  At the very start we memorize the first name received and add 1 to the counter.
  If the next name is not the same, we subtract 1 from our counter, otherwise we add 1 to our counter.
  When our counter reaches 0, we remove the name from memory and go back to the beginning on the next step (we memorize a new name and set the counter to 1).

  This algorithm will result in the majority element being in the memory at the end since it has to occur in the data stream more than $\frac{n}{2}$ times in order to be the majority, therefore even if there is another element in the data stream that is present in $\frac{n}{2}-1$ cases its counter will reach 0 at some point because the longest possible sequence for this element is $\frac{n}{2}-1$.

  % TODO (Sasha she/her)
  % я уже знаю решение 0_0
  % мы используем Misra-Gries Algorithm c k=2
  % то есть мы запоминаем какое то имя, и если видим его, то увеличиваем счётчик, а если не его, то уменьшаем
  % и когда он станет нулём, заменяем на друге имя

  % пусть тут ещё будет эта ссылка
  % https://cortan122.tk/latex_pdfs/сложность%20hw4.pdf
\end{document}
