\documentclass{article}

\usepackage{enumitem}
\usepackage{amsmath}
\usepackage{amsfonts}
\usepackage[dvipsnames]{xcolor}
\usepackage{amssymb}
\usepackage[margin=0.5in]{geometry}
\usepackage[hidelinks, bookmarks=false]{hyperref}

\let\oldemptyset\emptyset
\let\emptyset\varnothing
\let\oldepsilon\epsilon
\let\epsilon\varepsilon
\newcommand{\N}{\mathbb{N}}
\newcommand{\Z}{\mathbb{Z}}
\newcommand{\R}{\mathbb{R}}
\newcommand{\Q}{\mathbb{Q}}

\usepackage{environ}
\NewEnviron{centerframebox}{\begin{center}\fbox{\parbox{0.92\textwidth}{\BODY}}\end{center}}

\title{Combinatorial Optimization \\ Exercise Set 12 \\ Tuesday class}
\author{
  \AA{AAAAAAAAAA AAAAAAA}{6} \\
  \href{mailto:\AA{AAAAAAAAAAAAAAAAAAAA}{7}}{\AA{AAAAAAAAAAAAAAAAAAAA}{7}}
  \and
  Carola Ley \\
  \href{mailto:s6caleyy@uni-bonn.de}{s6caleyy@uni-bonn.de}
  \and
  Bailee Zacovic \\
  \href{mailto:s38bzaco@uni-bonn.de}{s38bzaco@uni-bonn.de}
}

\begin{document}
  \maketitle

  \setcounter{section}{12}
  \subsection{Making a Forest $2$-edge-connected}
  \begin{centerframebox}
    Let $G = (V, E)$ be a forest with $|V| \geq 2$.
    Let $p := |\{v \in V : |\delta_G(v)| = 1\}|$ and $q := |\{v \in V : |\delta_G(v)| = 0\}|$.
    Show that the minimum cardinality of a set $F \subseteq \{\{v,\, w\} : v,\, w \in V\}$
    of new edges that makes the graph $(V,\, E \cup F)$ 2-edge-connected is $\left\lceil\frac{1}{2} p\right\rceil + p$.
  \end{centerframebox}
  % Divide the graph $G$ into $n$ connected components.
  Let $n$ be the number of connected components of $G$.
  Because $G$ is a forest, each connected component $T$ will be a tree.
  All trees with $|V(T)| = 1$ have degree zero and are included in $q$,
  and all other trees with $|V(T)| > 1$ will have at least two vertices with degree $1$:
  if there is only one leaf, then the second degree $1$ vertex must be the root.
  To make the graph 2-edge-connected, we have to make sure every vertex is included in some cycle.

  Consider the following algorithm:
  \begin{enumerate}
    \item Connect all $q$ zero degree vertices $\{v \in V : |\delta_G(v)| = 0\}$ into one big cycle.
    This uses up exactly $q$ edges.
    We will use that cycle later.

    \item For each connected component $T_i$ of $G$, find two 1 degree vertices $a_i$ and $b_i$,
    such that the path $P(a_i,\, b_i)$ goes through the root of $T_i$.
    Connect them in a cycle with edges $\{a_i,\,b_{i+1 \mod n}\}$.
    This operation uses up $n$ edges and decreases the value of $p$ by $2n$.
    Every vertex in the union of the paths from $a_i$ to $b_i$ will be 2-edge-connected to every other vertex in that union.

    \item Now we can merge those two big cycles, by picking two edges $\{a,\, b\}$ and $\{c,\, d\}$,
    and replacing them by $\{a,\, c\}$ and $\{c,\, b\}$.
    This preserves the number of added edges.

    \item For each remaining 1 degree vertex, connect it to another 1 degree vertex in a different connected component.
    This will form another cycle, intersecting the Big Cycle we created in step 2,
    and the paths from those vertices to their corresponding tree roots will also be 2-edge-connected to each other.
    This process, along with step 2, uses up $\left\lfloor\frac{1}{2} p\right\rfloor$ edges.

    \item If $p$ was odd, we need to take care of an extra leftover vertex.
    We can connect it to the $a_i$ vertex of its connected component,
    and then it will also be 2-edge-connected to every other vertex in $G$.
  \end{enumerate}
  This algorithm produces a 2-edge-connected graph by adding exactly $\left\lceil\frac{1}{2} p\right\rceil + p$ edges.
  This is the minimum possible value, because each of the vertices counted in $p$ had to be included in one edge,
  and each vertex counted in $q$ has to be included in two edges.

  PS: consider the edge case of $|V| = 2$.
  In that case the only way to make $G$ 2-edge-connected will be to include a doubled up edge,
  which is technically not allowed.
  I think the condition in the exercise statement should've been written as $|V| \geq 2$.
  % TODO (Kotya)

  \subsection{$k$-edge-connected graph}
  \begin{centerframebox}
    Show that every edge-minimal $k$-edge-connected graph has two vertices of degree $k$.
  \end{centerframebox}
  TODO (Carola)

  \subsection{Survivable Network Design Problem}
  \begin{centerframebox}
    Consider the LP relaxation of the \textsc{Survivable Network Design Problem} described in the lecture.
    Assume that $f$ is obtained from an instance of the SNDP rather than being an arbitrary proper function.
    Show that the SNDLP can be formulated as an LP of polynomial size in this case.
    Specifically, show that there is an LP of polynomial size with variables $(x_e)_e \in E(G) \cup X$
    such that $x^* \in \R^{E(G)}$  is feasible for the SNDLP if and only if there is $y^* \in \R^X$
    such that $(x^*,\, y^*)$ is feasible for your LP.
  \end{centerframebox}
  TODO (Bailee)

  \subsection{Point-to-Point Connection Problem}
  \begin{centerframebox}
    Find a $2$-factor approximation algorithm for the \textsc{Point-to-Point Connection Problem}:
    Given an undirected graph $G$ with edge weights $c : E(G) \to \R_+$ and sets $S,\, T \subseteq V (G)$
    with $S \cap T = \emptyset$ and $|S| = |T| \geq 1$, find a set
    $F \subseteq E(G)$ of minimum cost such that there is a bijection $\pi: S \to T$ and paths
    from $s$ to $\pi(s)$ for all $s \in S$ in $(V(G),\, F)$.

    Hint: You may use that Jain's algorithm computes a $2$-approximation to the
    SNDIP for arbitrary proper functions $f$.
  \end{centerframebox}
  Let $f:\mathbb{2}^{V(G)}\rightarrow \mathbb{Z}_{\geq 0}$ be given as follows:
  $$f(U):=\begin{cases}
      1 & |U\cap S|\neq |U\cap T|\\
      0 & \text{else.}
  \end{cases}$$
  We claim $f$ is proper. First, observe that (1) $f(\emptyset)=0$ since $|\emptyset\cap S|=0=|\emptyset\cap T|.$ Secondly, $$|V\setminus U\cap S|+|U\cap S|=|S|=|T|=|V\setminus U\cap T|+|U\cap T|,$$i.e. $|U\cap S|=|U\cap T|$ if and only if $|V\setminus U\cap S|=|V\setminus U\cap T|.$ Consequently, $f(U)=f(V\setminus U).$ Finally, for $A,B\subseteq V,$ we have $$|(A\cup B)\cap S|=|A\cap S|+|B\cap S|-|A\cap B\cap S|$$and
  $$|(A\cup B)\cap T|=|A\cap T|+|B\cap T|-|A\cap B\cap T|.$$In the case that $f(A)=f(B)=0,$ we have $|A\cap S|=|A\cap T|$ and $|B\cap S|=|B\cap T|,$ hence $$|A\cap S|+|B\cap S|=|A\cap T|+|B\cap T|\implies |(A\cup B)\cap S|=|(A\cup B)\cap T|,$$and thus $f(A\cup B)=0=\text{max}\; \{f(A),f(B)\}.$ In all other cases, $f(A\cup B)\leq 1=\text{max}\; \{f(A),f(B)\}.$ This yields condition (3) for properness, and hence $f$ is a proper function. Via Jain's algorithm, we compute a $2$-approximation to the SNDIP for $f$ a proper function. In particular, we obtain $F:=E_{\text{sol}},$ and notice that the resulting bijection $\pi:S\rightarrow T$ may be given in the following way:

  Label the elements of $S$ as $s_0,\dots, s_m.$ Let $U_{0,0}:=\{s_0\}.$ Since $|U_{0,0}\cap S|=1\neq 0=|U_{0,0}\cap T|$ (by $S\cap T=\emptyset$), there exists $v_{0,0}\in \Gamma(U_{0,0})$ such that $\{s_0,v_{0,0}\}\in F.$ If $v_{0,0}\in T,$ set $\pi(s_0):=v_{0,0}.$ Else, set $U_{0,1}:=\{s_0,v_{0,0}\}.$ Again, since $|U_{0,1}\cap S|\geq 1\neq 0=|U_{0,1}\cap T|,$ there exists $v_{0,1}\in \Gamma(U_{0,1})$ such that $\{v_{0,1},v_{0,0}\}\in F$ or $\{v_{0,1},s_0\}\in F.$ If $v_{0,1}\in T,$ set $\pi(s_0):=v_{0,1}.$ Else, set $U_{0,2}:=\{s_0,v_{0,0},v_{0,1}\},$ and since $|S|,|T|$ are finite, this process will eventually terminate at some $v_{0,n_0}\in T,$ $n_0\in \mathbb{N},$ at which point we may set $\pi(s_0):=v_{0,n_0}.$

  We repeat this process for each $s_i\in S,$ recursively setting $U_{i,0}=\{s_i\}\cup U_{i-1,n_i}.$ This will generate unique images $\pi(s)\in T$ for each $s\in S.$ $\blacksquare$

\end{document}
